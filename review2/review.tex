\documentclass{article}
\usepackage[utf8]{inputenc}
\usepackage[UKenglish]{babel}
\usepackage[UKenglish]{isodate}
\usepackage{fullpage}
\usepackage[affil-it]{authblk}
\usepackage{pdfpages}
\usepackage{amsmath}
\usepackage{amsfonts}
\usepackage[capitalise]{cleveref}
\usepackage{mathrsfs}
\usepackage{complexity}

\title{Probabilistic Inference in Graphical and Relational Models}
\author{Paulius Dilkas \\[1ex] {\small Supervisors: Mr Vaishak Belle and Dr Ron
    Petrick}}
\affil{School of Informatics, University of Edinburgh}
% in total, maybe ~5 pages before references and appendices

\begin{document}
\maketitle

\section{Introduction} % TODO: gather ~50 most relevant references

\section{Background} % 2-3 pages

\subsection{Probabilities}
\subsection{Representations}
RDDL, probabilistic programming, ProbLog, MLNs, Markov networks, BNs, relational
BNs.
\subsection{Applications}
Chapter 1 of SRAI: applications of ProbLog, BNs, relational BNs
\subsection{Learning}
program induction, probabilistic program induction, structure learning for MLNs.
\subsection{Inference}
This is where my contribution lies. WMC, WMI, WFOMC, some kind of conditioning
algorithms.

\section{Progress To Date}

\begin{description}
\item[WP 1] (`On the Equivalence of Constants in Relational Knowledge Bases')
  was abandoned. While trying to take reviewer feedback into consideration as
  well as update and strengthen the paper, I found an important ambiguity: when
  defining what constants are `captured' and `transferred' by a clause, I fail
  to specify whether each relevant `spot' is occupied by a constant or a
  variable. In the former case, that makes the main theorem of the paper
  completely trivial. While in the latter case, the theorem becomes incorrect. I
  spent a couple of weeks looking for ways to transform the paper into something
  both correct and valuable. The best idea I could find was to use the
  perspective from this paper in the context of inductive logic programming;
  however, I did not want to explore this direction further.
\item[WP 2] (`Generating Random Logic Programs Using Constraint
  Programming') was revised (see \cref{app:cp}) and published and presented in
  CP~2020. I also gave three more talks about it: % TODO: how did it change?
  \begin{itemize}
  \item at the local AIAI seminar,
  \item for the Formal Analysis, Theory and Algorithms research section at the
    University of Glasgow,
  \item and at the FMAI~2021 workshop.
  \end{itemize}
\item[WP 3] morphed into \cref{app:uai} and was submitted to AAAI~2021 and
  UAI~2021. I was also invited to the program committee for UAI~2021.
\item[WP 4] was abandoned due to lack of contributions that could be made. % TODO: expand on this.
\item A new paper was written and submitted to SAT (see \cref{app:sat}). It is
  based on \textbf{WP 3}, addresses the same issue, and suggests an improved
  solution.
  % TODO: and on this.
\end{description}

% TODO: Emphasise the narrative (why I made the decisions that I have and how
% everything fits into a single picture)

\section{Future Goals}

\begin{itemize}
\item Parameterized Complexity of Weighted Model Counting in Theory and
  Practice 
\item Weighted Pseudo-Boolean Model Counting (if I can find a good application)
\end{itemize}

\bibliographystyle{acm}
\bibliography{review}

\includepdf[pages=-,pagecommand=\thispagestyle{plain},picturecommand*={%
  \put(70,750){%
    \parbox{\textwidth}{\appendix\section{Published Paper (CP 2020)}\label{app:cp}}
  }}]{../../published/random-logic-programs/paper/paper.pdf}
\includepdf[pages=-,pagecommand=\thispagestyle{plain},picturecommand*={%
  \put(70,750){%
    \parbox{\textwidth}{\section{Submitted Paper (AAAI 2021 and UAI 2021)}\label{app:uai}}
  }}]{../../conditional-wmc/doc/paper2/paper.pdf}
\includepdf[pages=-,pagecommand=\thispagestyle{plain},picturecommand*={%
  \put(70,750){%
    \parbox{\textwidth}{\section{Submitted Paper (SAT 2021)}\label{app:sat}}
  }}]{../../conditional-wmc/doc/paper3/paper.pdf}

\end{document}