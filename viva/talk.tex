\documentclass{beamer}
\usepackage[utf8]{inputenc}
\usepackage[UKenglish]{babel}
\usepackage[UKenglish]{isodate}
\usepackage{booktabs}
% \usepackage{amsmath}
\usepackage{mathtools}
\usepackage{tikz}
% \usepackage{complexity}
\usepackage{empheq}
\usepackage{spot}

% \usetikzlibrary{arrows}
% \usetikzlibrary{positioning}
\usetikzlibrary{overlay-beamer-styles}
\usetikzlibrary{arrows.meta}
\usetikzlibrary{shapes}
\usetikzlibrary{decorations.pathreplacing}

\tikzstyle{every picture}+=[remember picture]

\beamertemplatenavigationsymbolsempty

\definecolor{predicate}{HTML}{1b9e77}
\definecolor{probability}{HTML}{e7298a}
\definecolor{variable}{HTML}{d95f02}
\definecolor{constant}{HTML}{7570b3}
\definecolor{color5}{HTML}{66a61e}
\definecolor{color6}{HTML}{e6ab02}

\DeclareMathOperator{\ifff}{:-}
\DeclareMathOperator{\prob}{::}
\DeclareMathOperator{\negg}{\backslash+}

\author{Paulius Dilkas}
\title{Generalising Weighted Model Counting}
\institute{University of Edinburgh, UK}
\date{Viva Voce}

\begin{document}
%\addtobeamertemplate{block begin}{\setlength\abovedisplayskip{0pt}}

\begin{frame}[noframenumbering,plain]
  \tikz[remember picture,overlay]{
    \node at ([yshift=25pt,xshift=30pt]current page.south)
    {\includegraphics[height=40pt]{inf.png}};
    \node at ([yshift=25pt,xshift=75pt]current page.south)
    {\includegraphics[height=40pt]{ecr.jpg}};
    \node at ([yshift=20pt,xshift=140pt]current page.south)
    {\includegraphics[height=20pt]{epsrc.png}};
  }
  \titlepage
\end{frame}

\begin{frame}{Weighted Model Counting}
  \begin{example}
    We have a biased coin that has a probability \structure{$p \in [0, 1]$} of
    landing heads. What is the probability that it lands heads \alert{at least
      once} if we toss it \alert{three times}?
  \end{example}
  \begin{block}{In Propositional Logic\ldots}
    \begin{itemize}
      \item Formula: \structure{$x_{1} \lor x_{2} \lor x_{3}$}
      \item Weights: \structure{$w(x_{i}) = p$}, \structure{$w(\neg x_{i}) = 1-p$} for \structure{$i = 1, 2, 3$}
      \item Models: \structure{$\mathcal{P}(\{\, x_{1}, x_{2}, x_{3} \,\}) \setminus \{\, \emptyset \,\}$}
    \end{itemize}
  \end{block}
  \begin{block}{In First-Order Logic\ldots}
    \begin{itemize}
      \item Formula: \structure{$\exists x \in \{\,1, 2, 3\,\}\text{. }P(x)$}
      \item Weights: \structure{$w(P) = p$}, \structure{$w(\neg P) = 1 - p$}
      \item Models: \structure{$\mathcal{P}(\{\, P(1), P(2), P(3) \,\}) \setminus \{\, \emptyset \,\}$}
    \end{itemize}
  \end{block}
\end{frame}

\begin{frame}{Significance of WMC and This Work}
  \begin{block}{Applications}
  \begin{itemize}
    \item Probabilistic inference: graphical models, statistical relational models, probabilistic programming
    \item Neural-symbolic artificial intelligence
    \item Bioinformatics
    \item Robotics
    \item Natural language processing
    \item Enumerative combinatorics
  \end{itemize}
  \end{block}
  \begin{block}{Impact}
    \begin{columns}
      \begin{column}{0.50\textwidth}
        \begin{itemize}
          \item Suitable WMC algorithm
          \item Appropriate input format
          \item Lifted reasoning
          \item Expressive data structures
        \end{itemize}
      \end{column}
      \begin{column}{0.1\textwidth}
        \centering
        \includegraphics[width=\textwidth]{arrow.png}
      \end{column}
      \begin{column}{0.40\textwidth}
        \begin{itemize}
          \item provable tractability
          \item experimental speedup
        \end{itemize}
      \end{column}
    \end{columns}
  \end{block}
\end{frame}

\begin{frame}{Contributions}
  \begin{columns}[t]
    \begin{column}{0.46\textwidth}
      \centering
      \includegraphics[width=0.5\textwidth]{trees.png}
      \begin{block}{Generalising Representations}
        \begin{itemize}
          \item Beyond weights on literals
          \item Circuits for recursion
        \end{itemize}
      \end{block}
    \end{column}
    \begin{column}{0.54\textwidth}
      \centering
      \includegraphics[width=0.5\textwidth]{dice}
      \begin{block}{Random-Instance Experiments}
        \begin{itemize}
          \item Application-specific parameters%
          \begin{itemize}
            \item \textsc{ProbLog} predicates, arities
          \end{itemize}
          \item Parameters of hardness
          \begin{itemize}
            \item density, primal treewidth
          \end{itemize}
        \end{itemize}
      \end{block}
    \end{column}
  \end{columns}
\end{frame}

\begin{frame}
  \vfill
  \centering
  \begin{beamercolorbox}[sep=8pt,center,shadow=true,rounded=true]{title}
    \usebeamerfont{title}Generalising Representations\par%
  \end{beamercolorbox}
  \vfill
\end{frame}

\begin{frame}{WMC and Measures on Boolean Algebras}
  \begin{definition}
    A \alert{measure} is a function
    \structure{$\mu\colon \mathcal{P}(\mathcal{P}(X)) \to \mathbb{R}_{\ge 0}$}
    such that:
    \begin{itemize}
      \item \structure{$\mu(\bot) = 0$};
      \item \structure{$\mu(x \lor y) = \mu(x) + \mu(y)$} whenever
            \structure{$x \land y = \bot$}.
    \end{itemize}
  \end{definition}
  \begin{block}{Observation}
    WMC corresponds to the process of calculating the value of
    \structure{$\mu(x)$} for some
    \structure{$x \in \mathcal{P}(\mathcal{P}(X))$}.
  \end{block}
  \pause
  \begin{block}{Observation}
    Classical WMC is only able to evaluate \alert{factorable} measures (c.f., a
    collection of mutually independent random variables).
  \end{block}
  \begin{theorem}[Informal Version]
    It is always possible to add more variables to turn a non-factorable measure
    into a factorable measure.
  \end{theorem}
  However, that is not necessarily a good idea!
\end{frame}

% \begin{frame}{Experiments with Bayesian Networks}
%     \hspace*{-0.5cm}\input{scatter1}
% \end{frame}

\begin{frame}{Transforming Known WMC Encodings into PBP}
  For any propositional formula \structure{$\phi$} over a set of variables
  \structure{$X$} and \structure{$p, q \in \mathbb{R}$}, let
  \structure{$[\phi]^p_q\colon 2^X \to \mathbb{R}$} be the pseudo-Boolean
  function defined as
  \[
    [\phi]^p_q(Y) \coloneqq
    \begin{cases}
      p & \text{if } Y \models \phi \\
      q & \text{otherwise}
    \end{cases}
  \]
  for any \structure{$Y \subseteq X$}.
  \begin{example}
    \begin{center}
      \begin{tabular}{llll}
        \toprule
        Clauses & In CNF & Pseudo-Boolean Functions & \\
        \midrule
        $\neg x \Rightarrow p$ & $x \lor p$ & $[\neg x]_1^{0.2}$\tikz \coordinate (one); & \\
        $p \Rightarrow \neg x$ & $\neg x \lor \neg p$ & & \tikz \coordinate (two); $[x]^{0.8}_{0.2}$ \\
        $x \Rightarrow q$ & $\neg x \lor q$ & $[x]_1^{0.8}$ \tikz \coordinate (three); & \\
        $q \Rightarrow x$ & $x \lor \neg q$ & & \\
        $\neg x$ & $\neg x$ & $[\neg x]_0^1$ & $[\neg x]_0^1$ \\
        \bottomrule
      \end{tabular}
    \end{center}
  \end{example}
  \begin{tikzpicture}[remember picture,overlay]
    \draw[-latex,blue] (one) -- (two);
    \draw[-latex,blue] (three) -- (two);
  \end{tikzpicture}
\end{frame}

% \begin{frame}{Some Instances Become Tractable as a Result}
%   \centering
%   \includegraphics[width=\textwidth]{scatter2}
% \end{frame}

\begin{frame}{First-Order Logic and Recursive Computations}
  \begin{example}[Counting \structure{$P\colon M \to N$} Injections]
    \begin{columns}
      \begin{column}{0.7\textwidth}
        \begin{block}{Input Formula}
          \begin{gather*}
            \forall x \in M\text{. }\exists y \in N\text{. }P(x, y) \\
            \forall x \in M\text{. }\forall y, z \in N\text{. }P(x, y) \land P(x, z) \Rightarrow y=z \\
            \forall w, x \in M\text{. }\forall y \in N\text{. }P(w, y) \land P(x, y) \Rightarrow w=x
          \end{gather*}
        \end{block}
      \end{column}
      \begin{column}{0.3\textwidth}
        \centering
        \begin{tikzpicture}[every node/.style={draw,ellipse},edge from parent/.style={draw,-latex},sibling distance=10mm,level distance=10mm]
          \node[fill=red!50] (dr) {}
          child {node {}
            child {node[fill=red!50] {}
              child {node {}
                child {node {}}
                child {node[fill=red!50] (ref) {}}
              }}};
          \draw[-latex, bend right,red] (ref) to (dr);
        \end{tikzpicture}
      \end{column}
    \end{columns}
    \begin{block}{Recursive Solution}
    \[
      f(m, n) =
      \begin{cases}
        1 & \text{if } m = 0 \text{ and } n = 0 \\
        0 & \text{if } m > 0 \text{ and } n = 0 \\
        f(m, n-1) + m \cdot f(m-1, n-1) & \text{otherwise.}
      \end{cases}
    \]
    \end{block}
  \end{example}
\end{frame}

% \begin{frame}{First-Order Knowledge Compilation}
%   \begin{block}{Workflow Before}
%     \begin{enumerate}
%       \item Compile the formula to a \alert{circuit}
%       \item Evaluate to get the answer
%     \end{enumerate}
%   \end{block}
%   \begin{block}{Workflow After}
%     \begin{enumerate}
%       \item Compile the formula to a \alert{graph}
%       \item Extract the definitions of functions
%       \item Simplify
%       \item Supplement with \alert{base cases}
%       \item Evaluate to get the answer
%     \end{enumerate}
%   \end{block}
% \end{frame}

\begin{frame}{Resulting Improvements to Counting Functions}
  Let \structure{$M$} and \structure{$N$} be two sets with cardinalities
  \structure{$|M| = m$} and \structure{$|N| = n$}.

  The new compilation rules enable \textsc{ForcLift} to efficiently count
  \structure{$M \to N$} functions such as:
  \begin{itemize}
    \item injections in \structure{$\Theta(mn)$} time
          \begin{itemize}
            \item best: \structure{$\Theta(m)$}
          \end{itemize}
    \item partial injections in \structure{$\Theta(mn)$} time
          \begin{itemize}
            \item best: \structure{$\Theta({\min\{\, m, n \,\}}^2)$}
          \end{itemize}
          \item bijections in \structure{$\Theta(m)$} time
          \begin{itemize}
            \item \alert{optimal!}
          \end{itemize}
  \end{itemize}
\end{frame}

\begin{frame}
  \vfill
  \centering
  \begin{beamercolorbox}[sep=8pt,center,shadow=true,rounded=true]{title}
    \usebeamerfont{title}Random-Instance Experiments\par%
  \end{beamercolorbox}
  \vfill
\end{frame}

\begin{frame}{A Constraint Model for (Probabilistic) Logic Programs}
  \begin{columns}
    \hspace*{-0.7cm}\begin{column}{0.75\textwidth}
      \begin{empheq}[left ={\color{color5}\empheqlbrace}]{equation}
        \begin{align*}
      \textcolor{probability}{0.2} \prob &\textcolor{predicate}{\mathtt{stress}}(\textcolor{variable}{\mathsf{P}}) \ifff \textcolor{predicate}{\mathtt{person}}(\textcolor{variable}{\mathsf{P}}). \\
      \textcolor{probability}{0.3} \prob &\textcolor{predicate}{\mathtt{influences}}(\textcolor{variable}{\mathsf{P}_1}, \textcolor{variable}{\mathsf{P}_2}) \ifff \textcolor{predicate}{\mathtt{friend}}(\textcolor{variable}{\mathsf{P}_1}, \textcolor{variable}{\mathsf{P}_2}). \\
      \textcolor{probability}{0.1} \prob &\textcolor{predicate}{\mathtt{cancer\_spont}}(\textcolor{variable}{\mathsf{P}}) \ifff \textcolor{predicate}{\mathtt{person}}(\textcolor{variable}{\mathsf{P}}). \\
      \textcolor{probability}{0.3} \prob &\textcolor{predicate}{\mathtt{cancer\_smoke}}(\textcolor{variable}{\mathsf{P}}) \ifff \textcolor{predicate}{\mathtt{person}}(\textcolor{variable}{\mathsf{P}}). \\
                                         &\textcolor{predicate}{\mathtt{smokes}}(\textcolor{variable}{\mathsf{X}}) \ifff \textcolor{predicate}{\mathtt{stress}}(\textcolor{variable}{\mathsf{X}}). \\
                                         &\textcolor{predicate}{\mathtt{smokes}}(\textcolor{variable}{\mathsf{X}}) \ifff \textcolor{predicate}{\mathtt{smokes}}(\textcolor{variable}{\mathsf{Y}}), \textcolor{predicate}{\mathtt{influences}}(\textcolor{variable}{\mathsf{Y}}, \textcolor{variable}{\mathsf{X}}). \\
                                         &\textcolor{predicate}{\mathtt{cancer}}(\textcolor{variable}{\mathsf{P}}) \ifff \textcolor{predicate}{\mathtt{cancer\_spont}}(\textcolor{variable}{\mathsf{P}}). \\
                                         &\textcolor{predicate}{\mathtt{cancer}}(\textcolor{variable}{\mathsf{P}}) \ifff \tikz \coordinate (start);\textcolor{predicate}{\mathtt{smokes}}(\textcolor{variable}{\mathsf{P}}), \textcolor{predicate}{\mathtt{cancer\_smoke}}(\textcolor{variable}{\mathsf{P}})\tikz \coordinate (end);. \\
                                         &\textcolor{predicate}{\mathtt{person}}(\textcolor{constant}{\mathit{mary}}). \\
                                         &\textcolor{predicate}{\mathtt{person}}(\textcolor{constant}{\mathit{albert}}). \\
                                         &\textcolor{predicate}{\mathtt{friend}}(\textcolor{constant}{\mathit{albert}}, \textcolor{constant}{\mathit{mary}}).
        \end{align*}
      \end{empheq}
    \end{column}
    \begin{column}{0.25\textwidth}
      \begin{itemize}
      \item[\textcolor{predicate}{\textbullet}] predicates, arities
      \item[\textcolor{variable}{\textbullet}] variables
      \item[\textcolor{constant}{\textbullet}] constants
      \item[\textcolor{probability}{\textbullet}] probabilities
      \item[\textcolor{color5}{\textbullet}] length
      \item[\textcolor{color6}{\textbullet}] complexity
      \end{itemize}
    \end{column}
  \end{columns}
  \begin{tikzpicture}[overlay]
    \draw [decorate,decoration={brace,amplitude=10pt,mirror},color=color6] (start) -- (end);
  \end{tikzpicture}
\end{frame}

\begin{frame}{\textsc{ProbLog} Inference Algorithms on Random Instances}
  \centering
  % Created by tikzDevice version 0.12.3.1 on 2022-06-27 14:34:14
% !TEX encoding = UTF-8 Unicode
\begin{tikzpicture}[x=1pt,y=1pt]
\definecolor{fillColor}{RGB}{255,255,255}
\path[use as bounding box,fill=fillColor,fill opacity=0.00] (0,0) rectangle (411.94,224.04);
\begin{scope}
\path[clip] (  0.00,  0.00) rectangle (411.94,224.04);
\definecolor{drawColor}{RGB}{255,255,255}
\definecolor{fillColor}{RGB}{255,255,255}

\path[draw=drawColor,line width= 0.5pt,line join=round,line cap=round,fill=fillColor] (  0.00,  0.00) rectangle (411.94,224.04);
\end{scope}
\begin{scope}
\path[clip] ( 25.95, 25.11) rectangle (182.27,205.98);
\definecolor{fillColor}{RGB}{255,255,255}

\path[fill=fillColor] ( 25.95, 25.11) rectangle (182.27,205.98);
\definecolor{drawColor}{gray}{0.92}

\path[draw=drawColor,line width= 0.2pt,line join=round] ( 25.95, 61.00) --
	(182.27, 61.00);

\path[draw=drawColor,line width= 0.2pt,line join=round] ( 25.95,116.62) --
	(182.27,116.62);

\path[draw=drawColor,line width= 0.2pt,line join=round] ( 25.95,172.24) --
	(182.27,172.24);

\path[draw=drawColor,line width= 0.2pt,line join=round] ( 50.63, 25.11) --
	( 50.63,205.98);

\path[draw=drawColor,line width= 0.2pt,line join=round] ( 86.21, 25.11) --
	( 86.21,205.98);

\path[draw=drawColor,line width= 0.2pt,line join=round] (121.79, 25.11) --
	(121.79,205.98);

\path[draw=drawColor,line width= 0.2pt,line join=round] (157.37, 25.11) --
	(157.37,205.98);

\path[draw=drawColor,line width= 0.5pt,line join=round] ( 25.95, 33.19) --
	(182.27, 33.19);

\path[draw=drawColor,line width= 0.5pt,line join=round] ( 25.95, 88.81) --
	(182.27, 88.81);

\path[draw=drawColor,line width= 0.5pt,line join=round] ( 25.95,144.43) --
	(182.27,144.43);

\path[draw=drawColor,line width= 0.5pt,line join=round] ( 25.95,200.05) --
	(182.27,200.05);

\path[draw=drawColor,line width= 0.5pt,line join=round] ( 32.84, 25.11) --
	( 32.84,205.98);

\path[draw=drawColor,line width= 0.5pt,line join=round] ( 68.42, 25.11) --
	( 68.42,205.98);

\path[draw=drawColor,line width= 0.5pt,line join=round] (104.00, 25.11) --
	(104.00,205.98);

\path[draw=drawColor,line width= 0.5pt,line join=round] (139.58, 25.11) --
	(139.58,205.98);

\path[draw=drawColor,line width= 0.5pt,line join=round] (175.16, 25.11) --
	(175.16,205.98);
\definecolor{drawColor}{RGB}{27,158,119}

\path[draw=drawColor,line width= 0.6pt,line join=round] ( 33.05, 33.37) --
	( 33.27, 33.52) --
	( 33.69, 33.91) --
	( 34.26, 34.58) --
	( 34.55, 34.59) --
	( 36.25, 35.83) --
	( 43.62, 40.94) --
	( 43.83, 38.43) --
	( 44.26, 41.84) --
	( 47.07, 45.62) --
	( 54.19, 49.79) --
	( 54.40, 49.96) --
	( 54.83, 50.42) --
	( 64.97, 59.37) --
	( 65.40, 60.03) --
	( 75.54, 70.46) --
	( 75.75, 64.76) --
	( 75.96, 70.64) --
	( 76.18, 62.32) --
	( 77.03, 81.27) --
	( 86.53, 77.35) --
	( 97.10, 85.28) --
	(107.67, 88.88) --
	(118.23,127.89) --
	(118.66,107.19) --
	(119.51,105.00) --
	(161.14,139.37) --
	(162.00,136.90) --
	(175.16,167.44);
\definecolor{drawColor}{RGB}{217,95,2}

\path[draw=drawColor,line width= 0.6pt,dash pattern=on 2pt off 2pt ,line join=round] ( 33.05, 33.33) --
	( 33.27, 33.46) --
	( 33.69, 33.82) --
	( 34.26, 34.50) --
	( 34.55, 34.44) --
	( 36.25, 35.73) --
	( 43.62, 41.62) --
	( 43.83, 40.00) --
	( 44.26, 40.20) --
	( 47.07, 45.01) --
	( 54.19, 52.39) --
	( 54.40, 49.18) --
	( 54.83, 51.24) --
	( 64.97, 57.99) --
	( 65.40, 58.62) --
	( 75.54, 67.23) --
	( 75.75, 62.11) --
	( 75.96, 63.39) --
	( 76.18, 64.01) --
	( 77.03, 69.44) --
	( 86.53, 75.49) --
	( 97.10, 82.22) --
	(107.67, 89.78) --
	(118.23,129.12) --
	(118.66, 92.72) --
	(119.51,100.28) --
	(161.14,129.33) --
	(162.00,138.38) --
	(175.16,159.71);
\definecolor{drawColor}{RGB}{117,112,179}

\path[draw=drawColor,line width= 0.6pt,dash pattern=on 4pt off 2pt ,line join=round] ( 33.05, 33.38) --
	( 33.27, 33.64) --
	( 33.69, 34.01) --
	( 34.26, 34.65) --
	( 34.55, 34.66) --
	( 36.25, 36.07) --
	( 43.62, 40.82) --
	( 43.83, 39.50) --
	( 44.26, 41.47) --
	( 47.07, 45.90) --
	( 54.19, 53.47) --
	( 54.40, 51.03) --
	( 54.83, 49.18) --
	( 64.97, 58.96) --
	( 65.40, 60.32) --
	( 75.54, 63.98) --
	( 75.75, 62.13) --
	( 75.96, 66.98) --
	( 76.18, 63.17) --
	( 77.03, 64.95) --
	( 86.53, 77.27) --
	( 97.10, 87.94) --
	(107.67, 85.30) --
	(118.23,119.94) --
	(118.66,100.54) --
	(119.51,111.19) --
	(161.14,129.60) --
	(162.00,138.33) --
	(175.16,165.99);
\definecolor{drawColor}{RGB}{231,41,138}

\path[draw=drawColor,line width= 0.6pt,dash pattern=on 4pt off 4pt ,line join=round] ( 33.05, 33.37) --
	( 33.27, 33.53) --
	( 33.69, 33.97) --
	( 34.26, 34.71) --
	( 34.55, 34.58) --
	( 36.25, 36.25) --
	( 43.62, 41.37) --
	( 43.83, 40.30) --
	( 44.26, 42.18) --
	( 47.07, 47.37) --
	( 54.19, 58.32) --
	( 54.40, 51.70) --
	( 54.83, 47.71) --
	( 64.97, 60.10) --
	( 65.40, 63.85) --
	( 75.54, 73.21) --
	( 75.75, 73.00) --
	( 75.96, 73.21) --
	( 76.18, 76.81) --
	( 77.03, 74.66) --
	( 86.53, 79.83) --
	( 97.10, 88.93) --
	(107.67, 96.07) --
	(118.23,132.20) --
	(118.66,105.84) --
	(119.51, 93.28) --
	(161.14,151.35) --
	(162.00,152.87) --
	(175.16,185.57);
\definecolor{drawColor}{RGB}{102,166,30}

\path[draw=drawColor,line width= 0.6pt,dash pattern=on 1pt off 3pt ,line join=round] ( 33.05, 33.41) --
	( 33.27, 33.58) --
	( 33.69, 33.98) --
	( 34.26, 34.75) --
	( 34.55, 34.67) --
	( 36.25, 36.24) --
	( 43.62, 41.94) --
	( 43.83, 43.49) --
	( 44.26, 43.83) --
	( 47.07, 46.77) --
	( 54.19, 53.21) --
	( 54.40, 51.40) --
	( 54.83, 49.70) --
	( 64.97, 61.26) --
	( 65.40, 61.93) --
	( 75.54, 71.92) --
	( 75.75, 65.30) --
	( 75.96, 70.93) --
	( 76.18, 65.57) --
	( 77.03, 75.13) --
	( 86.53, 82.53) --
	( 97.10, 92.52) --
	(107.67, 97.91) --
	(118.23,145.49) --
	(118.66,114.18) --
	(119.51,112.35) --
	(161.14,141.46) --
	(162.00,141.66) --
	(175.16,180.91);
\definecolor{drawColor}{RGB}{230,171,2}

\path[draw=drawColor,line width= 0.6pt,dash pattern=on 1pt off 3pt on 4pt off 3pt ,line join=round] ( 33.05, 33.33) --
	( 33.27, 33.61) --
	( 33.69, 33.96) --
	( 34.26, 34.72) --
	( 34.55, 34.69) --
	( 36.25, 36.23) --
	( 43.62, 41.81) --
	( 43.83, 37.71) --
	( 44.26, 43.93) --
	( 47.07, 47.37) --
	( 54.19, 56.79) --
	( 54.40, 54.58) --
	( 54.83, 49.17) --
	( 64.97, 62.57) --
	( 65.40, 63.92) --
	( 75.54, 72.06) --
	( 75.75, 76.85) --
	( 75.96, 74.63) --
	( 76.18, 79.27) --
	( 77.03, 72.70) --
	( 86.53, 79.83) --
	( 97.10, 92.37) --
	(107.67,100.72) --
	(118.23,139.80) --
	(118.66,108.14) --
	(119.51,109.97) --
	(161.14,147.98) --
	(162.00,157.02) --
	(175.16,184.87);
\definecolor{drawColor}{gray}{0.20}

\path[draw=drawColor,line width= 0.5pt,line join=round,line cap=round] ( 25.95, 25.11) rectangle (182.27,205.98);
\end{scope}
\begin{scope}
\path[clip] (186.77, 25.11) rectangle (343.09,205.98);
\definecolor{fillColor}{RGB}{255,255,255}

\path[fill=fillColor] (186.77, 25.11) rectangle (343.09,205.98);
\definecolor{drawColor}{gray}{0.92}

\path[draw=drawColor,line width= 0.2pt,line join=round] (186.77, 61.00) --
	(343.09, 61.00);

\path[draw=drawColor,line width= 0.2pt,line join=round] (186.77,116.62) --
	(343.09,116.62);

\path[draw=drawColor,line width= 0.2pt,line join=round] (186.77,172.24) --
	(343.09,172.24);

\path[draw=drawColor,line width= 0.2pt,line join=round] (211.64, 25.11) --
	(211.64,205.98);

\path[draw=drawColor,line width= 0.2pt,line join=round] (247.17, 25.11) --
	(247.17,205.98);

\path[draw=drawColor,line width= 0.2pt,line join=round] (282.69, 25.11) --
	(282.69,205.98);

\path[draw=drawColor,line width= 0.2pt,line join=round] (318.22, 25.11) --
	(318.22,205.98);

\path[draw=drawColor,line width= 0.5pt,line join=round] (186.77, 33.19) --
	(343.09, 33.19);

\path[draw=drawColor,line width= 0.5pt,line join=round] (186.77, 88.81) --
	(343.09, 88.81);

\path[draw=drawColor,line width= 0.5pt,line join=round] (186.77,144.43) --
	(343.09,144.43);

\path[draw=drawColor,line width= 0.5pt,line join=round] (186.77,200.05) --
	(343.09,200.05);

\path[draw=drawColor,line width= 0.5pt,line join=round] (193.87, 25.11) --
	(193.87,205.98);

\path[draw=drawColor,line width= 0.5pt,line join=round] (229.40, 25.11) --
	(229.40,205.98);

\path[draw=drawColor,line width= 0.5pt,line join=round] (264.93, 25.11) --
	(264.93,205.98);

\path[draw=drawColor,line width= 0.5pt,line join=round] (300.46, 25.11) --
	(300.46,205.98);

\path[draw=drawColor,line width= 0.5pt,line join=round] (335.99, 25.11) --
	(335.99,205.98);
\definecolor{drawColor}{RGB}{27,158,119}

\path[draw=drawColor,line width= 0.6pt,line join=round] (193.87,157.23) --
	(198.95,171.27) --
	(209.10,173.13) --
	(214.18,154.17) --
	(217.56,177.01) --
	(234.48,166.16) --
	(239.55,154.85) --
	(254.78,160.85) --
	(270.01,172.52) --
	(275.08,152.64) --
	(280.16,159.77) --
	(288.62,145.00) --
	(290.31,150.87) --
	(295.38,146.14) --
	(300.46,169.51) --
	(305.53,175.62) --
	(310.61,172.64) --
	(325.84,151.72) --
	(330.91,163.01) --
	(335.99,174.38);
\definecolor{drawColor}{RGB}{217,95,2}

\path[draw=drawColor,line width= 0.6pt,dash pattern=on 2pt off 2pt ,line join=round] (193.87,145.07) --
	(198.95,156.90) --
	(209.10,165.05) --
	(214.18,151.81) --
	(217.56,166.07) --
	(234.48,156.98) --
	(239.55,145.37) --
	(254.78,155.22) --
	(270.01,167.53) --
	(275.08,143.18) --
	(280.16,156.65) --
	(288.62,145.98) --
	(290.31,143.25) --
	(295.38,146.20) --
	(300.46,160.59) --
	(305.53,171.18) --
	(310.61,166.90) --
	(325.84,149.46) --
	(330.91,155.36) --
	(335.99,162.09);
\definecolor{drawColor}{RGB}{117,112,179}

\path[draw=drawColor,line width= 0.6pt,dash pattern=on 4pt off 2pt ,line join=round] (193.87,161.43) --
	(198.95,162.33) --
	(209.10,169.77) --
	(214.18,152.37) --
	(217.56,172.94) --
	(234.48,164.74) --
	(239.55,157.69) --
	(254.78,160.22) --
	(270.01,173.36) --
	(275.08,147.24) --
	(280.16,159.97) --
	(288.62,143.54) --
	(290.31,151.70) --
	(295.38,152.85) --
	(300.46,163.90) --
	(305.53,171.18) --
	(310.61,174.17) --
	(325.84,154.34) --
	(330.91,162.96) --
	(335.99,172.18);
\definecolor{drawColor}{RGB}{231,41,138}

\path[draw=drawColor,line width= 0.6pt,dash pattern=on 4pt off 4pt ,line join=round] (193.87,177.27) --
	(198.95,178.69) --
	(209.10,197.76) --
	(214.18,175.05) --
	(217.56,194.04) --
	(234.48,191.09) --
	(239.55,174.02) --
	(254.78,184.14) --
	(270.01,195.54) --
	(275.08,159.62) --
	(280.16,181.32) --
	(288.62,163.66) --
	(290.31,166.71) --
	(295.38,156.97) --
	(300.46,184.31) --
	(305.53,193.27) --
	(310.61,193.07) --
	(325.84,164.71) --
	(330.91,178.22) --
	(335.99,189.16);
\definecolor{drawColor}{RGB}{102,166,30}

\path[draw=drawColor,line width= 0.6pt,dash pattern=on 1pt off 3pt ,line join=round] (193.87,164.65) --
	(198.95,188.00) --
	(209.10,181.00) --
	(214.18,165.31) --
	(217.56,187.20) --
	(234.48,179.63) --
	(239.55,166.93) --
	(254.78,180.22) --
	(270.01,189.33) --
	(275.08,159.43) --
	(280.16,174.40) --
	(288.62,170.07) --
	(290.31,166.16) --
	(295.38,168.97) --
	(300.46,180.18) --
	(305.53,185.51) --
	(310.61,187.91) --
	(325.84,168.55) --
	(330.91,171.82) --
	(335.99,186.12);
\definecolor{drawColor}{RGB}{230,171,2}

\path[draw=drawColor,line width= 0.6pt,dash pattern=on 1pt off 3pt on 4pt off 3pt ,line join=round] (193.87,166.10) --
	(198.95,179.45) --
	(209.10,197.49) --
	(214.18,165.22) --
	(217.56,193.27) --
	(234.48,188.36) --
	(239.55,176.71) --
	(254.78,181.33) --
	(270.01,194.97) --
	(275.08,165.87) --
	(280.16,182.46) --
	(288.62,157.95) --
	(290.31,168.16) --
	(295.38,166.36) --
	(300.46,182.60) --
	(305.53,187.91) --
	(310.61,193.82) --
	(325.84,165.74) --
	(330.91,176.90) --
	(335.99,189.73);
\definecolor{drawColor}{gray}{0.20}

\path[draw=drawColor,line width= 0.5pt,line join=round,line cap=round] (186.77, 25.11) rectangle (343.09,205.98);
\end{scope}
\begin{scope}
\path[clip] ( 25.95,205.98) rectangle (182.27,219.54);
\definecolor{drawColor}{gray}{0.20}
\definecolor{fillColor}{gray}{0.85}

\path[draw=drawColor,line width= 0.5pt,line join=round,line cap=round,fill=fillColor] ( 25.95,205.98) rectangle (182.27,219.54);
\definecolor{drawColor}{gray}{0.10}

\node[text=drawColor,anchor=base,inner sep=0pt, outer sep=0pt, scale=  0.72] at (104.11,210.28) {The number of facts ($\times 10^3$)};
\end{scope}
\begin{scope}
\path[clip] (186.77,205.98) rectangle (343.09,219.54);
\definecolor{drawColor}{gray}{0.20}
\definecolor{fillColor}{gray}{0.85}

\path[draw=drawColor,line width= 0.5pt,line join=round,line cap=round,fill=fillColor] (186.77,205.98) rectangle (343.09,219.54);
\definecolor{drawColor}{gray}{0.10}

\node[text=drawColor,anchor=base,inner sep=0pt, outer sep=0pt, scale=  0.72] at (264.93,210.28) {Proportion of independent pairs};
\end{scope}
\begin{scope}
\path[clip] (  0.00,  0.00) rectangle (411.94,224.04);
\definecolor{drawColor}{gray}{0.20}

\path[draw=drawColor,line width= 0.5pt,line join=round] ( 32.84, 22.86) --
	( 32.84, 25.11);

\path[draw=drawColor,line width= 0.5pt,line join=round] ( 68.42, 22.86) --
	( 68.42, 25.11);

\path[draw=drawColor,line width= 0.5pt,line join=round] (104.00, 22.86) --
	(104.00, 25.11);

\path[draw=drawColor,line width= 0.5pt,line join=round] (139.58, 22.86) --
	(139.58, 25.11);

\path[draw=drawColor,line width= 0.5pt,line join=round] (175.16, 22.86) --
	(175.16, 25.11);
\end{scope}
\begin{scope}
\path[clip] (  0.00,  0.00) rectangle (411.94,224.04);
\definecolor{drawColor}{gray}{0.30}

\node[text=drawColor,anchor=base,inner sep=0pt, outer sep=0pt, scale=  0.72] at ( 32.84, 16.10) {0};

\node[text=drawColor,anchor=base,inner sep=0pt, outer sep=0pt, scale=  0.72] at ( 68.42, 16.10) {25};

\node[text=drawColor,anchor=base,inner sep=0pt, outer sep=0pt, scale=  0.72] at (104.00, 16.10) {50};

\node[text=drawColor,anchor=base,inner sep=0pt, outer sep=0pt, scale=  0.72] at (139.58, 16.10) {75};

\node[text=drawColor,anchor=base,inner sep=0pt, outer sep=0pt, scale=  0.72] at (175.16, 16.10) {100};
\end{scope}
\begin{scope}
\path[clip] (  0.00,  0.00) rectangle (411.94,224.04);
\definecolor{drawColor}{gray}{0.20}

\path[draw=drawColor,line width= 0.5pt,line join=round] (193.87, 22.86) --
	(193.87, 25.11);

\path[draw=drawColor,line width= 0.5pt,line join=round] (229.40, 22.86) --
	(229.40, 25.11);

\path[draw=drawColor,line width= 0.5pt,line join=round] (264.93, 22.86) --
	(264.93, 25.11);

\path[draw=drawColor,line width= 0.5pt,line join=round] (300.46, 22.86) --
	(300.46, 25.11);

\path[draw=drawColor,line width= 0.5pt,line join=round] (335.99, 22.86) --
	(335.99, 25.11);
\end{scope}
\begin{scope}
\path[clip] (  0.00,  0.00) rectangle (411.94,224.04);
\definecolor{drawColor}{gray}{0.30}

\node[text=drawColor,anchor=base,inner sep=0pt, outer sep=0pt, scale=  0.72] at (193.87, 16.10) {0};

\node[text=drawColor,anchor=base,inner sep=0pt, outer sep=0pt, scale=  0.72] at (229.40, 16.10) {0.25};

\node[text=drawColor,anchor=base,inner sep=0pt, outer sep=0pt, scale=  0.72] at (264.93, 16.10) {0.5};

\node[text=drawColor,anchor=base,inner sep=0pt, outer sep=0pt, scale=  0.72] at (300.46, 16.10) {0.75};

\node[text=drawColor,anchor=base,inner sep=0pt, outer sep=0pt, scale=  0.72] at (335.99, 16.10) {1};
\end{scope}
\begin{scope}
\path[clip] (  0.00,  0.00) rectangle (411.94,224.04);
\definecolor{drawColor}{gray}{0.30}

\node[text=drawColor,anchor=base east,inner sep=0pt, outer sep=0pt, scale=  0.72] at ( 21.90, 30.71) {0};

\node[text=drawColor,anchor=base east,inner sep=0pt, outer sep=0pt, scale=  0.72] at ( 21.90, 86.33) {10};

\node[text=drawColor,anchor=base east,inner sep=0pt, outer sep=0pt, scale=  0.72] at ( 21.90,141.95) {20};

\node[text=drawColor,anchor=base east,inner sep=0pt, outer sep=0pt, scale=  0.72] at ( 21.90,197.57) {30};
\end{scope}
\begin{scope}
\path[clip] (  0.00,  0.00) rectangle (411.94,224.04);
\definecolor{drawColor}{gray}{0.20}

\path[draw=drawColor,line width= 0.5pt,line join=round] ( 23.70, 33.19) --
	( 25.95, 33.19);

\path[draw=drawColor,line width= 0.5pt,line join=round] ( 23.70, 88.81) --
	( 25.95, 88.81);

\path[draw=drawColor,line width= 0.5pt,line join=round] ( 23.70,144.43) --
	( 25.95,144.43);

\path[draw=drawColor,line width= 0.5pt,line join=round] ( 23.70,200.05) --
	( 25.95,200.05);
\end{scope}
\begin{scope}
\path[clip] (  0.00,  0.00) rectangle (411.94,224.04);
\definecolor{drawColor}{RGB}{0,0,0}

\node[text=drawColor,rotate= 90.00,anchor=base,inner sep=0pt, outer sep=0pt, scale=  0.90] at ( 10.70,115.54) {Mean inference time (s)};
\end{scope}
\begin{scope}
\path[clip] (  0.00,  0.00) rectangle (411.94,224.04);
\definecolor{fillColor}{RGB}{255,255,255}

\path[fill=fillColor] (352.09, 61.46) rectangle (407.44,169.63);
\end{scope}
\begin{scope}
\path[clip] (  0.00,  0.00) rectangle (411.94,224.04);
\definecolor{drawColor}{RGB}{0,0,0}

\node[text=drawColor,anchor=base west,inner sep=0pt, outer sep=0pt, scale=  0.90] at (356.59,158.06) {Algorithm};
\end{scope}
\begin{scope}
\path[clip] (  0.00,  0.00) rectangle (411.94,224.04);
\definecolor{fillColor}{RGB}{255,255,255}

\path[fill=fillColor] (356.59,138.23) rectangle (371.05,152.68);
\end{scope}
\begin{scope}
\path[clip] (  0.00,  0.00) rectangle (411.94,224.04);
\definecolor{drawColor}{RGB}{27,158,119}

\path[draw=drawColor,line width= 0.6pt,line join=round] (358.04,145.45) -- (369.60,145.45);
\end{scope}
\begin{scope}
\path[clip] (  0.00,  0.00) rectangle (411.94,224.04);
\definecolor{fillColor}{RGB}{255,255,255}

\path[fill=fillColor] (356.59,123.77) rectangle (371.05,138.23);
\end{scope}
\begin{scope}
\path[clip] (  0.00,  0.00) rectangle (411.94,224.04);
\definecolor{drawColor}{RGB}{217,95,2}

\path[draw=drawColor,line width= 0.6pt,dash pattern=on 2pt off 2pt ,line join=round] (358.04,131.00) -- (369.60,131.00);
\end{scope}
\begin{scope}
\path[clip] (  0.00,  0.00) rectangle (411.94,224.04);
\definecolor{fillColor}{RGB}{255,255,255}

\path[fill=fillColor] (356.59,109.32) rectangle (371.05,123.77);
\end{scope}
\begin{scope}
\path[clip] (  0.00,  0.00) rectangle (411.94,224.04);
\definecolor{drawColor}{RGB}{117,112,179}

\path[draw=drawColor,line width= 0.6pt,dash pattern=on 4pt off 2pt ,line join=round] (358.04,116.55) -- (369.60,116.55);
\end{scope}
\begin{scope}
\path[clip] (  0.00,  0.00) rectangle (411.94,224.04);
\definecolor{fillColor}{RGB}{255,255,255}

\path[fill=fillColor] (356.59, 94.86) rectangle (371.05,109.32);
\end{scope}
\begin{scope}
\path[clip] (  0.00,  0.00) rectangle (411.94,224.04);
\definecolor{drawColor}{RGB}{231,41,138}

\path[draw=drawColor,line width= 0.6pt,dash pattern=on 4pt off 4pt ,line join=round] (358.04,102.09) -- (369.60,102.09);
\end{scope}
\begin{scope}
\path[clip] (  0.00,  0.00) rectangle (411.94,224.04);
\definecolor{fillColor}{RGB}{255,255,255}

\path[fill=fillColor] (356.59, 80.41) rectangle (371.05, 94.86);
\end{scope}
\begin{scope}
\path[clip] (  0.00,  0.00) rectangle (411.94,224.04);
\definecolor{drawColor}{RGB}{102,166,30}

\path[draw=drawColor,line width= 0.6pt,dash pattern=on 1pt off 3pt ,line join=round] (358.04, 87.64) -- (369.60, 87.64);
\end{scope}
\begin{scope}
\path[clip] (  0.00,  0.00) rectangle (411.94,224.04);
\definecolor{fillColor}{RGB}{255,255,255}

\path[fill=fillColor] (356.59, 65.96) rectangle (371.05, 80.41);
\end{scope}
\begin{scope}
\path[clip] (  0.00,  0.00) rectangle (411.94,224.04);
\definecolor{drawColor}{RGB}{230,171,2}

\path[draw=drawColor,line width= 0.6pt,dash pattern=on 1pt off 3pt on 4pt off 3pt ,line join=round] (358.04, 73.18) -- (369.60, 73.18);
\end{scope}
\begin{scope}
\path[clip] (  0.00,  0.00) rectangle (411.94,224.04);
\definecolor{drawColor}{RGB}{0,0,0}

\node[text=drawColor,anchor=base west,inner sep=0pt, outer sep=0pt, scale=  0.72] at (375.55,142.97) {BDD};
\end{scope}
\begin{scope}
\path[clip] (  0.00,  0.00) rectangle (411.94,224.04);
\definecolor{drawColor}{RGB}{0,0,0}

\node[text=drawColor,anchor=base west,inner sep=0pt, outer sep=0pt, scale=  0.72] at (375.55,128.52) {d-DNNF};
\end{scope}
\begin{scope}
\path[clip] (  0.00,  0.00) rectangle (411.94,224.04);
\definecolor{drawColor}{RGB}{0,0,0}

\node[text=drawColor,anchor=base west,inner sep=0pt, outer sep=0pt, scale=  0.72] at (375.55,114.07) {K-Best};
\end{scope}
\begin{scope}
\path[clip] (  0.00,  0.00) rectangle (411.94,224.04);
\definecolor{drawColor}{RGB}{0,0,0}

\node[text=drawColor,anchor=base west,inner sep=0pt, outer sep=0pt, scale=  0.72] at (375.55, 99.61) {NNF};
\end{scope}
\begin{scope}
\path[clip] (  0.00,  0.00) rectangle (411.94,224.04);
\definecolor{drawColor}{RGB}{0,0,0}

\node[text=drawColor,anchor=base west,inner sep=0pt, outer sep=0pt, scale=  0.72] at (375.55, 85.16) {SDD};
\end{scope}
\begin{scope}
\path[clip] (  0.00,  0.00) rectangle (411.94,224.04);
\definecolor{drawColor}{RGB}{0,0,0}

\node[text=drawColor,anchor=base west,inner sep=0pt, outer sep=0pt, scale=  0.72] at (375.55, 70.70) {SDDX};
\end{scope}
\end{tikzpicture}

\end{frame}

\begin{frame}{Random WMC Instances}
  \begin{block}{Key Idea}
    Parameter \structure{$\rho \in [0, 1]$} biases the probability distribution
    towards adding variables that would introduce fewer new edges in the primal
    graph.
  \end{block}
  \begin{example}
    \begin{columns}
      \begin{column}{0.65\linewidth}
        Partially-constructed formula:
        \[
          (\neg x_{5} \lor x_{2} \lor x_{1}) \land (x_{5} \lor \alert{\mathord{?}} \lor \mathord{?}).
        \]
      \end{column}
      \begin{column}{0.25\linewidth}
        Its primal graph:

        \begin{tikzpicture}
          \node (a) at (0.5, 1) {$x_{1}$};
          \node (b) at (0, 0) {$x_{2}$};
          \node (c) at (1, 0) {\spot{$x_{5}$}};
          \node (d) at (2, 1) {$x_{3}$};
          \node (e) at (2, 0) {$x_{4}$};
          \draw (a) -- (b);
          \draw[blue,thick] (b) -- (c);
          \draw[blue,thick] (c) -- (a);
          \draw (current bounding box.north east) rectangle (current bounding box.south west);
        \end{tikzpicture}
      \end{column}
    \end{columns}

    Base probability of each variable being chosen:
    \[
      \frac{1 - \rho}{4}.
    \]

    Both \structure{$x_{1}$} and \structure{$x_{2}$} get a bonus probability of
    \structure{$\rho/2$} for each being the endpoint of \alert{one} out of
    the \alert{two} neighbourhood edges.
  \end{example}
\end{frame}

 \begin{frame}{How WMC Algorithms Scale w.r.t. Primal Treewidth}
  We fit the model \structure{$\ln t \sim \alpha w + \beta$}, i.e.,
  \structure{$t \sim e^{\beta}{(e^{\alpha})}^{w}$}, where \structure{$t$} is
  \alert{runtime}, and \structure{$w$} is \alert{primal treewidth}.

  \centering
  \includegraphics{linearbase.pdf}
\end{frame}

% Mention the applicability of WMC and the two main strands of my contributions
\begin{frame}{Summary}
  \begin{block}{What Have We Learned?}
    \begin{itemize}
      \item Pseudo-Boolean functions as an alternative to literal weights
      \item Cycles in graphs that encode recursive calls
%      \item WMC is not always the bottleneck in probabilistic inference
      \item WMC algorithms based on algebraic decision diagrams are fundamentally different:
      \begin{itemize}
        \item they can supports non-literal weights
        \item their running time depends on the numerical values of weights
        \item they peak at higher density
        \item they scale worse w.r.t.\ primal treewidth
      \end{itemize}
    \end{itemize}
  \end{block}
  \begin{block}{Future Directions}
    \begin{itemize}
      \item PBP\@: new encodings, kernelization, pseudo-Boolean solvers
      \item WFOMC\@: full automation and more liftable fragments
    \end{itemize}
  \end{block}
\end{frame}

% IMPORTANCE, ORIGINALITY, CORRECTNESS

% Originality:
% 1. Representations for WMC problems that go beyond the "CNF + literal weights" tradition (e.g., PBP)
% 2. Circuits with cycles + the 'recursive-algebraic' POV of WFOMC
% 3. Experimental studies of WMC algorithms on random instances with varying:
% a) application-specific parameters (e.g., predicate arities)
% b) parameters that determine the hardness of the problem (e.g., density and primal treewidth)

% Importance (why these things are important to study):
% What is the benefit for academic understanding, the general public, a specific avenue? Why this question in particular?
% 1. A simpler representation of a problem often leads to a faster solution. In my work, this is because ADDs are capable of representing more than just clauses.
% 2. Interpreting a circuit/graph as a collection of algebraically-defined functions highlights the kinds of computations that are not yet supported by WFOMC algorithms
% 3. Important observation: WMC is usually not the bottleneck of ProbLog inference
% 4. Important observation: ADD-based WMC algorithms are fundamentally different in that:
% a) their running time depends on the numerical values of weights
% b) they scale worse w.r.t. primal treewidth
% c) they peak at higher density

\end{document}
