\chapter{Example Programs}

In this appendix, we provide examples of probabilistic logic programs generated
by various combinations of parameters. In all cases, we use
\[
  \{\, 0.1, 0.2, \dots, 0.9, 1, 1, 1, 1, 1 \,\}
\]
as the multiset of probabilities. Each clause is written on a separate line and
ends with a full stop. The head and the body of each clause are separated with
\texttt{:-} (instead of $\gets$). The probability of each clause is prepended to
the clause, using \texttt{::} as a separator. Probabilities equal to one and
empty bodies of clauses can be omitted. Conjunction, disjunction, and negation
are denoted by commas, semicolons, and `\texttt{\textbackslash+}', respectively.
Parentheses are used to demonstrate precedence, although many of them are
redundant.

By setting $\predicates{} = [\texttt{p}]$, $\arities{} = [1]$, $\variables{} =
\{\, \texttt{X} \,\}$, $\constants{} = \emptyset$, $\maxNumNodes{} = 4$, and
$\maxNumClauses{} = 1$, we get fifteen one-line programs, six of which are
without negative cycles (as highlighted below). Only the last program has no
cycles at all.

\begin{enumerate}[label=\arabic*.]
\item
\begin{verbatim}
0.5 :: p(X) :- (\+(p(X))), (p(X)).
\end{verbatim}
\item
\begin{verbatim}
0.8 :: p(X) :- (\+(p(X))); (p(X)).
\end{verbatim}
\hlitem
\begin{verbatim}
0.8 :: p(X) :- (p(X)); (p(X)).
\end{verbatim}
\hlitem
\begin{verbatim}
0.7 :: p(X) :- (p(X)), (p(X)).
\end{verbatim}
\item
\begin{verbatim}
0.6 :: p(X) :- (p(X)), (\+(p(X))).
\end{verbatim}
\item
\begin{verbatim}
p(X) :- (p(X)); (\+(p(X))).
\end{verbatim}
\hlitem
\begin{verbatim}
0.1 :: p(X) :- (p(X)); (p(X)); (p(X)).
\end{verbatim}
\hlitem
\begin{verbatim}
0.8 :: p(X) :- (p(X)), (p(X)), (p(X)).
\end{verbatim}
\item
\begin{verbatim}
p(X) :- \+(p(X)).
\end{verbatim}
\item
\begin{verbatim}
0.1 :: p(X) :- \+(\+(p(X))).
\end{verbatim}
\item
\begin{verbatim}
p(X) :- \+((p(X)); (p(X))).
\end{verbatim}
\item
\begin{verbatim}
0.4 :: p(X) :- \+((p(X)), (p(X))).
\end{verbatim}
\item
\begin{verbatim}
0.4 :: p(X) :- \+(\+(\+(p(X)))).
\end{verbatim}
\hlitem
\begin{verbatim}
0.7 :: p(X) :- p(X).
\end{verbatim}
\hlitem
\begin{verbatim}
p(X).
\end{verbatim}
\end{enumerate}

Note that:
\begin{itemize}
\item A program such as Program~14, because of its cyclic definition, defines a
  predicate that has probability zero across all constants. This can more easily
  be seen as solving equation $0.7x = x$.
\item Programs 10 and 14 are not equivalent (i.e., double negation does not
  cancel out) because Program 10 has a negative cycle and is thus considered to
  be ill-defined.
\end{itemize}

To demonstrate variable symmetry reduction in action, we set $\predicates{} =
[\texttt{p}]$, $\arities{} = [3]$, $\variables{} = \{\, \texttt{X}, \texttt{Y},
\texttt{Z} \,\}$, $\constants{} = \emptyset$, $\maxNumNodes{} = 1$,
$\maxNumClauses{} = 1$, and forbid all cycles. This gives us the following five
programs:

\begin{itemize}
\item
\begin{verbatim}
0.8 :: p(Z, Z, Z).
\end{verbatim}
\item
\begin{verbatim}
p(Y, Y, Z).
\end{verbatim}
\item
\begin{verbatim}
p(Y, Z, Z).
\end{verbatim}
\item
\begin{verbatim}
p(Y, Z, Y).
\end{verbatim}
\item
\begin{verbatim}
0.1 :: p(X, Y, Z).
\end{verbatim}
\end{itemize}

This is one of many possible programs with
$\predicates{} = [\texttt{p}, \texttt{q}, \texttt{r}]$,
$\arities{} = [1, 2, 3]$,
$\variables{} = \{\, \texttt{X}, \texttt{Y}, \texttt{Z} \,\}$,
$\constants{} = \{\, \texttt{a}, \texttt{b}, \texttt{c} \,\}$,
$\maxNumNodes{} = 5$, $\maxNumClauses{} = 5$, and without negative cycles:

\begin{verbatim}
p(b) :- \+((q(a, b)), (q(X, Y)), (q(Z, X))).
0.4 :: q(X, X) :- \+(r(Y, Z, a)).
q(X, a) :- r(Y, Y, Z).
q(X, a) :- r(Y, b, Z).
r(Y, b, Z).
\end{verbatim}

Finally, we set $\predicates{} = [\texttt{p}, \texttt{q}, \texttt{r}]$,
$\arities{} = [1, 1, 1]$, $\variables{} = \emptyset$,
$\constants{} = \{\, \texttt{a} \,\}$, $\maxNumNodes{} = 3$,
$\maxNumClauses{} = 3$, forbid negative cycles, and constrain predicates
\texttt{p} and \texttt{q} to be independent. The resulting search space contains
thousands of programs such as:

\begin{itemize}
\item
\begin{verbatim}
0.5 :: p(a) :- (p(a)); (p(a)).
0.2 :: q(a) :- (q(a)), (q(a)).
0.4 :: r(a) :- \+(q(a)).
\end{verbatim}
\item
\begin{verbatim}
p(a) :- p(a).
0.5 :: q(a) :- (r(a)); (q(a)).
r(a) :- (r(a)); (r(a)).
\end{verbatim}
\item
\begin{verbatim}
p(a) :- (p(a)); (p(a)).
0.6 :: q(a) :- q(a).
0.7 :: r(a) :- \+(q(a)).
\end{verbatim}
\end{itemize}
