% 11-45 pages (27 on average)
% aim for 2-5 pages for each major section
\chapter{Background}

TODO: outside of specific sections, have a one-paragraph introduction before and a one-paragraph summary at the end.

\section{Propositional Logic} \label{sec:proplogic}

In this section, we briefly introduce the fundamentals of propositional logic and describe some logic-based computational problems. We refer the reader to the book by \cite{DBLP:books/daglib/0029942} for a more detailed introduction to logic and its role in computer science.

An \emph{atomic proposition} (also known as \emph{atom} and Boolean/logical variable) is a variable with two possible (truth) values: true and false. Unless specified otherwise, we will refer to atoms as \emph{variables}. A \emph{formula} is any well-formed expression that connects variables using the following Boolean/logical operators (and parentheses): negation ($\neg$), disjunction ($\lor$), conjunction ($\land$), (material) implication ($\Rightarrow$), and equivalence (i.e., material biconditional) ($\Leftrightarrow$). A \emph{literal} is either a variable or its negation, respectively called \emph{positive} and \emph{negative} literal. A \emph{clause} is a disjunction of literals.\footnote{In the context of logic programs, the word \emph{clause} is used differently (see \cref{sec:lp,chapter:randomlps}).} A formula is in \emph{conjunctive normal form} (CNF) if it is a conjunction of clauses, and it is in $k$-CNF if every clause has exactly $k$ literals. Many other normal forms and ways to represent propositional formulas are covered in \cref{sec:kc}.

An \emph{interpretation} (also known as a \emph{variable assignment}) of a formula $\phi$ is a map from the variables of $\phi$ to the set $\{\, \text{true}, \text{false} \,\}$. A \emph{model} is an interpretation under which $\phi$ evaluates to true. A formula is \emph{satisfiable} if it has at least one model.

Throughout the thesis, we use set-theoretic notation for many concepts in logic such as clauses and formulas in CNF (e.g., we write $c \in \phi$ to mean that clause $c$ is one of the clauses of formula $\phi$). However, this does not automatically mean that we assume no duplicates---whether or not that is the case is clarified on a case-by-case basis.

\begin{example} \label{example:logic}
  Formula $\phi \coloneqq (\neg a \lor b) \land a$ has two variables $a$ and $b$, is in CNF, and contains two clauses. The first clause $\neg a \lor b$ has a negative literal $\neg a$ and a positive literal $b$. Since $\phi$ has two variables, it also has four interpretations. Interpretation $\{\, a \mapsto \text{true}, b \mapsto \text{true} \,\}$ is a model, so $\phi$ is satisfiable. An equivalent set-theoretic representation of $\phi$ is $\{\, \{\, \neg a, b \,\}, \{\, a \,\} \,\}$.
\end{example}

\subsection{Logic-Based Computational Problems} \label{sec:logicproblems}

We begin with a description of \SAT{} and some of its extensions. Given a propositional formula\footnote{Unless stated otherwise, formulas for \SAT{} and other similar problems are assumed to be in CNF.}, \SAT{} asks whether the formula is satisfiable. \SAT{} (also known as \emph{propositional/Boolean satisfiability}) is the first problem shown to be \NP-complete \citep{DBLP:conf/stoc/Cook71,levin1973universal}. Motivated by many real-life problems that were found to be reducible to \SAT{}, research in \SAT{} solving produced algorithms that can efficiently tackle large instances despite the exponential worst-case time complexity \citep{DBLP:series/faia/2009-185}.

Instead of satisfying all clauses, one can attempt to find an interpretation that satisfies the maximum number of clauses---this problem is called Max\SAT{} \citep{bacchus2021maximum,DBLP:series/faia/LiM09}. It is an \NP-hard optimisation problem that (in its most general form) attaches a (potentially infinite) cost for failing to satisfy each clause and seeks to minise total cost.

\#\SAT{}, or \emph{(propositional) model counting}, asks to count the number of models of a formula \citep{DBLP:series/faia/GomesSS09}. \#\SAT{} is the canonical \#\P-complete problem with many applications in areas such as planning and probabilistic reasoning. $\#\exists\SAT{}$, or \emph{projected model counting}, selects a subset of variables called \emph{priority variables} \citep{DBLP:conf/sat/AzizCMS15}. The task is then to count the number of assignments of values to priority variables that can be extended to models. The extension of \#\SAT{} most relevant to our work is called \emph{weighted model counting} (WMC). Given a propositional formula $\phi$ and a \emph{weight function} $w$ from the literals of $\phi$ to non-negative real numbers, WMC asks to compute
\[
\mathrm{WMC}(\phi) = \sum_{\omega \models \phi} \prod_{\omega \models l} w(l),
\]
where the summation is over all models $\omega$ of $\phi$, and the product is over all literals of $\omega$ \citep{DBLP:journals/ai/ChaviraD08}. Lastly, both \#\SAT{} and WMC have been extended to first-order logic \citep{DBLP:conf/ijcai/BroeckTMDR11}---this is the topic of \cref{chapter:wfomc}.

\begin{example} \label{example:wmc1}
  The model count of the formula in \cref{example:logic} is equal to one. With a weight function $w \coloneqq \{\, a \mapsto 0.7, \neg a \mapsto 0.2, b \mapsto 0.8, \neg b \mapsto 0.7 \,\}$, the WMC of the same formula is $0.7 \times 0.8 = 0.56$.
\end{example}

\begin{example}
  With the same weight function $w$ as in \cref{example:wmc1}, the WMC of formula $a \lor b$ is $w(a)w(b) + w(a)w(\neg b) + w(\neg a)w(b) = 0.7 \times 0.8 + 0.7 \times 0.7 + 0.2 \times 0.8 = 1.21$, and the model count of this formula is 3.
\end{example}

There are a number of other computational problems that similarly use logical or algebraic constructs to encode problems from various domains. First, a propositional formula with prepended quantifiers for all of its variables is known as a \emph{quantified Boolean formula} \citep{DBLP:series/faia/BuningB09}. One can then ask whether the formula is true or false. \emph{Satisfiability module theories} considers \SAT{} in the context of a background theory \citep{DBLP:series/faia/BarrettSST09}. These theories can describe the properties of integer arithmetic, sets, trees, strings, and many commonly-used abstract data structures. \emph{Pseudo-Boolean} solvers consider decision and optimisation problems that can be expressed as linear inequalities over Boolean variables \citep{DBLP:series/faia/RousselM09}. \emph{Integer (linear) programming} instances encode integer optimisation problems under inequality constraints of a certain linear-algebraic form \citep{wolsey2020integer}. Finally, \emph{constraint programming} is a powerful paradigm for solving combinatorial search and optimisation problems with a much more expressive syntax \citep{DBLP:reference/fai/2}---we discuss constraint programming in more detail in \cref{sec:cp}.

\section{Declarative Programming}

In a declarative programming language, one describes \emph{what} is to be computed but not \emph{how}. Here we describe two declarative programming paradigms pertinent to our work: logic programming and constraint programming.

\subsection{Logic Programming} \label{sec:lp}

In this subsection, we give a brief introduction to logic programming. Specifically, we focus on Prolog---the most popular logic programming language to date. We do not, however, attempt to cover all (or even most) of the capabilities of Prolog but rather focus on the main concepts and ideas relevant to our work in \cref{chapter:randomlps}. Note that different descriptions of logic programming often use different (and mutually inconsistent) terminologies. Here we prioritise names and definitions that are sufficiently general for our needs and reasonably consistent with the terminology used in logic. For more details on logic programming and Prolog, we refer the reader to some of the numerous books on the subject \citep{DBLP:books/daglib/0041598,DBLP:books/daglib/0067951}.

A \emph{logic program} is a finite sequence\footnote{Although it is common to define logic programs as sets, the order is important for efficiency and can be the difference between finite and infinite running time.} of clauses. A \emph{clause} consists of a head and a body. If a clause has an empty body, it is a \emph{fact}, otherwise it is a \emph{rule}. The Prolog syntax for a fact and a rule is \verb+h.+ and \verb+h :- b.+, respectively, where \texttt{h} is the head and \texttt{b} is the body, although we often write $\texttt{h} \gets \texttt{b}$ instead.

The \emph{head} of a clause is an atom. An \emph{atom} (i.e., atomic formula) has the form $p(t_1, \dots, t_n)$, where $p$ is a \emph{predicate (symbol)}, and $(t_i)_{i=1}^n$ are terms. Here, $n \in \mathbb{N}_0$ is the \emph{arity} of $P$. Some built-in predicates such as equality can be written in infix notation and without parentheses, i.e., as $a = b$ instead of $=(a, b)$. A \emph{term} is either a \emph{(logical) variable} (i.e., a string that begins with a capital letter) or a \emph{constant} (i.e., any other string).

The \emph{body} of a clause is a formula.\footnote{In the literature, it is common to define clause bodies as conjunctions, but here we present a more general definition, given that such a generalisation is widely supported by the relevant software.} A \emph{formula} is any well-formed expression that connects atoms using conjunction, disjunction, and negation (as well as parentheses). Prolog syntax for these operators is different from the standard notation used in logic: we write `\verb+,+' instead of $\land$, `\verb+;+' instead of $\lor$, and `\verb#\+#' instead of $\neg$. Just like with the syntax for clauses, in most cases we continue to use logic-based syntax for convenience.

Finally, a \emph{query} is a formula to be evaluated. If the query has no variables, the evaluation returns either true or false. Otherwise, the logic programming engine tries to replace the variables of the query with constants such that the resulting formula is a logical consequence of the program. If successful, an example of such a mapping is returned; if not, the engine returns false.

\begin{example}
  Consider the following logic program.
\begin{verbatim}
parent(sky, will).
parent(will, zoe).
ancestor(X, Z) :- parent(X, Z); (parent(X, Y), ancestor(Y, Z)).
\end{verbatim}
In our alternative logic-based notation, the last clause could also be written as
\[
\texttt{ancestor(X, Z)} \gets \texttt{parent(X, Z)} \lor (\texttt{parent(X, Y)} \land \texttt{ancestor(Y, Z)}).
\]

This program has three clauses. The first two clauses are facts whereas the last clause is a rule. The program uses two predicates (\texttt{parent} and \texttt{ancestor}), three constants (\texttt{sky}, \texttt{will}, and \texttt{zoe}), and the last clauses uses three variables (\texttt{X}, \texttt{Y}, and \texttt{Z}). Both predicates are of arity 2.

Clause-by-clause, this program can be interpreted as:
\begin{itemize}
\item Sky is a parent of Will.
\item Will is a parent of Zoe.
\item \texttt{X} is an ancestor of \texttt{Z} if \texttt{X} is a parent of \texttt{Z} or there is a \texttt{Y} such that \texttt{X} is a parent of \texttt{Y}, and \texttt{Y} is an ancestor of \texttt{Z}.
\end{itemize}

The query \texttt{ancestor(sky, zoe)} returns true since Sky is a parent of a parent of Zoe, and thus an ancestor. The query \texttt{ancestor(X, sky)} returns false because we know nothing about the ancestors of Sky. Lastly, the query \texttt{ancestor(sky, X)} could return either $\{\, \texttt{X} \mapsto \texttt{will} \,\}$ or $\{\, \texttt{X} \mapsto \texttt{zoe} \,\}$ as both Will and Zoe have Sky as an ancestor.
\end{example}

% TODO: could also describe stratification in more detail (either here or in Chapter 3)
%% \paragraph{Things to mention.}
%% \begin{itemize}
%% \item we're not defining literals here
%% \item the generalisation of clauses affects the definitions of stratification and dependency graph as well
%% \item Stratification
%%   \begin{itemize}
%%   \item \emph{Stratification} is a condition necessary for probabilistic logic programs
%%     \citep{DBLP:conf/padl/MantadelisR17} and often enforced on logic programs
%%     \citep{DBLP:journals/tcs/Bidoit91} that helps to ensure a unique answer to every
%%     query. This is achieved by restricting the use of negation so that any program
%%     $\mathscr{P}$ can be partitioned into a sequence of programs $\mathscr{P} =
%%     \bigsqcup_{i=1}^n \mathscr{P}_i$ such that, for all $i$, the negative literals
%%     in $\mathscr{P}_i$ can only refer to predicates defined in $\mathscr{P}_j$ for
%%     $j \le i$ \citep{DBLP:journals/tcs/Bidoit91}.
%%   \item include the formal definition from the original paper \citep{DBLP:books/mk/minker88/AptBW88}
%%   \item also include a good example
%%   \item consider including the definition of a (predicate) dependency graph and the lemma that follows. I think the original definition is slightly different: it allows edges to be positive and negative at the same time.
%%   \item (the original paper) shown that stratified programs are always consistent (i.e., avoid paradoxical situations such as $p \gets \neg p$) \citep{DBLP:books/mk/minker88/AptBW88}
%%   \item only a sufficient condition for consistency
%%   \end{itemize}
%% \end{itemize}

\subsection{Constraint Programming} \label{sec:cp}

Constraint models are successfully used to tackle search problems in many domains such as bioinformatics, configuration, networks, planning, scheduling, and vehicle routing \citep{DBLP:reference/fai/2}. Here we briefly describe what a constraint satisfaction problem (CSP) is, how an algorithm might attempt to solve it, and how one can help the algorithm search efficiently.

\begin{definition}
  A \emph{CSP} is a triple $(X, D, C)$, where
  \begin{itemize}
  \item $X = (x_i)_{i=1}^n$ is an $n$-tuple of variables,
  \item $D = (D_i)_{i=1}^n$ is an $n$-tuple of (typically, finite) domains such that $x_i \in D_i$,
  \item and $C$ is a set of constraints.
  \end{itemize}
  A \emph{constraint} is a pair $(S, R)$, where $S \subseteq X$ is the \emph{scope} of the constraint, and $R \subseteq \prod_{x_i \in S} D_i$ is a relation specifying allowed combinations of values. Constraints can be specified either \emph{intensionally} (i.e., by describing a formula that must be satisfied) or \emph{extensionally} (i.e., by listing all tuples). A \emph{solution} to the CSP is an $n$-tuple $(a_i)_{i=1}^n$ such that $a_i \in D_i$ and the relevant $a_i$'s are in the relations of all the constraints in $C$.
\end{definition}

\begin{example}[$n$ queens]
  Imagine an $n \times n$ chess board. How can one place $n$ queens on the board so that no two queens threaten each other (i.e., are not on the same column, row, or diagonal)? This is the famous \emph{$n$ queens problem}---a common example in the constraint programming literature. The solution we describe here is adapted from a constraint modelling tutorial \citep{minizinc}.

  First, note each column (i.e, \emph{file}) must have exactly one queen. Let $(q_i)_{i=1}^n$ be variables with domains $q_i \in \{\, 1, \dots, n \,\}$, where we use $q_i = j$ to denote that the $i$th column queen is on row (i.e., \emph{rank}) $j$. Then the entire problem can be described by the following three constraints.

  \begin{constraint} \label{exampleconstraint:1}
    $\alldifferent(\{\,q_i\,\}_{i=1}^n)$
  \end{constraint}

  \begin{constraint} \label{exampleconstraint:2}
    $\alldifferent(\{\, q_i + i \mid i = 1, \dots, n \,\})$
  \end{constraint}

  \begin{constraint} \label{exampleconstraint:3}
    $\alldifferent(\{\, q_i - i \mid i = 1, \dots, n \,\})$
  \end{constraint}

  Here, $\alldifferent$ is a constraint on a set of variables (or `derivatives' of variables) that constrains them to be all different. \Cref{exampleconstraint:1} requires all queens to occupy different rows, and \cref{exampleconstraint:2,exampleconstraint:3} do the same for both diagonals.

  Note that, given one solution to the $n$-queens problem, we can easily find seven others just by rotating and flipping the board in every possible way (i.e., the symmetry group of a square has order 8). Thus, there is no reason for the constraint solver to find all eight symmetrical solutions independently. Avoiding this kind of excessive effort is the goal of \emph{symmetry breaking} constraints.

  While some symmetry breaking constraints can be expressed using variables $(q_i)_{i=1}^n$, others could benefit from a different representation. Specifically, let $\mathbf{B} = (b_{ij})$ be an $n \times n$ matrix, where each $b_{ij} \in \{\, \textrm{true}, \textrm{false} \,\}$ indicates whether the $(i,j)$-th square contains a queen. Constraints that connect different representations of the same problem are called \emph{channelling} constraints. In this case, the following constraint is sufficient.

  \begin{constraint}[Channelling]
    For all $i, j = 1, \dots, n$, we have that $b_{ij} \Leftrightarrow (q_i = j)$.
  \end{constraint}

  Finally, the following is an example of a symmetry breaking constraint.

  \begin{constraint}[Symmetry breaking]
    $\mathbf{B}$ is lexicographically smaller than or equal to $\mathbf{B}^\top$ (i.e., the transpose of $\mathbf{B}$).
  \end{constraint}
\end{example}

TODO: describe the solving process

\begin{itemize}
\item (chronological backtracking) search
  \begin{itemize}
  \item constraint propagation (also known as inference) (refer to the example)
    \begin{itemize}
    \item entailment (\coNP-complete)
    \end{itemize}
  \item heuristics: variable ordering heuristics, value ordering heuristics
  \item randomization and restarts
  \end{itemize}
\item thrashing (somewhere: maybe when describing search, maybe with the example)
\end{itemize}

\section{Representations of Probability Distributions}

(including the ideas behind inference)
\begin{itemize}
\item define what a distribution is
\item Probabilistic Graphical Models
  \begin{itemize}
  \item Bayesian Networks
  \item Markov Random Fields
  \item cite the PGM books
  \item maybe explain context-specific independence. Cite [Boutilier et al., 1996]
  \item Relational
    \begin{itemize}
    \item Markov Logic Networks
    \end{itemize}
  \end{itemize}
\item Probabilistic Programming
  \begin{itemize}
  \item Imperative and (Maybe) Functional, e.g., BLOG
  \item Probabilistic Logic Programming
    \begin{itemize}
    \item ProbLog (including some detail about how inference is defined)
    \end{itemize}
  \end{itemize}
\end{itemize}

\section{Knowledge Compilation} \label{sec:kc}

(including lots of detail about all the data structures)
\begin{itemize}
\item Boolean and Pseudo-Boolean Functions
\item NNF
\item d-DNNF
\item SDDs
\item BDDs
\item ADDs (with a brief mention of AADDs, XDDs, etc.)
\end{itemize}

\section{Applications}

of WMC?

\begin{itemize}
\item Statistical Relational Learning
\item Neuro-Symbolic Artificial Intelligence
\item Natural Language Processing
\item Robotics
\end{itemize}
