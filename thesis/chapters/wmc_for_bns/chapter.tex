\chapter{WMC with Conditional Weights for Bayesian Networks}\label{chapter:wmc1}

\section{Introduction}

% TODO: explain the difference between this and the next chapter

% While in this chapter we focus on measure-theoretic foundations and propose a
% new encoding for Bayesian networks, a related line of work---motivated by the
% same issue of WMC encodings having both more variables and more clauses as a
% consequence of having to conform to an unnecessarily restrictive
% format---shows how these variables and clauses can be removed from (most)
% already-existing Bayesian network encodings.

Weighted model counting (WMC), i.e., an extension of model counting (\mc{}) that
assigns a weight to every model \citep{DBLP:conf/aaai/SangBK05}, has emerged as
one of the most dominant and competitive approaches for handling inference tasks
in a wide range of formalisms including Bayesian networks
\citep{DBLP:conf/aaai/SangBK05,DBLP:books/daglib/0024906}, probabilistic
graphical models more generally \citep{DBLP:conf/ecsqaru/ChoiKD13}, and
probabilistic programs
\citep{DBLP:journals/tplp/FierensBRSGTJR15,DBLP:journals/corr/abs-2005-09089}.
Over the last fifteen years, WMC has been extended and generalised in many ways,
e.g., to handle continuous probability distributions
\citep{DBLP:conf/ijcai/BellePB15}, first-order probabilistic theories
\citep{DBLP:conf/ijcai/BroeckTMDR11,DBLP:journals/cacm/GogateD16}, and infinite
domains \citep{DBLP:conf/aaai/Belle17}. Furthermore, by generalising the notion
of weights to an arbitrary semiring, a range of other problems are also captured
\citep{DBLP:journals/japll/KimmigBR17}. Exact WMC solvers typically rely on
either knowledge compilation
\citep{DBLP:conf/ijcai/OztokD15,DBLP:conf/ijcai/LagniezM17} or exhaustive DPLL
search \citep{DBLP:conf/aaai/SangBK05}, whereas approximate solvers work by
sampling \citep{DBLP:conf/aaai/ChakrabortyFMSV14} and performing local search
\citep{DBLP:conf/sat/WeiS05}.

The most well-known version of WMC assigns weights to models based on
weights on literals, i.e., the weight of a model is the product of the weights
of all literals in it. This simplification is motivated by the fact that the
number of models scales exponentially with the number of atoms, so listing the
weight of every model is intractable. However, this also severely restricts what
probability distributions can be represented. A common way to overcome this
limitation is by adding more literals. While we show that this is always
possible, we demonstrate that it can be significantly more efficient to
encode weights in a more flexible format instead.

After briefly reviewing the background in \cref{sec:related}, in
\cref{sec:prelims} we  describe three equivalent perspectives on the subject
based on logic, set theory, and Boolean algebras. Furthermore, we describe the
space of functions on Boolean algebras and various operations on those
functions. \Cref{sec:wmc_as_measure} introduces WMC as the problem of computing
the value of a measure on a Boolean algebra. We show that not all measures can
be represented using literal-based WMC, but all Boolean algebras can be extended
to make any measure representable in such a manner.

This new perspective allows us to not only encode any discrete probability
distribution but also improve inference speed. In \cref{sec:bns} we demonstrate
this by developing a new WMC encoding for Bayesian networks that uses
\emph{conditional weights} on literals (in the spirit of conditional
probabilities) that have literal-based WMC as a special case. We prove the
correctness of the encoding and show how a state-of-the-art WMC solver
\textsf{ADDMC} \citep{DBLP:conf/aaai/DudekPV20} can be adapted to the new
format. \textsf{ADDMC} is a recently-proposed algorithm for WMC based on
manipulating functions on Boolean algebras using an efficient representation for
such functions known as algebraic decision diagrams (ADDs)
\citep{DBLP:journals/fmsd/BaharFGHMPS97}. \textsf{ADDMC} was already shown to be
capable of solving instances other solvers fail at and being the fastest solver
on the largest number of instances \citep{DBLP:conf/aaai/DudekPV20}. Our
experiments in \cref{sec:2experiments} focus on further improving the performance
of \textsf{ADDMC} on instances that originate from Bayesian networks. We show
how our new encoding improves inference on the vast majority of benchmark
instances, often by one or two orders of magnitude. We explain the performance
benefits by showing how our encoding has asymptotically fewer variables and
ADDs.

\section{Related Work}\label{sec:related}

Performing inference on Bayesian networks by encoding them into instances of WMC
is a well-established idea with a history of almost twenty years. Five encodings
have been proposed so far (we will identify them based on the initials of
authors as well as publications years): \texttt{d02}
\citep{DBLP:conf/kr/Darwiche02}, \texttt{sbk05} \citep{DBLP:conf/aaai/SangBK05},
\texttt{cd05} \citep{DBLP:conf/ijcai/ChaviraD05}, \texttt{cd06}
\citep{DBLP:conf/sat/ChaviraD06}, and \texttt{bklm16}
\citep{DBLP:conf/ecai/BartKLM16}\footnote{\citet{DBLP:conf/scai/VomlelT13} also
  propose an encoding, but only for networks of a particular bipartite structure
  and without any evaluation.}. Below we summarise the observed performance
differences among them.

\citet{DBLP:conf/aaai/SangBK05} claim that
\texttt{sbk05} is a smaller encoding than \texttt{d02} with respect to both the
number of clauses and the number of variables but provide no experimental
comparison. \citet{DBLP:conf/ijcai/ChaviraD05} compare \texttt{cd05} with
\texttt{d02} by measuring the time it takes to compile either encoding into an
arithmetic circuit. They show that
\texttt{cd05} always compiles faster and results in a smaller arithmetic circuit
(as measured by the number of edges). In their subsequent paper, the same
authors perform two sets of experiments (that are relevant to this summary)
\citep{DBLP:conf/sat/ChaviraD06}. First, they compile \texttt{cd05} and
\texttt{cd06} encodings into d-DNNF (i.e., deterministic decomposable negation
normal form \citep{DBLP:journals/jancl/Darwiche01}), measuring both compilation
time and numbers of edges in the d-DNNF diagram. The results are mostly in
favour of \texttt{cd06}. Second, they compare the inference time of
\texttt{sbk05} run with \textsf{Cachet} \citep{DBLP:conf/sat/SangBBKP04} with
the compile times of \texttt{cd05} and \texttt{cd06}, but only on five (types
of) instances. In these experiments, \texttt{cd06} is always faster than
\texttt{cd05}, while the comparison with \texttt{sbk05} is mixed.  The
performance difference between \texttt{sbk05} and \texttt{cd05} is even harder
to judge: \texttt{sbk05} is better on three out of five instances and worse on
the remaining two. Finally, \citet{DBLP:conf/ecai/BartKLM16} introduce
\texttt{bklm16} and show that it has both fewer variables and fewer clauses than
\texttt{cd06}. Their experiments show \texttt{bklm16} to be superior to
\texttt{cd06} with respect to both compilation time and encoding size when both
are compiled using
\textsf{c2d}\footnote{\url{http://reasoning.cs.ucla.edu/c2d/}}
\citep{DBLP:conf/ecai/Darwiche04} but inferior to \texttt{cd06} when
\texttt{cd06} is compiled using
\textsf{Ace}\footnote{\url{http://reasoning.cs.ucla.edu/ace/}} (which still uses
\textsf{c2d} but considers the structure of the Bayesian network along with its
encoding). Our experiments in \cref{sec:2experiments} confirm some of the
findings outlined in this section while also showing that the performance of
each encoding depends on the WMC algorithm in use, and smaller encodings are not
necessarily faster.

\section{Boolean Algebras, Power Sets, and Propositional
  Logic}\label{sec:prelims}

\begin{sidewaystable}
  \centering
  \begin{tabular}{lcc}
    \toprule
    Name in logic & Boolean-algebraic notation & Set-theoretic notation \\
    \midrule
    Atoms (elements of $U$) & $a, b$ & $a, b$ \\
    \rowcolor{gray!10} Models (elements of $2^U$) & $\neg a \land \neg b, a \land \neg b, \neg a \land b, a \land b$ & $\emptyset, \{\, a \,\}, \{\, b \,\}, \{\, a, b \,\}$ \\
    & $\top$ & $\{\, \emptyset, \{\, a \,\}, \{\, b \,\}, \{\, a, b \,\} \,\}$ \\
    & $\neg a \lor \neg b, a \to b$ & $\{\, \emptyset, \{\, a \,\}, \{\, b \,\} \,\}, \{\, \emptyset, \{\, b \,\}, \{\, a, b \,\} \,\}$ \\
    & $b \to a, a \lor b$ & $\{\, \emptyset, \{\, a \,\}, \{\, a, b \,\} \,\}, \{\, \{\, a \,\}, \{\, b \,\}, \{\, a, b \,\} \,\}$ \\
    & $\neg b, \neg a, a \leftrightarrow b$ & $\{\, \emptyset, \{\, a \,\} \,\}, \{\, \emptyset, \{\, b \,\} \,\}, \{\, \emptyset, \{\, a, b \,\} \,\}$ \\
    & $(a \land \neg b) \lor (b \land \neg a), a, b$ & $\{\, \{\, a \,\}, \{\, b \,\} \,\}, \{\, \{\, a \,\}, \{\, a, b \,\} \,\}, \{\, \{\, b \,\}, \{\, a, b \,\} \,\}$ \\
    & $\neg a \land \neg b, a \land \neg b, \neg a \land b, a \land b$ & $\{\, \emptyset \,\}, \{\, \{\, a \,\} \,\}, \{\, \{\, b \,\} \,\}, \{\, \{\, a, b \,\} \,\}$ \\
    \multirow{-7}{*}{Formulas (elements of $2^{2^U}$)} & $\bot$ & $\emptyset$ \\
    \bottomrule
  \end{tabular}
  \caption{Notation for a logic with two atoms. The elements in both columns are
    listed in the same order.}\label{tbl:notation_example}
\end{sidewaystable}

In this section, we give a brief introduction to two alternative ways to think
about logical constructs such as models and formulas. Let us consider a simple
example of a propositional logic $\mathcal{L}$ with only two atoms $a$ and $b$,
and let $U = \{\, a, b \,\}$. Then $2^U$, the power set of $U$, is the set of
all models of $\mathcal{L}$, and $2^{2^U}$ is the set of all formulas. These
sets can also be represented as Boolean algebras (e.g., using the syntax
$(2^{2^U}, \land, \lor, \neg, \bot, \top)$) with a partial order $\le$ that
corresponds to set inclusion $\subseteq$---see \cref{tbl:notation_example} for
examples of how various elements can be represented in both notations. Most
importantly, note that the word \emph{atom} has completely different meanings in
logic and Boolean algebras. An atom in $\mathcal{L}$ is an atomic formula, i.e.,
an element of $U$, whereas an atom in a Boolean algebra is (in set-theoretic
terms) a singleton set. For instance, an atom in $2^{2^U}$ corresponds to a
model of $\mathcal{L}$, i.e., an element of $2^U$. Unless referring specifically
to a logic, we will use the algebraic definition of an atom and refer to logical
atoms as \emph{variables}. In the rest of the paper, for any set $U$, we will
use set-theoretic notation for $2^U$ and Boolean-algebraic notation for
$2^{2^U}$, except for (Boolean) atoms in $2^{2^U}$ that are denoted as $\{x\}$
for some model $x \in 2^U$.

\subsection{Functions on Boolean Algebras}

We also consider the space of all functions from any Boolean algebra to
$\mathbb{R}_{\ge 0}$ together with some operations on those functions. They will
be instrumental in defining WMC as a measure in \cref{sec:wmc_as_measure} and
can be efficiently represented using ADDs. Furthermore, all of the operations
are supported by CUDD \citep{somenzi1998cudd}---a package used by \textsf{ADDMC}
for ADD manipulation \citep{DBLP:conf/aaai/DudekPV20}. The definitions of
multiplication and projection are as defined by
\citet{DBLP:conf/aaai/DudekPV20}, while others are new.

\begin{definition}[Operations on functions]
  Let $\alpha\colon 2^X \to \mathbb{R}_{\ge 0}$ and
  $\beta\colon 2^Y \to \mathbb{R}_{\ge 0}$ be functions,
  $p \in \mathbb{R}_{\ge 0}$, and $x \in X$. We define the following operations:
  \begin{description}
    \item[Addition:] $\alpha + \beta\colon 2^{X \cup Y} \to \mathbb{R}_{\ge 0}$ is
          such that $(\alpha + \beta)(T) = \alpha(T \cap X) + \beta(T \cap Y)$ for all
          $T \in 2^{X \cup Y}$.
    \item[Multiplication:] $\alpha \cdot \beta\colon 2^{X \cup Y} \to
          \mathbb{R}_{\ge 0}$ is such that $(\alpha \cdot \beta)(T) = \alpha(T \cap X)
          \cdot \beta(T \cap Y)$ for all $T \in 2^{X \cup Y}$.
    \item[Scalar multiplication:] $p\alpha\colon 2^X \to \mathbb{R}_{\ge 0}$ is
          such that $(p\alpha)(T) = p \cdot \alpha(T)$ for all $T \in 2^X$.
    \item[Complement:] $\overline{\alpha}\colon 2^X \to \mathbb{R}_{\ge 0}$ is
          such that $\overline{\alpha}(T) = 1 - \alpha(T)$ for all $T \in 2^X$.
    \item[Projection:] $\exists_x\alpha\colon 2^{X \setminus \{\, x \,\}} \to
          \mathbb{R}_{\ge 0}$ is such that $(\exists_x\alpha)(T) = \alpha(T) +
          \alpha(T \cup \{\, x \,\})$ for all $T \in 2^{X \setminus \{\, x \,\}}$. For any $Z =
          \{\, z_1, \dots, z_n \,\} \subseteq X$, we write $\exists_Z$ to mean
          $\exists_{z_1}\dots\exists_{z_n}$.
  \end{description}
\end{definition}

In summary, addition, multiplication, and scalar multiplication are defined
pointwise, while complement and projection interact with the algebraic structure
of the domains $2^X$ and $2^Y$. Specifically, note that both addition and
multiplication are both associative and commutative. We end the discussion on
function spaces by defining several special functions: unit $1\colon 2^\emptyset
\to \mathbb{R}_{\ge 0}$ defined as $1(\emptyset) = 1$, zero $0\colon 2^\emptyset
\to \mathbb{R}_{\ge 0}$ defined as $0(\emptyset) = 0$, and function $[a]\colon
2^{\{a\}} \to \mathbb{R}_{\ge 0}$ defined as $[a](\emptyset) = 0$, $[a](\{a\}) =
1$ for any $a$. Henceforth, for any function $\alpha\colon 2^X \to
\mathbb{R}_{\ge 0}$ and any set $T$, we will write $\alpha(T)$ to mean $\alpha(T
\cap X)$.

\section{WMC as a Measure on a Boolean Algebra}\label{sec:wmc_as_measure}

In this section, we introduce an alternative definition of WMC and demonstrate
how it relates to the standard one. Let $U$ be a set. A \emph{measure} is a
function $\mu\colon 2^{2^U} \to \mathbb{R}_{\ge 0}$ such that $\mu(\bot) = 0$,
and $\mu(a \lor b) = \mu(a) + \mu(b)$ for all $a, b \in 2^{2^U}$ whenever $a
\land b = \bot$ \citep{gaifman1964concerning,DBLP:books/daglib/0090259}. A
\emph{weight function} is a function $\nu\colon 2^U \to \mathbb{R}_{\ge 0}$. A
weight function is \emph{factored} if $\nu = \prod_{x \in U} \nu_x$ for some
functions $\nu_x\colon 2^{\{x\}} \to \mathbb{R}_{\ge 0}$, $x \in U$. We say that
a weight function $\nu\colon 2^U \to \mathbb{R}_{\ge 0}$ \emph{induces} a
measure $\mu_\nu\colon 2^{2^U} \to \mathbb{R}_{\ge 0}$ if $\mu_\nu(x) =
\sum_{\{u\} \le x} \nu(u)$.

\begin{theorem}\label{prop:measure}
  The function $\mu_\nu$ is a measure.
\end{theorem}
\begin{proof}
  Note that $\mu_\nu(\bot) = 0$ since there are no atoms below $\bot$. Let
  $a, b \in 2^{2^{U}}$ be such that $a \land b = \bot$. By elementary properties
  of Boolean algebras, all atoms below $a \lor b$ are either below $a$ or below
  $b$. Moreover, none of them can be below both $a$ and $b$ because then they
  would have to be below $a \land b = \bot$. Thus
  \begin{align*}
    \mu_\nu(a \lor b) &= \sum_{\{u\} \le a \lor b} \nu(u) = \sum_{\{u\} \le a} \nu(u) + \sum_{\{u\} \le b} \nu(u) \\
                      &= \mu_\nu(a) + \mu_\nu(b)
  \end{align*}
  as required.
\end{proof}

Finally, a measure $\mu\colon 2^{2^U} \to \mathbb{R}_{\ge 0}$ is
\emph{factorable} if there exists a factored weight function $\nu\colon 2^U \to
\mathbb{R}_{\ge 0}$ that induces $\mu$. In this formulation, WMC corresponds to
the process of calculating the value of $\mu_\nu(x)$ for some $x \in 2^{2^U}$
with a given definition of $\nu$.

% TODO: this paragraph needs to be rewritten, in particular to avoid terms
% 'axiom' and 'theory'
\paragraph*{Relation to the classical (logic-based) view of WMC.} Let
$\mathcal{L}$ be a propositional logic with two atoms $a$ and $b$ as in
\cref{sec:prelims} and
$w\colon \{\, a, b, \neg a, \neg b \,\} \to \mathbb{R}_{\ge 0}$ a weight
function defined as $w(a) = 0.3$, $w(\neg a) = 0.7$, $w(b) = 0.2$,
$w(\neg b) = 0.8$. Furthermore, let $\Delta$ be a theory in $\mathcal{L}$ with a
sole axiom $a$. Then $\Delta$ has two models: $\{\, a, b \,\}$ and
$\{\, a, \neg b \,\}$ and its WMC \citep{DBLP:journals/ai/ChaviraD08} is
\begin{equation} \label{eq:wmc_example}
  \begin{split}
    \WMC(\Delta) &= \sum_{\omega \models \Delta} \prod_{\omega \models l} w(l) \\
                 &= w(a)w(b) + w(a)w(\neg b) = 0.3.
  \end{split}
\end{equation}
Alternatively, we can define
$\nu_a\colon 2^{\{\, a \,\}} \to \mathbb{R}_{\ge 0}$ as
$\nu_a(\{\, a \,\}) = 0.3$, $\nu_a(\emptyset) = 0.7$ and
$\nu_b\colon 2^{\{\, b \,\}} \to \mathbb{R}_{\ge 0}$ as
$\nu_b(\{\, b \,\}) = 0.2$, $\nu_b(\emptyset) = 0.8$. Let $\mu$ be the measure
on $2^{2^U}$ induced by $\nu = \nu_a \cdot \nu_b$. Then, equivalently to
\cref{eq:wmc_example}, we can write
\begin{align*}
  \mu(a) &= \nu(\{\, a, b \,\}) + \nu(\{\, a \,\}) \\
         &= \nu_a(\{\, a \,\})\nu_b(\{\, b \,\}) + \nu_a(\{\, a \,\})\nu_b(\emptyset) = 0.3.
\end{align*}
Thus, one can equivalently think of WMC as summing over models of a theory or
over atoms below an element of a Boolean algebra.

\subsection{Not All Measures Are Factorable}

Using this new definition of WMC, we can show that WMC with weights defined on
literals is only able to capture a subset of all possible measures on a
Boolean algebra. This can be demonstrated with a simple example.

\begin{example}
  Let $U = \{a, b\}$ be a set of atoms and $\mu\colon 2^{2^U} \to
  \mathbb{R}_{\ge 0}$ a measure defined as $\mu(a \land b) = 0.72$, $\mu(a \land
  \neg b) = 0.18$, $\mu(\neg a \land b) = 0.07$, $\mu(\neg a \land \neg b) =
  0.03$.\footnote{The value of $\mu$ on any other element of $2^{2^U}$ can be
    deduced from the definition of a measure.} If $\mu$ could be represented
  using literal-weight (factored) WMC, we would have to find two weight
  functions $\nu_a\colon 2^{\{a\}} \to \mathbb{R}_{\ge 0}$ and $\nu_b\colon
  2^{\{b\}} \to \mathbb{R}_{\ge 0}$ such that $\nu = \nu_a \cdot \nu_b$ induces
  $\mu$, i.e., $\nu_a$ and $\nu_b$ would have to satisfy this system of
  equations:
  \begin{align*}
    \nu_a(\{a\}) \cdot \nu_b(\{b\}) &= 0.72 \\
    \nu_a(\{a\}) \cdot \nu_b(\emptyset) &= 0.18 \\
    \nu_a(\emptyset) \cdot \nu_b(\{b\}) &= 0.07 \\
    \nu_a(\emptyset) \cdot \nu_b(\emptyset) &= 0.03,
  \end{align*}
  which has no solutions.

  Alternatively, we can let $b$ depend on $a$ and consider weight functions
  $\nu_a\colon 2^{\{a\}} \to \mathbb{R}_{\ge 0}$ and $\nu_b\colon 2^{\{a, b\}}
  \to \mathbb{R}_{\ge 0}$ defined as $\nu_a(\{a\}) = 0.9$, $\nu_a(\emptyset) =
  0.1$, and $\nu_b(\{a, b\}) = 0.8$, $\nu_b(\{a\}) = 0.2$, $\nu_b(\{b\}) = 0.7$,
  $\nu_b(\emptyset) = 0.3$. One can easily check that with these definitions
  $\nu$ indeed induces $\mu$.
\end{example}

Note that in this case, we chose to interpret $\nu_b$ as $\Pr(b \mid a)$
while---with a different definition of $\nu_b$ that represents the joint
probability distribution $\Pr(a, b)$---$\nu_b$ by itself could induce $\mu$. In
general, however, factoring the full weight function into several smaller
functions often results in weight functions with smaller domains which leads to
increased efficiency and decreased memory usage
\citep{DBLP:conf/aaai/DudekPV20}. We can easily generalise this example further.

\begin{theorem}
  For any set $U$ such that $|U| \ge 2$, there exists a non-factorable measure
  $2^{2^{U}} \to \mathbb{R}_{\ge 0}$.
\end{theorem}

Since many measures of interest may not be factorable, a well-known way to
encode them into instances of WMC is by adding more literals
\citep{DBLP:journals/ai/ChaviraD08}. We can use the measure-theoretic
perspective on WMC to show that this is always possible, however, as ensuing
sections will demonstrate, it can make the inference task much harder in
practice.

\begin{theorem}
  For any set $U$ and measure $\mu\colon 2^{2^U} \to \mathbb{R}_{\ge 0}$, there
  exists a set $V \supseteq U$, a factorable measure $\mu'\colon 2^{2^V} \to
  \mathbb{R}_{\ge 0}$, and a formula $f \in 2^{2^V}$ such that $\mu(x) = \mu'(x
  \land f)$ for all formulas $x \in 2^{2^U}$.
\end{theorem}
\begin{proof}
  Let $V = U \cup \{ f_m \mid m \in 2^U \}$, and
  $f = \bigwedge_{m \in 2^U} \{ m \} \leftrightarrow f_m$. We define weight
  function $\nu\colon 2^V \to \mathbb{R}_{\ge 0}$ as
  $\nu = \prod_{v \in V} \nu_v$, where $\nu_v(\{v\}) = \mu(\{m\})$ if $v = f_m$
  for some $m \in 2^U$ and $\nu_v(x) = 1$ for all other $v \in V$ and
  $x \in 2^{\{v\}}$. Let $\mu'\colon 2^{2^V} \to \mathbb{R}_{\ge 0}$ be the
  measure induced by $\nu$. It is enough to show that $\mu$ and
  $x \mapsto \mu'(x \land f)$ agree on the atoms in $2^{2^U}$. For any
  $\{ a \} \in 2^{2^U}$,
  \begin{align*}
    \mu'(\{ a \} \land f) &= \sum_{\{ x \} \le \{ a \} \land f} \nu(x) = \nu(a \cup \{ f_a \}) \\
                          &= \nu_{f_a}(\{ f_a \}) = \mu(\{ a \})
  \end{align*}
  as required.
\end{proof}

\section{Encoding Bayesian Networks Using Conditional Weights}\label{sec:bns}

\begin{figure}[t]
  \centering
    \begin{tikzpicture}[edge from parent/.style={draw,-latex}]
      \node[draw,circle] {$W$}
      child {node[draw,circle] {$F$}}
      child {node[draw,circle] {$T$}};
    \end{tikzpicture}
    \\
    \begin{tabular}[t]{cc}
      \toprule
      $w$ & $\Pr(W = w)$ \\
      \midrule
      1 & 0.5 \\
      0 & 0.5 \\
      \bottomrule
    \end{tabular}
    \begin{tabular}[t]{ccc}
      \toprule
      $w$ & $f$ & $\Pr(F = f \mid W = w)$ \\
      \midrule
      1 & 1 & 0.6 \\
      1 & 0 & 0.4 \\
      0 & 1 & 0.1 \\
      0 & 0 & 0.9 \\
      \bottomrule
    \end{tabular}
    \begin{tabular}[t]{ccc}
      \toprule
      $w$ & $t$ & $\Pr(T = t \mid W = w)$ \\
      \midrule
      1 & $l$ & 0.2 \\
      1 & $m$ & 0.4 \\
      1 & $h$ & 0.4 \\
      0 & $l$ & 0.6 \\
      0 & $m$ & 0.3 \\
      0 & $h$ & 0.1 \\
      \bottomrule
    \end{tabular}
  \caption{An example Bayesian network with its CPTs.}\label{fig:example_bn}
\end{figure}

In this section, we describe a way to encode Bayesian networks into WMC without
restricting oneself to factorable measures and thus having to add extra
variables. We will refer to it as \texttt{cw}. A Bayesian network is a directed
acyclic graph with random variables as vertices that defines a probability
distribution over them. Let $\mathcal{V}$ denote this set of random variables.
For any random variable $X \in \mathcal{V}$, let $\Imm X$ denote its set of
values and $\mathrm{pa}(X)$ its set of parents. The full probability
distribution is then equal to
$\prod_{X \in \mathcal{V}} \Pr(X \mid \mathrm{pa}(X))$. For discrete Bayesian
networks (and we only consider discrete networks here), each factor of this
product can be represented by a CPT\@. See \cref{fig:example_bn} for an example
Bayesian network that we will refer to throughout this section. For this
network, $\mathcal{V} = \{ W, F, T \}$, $\mathrm{pa}(W) = \emptyset$,
$\mathrm{pa}(F) = \mathrm{pa}(T) = \{ W \}$, $\Imm W = \Imm F = \{0, 1 \}$, and
$\Imm T = \{ l, m, h \}$.

\begin{definition}[Indicator variables]
  Let $X \in \mathcal{V}$ be a random variable. If $X$ is binary (i.e.,
  $|\Imm X| = 2$), we can arbitrary identify one of the values as $1$ and the
  other one as $0$ (i.e, $\Imm X \cong \{ 0, 1 \}$). Then $X$ can be represented
  by a single \emph{indicator variable} $\lambda_{X=1}$. For notational
  simplicity, for any set $S$, we write $\lambda_{X=0} \in S$ or
  $S = \{ \lambda_{X=0}, \dots \}$ to mean $\lambda_{X=1} \not\in S$.

  On the other hand, if $X$ is not binary, we represent $X$ with $|\Imm X|$
  indicator variables, one for each value. We let
  \[
    \mathcal{E}(X) =
    \begin{cases}
      \{ \lambda_{X=1} \} & \text{if } |\Imm X| = 2 \\
      \{ \lambda_{X=x} \mid x \in \Imm X \} & \text{otherwise.}
    \end{cases}
  \]
  denote the set of indicator variables for $X$ and $\mathcal{E}^*(X) =
  \mathcal{E}(X) \cup \bigcup_{Y \in \mathrm{pa}(X)} \mathcal{E}(Y)$ denote the
  set of indicator variables for $X$ and its parents in the Bayesian network.
  Finally, let $U = \bigcup_{X \in \mathcal{V}} \mathcal{E}(X)$ denote the set
  of all indicator variables for all random variables in the Bayesian network.
  For example, in the Bayesian network from \cref{fig:example_bn},
  $\mathcal{E}^*(T) = \{ \lambda_{T=l}, \lambda_{T=m}, \lambda_{T=h},
  \lambda_{W=1} \}$.
\end{definition}

\begin{algorithm}[t]
  \caption{Encoding a Bayesian network.}\label{alg:encoding}
  \KwData{vertices $\mathcal{V}$, probability distribution $\Pr$}
  \KwResult{$\phi\colon 2^U \to \mathbb{R}_{\ge 0}$}
  $\phi \gets 1$\;
  \For{$X \in \mathcal{V}$}{
    \textit{let} $\mathrm{pa}(X) = \{ Y_1, \dots, Y_n \}$\;
    $\mathrm{CPT}_X \gets 0$\;
    \eIf{$|\Imm X| = 2$}{
      \For{$(y_i)_{i=1}^n \in \prod_{i = 1}^n \Imm Y_i$}{
        $p_1 \gets \Pr(X = 1 \mid y_1, \dots, y_n)$\;
        $p_0 \gets \Pr(X \ne 1 \mid y_1, \dots, y_n)$\;
        \nosemic$\mathrm{CPT}_X \gets \mathrm{CPT}_X$\;
        \pushline$+ p_1[\lambda_{X=1}] \cdot \prod_{i=1}^n [\lambda_{Y_i=y_i}]$\;
        \dosemic$+ p_0 \overline{[\lambda_{X=1}]} \cdot \prod_{i=1}^n [\lambda_{Y_i=y_i}]$\;
      }
    }{
      \textit{let} $\Imm X = \{ x_1, \dots, x_m \}$\;
      \For{$x \in \Imm X$ {\rm \textbf{and}} ${(y_i)}_{i=1}^n \in \prod_{i = 1}^n \Imm Y_i$}{
        $p_x \gets \Pr(X = x \mid y_1, \dots, y_n)$\;
        \nosemic$\mathrm{CPT}_X \gets \mathrm{CPT}_X$\;
        \pushline$+ p_x[\lambda_{X=x}] \cdot \prod_{i=1}^n [\lambda_{Y_i=y_i}]$\;
        \dosemic$+ \overline{[\lambda_{X=x}]} \cdot \prod_{i=1}^n [\lambda_{Y_i=y_i}]$\;
      }
      \nosemic$\mathrm{CPT}_X \gets \mathrm{CPT}_X \cdot \left( \sum_{i=1}^m [\lambda_{X = x_i}] \right)$\;
      \pushline\dosemic$\cdot \prod_{i=1}^m \prod_{j=i+1}^m (\overline{[\lambda_{X = x_i}]} + \overline{[\lambda_{X = x_j}]})$\;
    }
    $\phi \gets \phi \cdot \mathrm{CPT}_X$\;
  }
  \Return{$\phi$}\;
\end{algorithm}

\Cref{alg:encoding} shows how a Bayesian network with vertices $\mathcal{V}$
can be represented as a weight function $\phi\colon 2^U \to \mathbb{R}_{\ge 0}$.
The algorithm begins with the unit function and multiplies it by
$\mathrm{CPT}_X\colon 2^{\mathcal{E}^*(X)} \to \mathbb{R}_{\ge 0}$ for each
random variable $X \in \mathcal{V}$. We call each such function a
\emph{conditional weight function} as it represents a conditional probability
distribution. However, the distinction is primarily a semantic one: a function
$2^{\{a, b\}} \to \mathbb{R}_{\ge 0}$ can represent $\Pr(a \mid b)$, $\Pr(b \mid
a)$, or something else entirely, e.g., $\Pr(a \land b)$, $\Pr(a \lor b)$, etc.

For a binary random variable $X$, $\mathrm{CPT}_X$ is simply a sum of smaller
functions, one for each row of the CPT. If $X$ has more than two values, we also
multiply $\mathrm{CPT}_X$ by `clause' functions that restrict the value of
$\phi(T)$ to zero whenever $|\mathcal{E}(X) \cap T| \ne 1$, i.e., we add mutual
exclusivity constraints that ensure that each random variable is associated with
exactly one value. Note that \citet{DBLP:conf/ijcai/ChaviraD07} use the same ADD
representation of CPTs for their compilation algorithm based on variable
elimination. For the example Bayesian network in \cref{fig:example_bn}, we get:
\begin{align*}
  \mathrm{CPT_F} &= 0.6[\lambda_{F=1}] \cdot [\lambda_{W=1}] + 0.4[\lambda_{F=0}] \cdot [\lambda_{W=1}] \\
                 &+ 0.1[\lambda_{F=1}] \cdot [\lambda_{W=0}] + 0.9[\lambda_{F=0}] \cdot [\lambda_{W=0}], \\
  \mathrm{CPT_T} &= ([\lambda_{T=l}] + [\lambda_{T=m}] + [\lambda_{T=h}]) \\
                 &\cdot (\overline{[\lambda_{T=l}]} + \overline{[\lambda_{T=m}]}) \cdot (\overline{[\lambda_{T=l}]} + \overline{[\lambda_{T=h}]}) \\
                 &\cdot (\overline{[\lambda_{T=m}]} + \overline{[\lambda_{T=h}]}) \cdot \dots.
\end{align*}

\subsection{Correctness}

\Cref{alg:encoding} produces a function with a Boolean algebra as its domain.
This function can be represented by an ADD
\citep{DBLP:journals/fmsd/BaharFGHMPS97}. \textsf{ADDMC} takes an ADD
$\psi\colon 2^{U} \to \mathbb{R}_{\ge 0}$ (expressed as a product of smaller
ADDs) and returns $(\exists_U\psi)(\emptyset)$ \citep{DBLP:conf/aaai/DudekPV20}.
In this section, we prove that the function $\phi$ produced by
\cref{alg:encoding} can be used by \textsf{ADDMC} to correctly compute any
marginal probability of the Bayesian network that was encoded as
$\phi$.\footnote{Note that it can just as well compute \emph{any} probability
  expressed using the random variables in $\mathcal{V}$.} We begin with
\cref{lemma:cpt} which shows that any conditional weight function produces the
right answer when given a valid encoding of variable-value assignments relevant
to the CPT\@.

\begin{lemma}\label{lemma:cpt}
  Let $X \in \mathcal{V}$ be a random variable with parents
  $\mathrm{pa}(X) = \{ Y_1, \dots, Y_n \}$. Then
  $\mathrm{CPT}_X\colon 2^{\mathcal{E}^*(X)} \to \mathbb{R}_{\ge 0}$ is such
  that for any $x \in \Imm X$ and
  $(y_1, \dots, y_n) \in \prod_{i=1}^n \Imm Y_i$,
  \[
    \mathrm{CPT}_X (T) = \Pr(X = x \mid Y_1 = y_1, \dots, Y_n = y_n),
  \]
  where
  $T = \{ \lambda_{X=x} \} \cup \{ \lambda_{Y_i=y_i} \mid i = 1, \dots, n \}$.
\end{lemma}
\begin{proof}
  If $X$ is binary, then $\mathrm{CPT}_X$ is a sum of
  $2\prod_{i=1}^n |\Imm Y_i|$ terms, one for each possible assignment of values
  to variables $X, Y_1, \dots, Y_n$. Exactly one of these terms is nonzero when
  applied to $T$, and it is equal to
  $\Pr(X = x \mid Y_1 = y_1, \dots, Y_n = y_n)$ by definition.

  If $X$ is not binary, then
  $\left( \sum_{i=1}^m [\lambda_{X = x_i}] \right)(T) = 1$, and
  \[
    \left( \prod_{i=1}^m \prod_{j=i+1}^m (\overline{[\lambda_{X = x_i}]} + \overline{[\lambda_{X = x_j}]}) \right)(T) = 1,
  \]
  so $\mathrm{CPT}_X(T) = \Pr(X = x \mid Y_1 = y_1, \dots, Y_n = y_n)$ by a
  similar argument as before.
\end{proof}

Now, \cref{lemma:full_distribution} shows that $\phi$ represents the full
probability distribution of the Bayesian network, i.e., it gives the right
probabilities for the right inputs and zero otherwise. 

\begin{lemma}\label{lemma:full_distribution}
  Let $\mathcal{V} = \{X_1, \dots, X_n\}$. Then
  \[
    \phi(T) =
    \begin{cases}
      \Pr(x_1, \dots, x_n) &
                             \begin{aligned}
                               &\text{if } T = \{ \lambda_{X_i=x_i} \}_{i = 1}^n \text{ for} \\
                               &\text{some } \textstyle (x_i)_{i=1}^n \in \prod_{i=1}^n \Imm X_i
                             \end{aligned} \\
      0 & \text{otherwise,}
    \end{cases}
  \]
  for all $T \in 2^U$.
\end{lemma}
\begin{proof}
  If $T = \{ \lambda_{X=v_X} \mid X \in \mathcal{V} \}$ for some
  ${(v_X)}_{X \in \mathcal{V}} \in \prod_{X \in \mathcal{V}} \Imm X$, then
  \begin{align*}
    \phi(T) &= \prod_{X \in \mathcal{V}} \Pr \left( X=v_X \;\middle|\; \bigwedge_{Y \in \mathrm{pa}(X)} Y=v_Y \right) \\
            &= \Pr \left( \bigwedge_{X \in \mathcal{V}} X=v_X \right)
  \end{align*}
  by \cref{lemma:cpt} and the definition of a Bayesian network. Otherwise there
  must be some non-binary random variable $X \in \mathcal{V}$ such that
  $|\mathcal{E}(X) \cap T| \ne 1$. If $\mathcal{E}(X) \cap T = \emptyset$, then
  $\left( \sum_{i=1}^m [\lambda_{X = x_i}] \right)(T) = 0$, and so
  $\mathrm{CPT}_X(T) = 0$, and $\phi(T) = 0$. If $|\mathcal{E}(X) \cap T| > 1$,
  then we must have two different values $x_1, x_2 \in \Imm X$ such that
  $\{\lambda_{X=x_1}, \lambda_{X=x_2} \} \subseteq T$ which means that
  $(\overline{[\lambda_{X=x_1}]} + \overline{[\lambda_{X=x_2}]})(T) = 0$, and
  so, again, $\mathrm{CPT}_X(T) = 0$, and $\phi(T) = 0$.
\end{proof}

We end with \cref{thm:correctness} that shows how $\phi$ can be combined with an
encoding of a single variable-value assignment so that \textsf{ADDMC}
\citep{DBLP:conf/aaai/DudekPV20} would compute its marginal probability.

\begin{theorem}\label{thm:correctness}
  For any $X \in \mathcal{V}$ and $x \in \Imm X$,
  \[
    (\exists_U(\phi \cdot [\lambda_{X=x}]))(\emptyset) = \Pr(X = x).
  \]
\end{theorem}
\begin{proof}
  Let $\mathcal{V} = \{ X, Y_1, \dots, Y_n \}$. Then
  \begin{align*}
    (\exists_U (\phi \cdot [\lambda_{X=x}]))(\emptyset) &= \sum_{T \in 2^U} (\phi \cdot [\lambda_{X=x}])(T) \\
                                                        &= \sum_{\lambda_{X=x} \in T \in 2^U} \phi(T) \\
                                                        &= \sum_{\lambda_{X=x} \in T \in 2^U} \left( \prod_{Y \in \mathcal{V}} \mathrm{CPT}_Y \right)(T) \\
                                                        &= \sum_{{(y_i)}_{i=1}^n \in \prod_{i=1}^n \Imm Y_i} \Pr(x, y_1, \dots, y_n) \\
                                                        &= \Pr(X = x)
  \end{align*}
  by:
  \begin{itemize}
    \item the proof of Theorem~1 by \citet{DBLP:conf/aaai/DudekPV20};
    \item if $\lambda_{X=x} \not\in T \in 2^U$, then
    \[
      (\phi \cdot [\lambda_{X=x}])(T) = \phi(T) \cdot [\lambda_{X=x}](T \cap \{\lambda_{X=x} \}) = \phi(T) \cdot 0 = 0;
    \]
    \item \cref{lemma:full_distribution};
    \item marginalisation of a probability distribution.
  \end{itemize}
\end{proof}

\subsection{Textual Representation}\label{sec:textual_representation}

\Cref{alg:encoding} encodes a Bayesian network into a function on a Boolean
algebra, but how does it relate to the standard interpretation of a WMC encoding
as a formula in conjunctive normal form (CNF) together with a collection of
weights? The factors of $\phi$ that restrict the values of indicator variables
for non-binary random variables are already expressed as a product of sums of
0/1-valued functions, i.e., a kind of CNF\@. Disregarding these functions, each
conditional weight function $\mathrm{CPT}_X$ is represented by a sum with a term
for every subset of $\mathcal{E}^*(X)$. To encode these terms, we introduce
\emph{extended weight clauses} to the WMC format used by \textsf{Cachet}
\citep{DBLP:conf/sat/SangBBKP04}. For instance, here is a representation of the
Bayesian network from \cref{fig:example_bn}:
\[
  \begin{array}{lrrll}
    \lambda\sb{T=l} &\lambda\sb{T=m} &\lambda\sb{T=h} & &0 \\
                    &-\lambda\sb{T=l} &-\lambda\sb{T=m} & &0 \\
                    &-\lambda\sb{T=l} &-\lambda\sb{T=h} & &0 \\
                    &-\lambda\sb{T=m} &-\lambda\sb{T=h} & &0 \\
    w &\lambda\sb{W=1} & &0.5 &0.5 \\
    w &\lambda\sb{F=1} &\lambda\sb{W=1} &0.6 &0.4 \\
    w &\lambda\sb{F=1} &-\lambda\sb{W=1} &0.1 &0.9 \\
    w &\lambda\sb{T=l} &\lambda\sb{W=1} &0.2 &1 \\
    w &\lambda\sb{T=m} &\lambda\sb{W=1} &0.4 &1 \\
    w &\lambda\sb{T=h} &\lambda\sb{W=1} &0.4 &1 \\
    w &\lambda\sb{T=l} &-\lambda\sb{W=1} &0.6 &1 \\
    w &\lambda\sb{T=m} &-\lambda\sb{W=1} &0.3 &1 \\
    w &\lambda\sb{T=h} &-\lambda\sb{W=1} &0.1 &1
  \end{array}
\]
where each indicator variable is eventually replaced with a unique positive
integer. Each line prefixed with a $w$ can be split into four parts: the `main'
variable (always not negated), conditions (possibly none), and two weights. For
example, the line
\[
  \begin{array}{lrrll}
    w &\lambda\sb{T=m} &-\lambda\sb{W=1} &0.3 &1
  \end{array}
\]
encodes the function $0.3[\lambda_{T=m}] \cdot \overline{[\lambda_{W=1}]} +
1\overline{[\lambda_{T=m}]} \cdot \overline{[\lambda_{W=1}]}$ and can be
interpreted as defining two conditional weights: $\nu(T = m \mid W = 0) = 0.3$,
and $\nu(T \ne m \mid W = 0) = 1$, the former of which corresponds to a row in
the CPT of $T$ while the latter is artificially added as part of the encoding.
In our encoding of Bayesian networks, it is always the case that, in each weight
clause, either both weights sum to one, or the second weight is equal to one.
Finally, note that the measure induced by these weights is not probabilistic
(i.e., $\mu(\top) \ne 1$) by itself, but it becomes probabilistic when combined
with the additional clauses that restrict what combinations of indicator
variables can co-occur.

\subsection{Changes to \textsf{ADDMC}}

Here we describe two changes to
\textsf{ADDMC}\footnote{\url{https://github.com/vardigroup/ADDMC}}
\citep{DBLP:conf/aaai/DudekPV20} needed to adapt it to the new format.

First, \textsf{ADDMC} constructs the \emph{primal} (a.k.a. Gaifman) \emph{graph}
of the input CNF formula as an aid for the algorithm's heuristics. This graph
has as vertices the variables of the formula, and there is an edge between two
variables $u$ and $v$ if there is a clause in the formula that contains both $u$
and $v$. We extend this definition to functions on Boolean algebras, i.e., the
factors of $\phi$. For any pair of distinct variables $u, v \in U$, we draw an
edge between them in the primal graph if there is a function $\alpha\colon 2^X
\to \mathbb{R}_{\ge 0}$ that is a factor of $\phi$ such that $u, v \in X$. For
instance, a factor such as $\mathrm{CPT}_X$ will enable edges between all
distinct pairs of variables in $\mathcal{E}^*(X)$. Second, even though the
function $\phi$ produced by \cref{alg:encoding} is constructed to have $2^U$ as
its domain, sometimes the domain is effectively reduced to $2^V$ for some $V
\subset U$ by the ADD manipulation algorithms that optimise the ADD
representation of a function. For a simple example, consider $\alpha: 2^{\{a\}}
\to \mathbb{R}_{\ge 0}$ defined as $\alpha(\{a\}) = \alpha(\emptyset) = 0.5$.
Then $\alpha$ can be reduced to $\alpha'\colon 2^{\emptyset} \to \mathbb{R}_{\ge
  0}$ defined as $\alpha'(\emptyset) = 0.5$. To compensate for these reductions,
for the original WMC format with a weight function $w\colon U \cup \{ \neg u
\mid u \in U \} \to \mathbb{R}_{\ge 0}$, \textsf{ADDMC} would multiply its
computed answer by $\prod_{u \in U \setminus V} w(u) + w(\neg u)$. With the new
WMC format, we instead multiply the answer by $2^{|U \setminus V|}$. Each
`excluded' variable $u \in U \setminus V$ satisfies two properties: all weights
associated with $u$ are equal to $0.5$ (otherwise the corresponding CPT would
depend on $u$, and $u$ would not be excluded), and all other CPTs are
independent of $u$ (or they may have a trivial dependence, where the probability
stays the same if $u$ is replaced with its complement). Thus, the CPT that
corresponds to $u$ still multiplies the weight of every atom in the Boolean
algebra by $0.5$, but the number of atoms under consideration is halved. To
correct for this, we multiply the final answer by two for every $u \in U
\setminus V$.

\section{Experimental Results}\label{sec:2experiments}

We compare the six WMC encodings for Bayesian networks when run with both
\textsf{ADDMC} \citep{DBLP:conf/aaai/DudekPV20} and the WMC algorithms used in
the original papers.\footnote{Both \texttt{cd05} and \texttt{cd06} cannot be run
  with most WMC algorithms including \textsf{ADDMC} because these encodings
  allow for additional models that the WMC algorithm is supposed to ignore
  \citep{DBLP:conf/ijcai/ChaviraD05,DBLP:conf/sat/ChaviraD06}.} We compare the
encodings with respect to the total time it takes to encode a Bayesian network,
compile it or run a WMC algorithm on it, and extract the (numerical) answer.
Note that while all five papers that introduce other encodings include
experimental comparisons of encoding size, that is not feasible with
\textsf{ADDMC} as even instances that are fully solved in less than
\SI{0.1}{\second} are too big to build the full ADD within reasonable time and
memory limits. The experiments were run on a computing cluster with Intel Xeon
Gold 6138 and Intel Xeon E5-2630 processors\footnote{Each instance is run on the
  same processor for all encodings.} running Scientific~Linux~7 with a
\SI{32}{\gibi\byte} memory limit and a \SI{1000}{\second} timeout on both
encoding and inference. For inference, we use \textsf{Ace} for \texttt{cd05}
\citep{DBLP:conf/ijcai/ChaviraD05}, \texttt{cd06}
\citep{DBLP:conf/sat/ChaviraD06}, and \texttt{d02}
\citep{DBLP:conf/kr/Darwiche02};
\textsf{Cachet}\footnote{\url{https://cs.rochester.edu/u/kautz/Cachet/}}
\citep{DBLP:conf/sat/SangBBKP04} for \texttt{sbk05}
\citep{DBLP:conf/aaai/SangBK05}; and \textsf{c2d}
\citep{DBLP:conf/ecai/Darwiche04} for compilation and \textsf{query-dnnf}
\footnote{\url{http://www.cril.univ-artois.fr/kc/d-DNNF-reasoner.html}} for
answer computation for \texttt{bklm16} \citep{DBLP:conf/ecai/BartKLM16}. For
encoding, we use \textsf{bn2cnf}
\footnote{\url{http://www.cril.univ-artois.fr/KC/bn2cnf.html}} for
\texttt{bklm16}, and \textsf{Ace} for all other encodings (except for
\texttt{cw}, which is implemented in Python).

\textsf{Ace} was not used to encode evidence, as preliminary experiments
revealed that the evidence-encoding implementation contains bugs that can lead
to incorrect answers or a Java exception being thrown on some instances of the
data set (and the source code is not publicly available). Instead, we simply
list all the evidence as additional clauses in the encoding. Furthermore, to
ensure that \texttt{bklm16} \citep{DBLP:conf/ecai/BartKLM16} (whether run with
\textsf{ADDMC} \citep{DBLP:conf/aaai/DudekPV20} or \textsf{c2d}
\citep{DBLP:conf/ecai/Darwiche04}) returns correct answers on most instances, we
had to disable one of the improvements that \texttt{bklm16} brings over
\texttt{cd06} \citep{DBLP:conf/sat/ChaviraD06}, namely, the construction of a
scaling factor that `absorbs' one probability from each CDT
\citep{DBLP:conf/ecai/BartKLM16}. For realistic benchmark instances, this
scaling factor can easily be below $10^{-30}$, and thus would require
arbitrary-precision floating-point arithmetic to be usable. Even a toy Bayesian
network with seven binary independent variables with probabilities $0.1$ and
$0.9$ is enough for \textsf{bn2cnf} to output precisely zero as the scaling
factor. We note that this issue likely remained unnoticed because
\citet{DBLP:conf/ecai/BartKLM16} did not attempt to compute numerical answers in
their experiments.

For each Bayesian network, we need to choose a probability to compute. Whenever
a Bayesian network comes with an evidence file, we compute the probability of
evidence. Otherwise, let $X$ denote the last-mentioned vertex in the Bayesian
network. If $\true{} \in \Imm X$, then we compute the marginal probability of
$X = \true{}$. Otherwise, we pick the value of $X$ which is listed first and
calculate its marginal probability.

For experimental data, we use the Bayesian networks available with \textsf{Ace}
and \textsf{Cachet} \citep{DBLP:conf/sat/SangBBKP04}, most of which happen to be
binary. We classify them into the following seven categories:
\begin{itemize}
\item DQMR and
\item Grid networks as described by \citet{DBLP:conf/aaai/SangBK05},
\item Mastermind, and
\item Random Blocks from the work of \citet{DBLP:journals/ijar/ChaviraDJ06},
\item remaining binary Bayesian networks that include Plan Recognition
  \citep{DBLP:conf/aaai/SangBK05}, Friends and Smokers, Students and Professors
  \citep{DBLP:journals/ijar/ChaviraDJ06}, and \texttt{tcc4f}, and
\item non-binary classic Bayesian networks (\texttt{alarm}, \texttt{diabetes},
  \texttt{hailfinder}, \texttt{mildew}, \texttt{munin1}--\texttt{4},
  \texttt{pathfinder}, \texttt{pigs}, \texttt{water}).
\end{itemize}

\begin{figure}[t]
  \centering
  % Created by tikzDevice version 0.12.3.1 on 2021-02-09 19:25:55
% !TEX encoding = UTF-8 Unicode
\begin{tikzpicture}[x=1pt,y=1pt]
\definecolor{fillColor}{RGB}{255,255,255}
\path[use as bounding box,fill=fillColor,fill opacity=0.00] (0,0) rectangle (469.75,173.45);
\begin{scope}
\path[clip] (  0.00,  0.00) rectangle (469.75,173.45);
\definecolor{drawColor}{RGB}{255,255,255}
\definecolor{fillColor}{RGB}{255,255,255}

\path[draw=drawColor,line width= 0.6pt,line join=round,line cap=round,fill=fillColor] (  0.00,  0.00) rectangle (469.76,173.45);
\end{scope}
\begin{scope}
\path[clip] ( 40.51, 30.69) rectangle (332.10,167.95);
\definecolor{fillColor}{RGB}{255,255,255}

\path[fill=fillColor] ( 40.51, 30.69) rectangle (332.10,167.95);
\definecolor{drawColor}{gray}{0.87}

\path[draw=drawColor,line width= 0.1pt,line join=round] ( 40.51, 61.43) --
	(332.10, 61.43);

\path[draw=drawColor,line width= 0.1pt,line join=round] ( 40.51,110.63) --
	(332.10,110.63);

\path[draw=drawColor,line width= 0.1pt,line join=round] ( 40.51,159.84) --
	(332.10,159.84);

\path[draw=drawColor,line width= 0.1pt,line join=round] ( 92.20, 30.69) --
	( 92.20,167.95);

\path[draw=drawColor,line width= 0.1pt,line join=round] (156.60, 30.69) --
	(156.60,167.95);

\path[draw=drawColor,line width= 0.1pt,line join=round] (220.99, 30.69) --
	(220.99,167.95);

\path[draw=drawColor,line width= 0.1pt,line join=round] (285.39, 30.69) --
	(285.39,167.95);

\path[draw=drawColor,line width= 0.3pt,line join=round] ( 40.51, 36.83) --
	(332.10, 36.83);

\path[draw=drawColor,line width= 0.3pt,line join=round] ( 40.51, 86.03) --
	(332.10, 86.03);

\path[draw=drawColor,line width= 0.3pt,line join=round] ( 40.51,135.24) --
	(332.10,135.24);

\path[draw=drawColor,line width= 0.3pt,line join=round] ( 60.00, 30.69) --
	( 60.00,167.95);

\path[draw=drawColor,line width= 0.3pt,line join=round] (124.40, 30.69) --
	(124.40,167.95);

\path[draw=drawColor,line width= 0.3pt,line join=round] (188.80, 30.69) --
	(188.80,167.95);

\path[draw=drawColor,line width= 0.3pt,line join=round] (253.19, 30.69) --
	(253.19,167.95);

\path[draw=drawColor,line width= 0.3pt,line join=round] (317.59, 30.69) --
	(317.59,167.95);
\definecolor{drawColor}{RGB}{253,191,111}

\path[draw=drawColor,line width= 0.6pt,dash pattern=on 4pt off 4pt ,line join=round] (137.89, 37.02) --
	(138.06, 37.12) --
	(138.23, 37.22) --
	(138.57, 37.52) --
	(138.57, 37.32) --
	(138.74, 37.61) --
	(138.91, 37.81) --
	(138.91, 37.71) --
	(139.24, 38.11) --
	(139.24, 38.20) --
	(139.57, 38.30) --
	(139.73, 38.60) --
	(139.89, 38.79) --
	(140.05, 38.89) --
	(140.84, 39.29) --
	(140.84, 39.09) --
	(140.84, 39.39) --
	(140.99, 39.48) --
	(141.15, 39.68) --
	(141.30, 39.88) --
	(141.30, 39.98) --
	(141.60, 40.27) --
	(141.60, 40.76) --
	(141.76, 40.96) --
	(141.76, 41.06) --
	(141.91, 41.55) --
	(141.91, 41.65) --
	(142.05, 41.94) --
	(142.20, 42.24) --
	(142.35, 42.53) --
	(142.50, 43.03) --
	(142.50, 42.93) --
	(142.64, 43.81) --
	(142.79, 44.11) --
	(142.79, 44.01) --
	(142.93, 44.40) --
	(143.08, 44.70) --
	(143.08, 44.60) --
	(143.22, 45.09) --
	(143.36, 45.78) --
	(143.50, 46.47) --
	(143.64, 46.86) --
	(143.78, 47.06) --
	(143.92, 47.85) --
	(144.06, 48.24) --
	(144.20, 48.34) --
	(144.20, 48.64) --
	(144.34, 49.03) --
	(144.48, 49.62) --
	(144.61, 49.92) --
	(144.75, 50.11) --
	(144.75, 50.01) --
	(144.88, 50.51) --
	(145.02, 50.80) --
	(145.15, 51.00) --
	(145.28, 51.49) --
	(145.41, 51.78) --
	(145.55, 51.88) --
	(145.68, 52.08) --
	(145.68, 52.57) --
	(145.81, 53.16) --
	(145.94, 53.46) --
	(146.07, 53.85) --
	(146.19, 54.05) --
	(146.32, 54.34) --
	(146.58, 54.64) --
	(146.70, 54.84) --
	(146.70, 54.93) --
	(146.83, 55.23) --
	(146.95, 55.62) --
	(147.08, 55.92) --
	(147.20, 56.02) --
	(147.33, 56.21) --
	(147.45, 56.31) --
	(147.57, 56.51) --
	(147.69, 56.90) --
	(147.81, 57.10) --
	(148.18, 57.39) --
	(148.29, 57.49) --
	(148.41, 57.59) --
	(148.41, 57.79) --
	(148.53, 58.08) --
	(148.65, 58.58) --
	(148.77, 58.67) --
	(148.88, 58.77) --
	(149.00, 58.97) --
	(149.12, 59.07) --
	(149.35, 59.17) --
	(149.57, 59.26) --
	(149.69, 59.56) --
	(149.80, 59.85) --
	(149.91, 60.05) --
	(150.14, 60.15) --
	(150.25, 60.25) --
	(150.36, 60.35) --
	(150.47, 60.54) --
	(150.69, 60.64) --
	(150.80, 60.74) --
	(151.01, 60.94) --
	(151.12, 61.13) --
	(151.23, 61.23) --
	(151.34, 61.63) --
	(151.44, 61.92) --
	(151.55, 62.12) --
	(151.65, 62.22) --
	(151.65, 62.41) --
	(151.76, 62.51) --
	(151.87, 62.81) --
	(152.07, 63.20) --
	(152.18, 63.69) --
	(152.28, 64.18) --
	(152.38, 64.28) --
	(152.49, 64.58) --
	(152.59, 64.97) --
	(152.69, 65.07) --
	(152.79, 65.27) --
	(152.89, 65.37) --
	(153.09, 65.66) --
	(153.19, 65.86) --
	(153.29, 66.05) --
	(153.49, 66.25) --
	(153.59, 66.35) --
	(153.69, 66.45) --
	(153.79, 66.74) --
	(153.89, 66.84) --
	(153.98, 67.04) --
	(154.18, 67.24) --
	(154.27, 67.43) --
	(154.37, 67.63) --
	(154.56, 67.73) --
	(154.75, 67.83) --
	(154.84, 67.92) --
	(155.03, 68.32) --
	(155.12, 68.61) --
	(155.40, 68.71) --
	(155.49, 68.91) --
	(155.59, 69.20) --
	(155.77, 69.30) --
	(155.77, 69.40) --
	(155.95, 69.50) --
	(156.04, 69.60) --
	(156.22, 69.70) --
	(156.31, 69.79) --
	(156.40, 69.89) --
	(156.49, 69.99) --
	(156.58, 70.09) --
	(156.67, 70.29) --
	(156.75, 70.48) --
	(156.75, 70.38) --
	(156.84, 70.58) --
	(156.93, 70.68) --
	(157.10, 70.78) --
	(157.19, 70.88) --
	(157.19, 70.97) --
	(157.28, 71.07) --
	(157.36, 71.17) --
	(157.45, 71.27) --
	(157.53, 71.37) --
	(157.62, 71.47) --
	(157.79, 71.86) --
	(157.87, 71.96) --
	(157.96, 72.06) --
	(158.04, 72.45) --
	(158.13, 72.55) --
	(158.21, 72.65) --
	(158.38, 72.75) --
	(158.54, 72.84) --
	(158.54, 72.94) --
	(158.62, 73.04) --
	(158.62, 73.24) --
	(158.71, 73.34) --
	(158.79, 73.44) --
	(158.87, 73.63) --
	(158.87, 73.73) --
	(159.11, 73.83) --
	(159.19, 73.93) --
	(159.27, 74.03) --
	(159.44, 74.42) --
	(159.52, 74.52) --
	(159.52, 74.62) --
	(159.67, 74.71) --
	(159.75, 74.81) --
	(159.91, 75.01) --
	(159.91, 75.11) --
	(160.07, 75.21) --
	(160.15, 75.40) --
	(160.22, 75.70) --
	(160.22, 75.50) --
	(160.30, 75.99) --
	(160.30, 75.80) --
	(160.46, 76.09) --
	(160.61, 76.29) --
	(160.76, 76.58) --
	(160.91, 76.68) --
	(160.99, 76.78) --
	(160.99, 77.17) --
	(161.22, 77.27) --
	(161.36, 77.47) --
	(161.44, 77.67) --
	(161.51, 77.86) --
	(161.51, 77.96) --
	(161.74, 78.06) --
	(161.88, 78.16) --
	(161.88, 78.26) --
	(162.03, 78.55) --
	(162.10, 78.75) --
	(162.17, 78.95) --
	(162.25, 79.44) --
	(162.32, 79.63) --
	(162.39, 79.83) --
	(162.39, 79.93) --
	(162.46, 80.03) --
	(162.53, 80.13) --
	(162.60, 80.23) --
	(162.68, 80.52) --
	(162.68, 80.42) --
	(162.75, 80.62) --
	(162.82, 80.82) --
	(162.89, 80.91) --
	(162.96, 81.01) --
	(163.03, 81.11) --
	(163.17, 81.21) --
	(163.24, 81.41) --
	(163.31, 81.50) --
	(163.38, 81.60) --
	(163.38, 81.80) --
	(163.45, 82.00) --
	(163.59, 82.19) --
	(163.65, 82.29) --
	(163.72, 82.39) --
	(163.79, 82.59) --
	(163.93, 82.69) --
	(164.00, 82.88) --
	(164.06, 83.08) --
	(164.20, 83.28) --
	(164.33, 83.47) --
	(164.40, 83.57) --
	(164.53, 83.77) --
	(164.60, 83.87) --
	(164.67, 83.96) --
	(164.73, 84.26) --
	(164.87, 84.36) --
	(164.93, 84.65) --
	(165.00, 84.75) --
	(165.06, 84.85) --
	(165.19, 84.95) --
	(165.26, 85.15) --
	(165.26, 85.05) --
	(165.32, 85.24) --
	(165.45, 85.54) --
	(165.71, 85.74) --
	(165.77, 85.83) --
	(165.84, 85.93) --
	(165.96, 86.03) --
	(166.15, 86.23) --
	(166.28, 86.43) --
	(166.28, 86.52) --
	(166.40, 86.82) --
	(166.53, 87.11) --
	(166.59, 87.21) --
	(166.65, 87.41) --
	(166.65, 87.31) --
	(166.71, 87.70) --
	(166.83, 87.80) --
	(166.90, 87.90) --
	(166.96, 88.00) --
	(167.02, 88.10) --
	(167.08, 88.29) --
	(167.14, 88.39) --
	(167.20, 88.49) --
	(167.32, 88.69) --
	(167.32, 88.59) --
	(167.44, 88.79) --
	(167.50, 89.08) --
	(167.56, 89.18) --
	(167.68, 89.28) --
	(167.74, 89.38) --
	(167.80, 89.77) --
	(167.80, 89.48) --
	(167.92, 89.87) --
	(167.98, 90.16) --
	(168.03, 90.26) --
	(168.09, 90.46) --
	(168.15, 90.56) --
	(168.21, 90.66) --
	(168.27, 90.76) --
	(168.33, 90.85) --
	(168.38, 91.05) --
	(168.50, 91.44) --
	(168.56, 91.54) --
	(168.67, 91.64) --
	(168.73, 91.74) --
	(168.79, 91.84) --
	(168.90, 91.94) --
	(169.02, 92.03) --
	(169.07, 92.13) --
	(169.19, 92.23) --
	(169.24, 92.33) --
	(169.35, 92.53) --
	(169.41, 92.72) --
	(169.47, 92.82) --
	(169.52, 93.02) --
	(169.63, 93.12) --
	(169.69, 93.22) --
	(169.74, 93.31) --
	(169.80, 93.41) --
	(169.85, 93.61) --
	(169.91, 93.71) --
	(170.02, 93.90) --
	(170.07, 94.20) --
	(170.18, 94.30) --
	(170.24, 94.40) --
	(170.29, 94.49) --
	(170.35, 94.69) --
	(170.45, 94.89) --
	(170.51, 94.99) --
	(170.56, 95.09) --
	(170.61, 95.28) --
	(170.67, 95.38) --
	(170.77, 95.68) --
	(170.83, 95.87) --
	(170.88, 96.17) --
	(170.88, 95.97) --
	(170.93, 96.46) --
	(170.99, 96.56) --
	(171.04, 96.66) --
	(171.09, 96.76) --
	(171.20, 96.86) --
	(171.25, 96.96) --
	(171.35, 97.05) --
	(171.67, 97.25) --
	(171.77, 97.35) --
	(172.08, 97.45) --
	(172.13, 97.55) --
	(172.18, 97.64) --
	(172.23, 97.84) --
	(172.28, 97.94) --
	(172.38, 98.04) --
	(172.43, 98.14) --
	(172.48, 98.23) --
	(172.53, 98.33) --
	(172.58, 98.53) --
	(172.63, 98.63) --
	(172.68, 98.73) --
	(172.78, 98.82) --
	(172.83, 98.92) --
	(172.88, 99.02) --
	(172.98, 99.32) --
	(173.03, 99.61) --
	(173.07, 99.81) --
	(173.17,100.01) --
	(173.22,100.10) --
	(173.32,100.20) --
	(173.56,100.30) --
	(173.66,100.40) --
	(173.75,100.60) --
	(173.90,100.69) --
	(174.23,100.89) --
	(174.23,100.79) --
	(174.32,100.99) --
	(174.42,101.09) --
	(174.46,101.19) --
	(174.65,101.29) --
	(174.74,101.38) --
	(174.83,101.48) --
	(174.88,101.58) --
	(174.93,101.68) --
	(175.06,101.88) --
	(175.15,101.97) --
	(175.20,102.07) --
	(175.20,102.17) --
	(175.25,102.27) --
	(175.38,102.37) --
	(175.43,102.47) --
	(175.61,102.56) --
	(175.74,102.66) --
	(176.01,102.76) --
	(176.09,102.86) --
	(176.18,103.06) --
	(176.31,103.15) --
	(176.40,103.25) --
	(176.53,103.45) --
	(176.57,103.55) --
	(176.66,103.65) --
	(176.79,103.75) --
	(176.83,103.84) --
	(176.88,104.04) --
	(177.00,104.14) --
	(177.43,104.24) --
	(177.51,104.34) --
	(177.55,104.43) --
	(177.68,104.53) --
	(177.72,104.63) --
	(177.72,104.73) --
	(177.80,104.83) --
	(177.89,104.93) --
	(177.89,105.02) --
	(178.09,105.12) --
	(178.26,105.22) --
	(178.30,105.32) --
	(178.54,105.42) --
	(178.58,105.52) --
	(178.66,105.71) --
	(178.78,105.81) --
	(178.82,105.91) --
	(178.90,106.01) --
	(178.98,106.11) --
	(179.14,106.21) --
	(179.22,106.30) --
	(179.26,106.40) --
	(179.41,106.50) --
	(179.45,106.60) --
	(179.49,106.80) --
	(179.53,106.89) --
	(179.61,106.99) --
	(179.65,107.09) --
	(179.69,107.19) --
	(179.72,107.29) --
	(179.76,107.39) --
	(179.80,107.48) --
	(179.92,107.58) --
	(179.99,107.68) --
	(180.41,107.78) --
	(180.49,107.98) --
	(180.79,108.17) --
	(180.82,108.27) --
	(180.90,108.37) --
	(181.12,108.47) --
	(181.19,108.67) --
	(181.19,108.57) --
	(181.27,108.86) --
	(181.27,108.76) --
	(181.30,108.96) --
	(181.34,109.06) --
	(181.41,109.16) --
	(181.74,109.26) --
	(181.81,109.35) --
	(181.81,109.45) --
	(181.95,109.55) --
	(181.99,109.65) --
	(181.99,109.75) --
	(182.10,109.85) --
	(182.13,109.95) --
	(182.24,110.04) --
	(182.41,110.14) --
	(182.87,110.24) --
	(183.01,110.34) --
	(183.11,110.44) --
	(183.18,110.54) --
	(183.25,110.63) --
	(183.38,110.73) --
	(183.42,110.83) --
	(183.48,110.93) --
	(183.52,111.03) --
	(183.58,111.13) --
	(183.62,111.22) --
	(183.68,111.32) --
	(183.92,111.42) --
	(184.02,111.52) --
	(184.18,111.62) --
	(184.35,111.72) --
	(184.61,111.81) --
	(184.64,111.91) --
	(184.67,112.11) --
	(184.74,112.21) --
	(185.35,112.31) --
	(185.50,112.41) --
	(185.63,112.50) --
	(185.66,112.70) --
	(186.00,112.80) --
	(186.13,112.90) --
	(186.16,113.00) --
	(186.83,113.09) --
	(186.95,113.19) --
	(187.21,113.29) --
	(187.51,113.39) --
	(187.54,113.49) --
	(187.57,113.59) --
	(187.60,113.68) --
	(187.71,113.78) --
	(187.80,113.88) --
	(187.89,113.98) --
	(187.91,114.08) --
	(188.17,114.18) --
	(188.37,114.28) --
	(188.49,114.37) --
	(188.63,114.47) --
	(188.68,114.57) --
	(188.85,114.67) --
	(189.02,114.77) --
	(189.07,114.87) --
	(189.10,115.06) --
	(189.16,115.16) --
	(189.24,115.26) --
	(189.35,115.36) --
	(189.49,115.46) --
	(189.54,115.55) --
	(189.59,115.65) --
	(189.78,115.75) --
	(189.92,115.85) --
	(190.16,115.95) --
	(190.45,116.05) --
	(191.21,116.15) --
	(191.41,116.24) --
	(191.49,116.34) --
	(191.91,116.44) --
	(191.96,116.54) --
	(192.11,116.64) --
	(192.19,116.74) --
	(192.26,116.83) --
	(192.66,116.93) --
	(192.87,117.13) --
	(192.95,117.23) --
	(193.04,117.33) --
	(193.09,117.42) --
	(193.11,117.52) --
	(193.23,117.62) --
	(193.57,117.72) --
	(193.78,117.82) --
	(194.13,117.92) --
	(194.70,118.01) --
	(194.92,118.11) --
	(195.15,118.21) --
	(195.26,118.31) --
	(195.66,118.41) --
	(195.70,118.51) --
	(195.98,118.61) --
	(196.00,118.70) --
	(196.07,118.80) --
	(196.13,118.90) --
	(196.39,119.10) --
	(197.06,119.20) --
	(197.33,119.29) --
	(197.66,119.39) --
	(197.82,119.49) --
	(197.84,119.59) --
	(198.32,119.69) --
	(198.46,119.79) --
	(198.62,119.88) --
	(199.15,119.98) --
	(199.44,120.08) --
	(199.47,120.18) --
	(199.57,120.28) --
	(199.83,120.38) --
	(199.89,120.48) --
	(200.02,120.57) --
	(200.87,120.67) --
	(201.52,120.77) --
	(201.68,120.87) --
	(203.70,120.97) --
	(203.78,121.07) --
	(203.96,121.16) --
	(203.98,121.26) --
	(204.11,121.36) --
	(204.25,121.46) --
	(204.37,121.66) --
	(204.68,121.75) --
	(204.87,121.85) --
	(205.09,121.95) --
	(205.11,122.05) --
	(205.36,122.15) --
	(205.57,122.25) --
	(205.60,122.34) --
	(205.62,122.44) --
	(205.74,122.54) --
	(205.98,122.64) --
	(206.03,122.74) --
	(206.46,122.84) --
	(206.51,122.94) --
	(206.73,123.03) --
	(206.83,123.13) --
	(207.05,123.23) --
	(207.96,123.33) --
	(208.35,123.43) --
	(208.93,123.53) --
	(209.16,123.62) --
	(209.17,123.72) --
	(209.21,123.82) --
	(209.25,123.92) --
	(209.42,124.02) --
	(209.90,124.12) --
	(209.94,124.21) --
	(210.56,124.31) --
	(211.45,124.41) --
	(211.60,124.51) --
	(211.75,124.61) --
	(212.11,124.71) --
	(212.21,124.81) --
	(212.71,124.90) --
	(212.89,125.00) --
	(212.97,125.10) --
	(213.02,125.20) --
	(213.24,125.30) --
	(213.40,125.40) --
	(213.45,125.49) --
	(213.58,125.59) --
	(214.25,125.69) --
	(214.44,125.79) --
	(214.58,125.89) --
	(214.70,125.99) --
	(215.02,126.08) --
	(215.19,126.18) --
	(215.46,126.28) --
	(216.35,126.38) --
	(216.45,126.48) --
	(216.85,126.58) --
	(216.99,126.67) --
	(217.11,126.77) --
	(217.25,126.87) --
	(217.31,126.97) --
	(217.42,127.07) --
	(217.43,127.17) --
	(217.52,127.27) --
	(218.20,127.36) --
	(218.54,127.46) --
	(218.58,127.56) --
	(218.70,127.66) --
	(219.29,127.76) --
	(219.84,127.86) --
	(220.21,127.95) --
	(220.22,128.05) --
	(220.38,128.15) --
	(220.45,128.25) --
	(220.56,128.35) --
	(220.73,128.45) --
	(220.95,128.54) --
	(220.98,128.64) --
	(221.16,128.74) --
	(221.56,128.84) --
	(221.77,128.94) --
	(221.86,129.04) --
	(222.58,129.14) --
	(222.59,129.23) --
	(222.68,129.33) --
	(224.16,129.43) --
	(224.38,129.53) --
	(224.44,129.63) --
	(225.03,129.73) --
	(225.16,129.82) --
	(225.22,129.92) --
	(226.29,130.12) --
	(226.31,130.22) --
	(227.47,130.32) --
	(227.84,130.41) --
	(228.31,130.51) --
	(228.38,130.61) --
	(228.59,130.71) --
	(229.06,130.81) --
	(229.73,130.91) --
	(229.96,131.00) --
	(230.12,131.10) --
	(231.07,131.20) --
	(231.14,131.30) --
	(231.20,131.40) --
	(231.82,131.50) --
	(231.90,131.60) --
	(232.01,131.69) --
	(232.11,131.79) --
	(232.46,131.89) --
	(232.46,131.99) --
	(232.48,132.09) --
	(232.72,132.19) --
	(232.98,132.28) --
	(233.06,132.38) --
	(233.07,132.48) --
	(233.09,132.58) --
	(233.21,132.68) --
	(233.43,132.78) --
	(233.46,132.87) --
	(233.59,132.97) --
	(233.79,133.07) --
	(234.04,133.17) --
	(234.47,133.27) --
	(234.96,133.37) --
	(235.29,133.47) --
	(236.12,133.56) --
	(236.25,133.66) --
	(236.50,133.76) --
	(236.61,133.86) --
	(237.13,133.96) --
	(237.16,134.06) --
	(238.50,134.15) --
	(238.52,134.25) --
	(239.16,134.35) --
	(239.66,134.45) --
	(239.68,134.55) --
	(239.88,134.65) --
	(240.96,134.74) --
	(240.97,134.84) --
	(241.22,134.94) --
	(241.28,135.04) --
	(241.42,135.14) --
	(241.61,135.24) --
	(241.83,135.33) --
	(242.08,135.43) --
	(242.17,135.53) --
	(242.84,135.63) --
	(242.91,135.73) --
	(243.08,135.83) --
	(243.22,135.93) --
	(243.72,136.02) --
	(243.81,136.12) --
	(243.88,136.22) --
	(243.92,136.32) --
	(244.16,136.42) --
	(244.35,136.52) --
	(244.52,136.61) --
	(244.59,136.71) --
	(245.22,136.81) --
	(245.39,136.91) --
	(245.42,137.01) --
	(245.54,137.11) --
	(245.97,137.20) --
	(246.03,137.30) --
	(246.06,137.40) --
	(246.07,137.50) --
	(246.16,137.60) --
	(246.35,137.70) --
	(246.37,137.80) --
	(246.43,137.89) --
	(246.44,137.99) --
	(246.66,138.09) --
	(246.69,138.19) --
	(246.85,138.29) --
	(246.96,138.39) --
	(247.12,138.48) --
	(247.18,138.58) --
	(247.36,138.68) --
	(247.58,138.78) --
	(248.09,138.88) --
	(248.09,138.98) --
	(248.13,139.07) --
	(248.19,139.17) --
	(248.34,139.27) --
	(248.36,139.37) --
	(248.54,139.47) --
	(248.72,139.57) --
	(248.83,139.66) --
	(248.99,139.76) --
	(249.27,139.86) --
	(249.92,139.96) --
	(250.26,140.06) --
	(250.36,140.16) --
	(250.40,140.26) --
	(251.10,140.35) --
	(252.40,140.45) --
	(253.61,140.55) --
	(254.15,140.65) --
	(254.22,140.75) --
	(254.34,140.85) --
	(255.23,140.94) --
	(255.24,141.04) --
	(255.31,141.14) --
	(255.39,141.24) --
	(255.67,141.34) --
	(255.94,141.44) --
	(256.47,141.53) --
	(256.76,141.63) --
	(256.85,141.73) --
	(257.55,141.83) --
	(257.96,141.93) --
	(258.17,142.03) --
	(258.34,142.13) --
	(259.62,142.22) --
	(259.65,142.32) --
	(260.28,142.42) --
	(260.58,142.52) --
	(260.58,142.62) --
	(260.61,142.72) --
	(262.84,142.81) --
	(262.89,142.91) --
	(263.01,143.01) --
	(263.23,143.11) --
	(263.40,143.21) --
	(263.79,143.31) --
	(263.87,143.40) --
	(264.02,143.50) --
	(265.44,143.60) --
	(265.47,143.70) --
	(266.26,143.80) --
	(266.28,143.90) --
	(267.04,144.00) --
	(267.05,144.09) --
	(267.20,144.19) --
	(268.09,144.29) --
	(268.58,144.39) --
	(270.72,144.49) --
	(270.88,144.59) --
	(270.88,144.68) --
	(271.09,144.78) --
	(271.46,144.88) --
	(272.69,144.98) --
	(272.85,145.08) --
	(272.85,145.18) --
	(272.91,145.27) --
	(272.92,145.37) --
	(274.05,145.47) --
	(274.71,145.57) --
	(275.84,145.67) --
	(276.17,145.77) --
	(276.20,145.86) --
	(276.57,145.96) --
	(276.80,146.06) --
	(277.17,146.16) --
	(278.09,146.26) --
	(278.24,146.36) --
	(279.30,146.46) --
	(279.70,146.55) --
	(281.29,146.65) --
	(282.02,146.75) --
	(282.02,146.85) --
	(283.33,146.95) --
	(284.38,147.05) --
	(284.56,147.14) --
	(284.58,147.24) --
	(284.79,147.34) --
	(284.93,147.44) --
	(285.10,147.54) --
	(285.58,147.64) --
	(285.73,147.73) --
	(286.44,147.83) --
	(287.73,147.93) --
	(288.47,148.03) --
	(288.94,148.13) --
	(289.34,148.23) --
	(290.02,148.33) --
	(290.18,148.42) --
	(291.16,148.52) --
	(293.13,148.62) --
	(293.18,148.72) --
	(294.88,148.82) --
	(294.97,148.92) --
	(295.36,149.01) --
	(296.84,149.11) --
	(297.65,149.21) --
	(297.89,149.31) --
	(298.85,149.41) --
	(299.73,149.51) --
	(299.90,149.60) --
	(300.21,149.70) --
	(300.55,149.80) --
	(300.75,149.90) --
	(301.09,150.00) --
	(302.02,150.10) --
	(302.69,150.19) --
	(303.66,150.29) --
	(303.94,150.39) --
	(304.07,150.49) --
	(304.23,150.59) --
	(304.30,150.69) --
	(305.19,150.79) --
	(305.42,150.88) --
	(305.52,150.98) --
	(305.71,151.08) --
	(306.58,151.18) --
	(307.17,151.28) --
	(307.34,151.38) --
	(309.13,151.47) --
	(309.89,151.57) --
	(310.81,151.67) --
	(312.46,151.77) --
	(314.33,151.87) --
	(314.39,151.97) --
	(314.58,152.06) --
	(315.47,152.16) --
	(315.67,152.26) --
	(316.25,152.36) --
	(316.87,152.46) --
	(317.40,152.56);
\definecolor{drawColor}{RGB}{251,154,153}

\path[draw=drawColor,line width= 0.6pt,dash pattern=on 4pt off 4pt ,line join=round] (137.37, 36.93) --
	(137.54, 37.02) --
	(137.72, 37.22) --
	(137.89, 37.32) --
	(138.57, 37.42) --
	(139.24, 37.52) --
	(139.40, 37.61) --
	(139.57, 38.11) --
	(139.73, 38.40) --
	(139.89, 38.79) --
	(140.05, 38.89) --
	(140.21, 39.09) --
	(140.21, 38.99) --
	(140.37, 39.29) --
	(140.53, 39.48) --
	(140.53, 39.39) --
	(140.68, 39.88) --
	(140.84, 40.07) --
	(140.99, 40.27) --
	(141.15, 40.76) --
	(141.30, 40.96) --
	(141.45, 41.26) --
	(141.45, 41.06) --
	(141.60, 41.35) --
	(141.60, 41.55) --
	(141.76, 41.65) --
	(141.76, 41.75) --
	(141.91, 42.14) --
	(141.91, 42.04) --
	(142.05, 42.44) --
	(142.05, 42.53) --
	(142.20, 42.73) --
	(142.20, 42.83) --
	(142.35, 43.12) --
	(142.50, 43.32) --
	(142.50, 43.22) --
	(142.64, 43.52) --
	(142.79, 44.01) --
	(142.79, 43.81) --
	(142.93, 44.40) --
	(143.08, 44.99) --
	(143.08, 44.60) --
	(143.22, 45.29) --
	(143.36, 45.78) --
	(143.50, 46.18) --
	(143.64, 46.67) --
	(143.78, 46.96) --
	(143.92, 47.36) --
	(143.92, 47.45) --
	(144.06, 48.05) --
	(144.20, 48.34) --
	(144.20, 48.44) --
	(144.34, 48.73) --
	(144.48, 49.32) --
	(144.61, 49.72) --
	(144.75, 50.01) --
	(144.75, 50.11) --
	(144.88, 50.60) --
	(145.02, 51.00) --
	(145.15, 51.59) --
	(145.28, 51.88) --
	(145.41, 52.18) --
	(145.55, 52.57) --
	(145.68, 53.06) --
	(145.81, 53.26) --
	(145.94, 53.65) --
	(146.07, 54.15) --
	(146.19, 54.25) --
	(146.19, 54.34) --
	(146.32, 54.64) --
	(146.45, 54.93) --
	(146.58, 55.43) --
	(146.70, 55.62) --
	(146.83, 55.72) --
	(146.95, 56.11) --
	(147.08, 56.21) --
	(147.20, 56.31) --
	(147.33, 56.41) --
	(147.45, 56.51) --
	(147.69, 56.90) --
	(147.81, 57.39) --
	(147.94, 57.49) --
	(147.94, 57.69) --
	(148.06, 57.79) --
	(148.18, 58.18) --
	(148.29, 58.28) --
	(148.41, 58.38) --
	(148.53, 58.58) --
	(148.65, 58.67) --
	(148.77, 58.77) --
	(149.12, 59.17) --
	(149.23, 59.26) --
	(149.35, 59.36) --
	(149.35, 59.46) --
	(149.46, 59.56) --
	(149.57, 59.85) --
	(149.69, 59.95) --
	(149.80, 60.05) --
	(149.91, 60.25) --
	(150.03, 60.44) --
	(150.14, 60.54) --
	(150.25, 60.84) --
	(150.25, 60.64) --
	(150.36, 60.94) --
	(150.58, 61.23) --
	(150.69, 61.63) --
	(150.69, 61.33) --
	(150.91, 61.72) --
	(151.01, 61.92) --
	(151.12, 62.12) --
	(151.23, 62.31) --
	(151.23, 62.41) --
	(151.34, 62.61) --
	(151.44, 62.91) --
	(151.55, 63.00) --
	(151.65, 63.30) --
	(151.76, 63.59) --
	(151.87, 63.79) --
	(151.97, 63.89) --
	(152.07, 64.18) --
	(152.18, 64.38) --
	(152.28, 64.48) --
	(152.38, 64.87) --
	(152.49, 64.97) --
	(152.59, 65.17) --
	(152.69, 65.46) --
	(152.89, 65.76) --
	(153.09, 66.15) --
	(153.19, 66.25) --
	(153.29, 66.55) --
	(153.39, 66.74) --
	(153.49, 66.94) --
	(153.59, 67.24) --
	(153.79, 67.53) --
	(153.79, 67.43) --
	(153.89, 67.83) --
	(153.98, 68.12) --
	(154.08, 68.22) --
	(154.18, 68.51) --
	(154.18, 68.32) --
	(154.27, 68.61) --
	(154.37, 68.81) --
	(154.46, 69.01) --
	(154.46, 68.91) --
	(154.56, 69.20) --
	(154.56, 69.11) --
	(154.65, 69.40) --
	(154.75, 69.50) --
	(154.84, 69.70) --
	(154.94, 69.89) --
	(155.03, 69.99) --
	(155.12, 70.09) --
	(155.22, 70.19) --
	(155.31, 70.48) --
	(155.59, 70.68) --
	(155.68, 70.88) --
	(155.86, 70.97) --
	(156.04, 71.07) --
	(156.04, 71.17) --
	(156.22, 71.27) --
	(156.31, 71.37) --
	(156.49, 71.47) --
	(156.67, 71.76) --
	(156.75, 71.96) --
	(156.75, 71.86) --
	(156.84, 72.06) --
	(156.93, 72.16) --
	(157.02, 72.55) --
	(157.10, 72.75) --
	(157.10, 72.65) --
	(157.19, 72.94) --
	(157.19, 73.04) --
	(157.28, 73.14) --
	(157.53, 73.44) --
	(157.53, 73.24) --
	(157.62, 73.53) --
	(157.62, 73.63) --
	(157.79, 73.73) --
	(157.87, 74.03) --
	(157.87, 73.93) --
	(157.96, 74.12) --
	(157.96, 74.32) --
	(158.04, 74.42) --
	(158.21, 74.52) --
	(158.29, 74.62) --
	(158.29, 74.71) --
	(158.46, 74.91) --
	(158.54, 75.01) --
	(158.54, 75.11) --
	(158.62, 75.30) --
	(158.79, 75.60) --
	(158.87, 75.70) --
	(158.95, 75.80) --
	(159.11, 75.99) --
	(159.27, 76.19) --
	(159.36, 76.29) --
	(159.44, 76.49) --
	(159.67, 76.58) --
	(159.75, 76.78) --
	(159.83, 77.17) --
	(159.91, 77.37) --
	(159.91, 77.57) --
	(159.99, 77.67) --
	(160.07, 78.06) --
	(160.15, 78.36) --
	(160.22, 78.45) --
	(160.30, 78.55) --
	(160.38, 78.85) --
	(160.46, 79.14) --
	(160.53, 79.44) --
	(160.61, 79.73) --
	(160.61, 79.54) --
	(160.76, 80.13) --
	(160.84, 80.42) --
	(160.84, 80.32) --
	(160.91, 80.52) --
	(160.99, 80.62) --
	(161.22, 80.72) --
	(161.29, 80.82) --
	(161.36, 81.01) --
	(161.44, 81.11) --
	(161.59, 81.21) --
	(161.74, 81.31) --
	(161.81, 81.60) --
	(161.88, 81.80) --
	(161.88, 81.70) --
	(161.96, 81.90) --
	(162.10, 82.10) --
	(162.17, 82.29) --
	(162.17, 82.19) --
	(162.25, 82.49) --
	(162.39, 82.59) --
	(162.46, 82.69) --
	(162.46, 82.78) --
	(162.53, 82.88) --
	(162.60, 83.08) --
	(162.68, 83.28) --
	(162.89, 83.37) --
	(162.96, 83.47) --
	(163.03, 83.77) --
	(163.03, 83.57) --
	(163.10, 84.26) --
	(163.10, 83.87) --
	(163.17, 84.56) --
	(163.38, 84.65) --
	(163.38, 84.75) --
	(163.45, 84.85) --
	(163.52, 85.05) --
	(163.59, 85.24) --
	(163.59, 85.15) --
	(163.79, 85.34) --
	(163.86, 85.44) --
	(164.00, 85.74) --
	(164.06, 86.23) --
	(164.13, 86.33) --
	(164.13, 86.43) --
	(164.20, 86.52) --
	(164.27, 86.62) --
	(164.40, 86.72) --
	(164.47, 86.92) --
	(164.47, 87.02) --
	(164.53, 87.11) --
	(164.73, 87.21) --
	(164.93, 87.31) --
	(165.00, 87.41) --
	(165.06, 87.70) --
	(165.13, 87.90) --
	(165.19, 88.29) --
	(165.26, 88.39) --
	(165.39, 88.49) --
	(165.45, 88.89) --
	(165.52, 88.98) --
	(165.58, 89.08) --
	(165.71, 89.38) --
	(165.84, 89.48) --
	(165.96, 89.57) --
	(166.03, 89.67) --
	(166.09, 89.77) --
	(166.09, 89.87) --
	(166.15, 90.07) --
	(166.15, 89.97) --
	(166.21, 90.26) --
	(166.28, 90.36) --
	(166.34, 90.46) --
	(166.46, 90.76) --
	(166.59, 91.05) --
	(166.71, 91.15) --
	(166.77, 91.35) --
	(166.90, 91.84) --
	(166.96, 91.94) --
	(167.02, 92.03) --
	(167.14, 92.13) --
	(167.20, 92.23) --
	(167.26, 92.43) --
	(167.32, 92.63) --
	(167.38, 92.82) --
	(167.38, 92.72) --
	(167.50, 93.02) --
	(167.56, 93.22) --
	(167.62, 93.31) --
	(167.74, 93.51) --
	(167.80, 93.71) --
	(167.86, 93.90) --
	(167.86, 93.81) --
	(167.92, 94.10) --
	(167.98, 94.40) --
	(168.03, 94.49) --
	(168.09, 94.59) --
	(168.15, 94.79) --
	(168.15, 94.89) --
	(168.21, 95.18) --
	(168.27, 95.38) --
	(168.33, 95.58) --
	(168.38, 95.68) --
	(168.44, 95.77) --
	(168.50, 95.87) --
	(168.62, 95.97) --
	(168.62, 96.07) --
	(168.67, 96.17) --
	(168.79, 96.27) --
	(168.85, 96.36) --
	(168.96, 96.46) --
	(169.02, 96.66) --
	(169.07, 96.76) --
	(169.13, 96.86) --
	(169.19, 97.25) --
	(169.24, 97.35) --
	(169.30, 97.55) --
	(169.30, 97.45) --
	(169.35, 97.64) --
	(169.47, 97.74) --
	(169.52, 97.94) --
	(169.63, 98.04) --
	(169.69, 98.23) --
	(169.74, 98.33) --
	(169.80, 98.43) --
	(169.91, 98.53) --
	(169.96, 98.73) --
	(170.02, 98.92) --
	(170.07, 99.22) --
	(170.24, 99.32) --
	(170.35, 99.51) --
	(170.40, 99.61) --
	(170.51, 99.71) --
	(170.56, 99.81) --
	(170.67,100.20) --
	(170.67,100.01) --
	(170.83,100.30) --
	(170.88,100.40) --
	(170.93,100.60) --
	(171.04,100.69) --
	(171.09,100.79) --
	(171.15,100.89) --
	(171.20,100.99) --
	(171.25,101.09) --
	(171.30,101.38) --
	(171.41,101.58) --
	(171.46,101.68) --
	(171.56,101.78) --
	(171.61,101.88) --
	(171.77,101.97) --
	(171.87,102.07) --
	(171.92,102.27) --
	(172.02,102.37) --
	(172.08,102.47) --
	(172.13,102.56) --
	(172.18,102.86) --
	(172.38,102.96) --
	(172.48,103.06) --
	(172.53,103.15) --
	(172.63,103.25) --
	(172.73,103.35) --
	(172.78,103.45) --
	(172.83,103.55) --
	(172.98,103.65) --
	(173.12,103.75) --
	(173.22,103.84) --
	(173.27,104.04) --
	(173.32,104.14) --
	(173.37,104.34) --
	(173.61,104.43) --
	(173.75,104.63) --
	(173.80,104.73) --
	(173.90,104.83) --
	(174.04,104.93) --
	(174.09,105.02) --
	(174.23,105.12) --
	(174.28,105.22) --
	(174.37,105.32) --
	(174.51,105.52) --
	(174.65,105.62) --
	(174.69,105.71) --
	(174.79,105.81) --
	(174.83,105.91) --
	(175.20,106.01) --
	(175.25,106.11) --
	(175.29,106.21) --
	(175.34,106.30) --
	(175.38,106.40) --
	(175.43,106.50) --
	(175.52,106.60) --
	(175.61,106.70) --
	(175.70,106.80) --
	(175.83,106.89) --
	(175.92,106.99) --
	(175.92,107.09) --
	(175.96,107.19) --
	(176.18,107.39) --
	(176.23,107.48) --
	(176.27,107.68) --
	(176.27,107.58) --
	(176.36,107.78) --
	(176.40,107.88) --
	(176.44,108.08) --
	(176.44,107.98) --
	(176.49,108.17) --
	(176.57,108.27) --
	(176.75,108.37) --
	(176.83,108.47) --
	(176.88,108.57) --
	(176.92,108.67) --
	(177.05,108.76) --
	(177.22,109.06) --
	(177.34,109.16) --
	(177.43,109.35) --
	(177.60,109.45) --
	(177.76,109.55) --
	(177.89,109.65) --
	(177.89,109.75) --
	(178.01,109.85) --
	(178.05,109.95) --
	(178.26,110.04) --
	(178.30,110.14) --
	(178.34,110.24) --
	(178.50,110.34) --
	(178.54,110.44) --
	(178.58,110.54) --
	(178.66,110.63) --
	(178.78,110.73) --
	(178.86,110.83) --
	(179.06,110.93) --
	(179.14,111.03) --
	(179.22,111.13) --
	(179.30,111.22) --
	(179.33,111.32) --
	(179.45,111.42) --
	(179.65,111.62) --
	(179.65,111.72) --
	(179.72,111.81) --
	(179.84,111.91) --
	(180.37,112.01) --
	(180.41,112.11) --
	(180.64,112.21) --
	(180.71,112.31) --
	(180.75,112.41) --
	(180.82,112.50) --
	(181.08,112.70) --
	(181.19,112.80) --
	(181.23,112.90) --
	(181.27,113.00) --
	(181.45,113.09) --
	(181.59,113.19) --
	(181.85,113.39) --
	(181.88,113.49) --
	(181.99,113.59) --
	(182.06,113.68) --
	(182.10,113.78) --
	(182.10,113.88) --
	(182.13,113.98) --
	(182.17,114.08) --
	(182.62,114.18) --
	(182.66,114.28) --
	(182.94,114.37) --
	(183.14,114.47) --
	(183.21,114.67) --
	(183.35,114.77) --
	(183.45,114.87) --
	(183.52,114.96) --
	(183.55,115.06) --
	(183.82,115.16) --
	(183.85,115.26) --
	(184.02,115.36) --
	(184.05,115.46) --
	(184.08,115.55) --
	(184.45,115.65) --
	(184.74,115.75) --
	(185.00,115.95) --
	(185.25,116.05) --
	(185.28,116.24) --
	(185.35,116.34) --
	(185.38,116.54) --
	(185.38,116.44) --
	(185.54,116.64) --
	(185.66,116.74) --
	(185.82,116.93) --
	(186.10,117.03) --
	(186.25,117.13) --
	(186.31,117.23) --
	(186.43,117.33) --
	(186.64,117.42) --
	(186.71,117.52) --
	(187.24,117.62) --
	(187.45,117.72) --
	(187.54,117.92) --
	(187.54,117.82) --
	(187.62,118.01) --
	(187.71,118.11) --
	(187.89,118.31) --
	(187.91,118.41) --
	(188.06,118.51) --
	(188.09,118.61) --
	(188.17,118.80) --
	(188.20,119.00) --
	(188.40,119.10) --
	(188.49,119.20) --
	(188.74,119.29) --
	(189.21,119.39) --
	(189.27,119.49) --
	(189.29,119.59) --
	(189.89,119.69) --
	(190.08,119.79) --
	(190.21,119.88) --
	(190.50,119.98) --
	(190.61,120.08) --
	(190.90,120.18) --
	(191.05,120.28) --
	(191.31,120.38) --
	(191.36,120.48) --
	(191.71,120.57) --
	(191.76,120.67) --
	(191.84,120.77) --
	(192.73,120.87) --
	(192.90,120.97) --
	(192.97,121.07) --
	(193.19,121.16) --
	(193.23,121.26) --
	(193.78,121.36) --
	(193.99,121.46) --
	(194.01,121.56) --
	(194.03,121.66) --
	(194.22,121.75) --
	(194.29,121.95) --
	(194.36,122.05) --
	(194.49,122.15) --
	(194.52,122.25) --
	(194.58,122.34) --
	(194.74,122.44) --
	(194.79,122.54) --
	(194.95,122.64) --
	(195.28,122.74) --
	(195.35,122.84) --
	(195.59,123.03) --
	(195.79,123.13) --
	(195.85,123.23) --
	(195.92,123.33) --
	(196.22,123.43) --
	(196.47,123.53) --
	(196.52,123.62) --
	(196.62,123.72) --
	(197.08,123.92) --
	(197.17,124.02) --
	(197.48,124.12) --
	(198.18,124.21) --
	(198.22,124.31) --
	(198.30,124.41) --
	(198.32,124.51) --
	(198.62,124.61) --
	(198.80,124.71) --
	(199.01,124.81) --
	(199.17,124.90) --
	(199.21,125.00) --
	(199.53,125.10) --
	(199.83,125.20) --
	(199.89,125.30) --
	(200.02,125.40) --
	(200.13,125.49) --
	(200.19,125.59) --
	(200.26,125.69) --
	(200.49,125.79) --
	(200.69,125.89) --
	(200.80,125.99) --
	(201.11,126.08) --
	(201.41,126.18) --
	(201.48,126.38) --
	(201.48,126.28) --
	(202.56,126.48) --
	(202.61,126.58) --
	(202.88,126.67) --
	(203.27,126.77) --
	(203.34,126.87) --
	(203.40,127.07) --
	(203.62,127.17) --
	(204.60,127.27) --
	(204.76,127.36) --
	(205.14,127.46) --
	(205.34,127.56) --
	(205.68,127.66) --
	(205.71,127.76) --
	(205.76,127.86) --
	(205.79,127.95) --
	(205.83,128.05) --
	(205.91,128.15) --
	(205.95,128.25) --
	(205.98,128.35) --
	(206.00,128.45) --
	(206.04,128.54) --
	(206.43,128.64) --
	(207.23,128.74) --
	(207.39,128.84) --
	(207.41,128.94) --
	(207.71,129.04) --
	(207.83,129.14) --
	(207.84,129.23) --
	(208.00,129.33) --
	(208.11,129.43) --
	(208.31,129.53) --
	(208.35,129.73) --
	(208.35,129.63) --
	(209.16,129.82) --
	(209.20,129.92) --
	(209.73,130.02) --
	(210.33,130.12) --
	(210.35,130.32) --
	(210.47,130.41) --
	(210.72,130.51) --
	(211.21,130.61) --
	(211.31,130.71) --
	(211.62,130.81) --
	(211.86,130.91) --
	(211.87,131.00) --
	(212.43,131.10) --
	(212.56,131.20) --
	(212.83,131.30) --
	(213.03,131.40) --
	(213.04,131.50) --
	(213.23,131.60) --
	(213.75,131.69) --
	(214.07,131.79) --
	(214.13,131.89) --
	(214.25,131.99) --
	(214.35,132.09) --
	(214.40,132.19) --
	(214.58,132.28) --
	(215.07,132.38) --
	(215.26,132.48) --
	(215.36,132.58) --
	(215.42,132.68) --
	(215.43,132.78) --
	(215.45,132.87) --
	(215.68,132.97) --
	(215.81,133.07) --
	(216.00,133.17) --
	(216.08,133.27) --
	(216.52,133.37) --
	(217.19,133.47) --
	(217.40,133.56) --
	(217.46,133.66) --
	(217.87,133.86) --
	(217.87,133.76) --
	(218.08,133.96) --
	(218.10,134.06) --
	(218.18,134.15) --
	(218.43,134.25) --
	(218.76,134.35) --
	(218.78,134.45) --
	(218.85,134.55) --
	(218.93,134.65) --
	(219.05,134.84) --
	(219.16,134.94) --
	(219.20,135.04) --
	(219.34,135.14) --
	(219.85,135.24) --
	(219.87,135.33) --
	(220.00,135.43) --
	(220.05,135.53) --
	(220.29,135.63) --
	(220.44,135.73) --
	(220.53,135.83) --
	(220.71,135.93) --
	(220.82,136.02) --
	(220.85,136.12) --
	(220.87,136.22) --
	(221.10,136.42) --
	(221.17,136.52) --
	(221.31,136.61) --
	(221.52,136.71) --
	(221.62,136.81) --
	(221.82,136.91) --
	(222.06,137.01) --
	(222.35,137.11) --
	(222.44,137.20) --
	(222.88,137.30) --
	(223.22,137.40) --
	(223.24,137.50) --
	(223.65,137.60) --
	(223.70,137.70) --
	(224.66,137.80) --
	(225.00,137.89) --
	(225.13,137.99) --
	(225.40,138.09) --
	(225.72,138.19) --
	(225.95,138.29) --
	(226.74,138.39) --
	(228.07,138.48) --
	(228.23,138.58) --
	(228.40,138.68) --
	(229.50,138.78) --
	(229.56,138.88) --
	(229.74,138.98) --
	(229.79,139.07) --
	(230.60,139.17) --
	(230.64,139.37) --
	(230.78,139.47) --
	(230.98,139.57) --
	(231.27,139.66) --
	(231.47,139.76) --
	(231.71,139.86) --
	(232.29,139.96) --
	(232.45,140.06) --
	(232.75,140.16) --
	(232.77,140.26) --
	(233.69,140.35) --
	(234.13,140.45) --
	(234.19,140.55) --
	(234.24,140.65) --
	(234.24,140.75) --
	(235.05,140.85) --
	(235.34,140.94) --
	(235.54,141.04) --
	(235.58,141.14) --
	(236.49,141.24) --
	(236.58,141.34) --
	(236.59,141.44) --
	(236.95,141.53) --
	(237.45,141.63) --
	(238.03,141.73) --
	(238.07,141.83) --
	(238.16,141.93) --
	(238.45,142.03) --
	(238.93,142.13) --
	(241.13,142.22) --
	(241.32,142.32) --
	(241.34,142.42) --
	(242.13,142.52) --
	(242.16,142.62) --
	(242.17,142.72) --
	(242.42,142.81) --
	(242.45,142.91) --
	(243.05,143.01) --
	(243.10,143.11) --
	(243.17,143.21) --
	(243.42,143.31) --
	(243.46,143.40) --
	(243.51,143.50) --
	(244.09,143.60) --
	(244.25,143.70) --
	(244.28,143.80) --
	(244.44,143.90) --
	(244.51,144.00) --
	(244.78,144.09) --
	(245.02,144.19) --
	(245.09,144.29) --
	(245.11,144.39) --
	(245.68,144.49) --
	(245.70,144.59) --
	(245.72,144.68) --
	(245.79,144.78) --
	(245.89,144.88) --
	(245.93,144.98) --
	(246.29,145.08) --
	(246.83,145.18) --
	(246.97,145.27) --
	(247.06,145.37) --
	(247.57,145.47) --
	(247.87,145.57) --
	(247.97,145.67) --
	(248.11,145.77) --
	(248.13,145.86) --
	(248.19,145.96) --
	(248.21,146.06) --
	(248.24,146.16) --
	(248.28,146.26) --
	(248.62,146.36) --
	(248.63,146.46) --
	(248.67,146.55) --
	(248.85,146.65) --
	(248.85,146.75) --
	(249.02,146.85) --
	(249.21,146.95) --
	(249.23,147.05) --
	(249.32,147.14) --
	(249.51,147.24) --
	(249.58,147.34) --
	(249.68,147.44) --
	(249.72,147.54) --
	(249.73,147.64) --
	(249.91,147.73) --
	(249.92,147.83) --
	(249.92,147.93) --
	(249.98,148.03) --
	(250.13,148.13) --
	(250.28,148.23) --
	(250.32,148.33) --
	(250.40,148.42) --
	(250.53,148.52) --
	(251.37,148.62) --
	(251.44,148.72) --
	(252.03,148.82) --
	(252.50,148.92) --
	(252.65,149.01) --
	(252.93,149.11) --
	(252.94,149.21) --
	(252.95,149.31) --
	(253.26,149.41) --
	(253.54,149.51) --
	(253.63,149.60) --
	(254.22,149.70) --
	(254.54,149.80) --
	(254.71,149.90) --
	(254.81,150.00) --
	(255.06,150.10) --
	(255.24,150.19) --
	(255.29,150.29) --
	(255.31,150.39) --
	(255.50,150.49) --
	(255.52,150.59) --
	(255.98,150.69) --
	(256.54,150.79) --
	(256.84,150.88) --
	(257.08,150.98) --
	(257.85,151.08) --
	(258.28,151.18) --
	(258.57,151.28) --
	(258.79,151.38) --
	(259.26,151.47) --
	(259.51,151.57) --
	(260.28,151.67) --
	(260.32,151.77) --
	(260.38,151.87) --
	(260.58,151.97) --
	(260.63,152.06) --
	(260.65,152.16) --
	(261.20,152.26) --
	(262.28,152.36) --
	(262.37,152.46) --
	(262.75,152.56) --
	(262.86,152.66) --
	(262.88,152.75) --
	(263.02,152.85) --
	(263.03,152.95) --
	(263.58,153.05) --
	(263.68,153.15) --
	(263.73,153.25) --
	(264.02,153.34) --
	(264.13,153.44) --
	(264.15,153.54) --
	(264.25,153.64) --
	(264.32,153.74) --
	(264.36,153.84) --
	(264.69,153.93) --
	(265.00,154.03) --
	(265.13,154.13) --
	(265.15,154.23) --
	(265.63,154.33) --
	(265.97,154.43) --
	(266.09,154.52) --
	(266.18,154.62) --
	(266.95,154.72) --
	(267.32,154.82) --
	(267.38,154.92) --
	(267.59,155.02) --
	(267.74,155.12) --
	(268.29,155.21) --
	(268.49,155.31) --
	(268.87,155.41) --
	(269.87,155.51) --
	(269.87,155.61) --
	(269.87,155.71) --
	(270.12,155.80) --
	(270.39,155.90) --
	(272.06,156.00) --
	(272.26,156.10) --
	(273.29,156.20) --
	(274.63,156.30) --
	(275.43,156.39) --
	(276.95,156.49) --
	(277.01,156.59) --
	(277.20,156.69) --
	(278.23,156.79) --
	(278.69,156.89) --
	(279.30,156.99) --
	(279.50,157.08) --
	(279.78,157.18) --
	(282.32,157.28) --
	(282.47,157.38) --
	(283.24,157.48) --
	(283.43,157.58) --
	(284.00,157.67) --
	(284.32,157.77) --
	(284.60,157.87) --
	(286.15,157.97) --
	(286.62,158.07) --
	(286.92,158.17) --
	(287.77,158.26) --
	(287.85,158.36) --
	(288.92,158.46) --
	(290.09,158.56) --
	(290.84,158.66) --
	(291.74,158.76) --
	(292.84,158.85) --
	(292.89,158.95) --
	(293.49,159.05) --
	(294.52,159.15) --
	(294.67,159.25) --
	(294.78,159.35) --
	(295.21,159.45) --
	(295.44,159.54) --
	(295.78,159.64) --
	(296.26,159.74) --
	(296.26,159.84) --
	(296.40,159.94) --
	(298.86,160.04) --
	(299.03,160.13) --
	(300.24,160.23) --
	(300.59,160.33) --
	(300.78,160.43) --
	(303.93,160.53) --
	(306.31,160.63) --
	(306.66,160.72) --
	(307.41,160.82) --
	(308.51,160.92) --
	(311.46,161.02) --
	(312.11,161.12) --
	(313.04,161.22) --
	(315.54,161.32) --
	(315.89,161.41) --
	(315.92,161.51) --
	(317.05,161.61) --
	(318.68,161.71);
\definecolor{drawColor}{RGB}{178,223,138}

\path[draw=drawColor,line width= 0.6pt,dash pattern=on 4pt off 4pt ,line join=round] (135.74, 36.93) --
	(136.29, 37.02) --
	(136.48, 37.22) --
	(136.66, 37.32) --
	(137.01, 37.42) --
	(137.19, 37.52) --
	(137.37, 37.81) --
	(137.37, 37.61) --
	(137.54, 37.91) --
	(137.72, 38.01) --
	(137.89, 38.30) --
	(138.06, 38.40) --
	(138.23, 38.50) --
	(138.40, 38.60) --
	(138.57, 38.79) --
	(138.74, 38.89) --
	(138.91, 39.19) --
	(138.91, 38.99) --
	(139.07, 39.48) --
	(139.24, 39.58) --
	(139.40, 39.88) --
	(139.57, 40.17) --
	(139.57, 40.27) --
	(139.73, 40.47) --
	(139.89, 40.57) --
	(140.05, 40.66) --
	(140.37, 41.06) --
	(140.53, 41.35) --
	(140.68, 41.45) --
	(140.84, 41.65) --
	(141.15, 41.75) --
	(141.15, 42.14) --
	(141.30, 42.34) --
	(141.45, 42.44) --
	(141.60, 42.53) --
	(141.60, 42.73) --
	(141.76, 43.32) --
	(141.91, 43.72) --
	(142.05, 43.91) --
	(142.05, 43.81) --
	(142.20, 44.50) --
	(142.35, 44.90) --
	(142.50, 45.09) --
	(142.64, 45.39) --
	(142.79, 45.98) --
	(142.79, 45.59) --
	(142.93, 46.37) --
	(143.08, 46.57) --
	(143.08, 46.47) --
	(143.22, 46.67) --
	(143.36, 46.86) --
	(143.50, 47.36) --
	(143.64, 47.45) --
	(143.78, 47.75) --
	(143.92, 47.95) --
	(143.92, 48.05) --
	(144.20, 48.14) --
	(144.48, 48.44) --
	(144.61, 48.54) --
	(144.75, 48.73) --
	(144.88, 48.93) --
	(145.02, 49.13) --
	(145.15, 49.32) --
	(145.28, 49.52) --
	(145.55, 49.72) --
	(145.68, 49.82) --
	(146.07, 50.21) --
	(146.95, 50.41) --
	(147.08, 50.60) --
	(147.20, 50.80) --
	(147.45, 51.00) --
	(147.45, 50.90) --
	(147.57, 51.10) --
	(147.69, 51.19) --
	(147.81, 51.39) --
	(147.94, 51.49) --
	(148.06, 51.69) --
	(148.18, 51.78) --
	(148.29, 51.88) --
	(148.41, 52.08) --
	(148.53, 52.18) --
	(148.65, 52.28) --
	(148.88, 52.47) --
	(148.88, 52.38) --
	(149.12, 52.67) --
	(149.23, 52.77) --
	(149.23, 52.87) --
	(149.35, 52.97) --
	(149.46, 53.06) --
	(149.57, 53.26) --
	(149.69, 53.36) --
	(149.91, 53.46) --
	(150.03, 53.65) --
	(150.14, 53.75) --
	(150.25, 53.95) --
	(150.36, 54.25) --
	(150.69, 54.54) --
	(150.91, 54.64) --
	(151.01, 54.74) --
	(151.23, 54.84) --
	(151.34, 55.03) --
	(151.55, 55.23) --
	(151.55, 55.13) --
	(151.65, 55.33) --
	(151.76, 55.43) --
	(151.87, 55.52) --
	(151.97, 55.82) --
	(152.18, 55.92) --
	(152.28, 56.02) --
	(152.49, 56.11) --
	(152.59, 56.21) --
	(152.79, 56.41) --
	(152.89, 56.61) --
	(152.99, 56.71) --
	(153.09, 56.90) --
	(153.19, 57.00) --
	(153.29, 57.30) --
	(153.29, 57.10) --
	(153.39, 57.39) --
	(153.59, 57.49) --
	(153.69, 57.59) --
	(153.79, 57.69) --
	(153.98, 57.89) --
	(154.08, 57.98) --
	(154.18, 58.28) --
	(154.18, 58.18) --
	(154.56, 58.38) --
	(154.65, 58.58) --
	(154.75, 58.67) --
	(154.94, 58.77) --
	(155.22, 58.87) --
	(155.31, 59.17) --
	(155.40, 59.36) --
	(155.40, 59.26) --
	(155.49, 59.56) --
	(155.77, 59.66) --
	(155.95, 59.85) --
	(156.13, 59.95) --
	(156.22, 60.05) --
	(156.31, 60.25) --
	(156.40, 60.44) --
	(156.49, 60.54) --
	(156.58, 60.64) --
	(156.84, 60.74) --
	(156.84, 60.84) --
	(157.10, 61.04) --
	(157.19, 61.13) --
	(157.28, 61.33) --
	(157.28, 61.23) --
	(157.36, 61.43) --
	(157.53, 61.63) --
	(157.62, 61.72) --
	(157.62, 61.82) --
	(157.79, 61.92) --
	(157.87, 62.02) --
	(158.04, 62.22) --
	(158.13, 62.31) --
	(158.29, 62.41) --
	(158.46, 62.71) --
	(158.54, 62.81) --
	(158.54, 63.00) --
	(158.62, 63.20) --
	(158.62, 63.10) --
	(158.79, 63.69) --
	(158.87, 63.79) --
	(159.03, 63.89) --
	(159.11, 64.09) --
	(159.19, 64.28) --
	(159.27, 64.38) --
	(159.44, 64.58) --
	(159.52, 64.68) --
	(159.59, 64.78) --
	(159.67, 64.97) --
	(159.75, 65.07) --
	(159.83, 65.46) --
	(159.99, 65.56) --
	(160.07, 65.66) --
	(160.15, 65.96) --
	(160.30, 66.35) --
	(160.30, 66.05) --
	(160.46, 66.45) --
	(160.61, 66.64) --
	(160.69, 66.74) --
	(160.76, 66.94) --
	(160.84, 67.04) --
	(160.91, 67.14) --
	(160.99, 67.24) --
	(160.99, 67.33) --
	(161.06, 67.43) --
	(161.22, 67.53) --
	(161.29, 67.63) --
	(161.36, 67.73) --
	(161.59, 67.83) --
	(161.66, 67.92) --
	(161.74, 68.02) --
	(162.03, 68.12) --
	(162.10, 68.22) --
	(162.39, 68.61) --
	(162.39, 68.42) --
	(162.53, 68.71) --
	(162.68, 68.91) --
	(162.75, 69.01) --
	(162.82, 69.11) --
	(162.82, 69.40) --
	(163.10, 69.60) --
	(163.10, 69.50) --
	(163.17, 69.70) --
	(163.24, 69.99) --
	(163.45, 70.09) --
	(163.59, 70.19) --
	(163.65, 70.29) --
	(163.72, 70.38) --
	(163.79, 70.48) --
	(163.86, 70.68) --
	(163.93, 70.78) --
	(164.00, 70.88) --
	(164.06, 70.97) --
	(164.13, 71.17) --
	(164.20, 71.27) --
	(164.27, 71.37) --
	(164.40, 71.47) --
	(164.47, 71.57) --
	(164.53, 71.66) --
	(164.60, 71.76) --
	(164.73, 71.86) --
	(164.80, 71.96) --
	(164.93, 72.06) --
	(165.00, 72.16) --
	(165.06, 72.35) --
	(165.19, 72.65) --
	(165.32, 72.75) --
	(165.45, 72.84) --
	(165.58, 73.04) --
	(165.58, 72.94) --
	(165.64, 73.14) --
	(165.71, 73.34) --
	(165.77, 73.44) --
	(165.84, 73.53) --
	(165.90, 73.63) --
	(166.03, 73.73) --
	(166.15, 73.83) --
	(166.34, 73.93) --
	(166.34, 74.03) --
	(166.46, 74.12) --
	(166.53, 74.22) --
	(166.59, 74.42) --
	(166.65, 74.62) --
	(166.65, 74.52) --
	(166.71, 74.71) --
	(166.77, 74.91) --
	(166.83, 75.01) --
	(166.96, 75.21) --
	(167.08, 75.30) --
	(167.20, 75.60) --
	(167.26, 75.80) --
	(167.38, 75.99) --
	(167.44, 76.09) --
	(167.50, 76.19) --
	(167.62, 76.29) --
	(167.62, 76.39) --
	(167.68, 76.49) --
	(167.74, 76.58) --
	(167.80, 76.68) --
	(167.80, 76.78) --
	(167.92, 76.88) --
	(167.98, 76.98) --
	(168.09, 77.08) --
	(168.15, 77.17) --
	(168.15, 77.27) --
	(168.21, 77.37) --
	(168.27, 77.47) --
	(168.38, 77.57) --
	(168.44, 77.86) --
	(168.50, 77.96) --
	(168.56, 78.16) --
	(168.56, 78.26) --
	(168.67, 78.36) --
	(168.73, 78.55) --
	(168.85, 78.65) --
	(168.96, 78.75) --
	(169.02, 78.95) --
	(169.07, 79.04) --
	(169.13, 79.24) --
	(169.19, 79.44) --
	(169.30, 79.63) --
	(169.35, 79.83) --
	(169.41, 79.93) --
	(169.52, 80.03) --
	(169.58, 80.13) --
	(169.74, 80.23) --
	(170.35, 80.42) --
	(170.72, 80.62) --
	(170.99, 80.82) --
	(171.09, 80.91) --
	(171.20, 81.01) --
	(171.25, 81.21) --
	(171.46, 81.31) --
	(171.51, 81.41) --
	(171.56, 81.50) --
	(171.61, 81.60) --
	(171.77, 81.80) --
	(171.87, 82.00) --
	(171.92, 82.10) --
	(171.92, 82.19) --
	(172.02, 82.29) --
	(172.28, 82.39) --
	(172.78, 82.49) --
	(172.83, 82.69) --
	(172.88, 82.78) --
	(172.93, 82.88) --
	(172.98, 83.18) --
	(173.07, 83.28) --
	(173.12, 83.47) --
	(173.17, 83.57) --
	(173.37, 83.67) --
	(173.42, 83.77) --
	(173.66, 83.96) --
	(173.66, 83.87) --
	(173.75, 84.16) --
	(173.90, 84.26) --
	(173.94, 84.36) --
	(174.04, 84.46) --
	(174.13, 84.56) --
	(174.18, 84.65) --
	(174.28, 84.75) --
	(174.28, 84.95) --
	(174.32, 85.05) --
	(174.42, 85.15) --
	(174.51, 85.24) --
	(174.60, 85.34) --
	(174.69, 85.44) --
	(174.79, 85.54) --
	(174.88, 85.64) --
	(174.93, 85.74) --
	(175.15, 85.83) --
	(175.15, 85.93) --
	(175.25, 86.03) --
	(175.34, 86.23) --
	(175.43, 86.43) --
	(175.52, 86.52) --
	(175.83, 86.62) --
	(175.83, 86.72) --
	(176.14, 86.82) --
	(176.31, 86.92) --
	(176.36, 87.02) --
	(176.53, 87.11) --
	(176.57, 87.21) --
	(176.62, 87.31) --
	(176.79, 87.41) --
	(176.83, 87.51) --
	(177.13, 87.70) --
	(177.26, 87.90) --
	(177.26, 87.80) --
	(177.39, 88.00) --
	(177.47, 88.10) --
	(177.51, 88.20) --
	(177.64, 88.29) --
	(177.76, 88.39) --
	(177.93, 88.49) --
	(178.01, 88.59) --
	(178.05, 88.79) --
	(178.13, 88.89) --
	(178.17, 88.98) --
	(178.42, 89.08) --
	(178.46, 89.18) --
	(178.50, 89.38) --
	(178.54, 89.48) --
	(178.62, 89.57) --
	(178.66, 89.67) --
	(178.70, 89.77) --
	(178.94, 89.87) --
	(179.33, 89.97) --
	(179.45, 90.07) --
	(179.49, 90.16) --
	(179.53, 90.26) --
	(179.57, 90.36) --
	(179.69, 90.56) --
	(179.72, 90.66) --
	(179.96, 90.76) --
	(180.26, 90.85) --
	(180.30, 90.95) --
	(180.52, 91.05) --
	(180.60, 91.15) --
	(180.67, 91.35) --
	(180.67, 91.25) --
	(180.97, 91.44) --
	(181.12, 91.54) --
	(181.19, 91.64) --
	(181.52, 91.74) --
	(181.70, 91.84) --
	(181.92, 91.94) --
	(182.03, 92.03) --
	(182.06, 92.13) --
	(182.17, 92.23) --
	(182.34, 92.33) --
	(182.73, 92.43) --
	(182.97, 92.53) --
	(183.18, 92.63) --
	(183.38, 92.72) --
	(183.52, 92.82) --
	(183.62, 92.92) --
	(183.92, 93.12) --
	(184.12, 93.22) --
	(184.18, 93.31) --
	(184.54, 93.41) --
	(184.61, 93.51) --
	(184.74, 93.61) --
	(184.84, 93.81) --
	(185.12, 94.10) --
	(185.35, 94.20) --
	(185.44, 94.30) --
	(185.57, 94.40) --
	(185.72, 94.49) --
	(185.76, 94.59) --
	(185.82, 94.69) --
	(185.94, 94.89) --
	(185.97, 95.09) --
	(186.16, 95.18) --
	(186.25, 95.28) --
	(186.49, 95.38) --
	(186.80, 95.58) --
	(186.80, 95.48) --
	(186.92, 95.68) --
	(186.95, 95.77) --
	(186.98, 95.87) --
	(187.01, 95.97) --
	(187.09, 96.07) --
	(187.15, 96.17) --
	(187.18, 96.27) --
	(187.21, 96.36) --
	(187.51, 96.46) --
	(187.71, 96.56) --
	(187.74, 96.66) --
	(188.03, 96.76) --
	(188.23, 96.86) --
	(188.63, 96.96) --
	(188.68, 97.05) --
	(188.99, 97.25) --
	(189.05, 97.35) --
	(189.13, 97.45) --
	(189.16, 97.55) --
	(189.21, 97.64) --
	(189.32, 97.74) --
	(189.35, 97.84) --
	(189.38, 98.04) --
	(189.40, 98.14) --
	(189.51, 98.23) --
	(189.76, 98.33) --
	(189.92, 98.43) --
	(189.97, 98.53) --
	(190.08, 98.63) --
	(190.29, 98.73) --
	(190.32, 98.82) --
	(190.95, 98.92) --
	(190.95, 99.02) --
	(191.13, 99.12) --
	(191.26, 99.22) --
	(191.94, 99.32) --
	(192.14, 99.42) --
	(192.21, 99.51) --
	(192.24, 99.81) --
	(192.24, 99.61) --
	(192.34, 99.91) --
	(192.36,100.01) --
	(192.39,100.10) --
	(192.58,100.20) --
	(192.80,100.30) --
	(193.14,100.40) --
	(193.31,100.50) --
	(193.80,100.60) --
	(193.85,100.69) --
	(194.08,100.79) --
	(195.08,100.89) --
	(195.41,100.99) --
	(196.02,101.09) --
	(196.22,101.19) --
	(196.24,101.29) --
	(196.35,101.38) --
	(196.83,101.48) --
	(197.29,101.58) --
	(197.41,101.68) --
	(197.74,101.78) --
	(197.80,101.88) --
	(198.34,101.97) --
	(198.70,102.17) --
	(199.01,102.27) --
	(199.21,102.37) --
	(199.46,102.47) --
	(199.65,102.56) --
	(199.98,102.66) --
	(200.08,102.76) --
	(200.41,102.96) --
	(200.41,102.86) --
	(200.58,103.06) --
	(200.89,103.15) --
	(200.92,103.25) --
	(201.12,103.35) --
	(201.14,103.45) --
	(201.52,103.55) --
	(201.62,103.75) --
	(201.82,103.84) --
	(201.90,103.94) --
	(201.96,104.04) --
	(202.82,104.14) --
	(203.30,104.24) --
	(203.52,104.34) --
	(203.83,104.43) --
	(204.04,104.53) --
	(204.17,104.63) --
	(205.05,104.73) --
	(205.28,104.83) --
	(205.36,104.93) --
	(205.50,105.02) --
	(205.63,105.12) --
	(206.26,105.22) --
	(206.30,105.32) --
	(206.63,105.42) --
	(206.66,105.52) --
	(207.01,105.62) --
	(207.14,105.71) --
	(207.40,105.81) --
	(207.56,105.91) --
	(207.76,106.01) --
	(207.81,106.11) --
	(207.86,106.21) --
	(208.17,106.30) --
	(208.47,106.40) --
	(208.87,106.50) --
	(208.98,106.60) --
	(209.01,106.70) --
	(209.32,106.80) --
	(209.45,106.89) --
	(209.50,106.99) --
	(209.61,107.09) --
	(209.95,107.19) --
	(210.01,107.29) --
	(210.14,107.39) --
	(210.68,107.48) --
	(210.79,107.58) --
	(210.82,107.68) --
	(210.85,107.78) --
	(211.05,107.88) --
	(211.32,108.08) --
	(211.32,107.98) --
	(211.41,108.17) --
	(211.71,108.27) --
	(211.98,108.37) --
	(212.20,108.47) --
	(212.49,108.57) --
	(212.65,108.67) --
	(213.01,108.76) --
	(214.61,108.86) --
	(214.69,108.96) --
	(214.77,109.06) --
	(215.18,109.16) --
	(215.24,109.26) --
	(215.28,109.35) --
	(215.65,109.45) --
	(216.05,109.55) --
	(216.06,109.65) --
	(216.12,109.75) --
	(216.39,109.85) --
	(216.70,109.95) --
	(216.77,110.04) --
	(217.43,110.14) --
	(217.82,110.24) --
	(217.94,110.34) --
	(218.00,110.44) --
	(218.16,110.54) --
	(218.60,110.63) --
	(219.18,110.73) --
	(219.22,110.83) --
	(219.25,110.93) --
	(219.44,111.03) --
	(219.72,111.13) --
	(220.34,111.22) --
	(220.91,111.32) --
	(221.03,111.42) --
	(221.25,111.52) --
	(221.36,111.62) --
	(222.07,111.72) --
	(222.54,111.81) --
	(222.85,111.91) --
	(222.87,112.01) --
	(223.30,112.11) --
	(223.35,112.21) --
	(223.72,112.31) --
	(223.91,112.41) --
	(225.00,112.50) --
	(225.34,112.60) --
	(225.52,112.70) --
	(225.60,112.80) --
	(225.84,112.90) --
	(226.15,113.00) --
	(226.20,113.09) --
	(226.71,113.19) --
	(226.98,113.29) --
	(227.00,113.39) --
	(227.09,113.49) --
	(227.11,113.59) --
	(227.18,113.68) --
	(227.59,113.78) --
	(227.59,113.88) --
	(227.69,113.98) --
	(228.06,114.08) --
	(228.65,114.18) --
	(228.68,114.28) --
	(228.82,114.37) --
	(228.86,114.47) --
	(228.88,114.57) --
	(228.93,114.67) --
	(229.46,114.77) --
	(229.84,114.87) --
	(229.90,114.96) --
	(230.47,115.16) --
	(230.47,115.06) --
	(230.71,115.26) --
	(231.01,115.36) --
	(231.04,115.46) --
	(231.15,115.55) --
	(231.43,115.65) --
	(231.64,115.75) --
	(231.75,115.85) --
	(231.88,115.95) --
	(232.06,116.05) --
	(232.08,116.15) --
	(232.49,116.24) --
	(232.61,116.34) --
	(232.70,116.44) --
	(232.83,116.54) --
	(232.93,116.64) --
	(233.15,116.74) --
	(233.64,116.83) --
	(233.72,116.93) --
	(233.94,117.03) --
	(234.23,117.13) --
	(234.29,117.23) --
	(234.30,117.33) --
	(234.41,117.42) --
	(234.46,117.52) --
	(234.50,117.62) --
	(234.54,117.72) --
	(234.68,117.82) --
	(234.81,117.92) --
	(234.83,118.01) --
	(234.98,118.11) --
	(235.49,118.21) --
	(235.57,118.31) --
	(235.59,118.41) --
	(235.79,118.51) --
	(235.92,118.61) --
	(236.23,118.70) --
	(237.02,118.80) --
	(237.54,118.90) --
	(237.67,119.00) --
	(237.68,119.10) --
	(237.87,119.20) --
	(237.92,119.29) --
	(238.10,119.39) --
	(238.35,119.49) --
	(238.72,119.59) --
	(238.73,119.69) --
	(239.91,119.79) --
	(240.26,119.88) --
	(240.53,119.98) --
	(240.60,120.08) --
	(240.68,120.18) --
	(240.95,120.28) --
	(241.04,120.38) --
	(241.18,120.48) --
	(241.62,120.57) --
	(242.41,120.67) --
	(242.70,120.77) --
	(242.94,120.87) --
	(243.01,120.97) --
	(243.52,121.07) --
	(243.83,121.16) --
	(244.47,121.26) --
	(245.65,121.36) --
	(246.49,121.46) --
	(246.67,121.56) --
	(246.88,121.66) --
	(247.20,121.75) --
	(247.46,121.85) --
	(247.46,121.95) --
	(247.50,122.05) --
	(247.72,122.15) --
	(247.84,122.25) --
	(249.58,122.34) --
	(249.83,122.44) --
	(250.96,122.54) --
	(251.32,122.64) --
	(252.25,122.74) --
	(252.91,122.84) --
	(253.22,122.94) --
	(253.38,123.03) --
	(253.67,123.13) --
	(253.97,123.23) --
	(254.11,123.33) --
	(254.92,123.43) --
	(256.34,123.53) --
	(256.35,123.62) --
	(256.38,123.72) --
	(256.64,123.82) --
	(257.08,123.92) --
	(257.79,124.02) --
	(257.88,124.12) --
	(257.93,124.21) --
	(258.94,124.31) --
	(259.87,124.41) --
	(259.92,124.51) --
	(260.03,124.61) --
	(260.67,124.71) --
	(260.99,124.81) --
	(261.18,124.90) --
	(261.48,125.00) --
	(262.03,125.10) --
	(262.65,125.20) --
	(262.69,125.30) --
	(262.74,125.40) --
	(262.78,125.49) --
	(262.91,125.59) --
	(263.33,125.69) --
	(264.99,125.79) --
	(265.28,125.89) --
	(266.27,125.99) --
	(266.59,126.08) --
	(268.19,126.18) --
	(269.56,126.28) --
	(271.16,126.38) --
	(271.27,126.48) --
	(271.95,126.58) --
	(271.97,126.67) --
	(272.86,126.77) --
	(273.86,126.87) --
	(274.04,126.97) --
	(276.79,127.07) --
	(276.86,127.17) --
	(276.90,127.27) --
	(277.13,127.36) --
	(277.39,127.46) --
	(278.20,127.56) --
	(278.22,127.66) --
	(278.69,127.76) --
	(278.93,127.86) --
	(279.34,127.95) --
	(279.37,128.05) --
	(279.43,128.15) --
	(280.19,128.25) --
	(280.67,128.35) --
	(281.10,128.45) --
	(282.60,128.54) --
	(282.86,128.64) --
	(283.42,128.74) --
	(284.17,128.84) --
	(284.29,128.94) --
	(284.40,129.04) --
	(285.36,129.14) --
	(285.70,129.23) --
	(286.41,129.33) --
	(287.07,129.43) --
	(287.42,129.53) --
	(287.67,129.63) --
	(290.64,129.73) --
	(292.45,129.82) --
	(292.49,129.92) --
	(292.55,130.02) --
	(292.76,130.12) --
	(292.77,130.22) --
	(293.16,130.32) --
	(294.34,130.41) --
	(294.65,130.51) --
	(296.20,130.61) --
	(296.78,130.71) --
	(297.24,130.81) --
	(297.61,130.91) --
	(300.71,131.00) --
	(300.90,131.10) --
	(301.10,131.20) --
	(301.29,131.30) --
	(301.65,131.40) --
	(301.80,131.50) --
	(301.87,131.60) --
	(302.07,131.69) --
	(302.54,131.79) --
	(302.70,131.89) --
	(302.83,131.99) --
	(302.87,132.09) --
	(303.13,132.19) --
	(303.27,132.28) --
	(303.38,132.38) --
	(304.00,132.48) --
	(305.94,132.58) --
	(307.27,132.68) --
	(307.61,132.78) --
	(308.02,132.87) --
	(309.00,132.97) --
	(309.03,133.07) --
	(309.29,133.17) --
	(309.92,133.27) --
	(310.09,133.37) --
	(310.45,133.47) --
	(310.49,133.56) --
	(311.36,133.66) --
	(311.94,133.76) --
	(312.30,133.86) --
	(313.41,133.96) --
	(314.55,134.06) --
	(314.76,134.15) --
	(315.32,134.25) --
	(315.90,134.35) --
	(316.62,134.45) --
	(316.82,134.55) --
	(317.08,134.65) --
	(317.20,134.74) --
	(318.84,134.84);
\definecolor{drawColor}{RGB}{31,120,180}

\path[draw=drawColor,line width= 0.6pt,line join=round] ( 62.67, 37.02) --
	( 62.67, 36.93) --
	( 65.10, 37.22) --
	( 67.34, 44.21) --
	( 69.41, 46.47) --
	( 71.34, 46.86) --
	( 71.34, 49.52) --
	( 73.15, 53.85) --
	( 74.84, 59.76) --
	( 74.84, 59.85) --
	( 76.44, 62.41) --
	( 77.96, 66.45) --
	( 79.39, 66.64) --
	( 80.75, 66.84) --
	( 80.75, 67.14) --
	( 82.06, 67.33) --
	( 83.30, 67.63) --
	( 83.30, 67.73) --
	( 84.49, 67.83) --
	( 84.49, 67.92) --
	( 86.73, 68.02) --
	( 87.78, 68.71) --
	( 88.80, 70.38) --
	( 89.78, 71.07) --
	( 89.78, 70.58) --
	( 90.73, 71.57) --
	( 90.73, 71.76) --
	( 91.65, 72.25) --
	( 92.53, 72.35) --
	( 93.39, 72.65) --
	( 93.39, 72.45) --
	( 94.23, 73.04) --
	( 95.04, 73.34) --
	( 95.83, 73.63) --
	( 96.59, 73.83) --
	( 97.34, 74.03) --
	( 98.07, 74.32) --
	( 98.77, 74.81) --
	( 99.46, 75.30) --
	(100.14, 76.09) --
	(100.14, 75.70) --
	(100.80, 76.29) --
	(101.44, 76.39) --
	(102.07, 76.49) --
	(104.45, 76.68) --
	(105.01, 76.78) --
	(105.57, 77.08) --
	(110.11, 77.17) --
	(110.58, 77.37) --
	(111.92, 77.47) --
	(115.21, 77.57) --
	(115.98, 77.67) --
	(118.16, 77.77) --
	(118.85, 77.86) --
	(119.19, 77.96) --
	(119.85, 78.06) --
	(120.51, 78.16) --
	(125.23, 78.26) --
	(126.29, 78.36) --
	(128.06, 78.45) --
	(132.38, 78.65) --
	(133.00, 78.75) --
	(133.41, 78.85) --
	(134.98, 78.95) --
	(136.11, 79.04) --
	(136.29, 79.14) --
	(136.84, 79.24) --
	(137.37, 79.34) --
	(137.54, 79.44) --
	(138.74, 79.54) --
	(139.07, 79.63) --
	(139.89, 79.73) --
	(140.37, 79.83) --
	(141.45, 79.93) --
	(141.91, 80.03) --
	(143.22, 80.23) --
	(144.34, 80.32) --
	(145.41, 80.42) --
	(146.45, 80.52) --
	(146.58, 80.62) --
	(148.53, 80.72) --
	(148.77, 80.82) --
	(149.91, 80.91) --
	(150.25, 81.01) --
	(150.36, 81.11) --
	(151.34, 81.21) --
	(151.76, 81.31) --
	(152.18, 81.41) --
	(152.49, 81.50) --
	(153.69, 81.60) --
	(154.75, 81.70) --
	(159.59, 81.80) --
	(163.38, 81.90) --
	(163.59, 82.00) --
	(163.79, 82.10) --
	(165.71, 82.19) --
	(166.15, 82.29) --
	(168.38, 82.39) --
	(171.25, 82.49) --
	(171.51, 82.59) --
	(172.43, 82.69) --
	(172.48, 82.78) --
	(172.58, 82.88) --
	(173.03, 82.98) --
	(173.22, 83.08) --
	(174.97, 83.18) --
	(175.65, 83.28) --
	(176.14, 83.37) --
	(176.92, 83.47) --
	(177.05, 83.57) --
	(177.43, 83.67) --
	(177.84, 83.77) --
	(177.89, 83.87) --
	(179.61, 83.96) --
	(179.72, 84.06) --
	(179.76, 84.16) --
	(180.34, 84.26) --
	(181.67, 84.36) --
	(182.59, 84.46) --
	(182.69, 84.56) --
	(182.83, 84.65) --
	(183.07, 84.75) --
	(183.14, 84.85) --
	(183.45, 84.95) --
	(183.89, 85.05) --
	(184.22, 85.15) --
	(184.41, 85.24) --
	(184.80, 85.34) --
	(186.49, 85.44) --
	(186.52, 85.54) --
	(187.62, 85.64) --
	(189.95, 85.74) --
	(191.54, 85.83) --
	(192.11, 85.93) --
	(192.31, 86.03) --
	(192.43, 86.23) --
	(192.53, 86.33) --
	(192.70, 86.43) --
	(192.85, 86.52) --
	(193.02, 86.62) --
	(193.16, 86.72) --
	(193.33, 86.82) --
	(193.85, 86.92) --
	(194.90, 87.02) --
	(195.01, 87.11) --
	(195.19, 87.21) --
	(195.28, 87.31) --
	(196.58, 87.41) --
	(197.33, 87.51) --
	(197.39, 87.61) --
	(197.70, 87.70) --
	(198.10, 87.80) --
	(198.28, 87.90) --
	(199.13, 88.00) --
	(199.68, 88.10) --
	(199.72, 88.20) --
	(200.04, 88.29) --
	(200.78, 88.39) --
	(201.32, 88.49) --
	(202.53, 88.59) --
	(202.83, 88.69) --
	(203.04, 88.79) --
	(203.39, 88.89) --
	(203.67, 88.98) --
	(205.03, 89.08) --
	(205.11, 89.18) --
	(205.16, 89.28) --
	(205.28, 89.38) --
	(205.86, 89.48) --
	(206.72, 89.57) --
	(207.23, 89.67) --
	(210.01, 89.77) --
	(210.09, 89.87) --
	(210.49, 89.97) --
	(210.56, 90.07) --
	(210.73, 90.16) --
	(210.81, 90.26) --
	(211.30, 90.36) --
	(211.41, 90.46) --
	(211.80, 90.56) --
	(213.68, 90.66) --
	(217.02, 90.76) --
	(218.50, 90.85) --
	(218.86, 91.05) --
	(220.47, 91.15) --
	(221.46, 91.25) --
	(221.50, 91.35) --
	(227.40, 91.44) --
	(228.28, 91.54) --
	(232.91, 91.64) --
	(234.14, 91.74) --
	(234.66, 91.84) --
	(237.30, 91.94) --
	(240.13, 92.03) --
	(243.01, 92.13) --
	(243.83, 92.23) --
	(245.70, 92.33) --
	(248.33, 92.43) --
	(250.07, 92.53) --
	(250.09, 92.63) --
	(250.54, 92.72) --
	(251.79, 92.82) --
	(257.25, 92.92) --
	(257.87, 93.02) --
	(258.01, 93.12) --
	(259.92, 93.22) --
	(260.92, 93.31) --
	(263.19, 93.41) --
	(264.65, 93.51) --
	(266.48, 93.61) --
	(266.64, 93.71) --
	(267.54, 93.81) --
	(269.88, 93.90) --
	(270.41, 94.00) --
	(271.20, 94.10) --
	(271.46, 94.20) --
	(271.74, 94.30) --
	(272.62, 94.40) --
	(273.51, 94.49) --
	(276.17, 94.59) --
	(277.54, 94.69) --
	(278.29, 94.79) --
	(281.51, 94.89) --
	(282.01, 94.99) --
	(282.61, 95.09) --
	(283.02, 95.18) --
	(283.20, 95.28) --
	(284.86, 95.38) --
	(285.10, 95.48) --
	(286.64, 95.58) --
	(289.64, 95.68) --
	(289.94, 95.77) --
	(290.99, 95.87) --
	(291.27, 95.97) --
	(292.03, 96.07) --
	(295.43, 96.17) --
	(297.02, 96.27) --
	(297.66, 96.36) --
	(299.16, 96.46) --
	(299.96, 96.56) --
	(300.18, 96.66) --
	(300.81, 96.76) --
	(302.03, 96.86) --
	(309.81, 96.96) --
	(313.36, 97.05) --
	(313.49, 97.15) --
	(316.13, 97.25) --
	(316.16, 97.35) --
	(316.80, 97.45) --
	(317.53, 97.55);
\definecolor{drawColor}{RGB}{177,89,40}

\path[draw=drawColor,line width= 0.6pt,line join=round] ( 53.76, 36.93) --
	( 57.06, 37.32) --
	( 57.06, 37.42) --
	( 60.00, 38.01) --
	( 60.00, 37.71) --
	( 62.67, 45.49) --
	( 65.10, 46.08) --
	( 65.10, 52.28) --
	( 67.34, 62.22) --
	( 69.41, 66.05) --
	( 69.41, 63.30) --
	( 71.34, 70.48) --
	( 71.34, 70.19) --
	( 73.15, 71.37) --
	( 74.84, 72.06) --
	( 74.84, 71.86) --
	( 76.44, 72.25) --
	( 76.44, 72.35) --
	( 77.96, 74.52) --
	( 79.39, 75.80) --
	( 80.75, 75.90) --
	( 80.75, 76.29) --
	( 82.06, 77.17) --
	( 82.06, 77.67) --
	( 83.30, 78.16) --
	( 83.30, 77.77) --
	( 84.49, 78.55) --
	( 85.63, 78.95) --
	( 86.73, 79.54) --
	( 87.78, 79.83) --
	( 88.80, 80.32) --
	( 89.78, 81.21) --
	( 89.78, 80.52) --
	( 90.73, 81.50) --
	( 90.73, 81.70) --
	( 91.65, 81.80) --
	( 92.53, 82.98) --
	( 93.39, 83.37) --
	( 93.39, 84.36) --
	( 94.23, 84.95) --
	( 94.23, 84.56) --
	( 95.04, 85.05) --
	( 96.59, 85.34) --
	( 97.34, 85.64) --
	( 98.07, 85.83) --
	( 98.77, 86.03) --
	( 99.46, 86.43) --
	(100.14, 86.72) --
	(100.14, 86.82) --
	(100.80, 87.02) --
	(101.44, 87.11) --
	(102.07, 87.31) --
	(102.68, 87.51) --
	(102.68, 87.41) --
	(103.28, 87.61) --
	(104.45, 87.90) --
	(105.01, 88.00) --
	(105.57, 88.29) --
	(106.11, 88.49) --
	(107.17, 88.59) --
	(107.68, 88.69) --
	(108.18, 88.89) --
	(108.68, 89.08) --
	(109.17, 89.28) --
	(109.17, 89.48) --
	(109.64, 89.57) --
	(109.64, 89.67) --
	(110.11, 89.87) --
	(111.03, 90.07) --
	(111.48, 90.16) --
	(111.92, 90.26) --
	(112.35, 90.36) --
	(113.20, 90.56) --
	(114.02, 90.76) --
	(114.42, 90.95) --
	(114.82, 91.05) --
	(115.21, 91.25) --
	(115.60, 91.35) --
	(116.35, 91.44) --
	(116.72, 91.74) --
	(117.09, 91.84) --
	(117.45, 91.94) --
	(118.85, 92.23) --
	(119.19, 92.33) --
	(120.51, 92.43) --
	(120.51, 92.53) --
	(121.14, 92.72) --
	(122.97, 92.92) --
	(123.26, 93.02) --
	(124.12, 93.12) --
	(124.40, 93.31) --
	(124.68, 93.51) --
	(124.95, 93.61) --
	(125.23, 93.71) --
	(126.03, 93.81) --
	(126.55, 93.90) --
	(126.81, 94.00) --
	(127.07, 94.10) --
	(127.32, 94.40) --
	(127.32, 94.20) --
	(127.57, 94.49) --
	(128.79, 94.89) --
	(129.26, 94.99) --
	(129.50, 95.09) --
	(129.73, 95.18) --
	(129.96, 95.38) --
	(130.42, 95.48) --
	(131.08, 95.58) --
	(131.52, 95.68) --
	(132.38, 95.77) --
	(133.61, 95.87) --
	(133.61, 95.97) --
	(133.81, 96.17) --
	(134.40, 96.27) --
	(134.60, 96.46) --
	(134.98, 96.56) --
	(135.74, 96.66) --
	(135.93, 96.76) --
	(136.29, 96.86) --
	(136.66, 96.96) --
	(137.01, 97.25) --
	(137.01, 97.05) --
	(137.37, 97.45) --
	(137.37, 97.35) --
	(137.54, 97.55) --
	(137.72, 97.64) --
	(138.23, 97.74) --
	(138.40, 98.04) --
	(138.91, 98.14) --
	(139.07, 98.33) --
	(140.68, 98.43) --
	(141.15, 98.53) --
	(141.30, 98.63) --
	(141.60, 98.82) --
	(141.60, 98.73) --
	(141.76, 99.02) --
	(141.76, 99.12) --
	(142.05, 99.22) --
	(142.20, 99.32) --
	(142.35, 99.51) --
	(142.50, 99.61) --
	(142.93, 99.71) --
	(143.08, 99.91) --
	(143.50,100.01) --
	(144.34,100.10) --
	(144.88,100.20) --
	(145.15,100.30) --
	(145.94,100.50) --
	(146.19,100.60) --
	(146.32,100.69) --
	(146.45,100.89) --
	(146.58,101.09) --
	(147.69,101.29) --
	(148.29,101.38) --
	(148.65,101.48) --
	(148.77,101.58) --
	(148.88,101.68) --
	(149.00,101.88) --
	(149.23,101.97) --
	(149.35,102.07) --
	(150.03,102.17) --
	(150.25,102.27) --
	(150.36,102.37) --
	(150.69,102.47) --
	(151.34,102.56) --
	(152.38,102.66) --
	(152.49,102.76) --
	(152.69,102.86) --
	(152.89,102.96) --
	(152.99,103.15) --
	(153.69,103.25) --
	(154.18,103.35) --
	(155.31,103.45) --
	(156.93,103.55) --
	(157.28,103.65) --
	(157.79,103.75) --
	(159.75,103.84) --
	(159.83,103.94) --
	(160.61,104.04) --
	(160.84,104.14) --
	(161.14,104.24) --
	(161.36,104.34) --
	(161.44,104.43) --
	(161.51,104.53) --
	(161.74,104.63) --
	(163.31,104.73) --
	(163.59,104.83) --
	(164.60,104.93) --
	(166.65,105.02) --
	(167.44,105.12) --
	(167.68,105.22) --
	(167.86,105.32) --
	(168.50,105.42) --
	(168.67,105.52) --
	(168.90,105.62) --
	(169.47,105.71) --
	(170.45,105.81) --
	(171.09,105.91) --
	(171.25,106.01) --
	(171.92,106.11) --
	(172.08,106.30) --
	(172.43,106.40) --
	(172.58,106.50) --
	(172.83,106.60) --
	(172.88,106.70) --
	(173.80,106.80) --
	(175.70,106.89) --
	(176.57,106.99) --
	(176.62,107.09) --
	(176.66,107.19) --
	(176.79,107.29) --
	(176.96,107.39) --
	(177.09,107.48) --
	(177.60,107.58) --
	(177.84,107.68) --
	(178.94,107.78) --
	(179.57,107.88) --
	(179.76,107.98) --
	(179.88,108.08) --
	(180.56,108.17) --
	(180.79,108.27) --
	(180.86,108.37) --
	(181.12,108.47) --
	(181.59,108.57) --
	(182.06,108.67) --
	(182.62,108.76) --
	(184.02,108.86) --
	(184.12,108.96) --
	(184.64,109.06) --
	(186.80,109.16) --
	(187.60,109.26) --
	(187.65,109.35) --
	(188.32,109.45) --
	(189.24,109.55) --
	(189.46,109.65) --
	(190.50,109.75) --
	(191.08,109.85) --
	(191.59,109.95) --
	(191.66,110.04) --
	(192.14,110.14) --
	(192.70,110.24) --
	(193.19,110.34) --
	(193.21,110.44) --
	(193.38,110.54) --
	(193.92,110.63) --
	(194.47,110.73) --
	(195.08,110.83) --
	(195.30,110.93) --
	(195.52,111.03) --
	(196.47,111.13) --
	(196.50,111.22) --
	(198.64,111.32) --
	(199.19,111.42) --
	(200.00,111.52) --
	(200.23,111.62) --
	(201.23,111.72) --
	(201.29,111.81) --
	(201.32,111.91) --
	(201.36,112.01) --
	(201.48,112.11) --
	(203.45,112.21) --
	(204.11,112.31) --
	(204.73,112.41) --
	(206.89,112.50) --
	(207.31,112.60) --
	(209.85,112.70) --
	(209.86,112.80) --
	(210.32,112.90) --
	(211.68,113.00) --
	(211.87,113.09) --
	(212.08,113.19) --
	(212.87,113.29) --
	(213.56,113.39) --
	(213.58,113.49) --
	(213.64,113.59) --
	(214.94,113.68) --
	(216.01,113.78) --
	(216.16,113.88) --
	(216.50,113.98) --
	(217.28,114.08) --
	(217.39,114.18) --
	(217.57,114.28) --
	(218.11,114.37) --
	(218.58,114.47) --
	(219.25,114.57) --
	(220.51,114.67) --
	(220.99,114.77) --
	(221.09,114.87) --
	(221.78,114.96) --
	(221.80,115.06) --
	(221.85,115.16) --
	(222.31,115.26) --
	(222.36,115.36) --
	(223.14,115.46) --
	(223.43,115.55) --
	(223.73,115.65) --
	(223.99,115.75) --
	(224.18,115.85) --
	(224.49,115.95) --
	(224.83,116.05) --
	(225.10,116.15) --
	(225.47,116.24) --
	(225.93,116.34) --
	(226.23,116.44) --
	(227.47,116.54) --
	(228.86,116.64) --
	(228.87,116.74) --
	(230.33,116.83) --
	(230.43,116.93) --
	(230.80,117.03) --
	(231.39,117.13) --
	(233.05,117.23) --
	(235.24,117.33) --
	(235.82,117.42) --
	(236.94,117.52) --
	(237.01,117.62) --
	(237.30,117.72) --
	(239.75,117.82) --
	(239.87,117.92) --
	(240.13,118.01) --
	(240.64,118.11) --
	(240.90,118.21) --
	(241.40,118.31) --
	(241.65,118.41) --
	(241.83,118.51) --
	(242.23,118.61) --
	(243.10,118.70) --
	(244.58,118.80) --
	(244.69,118.90) --
	(245.35,119.00) --
	(245.74,119.10) --
	(246.45,119.20) --
	(246.76,119.29) --
	(247.12,119.39) --
	(248.24,119.49) --
	(249.04,119.59) --
	(249.17,119.69) --
	(249.57,119.79) --
	(249.88,119.88) --
	(251.20,119.98) --
	(251.56,120.08) --
	(252.20,120.18) --
	(252.90,120.28) --
	(253.23,120.38) --
	(254.21,120.48) --
	(255.14,120.57) --
	(256.30,120.67) --
	(256.43,120.77) --
	(256.93,120.87) --
	(258.01,120.97) --
	(259.92,121.07) --
	(260.08,121.16) --
	(261.64,121.26) --
	(263.89,121.36) --
	(264.17,121.46) --
	(264.90,121.56) --
	(265.08,121.66) --
	(265.63,121.75) --
	(268.36,121.85) --
	(270.46,121.95) --
	(271.75,122.05) --
	(272.20,122.15) --
	(272.22,122.25) --
	(272.52,122.34) --
	(272.70,122.44) --
	(273.48,122.54) --
	(274.81,122.64) --
	(274.85,122.74) --
	(276.21,122.84) --
	(280.22,122.94) --
	(280.31,123.03) --
	(280.66,123.13) --
	(281.20,123.23) --
	(282.35,123.33) --
	(283.19,123.43) --
	(283.39,123.53) --
	(283.51,123.62) --
	(284.21,123.72) --
	(284.41,123.82) --
	(285.22,123.92) --
	(286.48,124.02) --
	(286.79,124.12) --
	(287.69,124.21) --
	(288.27,124.31) --
	(288.79,124.41) --
	(290.98,124.51) --
	(291.66,124.61) --
	(292.27,124.71) --
	(293.19,124.81) --
	(295.07,124.90) --
	(295.68,125.00) --
	(296.54,125.10) --
	(297.33,125.20) --
	(297.52,125.30) --
	(297.91,125.40) --
	(299.40,125.49) --
	(299.46,125.59) --
	(302.12,125.69) --
	(306.64,125.79) --
	(308.08,125.89) --
	(309.00,125.99) --
	(309.68,126.08) --
	(311.85,126.18) --
	(312.37,126.28) --
	(312.88,126.38) --
	(313.08,126.48) --
	(313.27,126.58) --
	(313.62,126.67) --
	(314.99,126.77) --
	(315.00,126.87) --
	(315.39,126.97) --
	(315.72,127.07) --
	(316.07,127.17) --
	(316.51,127.27) --
	(316.73,127.36) --
	(317.03,127.46);
\definecolor{drawColor}{RGB}{51,160,44}

\path[draw=drawColor,line width= 0.6pt,line join=round] (116.35, 36.93) --
	(123.55, 37.02) --
	(124.12, 37.12) --
	(124.40, 37.32) --
	(124.68, 37.61) --
	(124.95, 38.01) --
	(125.23, 38.30) --
	(125.50, 38.50) --
	(125.76, 38.89) --
	(126.03, 39.39) --
	(126.29, 40.37) --
	(126.55, 40.66) --
	(126.81, 41.35) --
	(127.07, 42.14) --
	(127.32, 43.12) --
	(127.57, 44.31) --
	(127.82, 44.80) --
	(128.06, 45.19) --
	(128.31, 45.78) --
	(128.55, 46.57) --
	(128.79, 47.26) --
	(129.03, 47.75) --
	(129.26, 48.44) --
	(129.26, 47.95) --
	(129.50, 49.32) --
	(129.73, 49.92) --
	(129.73, 49.82) --
	(129.96, 50.01) --
	(130.19, 50.41) --
	(130.19, 50.51) --
	(130.42, 50.70) --
	(130.64, 51.00) --
	(130.86, 51.10) --
	(131.08, 51.49) --
	(131.30, 52.08) --
	(131.52, 52.47) --
	(131.74, 53.56) --
	(131.95, 54.84) --
	(132.16, 55.82) --
	(132.38, 56.61) --
	(132.58, 57.59) --
	(132.58, 56.71) --
	(132.79, 58.87) --
	(133.00, 60.35) --
	(133.20, 60.44) --
	(133.20, 61.04) --
	(133.41, 61.63) --
	(133.61, 61.92) --
	(134.01, 62.12) --
	(134.21, 62.22) --
	(134.21, 63.00) --
	(134.40, 63.40) --
	(134.60, 63.89) --
	(134.79, 64.09) --
	(134.79, 64.97) --
	(134.98, 65.56) --
	(134.98, 65.17) --
	(135.17, 65.96) --
	(135.36, 66.25) --
	(135.36, 66.45) --
	(135.55, 66.94) --
	(135.74, 67.24) --
	(135.93, 67.33) --
	(135.93, 67.43) --
	(136.29, 68.02) --
	(136.29, 67.53) --
	(136.48, 68.12) --
	(136.84, 68.22) --
	(137.01, 68.32) --
	(137.19, 68.51) --
	(137.37, 68.61) --
	(137.37, 68.71) --
	(137.72, 68.81) --
	(137.72, 68.91) --
	(138.23, 69.01) --
	(138.40, 69.11) --
	(138.91, 69.30) --
	(139.07, 69.79) --
	(139.24, 69.99) --
	(139.24, 69.89) --
	(139.40, 70.09) --
	(139.57, 70.19) --
	(139.73, 70.29) --
	(139.89, 70.38) --
	(140.21, 70.78) --
	(140.53, 71.17) --
	(140.68, 71.27) --
	(140.84, 71.66) --
	(140.99, 71.76) --
	(140.99, 71.86) --
	(141.15, 72.16) --
	(141.15, 71.96) --
	(141.30, 72.25) --
	(141.45, 72.84) --
	(141.60, 72.94) --
	(141.91, 73.14) --
	(141.91, 73.04) --
	(142.05, 73.63) --
	(142.20, 73.83) --
	(142.35, 74.32) --
	(142.35, 74.22) --
	(142.50, 74.81) --
	(142.50, 74.91) --
	(142.79, 75.21) --
	(142.93, 75.40) --
	(142.93, 75.30) --
	(143.08, 75.50) --
	(143.22, 75.60) --
	(143.36, 75.90) --
	(143.50, 75.99) --
	(143.64, 76.09) --
	(143.78, 76.19) --
	(144.20, 76.29) --
	(144.34, 76.39) --
	(144.48, 76.49) --
	(144.88, 76.68) --
	(145.28, 76.88) --
	(145.41, 77.27) --
	(145.55, 77.37) --
	(145.68, 77.47) --
	(145.81, 77.57) --
	(146.07, 77.67) --
	(146.19, 77.77) --
	(147.20, 77.86) --
	(147.33, 77.96) --
	(147.57, 78.06) --
	(147.69, 78.16) --
	(148.06, 78.26) --
	(148.18, 78.36) --
	(148.65, 78.45) --
	(148.88, 78.55) --
	(149.23, 78.65) --
	(149.46, 78.85) --
	(149.80, 78.95) --
	(151.12, 79.04) --
	(151.23, 79.14) --
	(151.65, 79.24) --
	(151.76, 79.34) --
	(152.28, 79.44) --
	(152.59, 79.54) --
	(152.79, 79.63) --
	(152.89, 79.83) --
	(152.99, 79.93) --
	(153.59, 80.03) --
	(154.08, 80.13) --
	(154.65, 80.23) --
	(154.84, 80.32) --
	(154.94, 80.42) --
	(155.22, 80.52) --
	(155.31, 80.62) --
	(155.40, 80.72) --
	(155.77, 80.82) --
	(155.95, 80.91) --
	(156.04, 81.01) --
	(156.13, 81.11) --
	(156.22, 81.31) --
	(156.67, 81.41) --
	(156.75, 81.50) --
	(156.84, 81.60) --
	(156.93, 81.70) --
	(157.28, 81.90) --
	(157.28, 81.80) --
	(158.04, 82.00) --
	(158.13, 82.10) --
	(158.46, 82.19) --
	(159.11, 82.39) --
	(160.07, 82.49) --
	(160.38, 82.59) --
	(160.46, 82.69) --
	(160.61, 82.78) --
	(160.84, 82.88) --
	(160.91, 83.08) --
	(161.44, 83.28) --
	(161.44, 83.18) --
	(161.88, 83.37) --
	(162.10, 83.47) --
	(162.25, 83.67) --
	(162.39, 83.77) --
	(163.45, 83.87) --
	(163.52, 83.96) --
	(163.86, 84.16) --
	(163.93, 84.26) --
	(164.40, 84.36) --
	(164.47, 84.46) --
	(164.73, 84.65) --
	(164.93, 84.75) --
	(165.13, 84.85) --
	(165.77, 84.95) --
	(166.03, 85.05) --
	(166.09, 85.24) --
	(166.09, 85.15) --
	(166.34, 85.34) --
	(166.40, 85.44) --
	(166.77, 85.54) --
	(166.83, 85.64) --
	(167.20, 85.74) --
	(167.62, 85.83) --
	(167.80, 86.03) --
	(167.80, 85.93) --
	(168.27, 86.23) --
	(168.27, 86.13) --
	(168.56, 86.33) --
	(168.85, 86.43) --
	(169.41, 86.62) --
	(169.74, 86.72) --
	(169.85, 86.82) --
	(169.96, 86.92) --
	(170.18, 87.02) --
	(170.67, 87.11) --
	(170.88, 87.21) --
	(171.09, 87.31) --
	(171.25, 87.51) --
	(171.35, 87.61) --
	(171.61, 87.70) --
	(171.97, 87.80) --
	(172.28, 87.90) --
	(172.58, 88.00) --
	(172.78, 88.10) --
	(172.98, 88.20) --
	(173.32, 88.29) --
	(173.66, 88.39) --
	(173.90, 88.49) --
	(173.94, 88.59) --
	(174.69, 88.69) --
	(174.74, 88.79) --
	(176.36, 88.89) --
	(176.62, 88.98) --
	(177.39, 89.08) --
	(178.42, 89.18) --
	(178.94, 89.28) --
	(178.98, 89.38) --
	(179.65, 89.48) --
	(181.23, 89.57) --
	(181.27, 89.67) --
	(181.67, 89.77) --
	(182.10, 89.87) --
	(183.48, 89.97) --
	(183.75, 90.07) --
	(183.82, 90.16) --
	(185.25, 90.26) --
	(185.44, 90.36) --
	(189.07, 90.46) --
	(189.95, 90.56) --
	(191.15, 90.66) --
	(193.26, 90.76) --
	(194.58, 90.85) --
	(195.41, 90.95) --
	(195.55, 91.05) --
	(196.33, 91.25) --
	(196.56, 91.35) --
	(197.76, 91.44) --
	(197.92, 91.54) --
	(198.17, 91.64) --
	(198.60, 91.74) --
	(198.72, 91.84) --
	(199.63, 91.94) --
	(200.10, 92.03) --
	(200.67, 92.13) --
	(201.03, 92.23) --
	(203.37, 92.33) --
	(204.62, 92.43) --
	(206.06, 92.53) --
	(206.61, 92.63) --
	(207.33, 92.72) --
	(207.56, 92.82) --
	(207.90, 92.92) --
	(209.48, 93.02) --
	(209.99, 93.12) --
	(210.36, 93.22) --
	(210.36, 93.31) --
	(211.19, 93.41) --
	(211.27, 93.51) --
	(211.55, 93.61) --
	(212.08, 93.71) --
	(212.45, 93.81) --
	(212.74, 93.90) --
	(213.97, 94.00) --
	(214.42, 94.10) --
	(214.90, 94.20) --
	(215.73, 94.30) --
	(216.86, 94.40) --
	(217.27, 94.49) --
	(217.75, 94.59) --
	(218.57, 94.69) --
	(219.41, 94.79) --
	(219.80, 94.89) --
	(221.83, 94.99) --
	(221.85, 95.09) --
	(222.76, 95.18) --
	(222.86, 95.28) --
	(223.46, 95.38) --
	(225.90, 95.48) --
	(225.98, 95.58) --
	(227.18, 95.68) --
	(227.64, 95.77) --
	(227.78, 95.87) --
	(228.95, 95.97) --
	(229.46, 96.07) --
	(230.80, 96.17) --
	(230.83, 96.27) --
	(231.43, 96.36) --
	(231.47, 96.46) --
	(232.25, 96.56) --
	(232.39, 96.66) --
	(232.48, 96.76) --
	(232.91, 96.86) --
	(233.43, 96.96) --
	(234.43, 97.05) --
	(234.88, 97.15) --
	(235.28, 97.25) --
	(236.14, 97.35) --
	(236.47, 97.45) --
	(236.63, 97.55) --
	(239.55, 97.64) --
	(239.67, 97.74) --
	(240.75, 97.84) --
	(242.70, 97.94) --
	(243.30, 98.04) --
	(243.39, 98.14) --
	(243.85, 98.23) --
	(244.19, 98.33) --
	(245.58, 98.43) --
	(246.68, 98.53) --
	(247.05, 98.63) --
	(248.41, 98.73) --
	(249.31, 98.82) --
	(250.41, 98.92) --
	(250.78, 99.02) --
	(251.27, 99.12) --
	(252.26, 99.22) --
	(252.45, 99.32) --
	(253.04, 99.42) --
	(253.76, 99.51) --
	(254.09, 99.61) --
	(255.14, 99.71) --
	(256.56, 99.81) --
	(257.23, 99.91) --
	(257.35,100.01) --
	(259.66,100.10) --
	(260.19,100.20) --
	(263.72,100.30) --
	(263.90,100.40) --
	(264.77,100.50) --
	(264.93,100.60) --
	(266.99,100.69) --
	(267.78,100.79) --
	(268.10,100.89) --
	(273.74,100.99) --
	(275.98,101.09) --
	(278.98,101.19) --
	(280.60,101.29) --
	(280.62,101.38) --
	(283.38,101.48) --
	(283.86,101.58) --
	(283.95,101.68) --
	(284.01,101.78) --
	(284.69,101.88) --
	(285.33,101.97) --
	(285.97,102.07) --
	(287.08,102.17) --
	(287.19,102.27) --
	(287.63,102.37) --
	(288.36,102.47) --
	(289.27,102.56) --
	(289.39,102.66) --
	(290.02,102.76) --
	(291.75,102.86) --
	(293.29,102.96) --
	(294.81,103.06) --
	(295.96,103.15) --
	(296.49,103.25) --
	(297.16,103.35) --
	(297.28,103.45) --
	(297.78,103.55) --
	(300.10,103.65) --
	(303.30,103.75) --
	(303.37,103.84) --
	(304.30,103.94) --
	(309.02,104.04) --
	(309.19,104.14) --
	(309.25,104.24) --
	(310.07,104.34) --
	(310.08,104.43) --
	(310.48,104.53) --
	(311.62,104.63) --
	(311.64,104.73) --
	(311.77,104.83) --
	(312.22,104.93) --
	(312.44,105.02) --
	(313.01,105.12) --
	(314.55,105.22) --
	(314.86,105.32) --
	(314.99,105.42) --
	(315.40,105.52) --
	(315.55,105.62) --
	(315.69,105.71) --
	(315.76,105.81) --
	(316.06,105.91) --
	(317.05,106.01) --
	(317.31,106.11) --
	(317.42,106.21);
\definecolor{drawColor}{RGB}{106,61,154}

\path[draw=drawColor,line width= 0.6pt,line join=round] (117.81, 36.93) --
	(122.67, 37.12) --
	(122.97, 37.22) --
	(123.26, 37.42) --
	(123.55, 37.61) --
	(123.83, 37.81) --
	(124.12, 38.20) --
	(124.40, 38.60) --
	(124.68, 38.79) --
	(124.95, 38.89) --
	(125.23, 39.78) --
	(125.50, 40.27) --
	(125.76, 41.75) --
	(126.03, 42.44) --
	(126.29, 43.62) --
	(126.55, 44.60) --
	(126.81, 45.78) --
	(127.07, 46.47) --
	(127.32, 47.16) --
	(127.57, 47.85) --
	(127.82, 48.64) --
	(128.06, 49.03) --
	(128.31, 49.52) --
	(128.55, 49.92) --
	(128.79, 50.21) --
	(129.03, 50.31) --
	(129.03, 50.80) --
	(129.26, 50.90) --
	(129.26, 51.00) --
	(129.50, 51.49) --
	(129.73, 51.88) --
	(129.96, 53.06) --
	(130.19, 53.65) --
	(130.42, 54.74) --
	(130.64, 56.41) --
	(130.86, 57.30) --
	(131.08, 58.58) --
	(131.30, 59.26) --
	(131.52, 60.54) --
	(131.74, 61.82) --
	(131.95, 62.51) --
	(132.16, 63.10) --
	(132.38, 63.59) --
	(132.58, 64.09) --
	(132.79, 64.28) --
	(133.00, 64.38) --
	(133.00, 65.07) --
	(133.20, 65.76) --
	(133.41, 66.05) --
	(133.41, 65.86) --
	(133.61, 67.04) --
	(133.81, 67.14) --
	(133.81, 67.24) --
	(134.01, 67.33) --
	(134.21, 67.73) --
	(134.21, 67.43) --
	(134.40, 67.92) --
	(134.60, 68.22) --
	(134.60, 68.02) --
	(134.79, 68.42) --
	(134.98, 68.61) --
	(134.98, 68.71) --
	(135.36, 68.81) --
	(135.55, 69.11) --
	(135.55, 69.20) --
	(135.74, 69.30) --
	(135.93, 69.60) --
	(135.93, 69.40) --
	(135.93, 69.50) --
	(136.11, 69.70) --
	(136.29, 69.79) --
	(136.84, 69.89) --
	(137.19, 69.99) --
	(137.89, 70.29) --
	(138.06, 70.38) --
	(138.40, 70.58) --
	(138.40, 70.68) --
	(138.57, 71.17) --
	(138.74, 71.66) --
	(138.74, 71.76) --
	(138.91, 71.96) --
	(138.91, 71.86) --
	(139.07, 72.35) --
	(139.07, 72.25) --
	(139.40, 72.45) --
	(139.40, 72.55) --
	(139.57, 72.84) --
	(139.73, 73.04) --
	(139.89, 73.24) --
	(140.05, 73.53) --
	(140.21, 73.63) --
	(140.37, 73.83) --
	(140.53, 74.22) --
	(140.53, 73.93) --
	(140.68, 74.42) --
	(140.68, 74.52) --
	(140.84, 74.81) --
	(140.84, 75.11) --
	(140.99, 75.21) --
	(140.99, 75.30) --
	(141.15, 75.50) --
	(141.30, 75.70) --
	(141.30, 75.60) --
	(141.45, 76.39) --
	(141.76, 76.58) --
	(141.91, 76.68) --
	(142.05, 76.98) --
	(143.22, 77.08) --
	(143.50, 77.27) --
	(143.64, 77.47) --
	(143.78, 77.57) --
	(143.92, 77.86) --
	(144.06, 78.26) --
	(144.20, 78.36) --
	(144.48, 78.45) --
	(145.94, 78.55) --
	(146.32, 78.65) --
	(146.95, 78.75) --
	(147.20, 78.85) --
	(147.81, 78.95) --
	(147.94, 79.04) --
	(148.53, 79.14) --
	(148.77, 79.24) --
	(149.23, 79.34) --
	(149.46, 79.44) --
	(151.34, 79.54) --
	(151.76, 79.63) --
	(151.87, 79.73) --
	(151.97, 79.83) --
	(152.07, 79.93) --
	(152.18, 80.03) --
	(152.38, 80.13) --
	(152.59, 80.42) --
	(152.59, 80.23) --
	(153.19, 80.52) --
	(154.08, 80.62) --
	(154.27, 80.72) --
	(155.49, 80.82) --
	(155.68, 81.11) --
	(155.68, 80.91) --
	(156.04, 81.21) --
	(156.22, 81.31) --
	(156.31, 81.41) --
	(156.49, 81.50) --
	(156.58, 81.60) --
	(156.84, 81.70) --
	(156.93, 81.80) --
	(157.10, 81.90) --
	(157.45, 82.00) --
	(157.70, 82.10) --
	(158.29, 82.29) --
	(158.29, 82.19) --
	(158.54, 82.39) --
	(158.62, 82.49) --
	(158.79, 82.59) --
	(159.59, 82.69) --
	(159.67, 82.78) --
	(160.38, 82.98) --
	(160.69, 83.08) --
	(160.76, 83.18) --
	(161.36, 83.28) --
	(161.44, 83.37) --
	(161.51, 83.47) --
	(162.32, 83.57) --
	(162.68, 83.67) --
	(162.82, 83.77) --
	(162.96, 83.87) --
	(163.03, 83.96) --
	(163.24, 84.06) --
	(163.31, 84.16) --
	(164.13, 84.26) --
	(164.20, 84.36) --
	(164.53, 84.46) --
	(164.73, 84.56) --
	(164.80, 84.65) --
	(164.87, 84.75) --
	(164.93, 84.85) --
	(165.19, 84.95) --
	(165.45, 85.15) --
	(165.77, 85.34) --
	(165.77, 85.24) --
	(165.84, 85.54) --
	(165.90, 85.64) --
	(166.09, 85.74) --
	(166.53, 85.83) --
	(167.02, 85.93) --
	(167.20, 86.03) --
	(167.50, 86.13) --
	(167.74, 86.23) --
	(168.15, 86.33) --
	(168.21, 86.43) --
	(168.44, 86.62) --
	(168.85, 86.72) --
	(169.35, 86.82) --
	(169.80, 87.02) --
	(169.85, 87.11) --
	(170.02, 87.21) --
	(171.09, 87.31) --
	(171.15, 87.51) --
	(171.35, 87.61) --
	(171.56, 87.70) --
	(171.87, 87.90) --
	(172.33, 88.00) --
	(172.43, 88.10) --
	(172.48, 88.20) --
	(172.98, 88.29) --
	(174.37, 88.39) --
	(175.34, 88.49) --
	(175.56, 88.59) --
	(175.92, 88.69) --
	(176.01, 88.79) --
	(176.75, 88.89) --
	(177.60, 88.98) --
	(178.50, 89.08) --
	(178.54, 89.18) --
	(179.18, 89.28) --
	(179.30, 89.38) --
	(179.76, 89.48) --
	(180.79, 89.57) --
	(183.75, 89.67) --
	(183.95, 89.77) --
	(184.74, 89.87) --
	(185.28, 89.97) --
	(188.60, 90.07) --
	(188.68, 90.16) --
	(189.24, 90.26) --
	(189.38, 90.36) --
	(191.96, 90.46) --
	(192.31, 90.56) --
	(192.92, 90.66) --
	(193.45, 90.76) --
	(193.52, 90.85) --
	(194.79, 90.95) --
	(196.28, 91.05) --
	(196.35, 91.15) --
	(196.88, 91.25) --
	(197.86, 91.35) --
	(197.88, 91.44) --
	(198.93, 91.54) --
	(200.12, 91.64) --
	(201.48, 91.74) --
	(202.13, 91.84) --
	(205.56, 91.94) --
	(207.05, 92.03) --
	(207.67, 92.23) --
	(207.76, 92.33) --
	(210.19, 92.43) --
	(211.01, 92.53) --
	(211.78, 92.63) --
	(211.89, 92.72) --
	(211.98, 92.82) --
	(213.55, 92.92) --
	(213.86, 93.02) --
	(214.03, 93.12) --
	(214.31, 93.22) --
	(214.99, 93.31) --
	(215.03, 93.41) --
	(216.09, 93.51) --
	(216.43, 93.61) --
	(216.96, 93.71) --
	(217.00, 93.81) --
	(217.26, 93.90) --
	(218.09, 94.00) --
	(218.77, 94.10) --
	(219.09, 94.20) --
	(219.19, 94.30) --
	(219.25, 94.40) --
	(219.85, 94.49) --
	(219.93, 94.59) --
	(219.94, 94.69) --
	(220.51, 94.79) --
	(220.57, 94.89) --
	(220.59, 94.99) --
	(220.67, 95.09) --
	(220.71, 95.18) --
	(220.83, 95.28) --
	(221.49, 95.38) --
	(221.82, 95.48) --
	(222.40, 95.58) --
	(222.55, 95.68) --
	(222.74, 95.77) --
	(222.77, 95.87) --
	(223.27, 95.97) --
	(223.40, 96.07) --
	(223.64, 96.17) --
	(224.27, 96.27) --
	(225.22, 96.36) --
	(225.29, 96.46) --
	(225.32, 96.56) --
	(226.03, 96.66) --
	(227.23, 96.76) --
	(227.37, 96.86) --
	(227.50, 96.96) --
	(227.56, 97.05) --
	(227.88, 97.15) --
	(228.98, 97.25) --
	(228.98, 97.35) --
	(229.80, 97.45) --
	(229.87, 97.55) --
	(230.12, 97.64) --
	(230.59, 97.74) --
	(231.39, 97.84) --
	(231.46, 97.94) --
	(231.93, 98.04) --
	(232.24, 98.14) --
	(232.25, 98.23) --
	(232.38, 98.33) --
	(234.13, 98.43) --
	(234.78, 98.53) --
	(235.00, 98.63) --
	(235.33, 98.73) --
	(235.37, 98.82) --
	(235.94, 98.92) --
	(237.12, 99.02) --
	(238.99, 99.12) --
	(239.58, 99.22) --
	(240.03, 99.32) --
	(240.14, 99.42) --
	(245.76, 99.51) --
	(245.98, 99.61) --
	(246.27, 99.71) --
	(246.43, 99.81) --
	(247.14, 99.91) --
	(247.49,100.01) --
	(247.60,100.10) --
	(248.99,100.20) --
	(249.13,100.30) --
	(252.20,100.40) --
	(254.31,100.50) --
	(258.47,100.60) --
	(261.03,100.69) --
	(262.57,100.79) --
	(263.46,100.89) --
	(263.77,100.99) --
	(265.17,101.09) --
	(266.08,101.19) --
	(269.22,101.29) --
	(270.17,101.38) --
	(272.01,101.48) --
	(273.33,101.58) --
	(273.50,101.68) --
	(274.37,101.78) --
	(276.83,101.88) --
	(278.22,101.97) --
	(278.66,102.07) --
	(278.74,102.17) --
	(278.97,102.27) --
	(280.52,102.37) --
	(281.14,102.47) --
	(282.03,102.56) --
	(282.39,102.66) --
	(283.22,102.76) --
	(283.42,102.86) --
	(283.88,102.96) --
	(284.14,103.06) --
	(284.29,103.15) --
	(284.84,103.25) --
	(284.93,103.35) --
	(285.15,103.45) --
	(285.64,103.55) --
	(285.71,103.65) --
	(285.84,103.75) --
	(285.89,103.84) --
	(286.00,103.94) --
	(286.01,104.04) --
	(286.43,104.14) --
	(286.68,104.24) --
	(286.99,104.34) --
	(287.41,104.43) --
	(287.70,104.53) --
	(288.56,104.63) --
	(289.06,104.73) --
	(289.31,104.83) --
	(289.80,104.93) --
	(290.29,105.02) --
	(290.70,105.12) --
	(293.24,105.22) --
	(295.51,105.32) --
	(295.94,105.42) --
	(296.00,105.52) --
	(296.03,105.62) --
	(297.21,105.71) --
	(297.82,105.81) --
	(297.83,105.91) --
	(298.46,106.01) --
	(301.44,106.11) --
	(303.14,106.21) --
	(303.99,106.30) --
	(304.62,106.40) --
	(304.98,106.50) --
	(305.24,106.60) --
	(306.14,106.70) --
	(308.22,106.80) --
	(308.31,106.89) --
	(310.07,106.99) --
	(310.99,107.09) --
	(311.66,107.19) --
	(311.78,107.29) --
	(311.90,107.39) --
	(311.96,107.48) --
	(312.17,107.58) --
	(312.24,107.68) --
	(313.03,107.78) --
	(313.10,107.88) --
	(313.22,107.98) --
	(313.75,108.08) --
	(314.08,108.17) --
	(314.23,108.27) --
	(314.33,108.37) --
	(314.92,108.47) --
	(316.05,108.57) --
	(316.74,108.67) --
	(317.03,108.76);
\definecolor{drawColor}{RGB}{202,178,214}

\path[draw=drawColor,line width= 0.6pt,dash pattern=on 4pt off 4pt ,line join=round] ( 87.78, 36.93) --
	(120.51, 37.02) --
	(121.76, 37.12) --
	(122.97, 37.22) --
	(124.12, 37.32) --
	(124.40, 37.42) --
	(124.68, 37.52) --
	(125.50, 37.81) --
	(125.76, 38.01) --
	(126.29, 38.30) --
	(126.81, 38.50) --
	(127.82, 38.60) --
	(127.82, 38.70) --
	(128.31, 38.99) --
	(128.55, 39.09) --
	(128.79, 39.29) --
	(129.03, 39.68) --
	(129.26, 39.78) --
	(129.50, 39.98) --
	(129.50, 40.17) --
	(129.96, 40.37) --
	(130.19, 40.47) --
	(130.42, 41.16) --
	(130.64, 41.26) --
	(130.86, 41.45) --
	(130.86, 41.35) --
	(131.08, 41.55) --
	(131.30, 41.65) --
	(131.52, 42.34) --
	(131.95, 42.44) --
	(132.16, 42.53) --
	(132.16, 42.73) --
	(132.38, 43.03) --
	(132.58, 43.72) --
	(132.79, 43.81) --
	(133.20, 43.91) --
	(133.61, 44.01) --
	(133.81, 44.21) --
	(134.01, 44.70) --
	(134.01, 44.31) --
	(134.21, 45.49) --
	(134.60, 45.88) --
	(134.79, 46.27) --
	(134.79, 46.08) --
	(134.98, 46.47) --
	(134.98, 46.57) --
	(135.17, 46.86) --
	(135.36, 47.16) --
	(135.55, 47.26) --
	(135.74, 47.85) --
	(135.93, 47.95) --
	(136.11, 48.14) --
	(136.29, 48.44) --
	(136.29, 48.24) --
	(136.48, 48.73) --
	(136.66, 49.23) --
	(136.84, 49.62) --
	(137.01, 49.72) --
	(137.19, 49.82) --
	(137.37, 50.01) --
	(137.54, 50.11) --
	(137.54, 50.31) --
	(137.72, 50.90) --
	(137.72, 50.41) --
	(137.89, 51.19) --
	(138.06, 51.29) --
	(138.06, 51.39) --
	(138.40, 51.49) --
	(138.40, 51.69) --
	(138.57, 51.98) --
	(138.57, 51.78) --
	(138.74, 52.18) --
	(138.74, 52.08) --
	(138.91, 52.47) --
	(139.07, 52.57) --
	(139.07, 52.77) --
	(139.24, 52.87) --
	(139.24, 52.97) --
	(139.40, 53.16) --
	(139.40, 53.26) --
	(139.57, 53.36) --
	(139.73, 53.56) --
	(139.89, 53.75) --
	(140.05, 53.95) --
	(140.21, 54.54) --
	(140.37, 54.64) --
	(140.53, 54.74) --
	(140.68, 55.03) --
	(140.84, 55.13) --
	(140.99, 55.43) --
	(141.15, 55.52) --
	(141.15, 55.62) --
	(141.45, 55.82) --
	(141.45, 55.72) --
	(141.60, 55.92) --
	(141.60, 56.11) --
	(141.76, 56.21) --
	(141.91, 56.41) --
	(141.91, 56.51) --
	(142.05, 56.71) --
	(142.20, 56.90) --
	(142.50, 57.00) --
	(142.79, 57.10) --
	(142.93, 57.30) --
	(142.93, 57.49) --
	(143.22, 57.59) --
	(143.36, 57.69) --
	(143.50, 57.79) --
	(143.64, 57.98) --
	(143.78, 58.08) --
	(143.92, 58.48) --
	(143.92, 58.28) --
	(144.06, 58.67) --
	(144.34, 58.77) --
	(144.48, 59.07) --
	(144.61, 59.26) --
	(144.75, 59.46) --
	(144.75, 59.36) --
	(144.88, 59.56) --
	(145.15, 59.66) --
	(145.28, 59.85) --
	(145.41, 60.15) --
	(145.55, 60.25) --
	(145.68, 60.35) --
	(145.81, 60.44) --
	(145.94, 60.54) --
	(146.07, 60.64) --
	(146.19, 60.74) --
	(146.19, 60.84) --
	(146.32, 61.04) --
	(146.58, 61.13) --
	(146.70, 61.23) --
	(146.83, 61.33) --
	(146.95, 61.43) --
	(147.08, 61.53) --
	(147.20, 61.63) --
	(147.20, 61.72) --
	(147.33, 61.82) --
	(147.45, 62.02) --
	(147.45, 61.92) --
	(147.57, 62.31) --
	(147.69, 62.51) --
	(147.69, 62.41) --
	(147.81, 62.71) --
	(147.94, 62.81) --
	(148.18, 62.91) --
	(148.29, 63.00) --
	(148.41, 63.20) --
	(148.41, 63.10) --
	(148.53, 63.30) --
	(148.88, 63.40) --
	(149.12, 63.50) --
	(149.57, 63.59) --
	(149.69, 63.79) --
	(149.91, 63.99) --
	(150.03, 64.28) --
	(150.14, 64.38) --
	(150.25, 64.48) --
	(150.58, 64.58) --
	(150.91, 64.68) --
	(151.12, 64.78) --
	(151.12, 64.97) --
	(152.38, 65.07) --
	(152.99, 65.27) --
	(153.09, 65.37) --
	(153.19, 65.46) --
	(153.29, 65.56) --
	(153.59, 65.66) --
	(153.69, 65.76) --
	(154.18, 65.96) --
	(154.46, 66.05) --
	(154.56, 66.15) --
	(154.65, 66.25) --
	(154.94, 66.45) --
	(155.03, 66.55) --
	(155.12, 66.64) --
	(155.22, 66.74) --
	(155.68, 66.84) --
	(155.77, 66.94) --
	(155.86, 67.04) --
	(156.04, 67.14) --
	(156.13, 67.24) --
	(156.75, 67.33) --
	(157.02, 67.53) --
	(157.70, 67.63) --
	(157.87, 67.73) --
	(157.96, 67.83) --
	(157.96, 67.92) --
	(158.62, 68.02) --
	(158.95, 68.12) --
	(159.11, 68.32) --
	(159.44, 68.42) --
	(159.67, 68.51) --
	(160.30, 68.61) --
	(160.38, 68.71) --
	(160.76, 68.81) --
	(161.29, 68.91) --
	(161.44, 69.20) --
	(161.66, 69.30) --
	(162.17, 69.40) --
	(162.39, 69.50) --
	(162.39, 69.70) --
	(162.46, 69.79) --
	(162.46, 69.89) --
	(162.53, 69.99) --
	(162.68, 70.09) --
	(162.82, 70.19) --
	(163.17, 70.29) --
	(163.52, 70.48) --
	(163.79, 70.58) --
	(164.13, 70.68) --
	(164.40, 70.78) --
	(164.60, 70.88) --
	(164.73, 71.07) --
	(164.80, 71.17) --
	(164.93, 71.37) --
	(165.06, 71.47) --
	(165.13, 71.57) --
	(165.32, 71.66) --
	(165.64, 71.76) --
	(165.84, 71.86) --
	(165.90, 72.06) --
	(165.96, 72.25) --
	(166.09, 72.35) --
	(166.65, 72.45) --
	(166.90, 72.55) --
	(167.14, 72.65) --
	(167.32, 72.75) --
	(167.74, 72.84) --
	(167.86, 73.24) --
	(167.86, 72.94) --
	(167.92, 73.34) --
	(168.09, 73.44) --
	(168.27, 73.63) --
	(168.44, 73.73) --
	(168.50, 73.83) --
	(168.56, 73.93) --
	(168.90, 74.03) --
	(168.96, 74.12) --
	(169.19, 74.22) --
	(169.24, 74.32) --
	(169.30, 74.42) --
	(169.41, 74.52) --
	(169.52, 74.62) --
	(169.74, 74.71) --
	(169.80, 74.91) --
	(170.02, 75.01) --
	(170.02, 75.11) --
	(170.29, 75.21) --
	(170.40, 75.30) --
	(170.56, 75.40) --
	(170.77, 75.50) --
	(170.88, 75.60) --
	(171.04, 75.80) --
	(171.09, 75.90) --
	(171.61, 76.09) --
	(171.87, 76.19) --
	(172.02, 76.29) --
	(172.02, 76.39) --
	(172.08, 76.49) --
	(172.18, 76.58) --
	(172.33, 76.68) --
	(172.43, 76.78) --
	(173.12, 76.98) --
	(173.37, 77.08) --
	(173.51, 77.37) --
	(173.71, 77.47) --
	(173.85, 77.57) --
	(173.85, 77.67) --
	(173.90, 77.77) --
	(174.65, 77.96) --
	(174.83, 78.06) --
	(175.20, 78.16) --
	(175.34, 78.26) --
	(175.52, 78.36) --
	(175.61, 78.45) --
	(175.65, 78.55) --
	(175.70, 78.65) --
	(175.83, 78.75) --
	(175.87, 78.85) --
	(176.01, 79.04) --
	(176.23, 79.14) --
	(176.27, 79.24) --
	(176.31, 79.34) --
	(176.40, 79.44) --
	(176.70, 79.54) --
	(176.83, 79.63) --
	(177.17, 79.73) --
	(177.64, 79.83) --
	(177.76, 79.93) --
	(177.93, 80.03) --
	(178.21, 80.13) --
	(178.34, 80.23) --
	(178.94, 80.32) --
	(179.02, 80.42) --
	(179.22, 80.52) --
	(179.53, 80.62) --
	(179.61, 80.72) --
	(179.69, 80.91) --
	(180.52, 81.01) --
	(180.67, 81.11) --
	(181.19, 81.21) --
	(181.52, 81.31) --
	(181.59, 81.41) --
	(182.27, 81.50) --
	(182.41, 81.60) --
	(182.45, 81.70) --
	(182.73, 81.80) --
	(182.80, 81.90) --
	(182.94, 82.00) --
	(183.35, 82.10) --
	(183.38, 82.29) --
	(183.45, 82.39) --
	(183.52, 82.49) --
	(184.05, 82.69) --
	(184.38, 82.78) --
	(184.48, 82.88) --
	(186.19, 82.98) --
	(186.22, 83.18) --
	(186.83, 83.28) --
	(186.92, 83.47) --
	(187.09, 83.57) --
	(187.12, 83.67) --
	(187.36, 83.77) --
	(187.39, 83.87) --
	(188.46, 83.96) --
	(188.71, 84.06) --
	(188.93, 84.16) --
	(189.02, 84.36) --
	(189.07, 84.46) --
	(189.18, 84.56) --
	(190.32, 84.65) --
	(190.48, 84.85) --
	(190.66, 84.95) --
	(191.02, 85.05) --
	(191.18, 85.15) --
	(191.26, 85.24) --
	(191.64, 85.34) --
	(193.49, 85.44) --
	(193.52, 85.54) --
	(193.64, 85.64) --
	(193.71, 85.74) --
	(193.85, 85.83) --
	(193.89, 85.93) --
	(194.20, 86.03) --
	(194.52, 86.13) --
	(194.54, 86.23) --
	(194.81, 86.33) --
	(195.21, 86.43) --
	(195.28, 86.52) --
	(195.30, 86.62) --
	(195.39, 86.72) --
	(195.41, 86.82) --
	(195.52, 86.92) --
	(195.57, 87.02) --
	(195.68, 87.21) --
	(195.70, 87.31) --
	(195.74, 87.51) --
	(195.81, 87.61) --
	(195.85, 87.70) --
	(196.05, 87.80) --
	(196.13, 87.90) --
	(196.64, 88.00) --
	(196.77, 88.10) --
	(196.90, 88.20) --
	(197.02, 88.29) --
	(197.08, 88.39) --
	(197.23, 88.49) --
	(197.54, 88.59) --
	(198.64, 88.69) --
	(198.66, 88.79) --
	(198.80, 88.89) --
	(198.82, 88.98) --
	(198.84, 89.08) --
	(199.11, 89.18) --
	(199.11, 89.28) --
	(199.13, 89.38) --
	(199.46, 89.48) --
	(199.55, 89.57) --
	(199.85, 89.67) --
	(200.26, 89.77) --
	(200.98, 89.87) --
	(201.05, 89.97) --
	(201.78, 90.07) --
	(201.90, 90.16) --
	(202.44, 90.26) --
	(202.48, 90.36) --
	(202.49, 90.46) --
	(202.60, 90.56) --
	(202.75, 90.66) --
	(202.90, 90.76) --
	(203.04, 90.85) --
	(203.10, 90.95) --
	(203.29, 91.05) --
	(203.78, 91.15) --
	(204.32, 91.35) --
	(204.49, 91.54) --
	(204.70, 91.74) --
	(204.75, 91.84) --
	(204.87, 91.94) --
	(204.97, 92.03) --
	(205.00, 92.13) --
	(205.22, 92.23) --
	(205.74, 92.33) --
	(206.36, 92.53) --
	(206.60, 92.63) --
	(206.72, 92.72) --
	(206.97, 92.82) --
	(207.02, 92.92) --
	(207.20, 93.02) --
	(207.33, 93.12) --
	(207.49, 93.22) --
	(207.57, 93.31) --
	(207.59, 93.41) --
	(207.70, 93.61) --
	(207.83, 93.71) --
	(207.97, 93.90) --
	(208.14, 94.00) --
	(208.15, 94.10) --
	(208.25, 94.20) --
	(208.35, 94.30) --
	(208.57, 94.40) --
	(208.62, 94.49) --
	(208.69, 94.59) --
	(208.90, 94.69) --
	(208.94, 94.79) --
	(209.03, 94.89) --
	(209.40, 94.99) --
	(209.52, 95.09) --
	(209.61, 95.28) --
	(209.88, 95.38) --
	(210.32, 95.48) --
	(210.41, 95.58) --
	(210.45, 95.68) --
	(210.54, 95.77) --
	(210.81, 95.87) --
	(210.82, 95.97) --
	(210.88, 96.07) --
	(211.54, 96.17) --
	(211.64, 96.27) --
	(211.65, 96.36) --
	(211.89, 96.46) --
	(211.91, 96.56) --
	(211.98, 96.76) --
	(212.08, 96.86) --
	(212.14, 96.96) --
	(212.29, 97.05) --
	(212.32, 97.15) --
	(212.49, 97.25) --
	(212.84, 97.35) --
	(212.87, 97.45) --
	(213.03, 97.55) --
	(213.14, 97.64) --
	(213.26, 97.74) --
	(213.33, 97.84) --
	(213.60, 97.94) --
	(213.84, 98.04) --
	(214.12, 98.23) --
	(214.14, 98.33) --
	(214.42, 98.43) --
	(214.47, 98.53) --
	(214.75, 98.63) --
	(214.92, 98.73) --
	(215.00, 98.82) --
	(215.02, 98.92) --
	(215.29, 99.02) --
	(215.46, 99.12) --
	(215.56, 99.22) --
	(215.61, 99.32) --
	(215.78, 99.42) --
	(215.81, 99.51) --
	(216.04, 99.61) --
	(216.17, 99.71) --
	(216.91, 99.81) --
	(216.93, 99.91) --
	(217.13,100.10) --
	(217.33,100.20) --
	(217.34,100.30) --
	(217.40,100.40) --
	(217.43,100.50) --
	(217.80,100.60) --
	(217.94,100.69) --
	(217.95,100.79) --
	(218.02,100.89) --
	(218.10,100.99) --
	(218.11,101.09) --
	(218.21,101.19) --
	(218.56,101.29) --
	(218.67,101.38) --
	(218.88,101.48) --
	(218.98,101.58) --
	(219.07,101.68) --
	(219.18,101.78) --
	(219.32,101.88) --
	(219.41,101.97) --
	(219.50,102.07) --
	(219.62,102.17) --
	(219.74,102.37) --
	(220.34,102.47) --
	(220.37,102.56) --
	(220.63,102.66) --
	(220.99,102.76) --
	(221.03,102.86) --
	(221.21,102.96) --
	(221.32,103.06) --
	(221.39,103.15) --
	(221.53,103.25) --
	(221.56,103.35) --
	(221.59,103.45) --
	(222.37,103.55) --
	(222.38,103.75) --
	(222.38,103.65) --
	(222.45,103.84) --
	(222.75,103.94) --
	(223.01,104.04) --
	(223.02,104.14) --
	(223.11,104.24) --
	(223.56,104.34) --
	(223.69,104.43) --
	(223.77,104.63) --
	(223.80,104.73) --
	(224.27,104.83) --
	(224.48,104.93) --
	(224.49,105.02) --
	(224.54,105.12) --
	(224.56,105.22) --
	(224.56,105.32) --
	(224.68,105.42) --
	(224.70,105.52) --
	(224.82,105.62) --
	(224.90,105.71) --
	(225.01,105.81) --
	(225.23,105.91) --
	(225.69,106.01) --
	(225.78,106.11) --
	(225.85,106.21) --
	(225.92,106.30) --
	(226.03,106.40) --
	(226.06,106.50) --
	(226.55,106.60) --
	(226.60,106.70) --
	(226.90,106.80) --
	(227.23,106.89) --
	(227.27,106.99) --
	(227.38,107.09) --
	(227.49,107.19) --
	(227.78,107.29) --
	(227.82,107.39) --
	(228.26,107.48) --
	(228.27,107.58) --
	(228.29,107.68) --
	(228.41,107.78) --
	(228.43,107.88) --
	(228.63,107.98) --
	(228.71,108.08) --
	(228.80,108.17) --
	(228.89,108.27) --
	(228.96,108.37) --
	(229.12,108.47) --
	(229.33,108.57) --
	(229.37,108.67) --
	(229.43,108.76) --
	(229.49,108.86) --
	(229.57,108.96) --
	(229.61,109.06) --
	(229.63,109.16) --
	(229.67,109.26) --
	(229.82,109.35) --
	(229.92,109.45) --
	(230.01,109.55) --
	(230.19,109.65) --
	(230.26,109.75) --
	(230.60,109.85) --
	(230.82,109.95) --
	(231.44,110.04) --
	(231.58,110.14) --
	(231.63,110.24) --
	(231.72,110.34) --
	(231.93,110.44) --
	(231.97,110.54) --
	(232.00,110.63) --
	(232.20,110.73) --
	(232.26,110.83) --
	(232.46,110.93) --
	(232.73,111.03) --
	(232.73,111.13) --
	(232.92,111.22) --
	(233.07,111.32) --
	(233.13,111.42) --
	(233.26,111.52) --
	(233.30,111.62) --
	(233.54,111.72) --
	(233.61,111.81) --
	(233.95,111.91) --
	(234.50,112.01) --
	(234.63,112.11) --
	(235.38,112.21) --
	(235.47,112.31) --
	(235.49,112.41) --
	(235.51,112.50) --
	(235.54,112.60) --
	(235.65,112.70) --
	(236.24,112.80) --
	(236.42,112.90) --
	(236.60,113.00) --
	(236.77,113.09) --
	(236.82,113.19) --
	(236.84,113.29) --
	(236.85,113.39) --
	(236.95,113.49) --
	(237.04,113.59) --
	(237.16,113.68) --
	(237.25,113.78) --
	(237.39,113.88) --
	(237.56,113.98) --
	(237.66,114.08) --
	(237.68,114.18) --
	(237.79,114.28) --
	(237.94,114.37) --
	(238.53,114.47) --
	(239.13,114.57) --
	(239.66,114.67) --
	(240.14,114.77) --
	(240.34,114.87) --
	(240.76,114.96) --
	(240.88,115.06) --
	(240.91,115.16) --
	(240.92,115.26) --
	(241.34,115.36) --
	(241.40,115.46) --
	(241.47,115.55) --
	(241.73,115.65) --
	(242.21,115.75) --
	(242.93,115.85) --
	(242.94,115.95) --
	(243.06,116.05) --
	(243.20,116.15) --
	(243.60,116.24) --
	(243.73,116.34) --
	(243.94,116.44) --
	(244.56,116.54) --
	(244.58,116.64) --
	(244.69,116.74) --
	(245.08,116.83) --
	(245.30,116.93) --
	(245.35,117.03) --
	(245.69,117.13) --
	(245.95,117.23) --
	(246.23,117.33) --
	(246.59,117.42) --
	(246.82,117.52) --
	(246.86,117.62) --
	(247.17,117.72) --
	(247.96,117.82) --
	(248.41,117.92) --
	(248.79,118.01) --
	(249.06,118.11) --
	(249.23,118.21) --
	(249.42,118.31) --
	(249.61,118.41) --
	(250.20,118.51) --
	(250.53,118.61) --
	(251.03,118.70) --
	(251.14,118.80) --
	(251.28,118.90) --
	(251.57,119.00) --
	(251.91,119.10) --
	(251.93,119.20) --
	(252.13,119.29) --
	(252.18,119.39) --
	(252.45,119.49) --
	(252.88,119.59) --
	(252.89,119.69) --
	(253.17,119.79) --
	(253.49,119.88) --
	(253.91,119.98) --
	(254.54,120.08) --
	(254.58,120.18) --
	(254.91,120.28) --
	(255.42,120.48) --
	(256.02,120.57) --
	(256.05,120.67) --
	(256.54,120.77) --
	(256.54,120.87) --
	(256.78,120.97) --
	(257.08,121.07) --
	(257.14,121.16) --
	(257.17,121.26) --
	(257.24,121.36) --
	(257.36,121.46) --
	(257.38,121.56) --
	(257.70,121.66) --
	(257.73,121.75) --
	(257.74,121.85) --
	(258.27,121.95) --
	(258.33,122.05) --
	(258.44,122.15) --
	(258.52,122.25) --
	(258.70,122.34) --
	(258.73,122.44) --
	(258.77,122.54) --
	(258.92,122.64) --
	(259.05,122.74) --
	(259.34,122.84) --
	(259.35,122.94) --
	(259.68,123.03) --
	(260.06,123.13) --
	(260.41,123.23) --
	(260.66,123.33) --
	(260.92,123.43) --
	(261.19,123.53) --
	(261.20,123.62) --
	(261.36,123.72) --
	(261.64,123.82) --
	(261.86,123.92) --
	(262.15,124.02) --
	(262.26,124.12) --
	(262.32,124.21) --
	(262.55,124.31) --
	(262.74,124.41) --
	(262.79,124.51) --
	(263.06,124.61) --
	(263.23,124.71) --
	(263.53,124.81) --
	(263.66,124.90) --
	(264.44,125.00) --
	(264.71,125.10) --
	(264.74,125.20) --
	(264.84,125.30) --
	(265.26,125.40) --
	(265.44,125.49) --
	(265.62,125.59) --
	(266.22,125.69) --
	(266.41,125.79) --
	(266.42,125.89) --
	(266.53,125.99) --
	(266.68,126.08) --
	(267.23,126.18) --
	(267.78,126.28) --
	(268.63,126.38) --
	(268.92,126.48) --
	(269.70,126.58) --
	(269.70,126.67) --
	(270.15,126.77) --
	(270.51,126.87) --
	(270.73,126.97) --
	(270.82,127.07) --
	(270.93,127.17) --
	(271.60,127.27) --
	(271.65,127.36) --
	(271.72,127.46) --
	(272.09,127.56) --
	(272.32,127.66) --
	(272.37,127.76) --
	(272.42,127.86) --
	(273.09,127.95) --
	(273.18,128.05) --
	(275.05,128.15) --
	(275.15,128.25) --
	(276.78,128.35) --
	(277.99,128.45) --
	(278.08,128.54) --
	(278.29,128.64) --
	(278.30,128.74) --
	(279.90,128.84) --
	(280.80,128.94) --
	(280.84,129.04) --
	(280.91,129.14) --
	(281.77,129.23) --
	(281.81,129.33) --
	(282.25,129.43) --
	(282.35,129.53) --
	(282.60,129.63) --
	(283.30,129.73) --
	(283.50,129.82) --
	(283.94,129.92) --
	(284.36,130.02) --
	(284.44,130.12) --
	(284.45,130.22) --
	(285.04,130.32) --
	(285.83,130.41) --
	(285.99,130.51) --
	(286.24,130.61) --
	(286.30,130.71) --
	(286.33,130.81) --
	(286.88,130.91) --
	(287.06,131.00) --
	(287.29,131.10) --
	(287.62,131.20) --
	(287.75,131.30) --
	(287.86,131.40) --
	(288.17,131.50) --
	(288.29,131.60) --
	(289.11,131.69) --
	(289.41,131.79) --
	(289.48,131.89) --
	(289.72,131.99) --
	(290.00,132.09) --
	(291.02,132.19) --
	(291.10,132.28) --
	(291.61,132.38) --
	(292.13,132.48) --
	(292.27,132.58) --
	(292.41,132.68) --
	(293.26,132.78) --
	(294.40,132.87) --
	(298.30,132.97) --
	(298.69,133.07) --
	(300.72,133.17) --
	(301.60,133.27) --
	(303.32,133.37) --
	(304.73,133.47) --
	(305.01,133.56) --
	(305.72,133.66) --
	(306.02,133.76) --
	(306.42,133.86) --
	(307.58,133.96) --
	(307.95,134.06) --
	(308.61,134.15) --
	(308.95,134.25) --
	(309.52,134.35) --
	(309.59,134.45) --
	(310.56,134.55) --
	(310.67,134.65) --
	(311.05,134.74) --
	(311.21,134.84) --
	(311.23,134.94) --
	(311.78,135.04) --
	(311.90,135.14) --
	(312.06,135.24) --
	(312.12,135.33) --
	(312.13,135.43) --
	(312.37,135.53) --
	(313.08,135.63) --
	(313.49,135.73) --
	(313.52,135.83) --
	(313.64,135.93) --
	(313.74,136.02) --
	(313.79,136.12) --
	(314.05,136.22) --
	(314.20,136.32) --
	(314.99,136.42) --
	(315.11,136.52) --
	(315.25,136.61) --
	(315.26,136.71) --
	(315.45,136.81) --
	(315.82,136.91) --
	(315.82,137.01) --
	(315.91,137.11) --
	(315.98,137.20) --
	(315.99,137.30) --
	(316.81,137.40) --
	(317.60,137.50);
\definecolor{drawColor}{RGB}{166,206,227}

\path[draw=drawColor,line width= 0.6pt,dash pattern=on 4pt off 4pt ,line join=round] (118.51, 36.93) --
	(122.67, 37.22) --
	(122.97, 37.32) --
	(123.26, 37.52) --
	(123.55, 37.61) --
	(123.83, 37.91) --
	(124.12, 38.20) --
	(124.40, 38.30) --
	(124.68, 38.50) --
	(124.95, 39.09) --
	(125.23, 39.48) --
	(125.50, 40.17) --
	(125.76, 40.57) --
	(126.03, 41.55) --
	(126.29, 42.14) --
	(126.55, 42.63) --
	(126.81, 43.32) --
	(127.07, 43.42) --
	(127.32, 43.81) --
	(127.57, 44.50) --
	(127.82, 45.09) --
	(127.82, 44.60) --
	(128.06, 45.19) --
	(128.31, 45.39) --
	(128.31, 45.49) --
	(128.55, 45.78) --
	(128.55, 45.59) --
	(128.79, 45.98) --
	(129.03, 46.57) --
	(129.03, 46.77) --
	(129.26, 47.26) --
	(129.50, 48.14) --
	(129.50, 47.95) --
	(129.73, 48.64) --
	(129.96, 49.03) --
	(130.19, 50.01) --
	(130.42, 51.10) --
	(130.64, 51.88) --
	(130.86, 52.47) --
	(131.08, 53.06) --
	(131.30, 54.05) --
	(131.30, 53.16) --
	(131.52, 54.54) --
	(131.52, 54.15) --
	(131.74, 54.74) --
	(131.95, 55.13) --
	(132.16, 55.52) --
	(132.38, 55.82) --
	(132.58, 56.02) --
	(132.79, 56.51) --
	(133.00, 56.71) --
	(133.20, 57.30) --
	(133.20, 57.00) --
	(133.41, 57.59) --
	(133.41, 57.39) --
	(133.61, 57.69) --
	(134.01, 57.89) --
	(134.21, 58.08) --
	(134.40, 58.28) --
	(134.40, 58.18) --
	(134.60, 58.67) --
	(134.60, 58.48) --
	(134.79, 58.77) --
	(134.98, 59.17) --
	(134.98, 59.07) --
	(135.74, 59.36) --
	(135.93, 59.56) --
	(136.11, 59.66) --
	(136.66, 59.95) --
	(136.84, 60.05) --
	(137.19, 60.25) --
	(137.54, 60.64) --
	(137.54, 60.44) --
	(137.72, 60.94) --
	(137.89, 61.13) --
	(137.89, 61.04) --
	(138.06, 61.23) --
	(138.23, 61.43) --
	(138.40, 61.53) --
	(138.57, 61.82) --
	(138.57, 61.92) --
	(138.74, 62.02) --
	(138.91, 62.12) --
	(139.07, 62.31) --
	(139.24, 62.61) --
	(139.40, 62.71) --
	(139.57, 62.91) --
	(139.73, 63.20) --
	(139.73, 63.10) --
	(139.89, 63.30) --
	(140.05, 63.50) --
	(140.21, 63.59) --
	(140.37, 63.79) --
	(140.53, 63.89) --
	(140.68, 64.09) --
	(140.99, 64.18) --
	(141.15, 64.48) --
	(141.15, 64.28) --
	(141.30, 64.87) --
	(141.45, 64.97) --
	(141.45, 65.07) --
	(141.60, 65.37) --
	(141.76, 65.56) --
	(141.76, 65.66) --
	(141.91, 65.86) --
	(142.05, 65.96) --
	(142.20, 66.05) --
	(142.50, 66.35) --
	(142.64, 66.45) --
	(142.79, 66.74) --
	(142.79, 66.84) --
	(143.36, 66.94) --
	(143.50, 67.04) --
	(143.64, 67.14) --
	(143.64, 67.24) --
	(144.06, 67.43) --
	(144.20, 67.63) --
	(144.34, 67.83) --
	(144.48, 67.92) --
	(144.61, 68.02) --
	(144.88, 68.22) --
	(145.02, 68.32) --
	(145.41, 68.51) --
	(145.55, 68.71) --
	(145.68, 68.81) --
	(145.68, 69.01) --
	(145.94, 69.11) --
	(146.07, 69.30) --
	(146.32, 69.40) --
	(146.45, 69.60) --
	(146.58, 69.79) --
	(146.70, 69.99) --
	(146.83, 70.19) --
	(146.95, 70.29) --
	(147.08, 70.38) --
	(147.33, 70.78) --
	(147.57, 70.97) --
	(147.69, 71.37) --
	(147.81, 71.57) --
	(147.94, 71.86) --
	(148.06, 71.96) --
	(148.18, 72.06) --
	(148.29, 72.16) --
	(148.53, 72.25) --
	(148.65, 72.45) --
	(148.65, 72.35) --
	(148.77, 72.65) --
	(148.77, 72.55) --
	(148.88, 72.75) --
	(149.12, 72.84) --
	(149.23, 72.94) --
	(149.35, 73.14) --
	(149.46, 73.24) --
	(149.91, 73.34) --
	(150.03, 73.53) --
	(150.14, 73.73) --
	(150.14, 73.63) --
	(150.25, 73.93) --
	(150.36, 74.03) --
	(150.47, 74.22) --
	(150.58, 74.32) --
	(150.69, 74.42) --
	(150.80, 74.52) --
	(150.91, 74.71) --
	(151.01, 75.01) --
	(151.23, 75.21) --
	(151.34, 75.40) --
	(151.44, 75.50) --
	(151.55, 75.80) --
	(151.55, 75.60) --
	(151.65, 75.90) --
	(151.76, 75.99) --
	(152.07, 76.09) --
	(152.18, 76.19) --
	(152.28, 76.39) --
	(152.38, 76.49) --
	(152.59, 76.58) --
	(152.59, 76.68) --
	(152.69, 76.78) --
	(152.89, 76.88) --
	(152.99, 77.27) --
	(153.09, 77.37) --
	(153.19, 77.57) --
	(153.39, 77.67) --
	(153.49, 78.06) --
	(153.49, 77.77) --
	(153.59, 78.55) --
	(153.59, 78.45) --
	(153.69, 78.65) --
	(153.79, 78.85) --
	(153.79, 78.75) --
	(153.89, 79.24) --
	(153.98, 79.34) --
	(154.08, 79.63) --
	(154.08, 79.44) --
	(154.18, 79.83) --
	(154.27, 80.03) --
	(154.37, 80.13) --
	(154.46, 80.32) --
	(154.65, 80.42) --
	(154.65, 80.52) --
	(154.75, 80.72) --
	(154.94, 81.11) --
	(155.03, 81.31) --
	(155.22, 81.41) --
	(155.31, 81.70) --
	(155.40, 81.80) --
	(155.49, 82.10) --
	(155.49, 81.90) --
	(155.59, 82.19) --
	(155.68, 82.59) --
	(155.68, 82.69) --
	(155.77, 82.78) --
	(155.95, 82.88) --
	(156.04, 83.08) --
	(156.13, 83.18) --
	(156.22, 83.37) --
	(156.31, 83.57) --
	(156.40, 83.87) --
	(156.40, 83.67) --
	(156.49, 83.96) --
	(156.67, 84.06) --
	(156.84, 84.26) --
	(157.19, 84.36) --
	(157.45, 84.46) --
	(157.53, 84.56) --
	(157.70, 84.95) --
	(157.70, 84.65) --
	(157.79, 85.15) --
	(157.87, 85.24) --
	(157.87, 85.34) --
	(158.04, 85.54) --
	(158.13, 85.64) --
	(158.21, 85.74) --
	(158.29, 85.93) --
	(158.38, 86.03) --
	(158.38, 86.13) --
	(158.46, 86.33) --
	(158.54, 86.43) --
	(158.79, 86.72) --
	(158.87, 86.92) --
	(158.87, 86.82) --
	(158.95, 87.02) --
	(159.03, 87.11) --
	(159.19, 87.21) --
	(159.19, 87.31) --
	(159.27, 87.51) --
	(159.52, 87.61) --
	(159.59, 87.70) --
	(160.15, 87.80) --
	(160.15, 87.90) --
	(160.22, 88.00) --
	(160.22, 88.29) --
	(160.30, 88.49) --
	(160.38, 88.69) --
	(160.38, 88.59) --
	(160.46, 88.89) --
	(160.69, 88.98) --
	(160.69, 89.18) --
	(160.76, 89.28) --
	(160.84, 89.38) --
	(161.29, 89.48) --
	(161.44, 89.57) --
	(161.96, 89.77) --
	(162.17, 89.87) --
	(162.25, 89.97) --
	(162.32, 90.07) --
	(162.39, 90.26) --
	(162.75, 90.36) --
	(162.89, 90.46) --
	(163.10, 90.56) --
	(163.24, 90.66) --
	(163.31, 90.76) --
	(163.38, 90.85) --
	(163.59, 90.95) --
	(163.86, 91.05) --
	(164.06, 91.15) --
	(164.13, 91.25) --
	(164.20, 91.35) --
	(164.27, 91.44) --
	(164.33, 91.54) --
	(164.40, 91.64) --
	(164.73, 91.84) --
	(165.00, 91.94) --
	(165.06, 92.03) --
	(165.19, 92.23) --
	(165.26, 92.33) --
	(165.32, 92.43) --
	(165.39, 92.53) --
	(165.45, 92.63) --
	(165.52, 92.72) --
	(165.77, 92.82) --
	(165.90, 92.92) --
	(165.90, 93.02) --
	(165.96, 93.12) --
	(166.03, 93.31) --
	(166.03, 93.22) --
	(166.53, 93.41) --
	(166.59, 93.61) --
	(166.65, 93.81) --
	(166.65, 93.71) --
	(166.83, 93.90) --
	(166.90, 94.10) --
	(167.08, 94.20) --
	(167.14, 94.30) --
	(167.32, 94.59) --
	(167.32, 94.40) --
	(167.38, 94.79) --
	(167.86, 94.99) --
	(168.03, 95.09) --
	(168.15, 95.18) --
	(168.27, 95.38) --
	(168.27, 95.48) --
	(168.33, 95.58) --
	(168.38, 95.68) --
	(168.50, 95.77) --
	(168.73, 95.87) --
	(168.79, 95.97) --
	(168.79, 96.07) --
	(168.96, 96.17) --
	(169.02, 96.27) --
	(169.47, 96.36) --
	(169.74, 96.46) --
	(169.85, 96.56) --
	(169.91, 96.66) --
	(170.02, 96.86) --
	(170.18, 96.96) --
	(170.51, 97.05) --
	(170.99, 97.15) --
	(171.09, 97.25) --
	(171.20, 97.35) --
	(171.30, 97.45) --
	(171.35, 97.55) --
	(171.41, 97.64) --
	(171.56, 97.74) --
	(171.67, 97.84) --
	(171.72, 97.94) --
	(172.02, 98.04) --
	(172.13, 98.14) --
	(172.23, 98.23) --
	(172.48, 98.43) --
	(172.73, 98.53) --
	(172.73, 98.63) --
	(172.83, 98.73) --
	(173.03, 98.82) --
	(173.22, 98.92) --
	(173.27, 99.02) --
	(173.42, 99.22) --
	(173.46, 99.32) --
	(173.61, 99.42) --
	(174.69, 99.51) --
	(174.88, 99.61) --
	(174.93, 99.71) --
	(174.97, 99.81) --
	(175.02, 99.91) --
	(175.11,100.01) --
	(175.20,100.10) --
	(175.92,100.20) --
	(176.14,100.30) --
	(176.23,100.40) --
	(176.27,100.50) --
	(176.36,100.60) --
	(176.53,100.69) --
	(176.70,100.79) --
	(176.88,100.89) --
	(177.05,100.99) --
	(177.09,101.09) --
	(177.80,101.19) --
	(178.17,101.29) --
	(178.46,101.38) --
	(179.22,101.48) --
	(179.33,101.58) --
	(179.53,101.68) --
	(179.72,101.78) --
	(179.84,101.88) --
	(180.97,101.97) --
	(181.05,102.07) --
	(181.12,102.17) --
	(181.52,102.27) --
	(181.85,102.37) --
	(182.31,102.47) --
	(182.66,102.56) --
	(182.80,102.66) --
	(182.90,102.76) --
	(182.94,102.86) --
	(183.72,102.96) --
	(183.99,103.06) --
	(184.67,103.15) --
	(184.77,103.25) --
	(185.12,103.35) --
	(185.22,103.45) --
	(185.76,103.55) --
	(185.82,103.65) --
	(185.85,103.75) --
	(185.88,103.84) --
	(186.00,103.94) --
	(186.03,104.04) --
	(186.10,104.14) --
	(186.49,104.24) --
	(187.03,104.34) --
	(187.18,104.43) --
	(187.27,104.53) --
	(187.36,104.63) --
	(187.39,104.83) --
	(187.42,104.93) --
	(187.80,105.02) --
	(187.89,105.12) --
	(187.89,105.22) --
	(188.17,105.32) --
	(188.32,105.42) --
	(188.57,105.52) --
	(188.68,105.62) --
	(188.74,105.71) --
	(188.82,105.81) --
	(188.85,105.91) --
	(188.96,106.01) --
	(189.38,106.21) --
	(189.49,106.30) --
	(190.27,106.40) --
	(190.79,106.50) --
	(190.90,106.60) --
	(191.00,106.70) --
	(191.10,106.80) --
	(191.15,106.89) --
	(191.21,106.99) --
	(191.38,107.09) --
	(191.76,107.19) --
	(191.79,107.29) --
	(192.66,107.39) --
	(192.73,107.48) --
	(192.95,107.58) --
	(192.97,107.68) --
	(192.99,107.78) --
	(193.16,107.88) --
	(193.40,108.08) --
	(193.40,107.98) --
	(193.52,108.17) --
	(194.45,108.27) --
	(194.70,108.37) --
	(195.06,108.47) --
	(195.35,108.57) --
	(195.46,108.67) --
	(196.60,108.76) --
	(196.90,108.86) --
	(197.17,108.96) --
	(197.70,109.06) --
	(197.80,109.16) --
	(197.88,109.26) --
	(199.01,109.35) --
	(199.32,109.45) --
	(199.47,109.55) --
	(199.65,109.65) --
	(200.34,109.75) --
	(201.34,109.85) --
	(201.45,109.95) --
	(201.68,110.04) --
	(204.11,110.14) --
	(204.12,110.24) --
	(204.57,110.34) --
	(204.83,110.44) --
	(205.98,110.54) --
	(206.85,110.63) --
	(206.95,110.73) --
	(207.77,110.83) --
	(208.07,110.93) --
	(208.12,111.03) --
	(208.72,111.13) --
	(209.02,111.22) --
	(209.54,111.32) --
	(210.58,111.42) --
	(210.82,111.52) --
	(212.15,111.62) --
	(212.59,111.72) --
	(212.79,111.81) --
	(212.83,111.91) --
	(213.36,112.01) --
	(213.76,112.11) --
	(213.87,112.21) --
	(214.36,112.31) --
	(215.42,112.41) --
	(215.75,112.50) --
	(216.02,112.60) --
	(216.30,112.70) --
	(216.49,112.80) --
	(216.76,112.90) --
	(216.83,113.00) --
	(216.95,113.09) --
	(217.33,113.19) --
	(217.49,113.29) --
	(217.50,113.39) --
	(217.84,113.49) --
	(217.96,113.59) --
	(218.04,113.68) --
	(218.25,113.78) --
	(218.50,113.88) --
	(218.73,113.98) --
	(218.77,114.08) --
	(219.06,114.18) --
	(219.80,114.28) --
	(220.49,114.37) --
	(220.84,114.47) --
	(221.12,114.57) --
	(221.23,114.67) --
	(221.41,114.77) --
	(221.90,114.87) --
	(221.90,114.96) --
	(222.47,115.06) --
	(222.65,115.16) --
	(223.44,115.26) --
	(223.85,115.36) --
	(224.40,115.46) --
	(224.98,115.55) --
	(225.57,115.65) --
	(226.26,115.75) --
	(226.97,115.85) --
	(228.33,115.95) --
	(229.31,116.05) --
	(229.85,116.15) --
	(230.68,116.24) --
	(230.80,116.34) --
	(230.84,116.44) --
	(231.90,116.54) --
	(231.92,116.64) --
	(233.14,116.74) --
	(234.34,116.83) --
	(235.27,116.93) --
	(235.46,117.03) --
	(235.51,117.13) --
	(235.99,117.23) --
	(236.00,117.33) --
	(236.70,117.42) --
	(239.34,117.52) --
	(239.54,117.62) --
	(240.43,117.72) --
	(240.51,117.82) --
	(240.52,117.92) --
	(241.13,118.01) --
	(241.56,118.11) --
	(241.67,118.21) --
	(242.57,118.31) --
	(243.49,118.41) --
	(243.82,118.51) --
	(247.08,118.61) --
	(247.82,118.70) --
	(248.00,118.80) --
	(248.21,118.90) --
	(248.59,119.00) --
	(249.45,119.10) --
	(250.33,119.20) --
	(250.45,119.29) --
	(251.93,119.39) --
	(252.29,119.49) --
	(252.74,119.59) --
	(252.80,119.69) --
	(252.91,119.79) --
	(253.32,119.88) --
	(253.48,119.98) --
	(253.95,120.08) --
	(254.41,120.18) --
	(254.55,120.28) --
	(254.80,120.38) --
	(254.89,120.48) --
	(255.07,120.57) --
	(255.48,120.67) --
	(255.83,120.77) --
	(255.89,120.87) --
	(256.00,120.97) --
	(256.10,121.07) --
	(256.48,121.16) --
	(256.80,121.26) --
	(256.85,121.36) --
	(256.96,121.46) --
	(256.98,121.56) --
	(257.87,121.66) --
	(258.37,121.75) --
	(258.57,121.85) --
	(259.50,121.95) --
	(261.29,122.05) --
	(262.35,122.15) --
	(263.60,122.25) --
	(263.86,122.34) --
	(264.36,122.44) --
	(265.20,122.54) --
	(265.52,122.64) --
	(266.05,122.74) --
	(266.57,122.84) --
	(267.57,122.94) --
	(268.14,123.03) --
	(268.28,123.13) --
	(268.33,123.23) --
	(268.38,123.33) --
	(269.35,123.43) --
	(270.07,123.53) --
	(270.60,123.62) --
	(271.22,123.72) --
	(271.73,123.82) --
	(271.75,123.92) --
	(272.39,124.02) --
	(272.83,124.12) --
	(273.29,124.21) --
	(274.81,124.31) --
	(276.62,124.41) --
	(276.76,124.51) --
	(277.19,124.61) --
	(277.53,124.71) --
	(277.96,124.81) --
	(278.33,124.90) --
	(278.75,125.00) --
	(278.84,125.10) --
	(279.17,125.20) --
	(279.51,125.30) --
	(279.54,125.40) --
	(279.62,125.49) --
	(279.74,125.59) --
	(279.82,125.69) --
	(280.45,125.79) --
	(282.84,125.89) --
	(283.29,125.99) --
	(283.95,126.08) --
	(285.04,126.18) --
	(285.21,126.28) --
	(286.82,126.38) --
	(288.48,126.48) --
	(288.61,126.58) --
	(289.29,126.67) --
	(290.45,126.77) --
	(291.11,126.87) --
	(291.67,126.97) --
	(293.90,127.07) --
	(295.51,127.17) --
	(295.77,127.27) --
	(297.52,127.36) --
	(297.56,127.46) --
	(298.39,127.56) --
	(301.78,127.66) --
	(304.22,127.76) --
	(305.84,127.86) --
	(309.93,127.95) --
	(311.39,128.05) --
	(312.72,128.15) --
	(313.44,128.25) --
	(313.88,128.35) --
	(315.95,128.45);
\definecolor{drawColor}{RGB}{152,152,152}

\path[draw=drawColor,line width= 0.6pt,line join=round,line cap=round] ( 40.62, 30.69) -- ( 40.62, 36.38);

\path[draw=drawColor,line width= 0.6pt,line join=round,line cap=round] ( 45.72, 30.69) -- ( 45.72, 33.53);

\path[draw=drawColor,line width= 0.6pt,line join=round,line cap=round] ( 50.03, 30.69) -- ( 50.03, 33.53);

\path[draw=drawColor,line width= 0.6pt,line join=round,line cap=round] ( 53.76, 30.69) -- ( 53.76, 33.53);

\path[draw=drawColor,line width= 0.6pt,line join=round,line cap=round] ( 57.06, 30.69) -- ( 57.06, 33.53);

\path[draw=drawColor,line width= 0.6pt,line join=round,line cap=round] ( 60.00, 30.69) -- ( 60.00, 39.22);

\path[draw=drawColor,line width= 0.6pt,line join=round,line cap=round] ( 79.39, 30.69) -- ( 79.39, 33.53);

\path[draw=drawColor,line width= 0.6pt,line join=round,line cap=round] ( 90.73, 30.69) -- ( 90.73, 33.53);

\path[draw=drawColor,line width= 0.6pt,line join=round,line cap=round] ( 98.77, 30.69) -- ( 98.77, 33.53);

\path[draw=drawColor,line width= 0.6pt,line join=round,line cap=round] (105.01, 30.69) -- (105.01, 36.38);

\path[draw=drawColor,line width= 0.6pt,line join=round,line cap=round] (110.11, 30.69) -- (110.11, 33.53);

\path[draw=drawColor,line width= 0.6pt,line join=round,line cap=round] (114.42, 30.69) -- (114.42, 33.53);

\path[draw=drawColor,line width= 0.6pt,line join=round,line cap=round] (118.16, 30.69) -- (118.16, 33.53);

\path[draw=drawColor,line width= 0.6pt,line join=round,line cap=round] (121.45, 30.69) -- (121.45, 33.53);

\path[draw=drawColor,line width= 0.6pt,line join=round,line cap=round] (124.40, 30.69) -- (124.40, 39.22);

\path[draw=drawColor,line width= 0.6pt,line join=round,line cap=round] (143.78, 30.69) -- (143.78, 33.53);

\path[draw=drawColor,line width= 0.6pt,line join=round,line cap=round] (155.12, 30.69) -- (155.12, 33.53);

\path[draw=drawColor,line width= 0.6pt,line join=round,line cap=round] (163.17, 30.69) -- (163.17, 33.53);

\path[draw=drawColor,line width= 0.6pt,line join=round,line cap=round] (169.41, 30.69) -- (169.41, 36.38);

\path[draw=drawColor,line width= 0.6pt,line join=round,line cap=round] (174.51, 30.69) -- (174.51, 33.53);

\path[draw=drawColor,line width= 0.6pt,line join=round,line cap=round] (178.82, 30.69) -- (178.82, 33.53);

\path[draw=drawColor,line width= 0.6pt,line join=round,line cap=round] (182.55, 30.69) -- (182.55, 33.53);

\path[draw=drawColor,line width= 0.6pt,line join=round,line cap=round] (185.85, 30.69) -- (185.85, 33.53);

\path[draw=drawColor,line width= 0.6pt,line join=round,line cap=round] (188.80, 30.69) -- (188.80, 39.22);

\path[draw=drawColor,line width= 0.6pt,line join=round,line cap=round] (208.18, 30.69) -- (208.18, 33.53);

\path[draw=drawColor,line width= 0.6pt,line join=round,line cap=round] (219.52, 30.69) -- (219.52, 33.53);

\path[draw=drawColor,line width= 0.6pt,line join=round,line cap=round] (227.56, 30.69) -- (227.56, 33.53);

\path[draw=drawColor,line width= 0.6pt,line join=round,line cap=round] (233.81, 30.69) -- (233.81, 36.38);

\path[draw=drawColor,line width= 0.6pt,line join=round,line cap=round] (238.90, 30.69) -- (238.90, 33.53);

\path[draw=drawColor,line width= 0.6pt,line join=round,line cap=round] (243.22, 30.69) -- (243.22, 33.53);

\path[draw=drawColor,line width= 0.6pt,line join=round,line cap=round] (246.95, 30.69) -- (246.95, 33.53);

\path[draw=drawColor,line width= 0.6pt,line join=round,line cap=round] (250.24, 30.69) -- (250.24, 33.53);

\path[draw=drawColor,line width= 0.6pt,line join=round,line cap=round] (253.19, 30.69) -- (253.19, 39.22);

\path[draw=drawColor,line width= 0.6pt,line join=round,line cap=round] (272.58, 30.69) -- (272.58, 33.53);

\path[draw=drawColor,line width= 0.6pt,line join=round,line cap=round] (283.91, 30.69) -- (283.91, 33.53);

\path[draw=drawColor,line width= 0.6pt,line join=round,line cap=round] (291.96, 30.69) -- (291.96, 33.53);

\path[draw=drawColor,line width= 0.6pt,line join=round,line cap=round] (298.20, 30.69) -- (298.20, 36.38);

\path[draw=drawColor,line width= 0.6pt,line join=round,line cap=round] (303.30, 30.69) -- (303.30, 33.53);

\path[draw=drawColor,line width= 0.6pt,line join=round,line cap=round] (307.61, 30.69) -- (307.61, 33.53);

\path[draw=drawColor,line width= 0.6pt,line join=round,line cap=round] (311.34, 30.69) -- (311.34, 33.53);

\path[draw=drawColor,line width= 0.6pt,line join=round,line cap=round] (314.64, 30.69) -- (314.64, 33.53);

\path[draw=drawColor,line width= 0.6pt,line join=round,line cap=round] (317.59, 30.69) -- (317.59, 39.22);
\definecolor{drawColor}{gray}{0.70}

\path[draw=drawColor,line width= 0.6pt,line join=round,line cap=round] ( 40.51, 30.69) rectangle (332.10,167.95);
\end{scope}
\begin{scope}
\path[clip] (  0.00,  0.00) rectangle (469.75,173.45);
\definecolor{drawColor}{gray}{0.30}

\node[text=drawColor,anchor=base east,inner sep=0pt, outer sep=0pt, scale=  0.88] at ( 35.56, 33.80) {0};

\node[text=drawColor,anchor=base east,inner sep=0pt, outer sep=0pt, scale=  0.88] at ( 35.56, 83.00) {500};

\node[text=drawColor,anchor=base east,inner sep=0pt, outer sep=0pt, scale=  0.88] at ( 35.56,132.21) {1000};
\end{scope}
\begin{scope}
\path[clip] (  0.00,  0.00) rectangle (469.75,173.45);
\definecolor{drawColor}{gray}{0.70}

\path[draw=drawColor,line width= 0.3pt,line join=round] ( 37.76, 36.83) --
	( 40.51, 36.83);

\path[draw=drawColor,line width= 0.3pt,line join=round] ( 37.76, 86.03) --
	( 40.51, 86.03);

\path[draw=drawColor,line width= 0.3pt,line join=round] ( 37.76,135.24) --
	( 40.51,135.24);
\end{scope}
\begin{scope}
\path[clip] (  0.00,  0.00) rectangle (469.75,173.45);
\definecolor{drawColor}{gray}{0.70}

\path[draw=drawColor,line width= 0.3pt,line join=round] ( 60.00, 27.94) --
	( 60.00, 30.69);

\path[draw=drawColor,line width= 0.3pt,line join=round] (124.40, 27.94) --
	(124.40, 30.69);

\path[draw=drawColor,line width= 0.3pt,line join=round] (188.80, 27.94) --
	(188.80, 30.69);

\path[draw=drawColor,line width= 0.3pt,line join=round] (253.19, 27.94) --
	(253.19, 30.69);

\path[draw=drawColor,line width= 0.3pt,line join=round] (317.59, 27.94) --
	(317.59, 30.69);
\end{scope}
\begin{scope}
\path[clip] (  0.00,  0.00) rectangle (469.75,173.45);
\definecolor{drawColor}{gray}{0.30}

\node[text=drawColor,anchor=base,inner sep=0pt, outer sep=0pt, scale=  0.88] at ( 60.00, 19.68) {0.1};

\node[text=drawColor,anchor=base,inner sep=0pt, outer sep=0pt, scale=  0.88] at (124.40, 19.68) {1};

\node[text=drawColor,anchor=base,inner sep=0pt, outer sep=0pt, scale=  0.88] at (188.80, 19.68) {10};

\node[text=drawColor,anchor=base,inner sep=0pt, outer sep=0pt, scale=  0.88] at (253.19, 19.68) {100};

\node[text=drawColor,anchor=base,inner sep=0pt, outer sep=0pt, scale=  0.88] at (317.59, 19.68) {1000};
\end{scope}
\begin{scope}
\path[clip] (  0.00,  0.00) rectangle (469.75,173.45);
\definecolor{drawColor}{RGB}{0,0,0}

\node[text=drawColor,anchor=base,inner sep=0pt, outer sep=0pt, scale=  1.10] at (186.30,  7.64) {Time (s)};
\end{scope}
\begin{scope}
\path[clip] (  0.00,  0.00) rectangle (469.75,173.45);
\definecolor{drawColor}{RGB}{0,0,0}

\node[text=drawColor,rotate= 90.00,anchor=base,inner sep=0pt, outer sep=0pt, scale=  1.10] at ( 13.08, 99.32) {Instances solved};
\end{scope}
\begin{scope}
\path[clip] (  0.00,  0.00) rectangle (469.75,173.45);
\definecolor{fillColor}{RGB}{255,255,255}

\path[fill=fillColor] (343.10, 21.17) rectangle (464.25,177.47);
\end{scope}
\begin{scope}
\path[clip] (  0.00,  0.00) rectangle (469.75,173.45);
\definecolor{drawColor}{RGB}{0,0,0}

\node[text=drawColor,anchor=base west,inner sep=0pt, outer sep=0pt, scale=  1.10] at (348.60,163.32) {Algorithm \& Encoding};
\end{scope}
\begin{scope}
\path[clip] (  0.00,  0.00) rectangle (469.75,173.45);
\definecolor{fillColor}{RGB}{255,255,255}

\path[fill=fillColor] (348.60,142.30) rectangle (363.05,156.75);
\end{scope}
\begin{scope}
\path[clip] (  0.00,  0.00) rectangle (469.75,173.45);
\definecolor{drawColor}{RGB}{253,191,111}

\path[draw=drawColor,line width= 0.6pt,dash pattern=on 4pt off 4pt ,line join=round] (350.04,149.53) -- (361.61,149.53);
\end{scope}
\begin{scope}
\path[clip] (  0.00,  0.00) rectangle (469.75,173.45);
\definecolor{fillColor}{RGB}{255,255,255}

\path[fill=fillColor] (348.60,127.84) rectangle (363.05,142.30);
\end{scope}
\begin{scope}
\path[clip] (  0.00,  0.00) rectangle (469.75,173.45);
\definecolor{drawColor}{RGB}{251,154,153}

\path[draw=drawColor,line width= 0.6pt,dash pattern=on 4pt off 4pt ,line join=round] (350.04,135.07) -- (361.61,135.07);
\end{scope}
\begin{scope}
\path[clip] (  0.00,  0.00) rectangle (469.75,173.45);
\definecolor{fillColor}{RGB}{255,255,255}

\path[fill=fillColor] (348.60,113.39) rectangle (363.05,127.84);
\end{scope}
\begin{scope}
\path[clip] (  0.00,  0.00) rectangle (469.75,173.45);
\definecolor{drawColor}{RGB}{178,223,138}

\path[draw=drawColor,line width= 0.6pt,dash pattern=on 4pt off 4pt ,line join=round] (350.04,120.62) -- (361.61,120.62);
\end{scope}
\begin{scope}
\path[clip] (  0.00,  0.00) rectangle (469.75,173.45);
\definecolor{fillColor}{RGB}{255,255,255}

\path[fill=fillColor] (348.60, 98.94) rectangle (363.05,113.39);
\end{scope}
\begin{scope}
\path[clip] (  0.00,  0.00) rectangle (469.75,173.45);
\definecolor{drawColor}{RGB}{31,120,180}

\path[draw=drawColor,line width= 0.6pt,line join=round] (350.04,106.16) -- (361.61,106.16);
\end{scope}
\begin{scope}
\path[clip] (  0.00,  0.00) rectangle (469.75,173.45);
\definecolor{fillColor}{RGB}{255,255,255}

\path[fill=fillColor] (348.60, 84.48) rectangle (363.05, 98.94);
\end{scope}
\begin{scope}
\path[clip] (  0.00,  0.00) rectangle (469.75,173.45);
\definecolor{drawColor}{RGB}{177,89,40}

\path[draw=drawColor,line width= 0.6pt,line join=round] (350.04, 91.71) -- (361.61, 91.71);
\end{scope}
\begin{scope}
\path[clip] (  0.00,  0.00) rectangle (469.75,173.45);
\definecolor{fillColor}{RGB}{255,255,255}

\path[fill=fillColor] (348.60, 70.03) rectangle (363.05, 84.48);
\end{scope}
\begin{scope}
\path[clip] (  0.00,  0.00) rectangle (469.75,173.45);
\definecolor{drawColor}{RGB}{51,160,44}

\path[draw=drawColor,line width= 0.6pt,line join=round] (350.04, 77.26) -- (361.61, 77.26);
\end{scope}
\begin{scope}
\path[clip] (  0.00,  0.00) rectangle (469.75,173.45);
\definecolor{fillColor}{RGB}{255,255,255}

\path[fill=fillColor] (348.60, 55.57) rectangle (363.05, 70.03);
\end{scope}
\begin{scope}
\path[clip] (  0.00,  0.00) rectangle (469.75,173.45);
\definecolor{drawColor}{RGB}{106,61,154}

\path[draw=drawColor,line width= 0.6pt,line join=round] (350.04, 62.80) -- (361.61, 62.80);
\end{scope}
\begin{scope}
\path[clip] (  0.00,  0.00) rectangle (469.75,173.45);
\definecolor{fillColor}{RGB}{255,255,255}

\path[fill=fillColor] (348.60, 41.12) rectangle (363.05, 55.57);
\end{scope}
\begin{scope}
\path[clip] (  0.00,  0.00) rectangle (469.75,173.45);
\definecolor{drawColor}{RGB}{202,178,214}

\path[draw=drawColor,line width= 0.6pt,dash pattern=on 4pt off 4pt ,line join=round] (350.04, 48.35) -- (361.61, 48.35);
\end{scope}
\begin{scope}
\path[clip] (  0.00,  0.00) rectangle (469.75,173.45);
\definecolor{fillColor}{RGB}{255,255,255}

\path[fill=fillColor] (348.60, 26.67) rectangle (363.05, 41.12);
\end{scope}
\begin{scope}
\path[clip] (  0.00,  0.00) rectangle (469.75,173.45);
\definecolor{drawColor}{RGB}{166,206,227}

\path[draw=drawColor,line width= 0.6pt,dash pattern=on 4pt off 4pt ,line join=round] (350.04, 33.89) -- (361.61, 33.89);
\end{scope}
\begin{scope}
\path[clip] (  0.00,  0.00) rectangle (469.75,173.45);
\definecolor{drawColor}{RGB}{0,0,0}

\node[text=drawColor,anchor=base west,inner sep=0pt, outer sep=0pt, scale=  0.88] at (368.55,146.50) {\textsf{Ace} + \texttt{cd05}};
\end{scope}
\begin{scope}
\path[clip] (  0.00,  0.00) rectangle (469.75,173.45);
\definecolor{drawColor}{RGB}{0,0,0}

\node[text=drawColor,anchor=base west,inner sep=0pt, outer sep=0pt, scale=  0.88] at (368.55,132.04) {\textsf{Ace} + \texttt{cd06}};
\end{scope}
\begin{scope}
\path[clip] (  0.00,  0.00) rectangle (469.75,173.45);
\definecolor{drawColor}{RGB}{0,0,0}

\node[text=drawColor,anchor=base west,inner sep=0pt, outer sep=0pt, scale=  0.88] at (368.55,117.59) {\textsf{Ace} + \texttt{d02}};
\end{scope}
\begin{scope}
\path[clip] (  0.00,  0.00) rectangle (469.75,173.45);
\definecolor{drawColor}{RGB}{0,0,0}

\node[text=drawColor,anchor=base west,inner sep=0pt, outer sep=0pt, scale=  0.88] at (368.55,103.13) {\textsf{ADDMC} + \texttt{bklm16}};
\end{scope}
\begin{scope}
\path[clip] (  0.00,  0.00) rectangle (469.75,173.45);
\definecolor{drawColor}{RGB}{0,0,0}

\node[text=drawColor,anchor=base west,inner sep=0pt, outer sep=0pt, scale=  0.88] at (368.55, 88.68) {\textsf{ADDMC} + \texttt{cw}};
\end{scope}
\begin{scope}
\path[clip] (  0.00,  0.00) rectangle (469.75,173.45);
\definecolor{drawColor}{RGB}{0,0,0}

\node[text=drawColor,anchor=base west,inner sep=0pt, outer sep=0pt, scale=  0.88] at (368.55, 74.23) {\textsf{ADDMC} + \texttt{d02}};
\end{scope}
\begin{scope}
\path[clip] (  0.00,  0.00) rectangle (469.75,173.45);
\definecolor{drawColor}{RGB}{0,0,0}

\node[text=drawColor,anchor=base west,inner sep=0pt, outer sep=0pt, scale=  0.88] at (368.55, 59.77) {\textsf{ADDMC} + \texttt{sbk05}};
\end{scope}
\begin{scope}
\path[clip] (  0.00,  0.00) rectangle (469.75,173.45);
\definecolor{drawColor}{RGB}{0,0,0}

\node[text=drawColor,anchor=base west,inner sep=0pt, outer sep=0pt, scale=  0.88] at (368.55, 45.32) {\textsf{c2d} + \texttt{bklm16}};
\end{scope}
\begin{scope}
\path[clip] (  0.00,  0.00) rectangle (469.75,173.45);
\definecolor{drawColor}{RGB}{0,0,0}

\node[text=drawColor,anchor=base west,inner sep=0pt, outer sep=0pt, scale=  0.88] at (368.55, 30.86) {\textsf{Cachet} + \texttt{sbk05}};
\end{scope}
\end{tikzpicture}
%
  \caption{Cumulative numbers of instances solved by combinations of algorithms
    and encodings over time.}\label{fig:cumulative}
\end{figure}

\begin{table}[t]
  \centering
  \begin{tabular}{lrrr}
    \toprule
    Algorithm \& Encoding & Unique & Fastest & Total \\
    \midrule
    $\textsf{Ace} + \texttt{cd05}$ & \textcolor{gray}{0} & 55 & 1169 \\
    $\textsf{Ace} + \texttt{cd06}$ & \textbf{34} & 218 & \textbf{1259} \\
    $\textsf{Ace} + \texttt{d02}$ & \textcolor{gray}{0} & 46 & 993 \\
    $\textsf{ADDMC} + \texttt{bklm16}$ & \textcolor{gray}{0} & 29 & \textcolor{gray}{617} \\
    $\textsf{ADDMC} + \texttt{cw}$ & 14 & \textbf{770} & 919 \\
    $\textsf{ADDMC} + \texttt{d02}$ & \textcolor{gray}{0} & \textcolor{gray}{0} & 703 \\
    $\textsf{ADDMC} + \texttt{sbk05}$ & \textcolor{gray}{0} & \textcolor{gray}{0} & 729 \\
    $\textsf{c2d} + \texttt{bklm16}$ & \textcolor{gray}{0} & 3 & 1017 \\
    $\textsf{Cachet} + \texttt{sbk05}$ & 13 & 229 & 928 \\
    \bottomrule
  \end{tabular}
  \caption{The numbers of instances (out of 1466) solved by each combination of
    algorithm and encoding (uniquely, faster than others, and in
    total).}\label{tbl:tallies}
\end{table}

\begin{figure}[t]
  \centering
  % Created by tikzDevice version 0.12.3.1 on 2022-06-27 11:24:42
% !TEX encoding = UTF-8 Unicode
\documentclass[10pt]{article}
\usepackage{tikz}

\usepackage[active,tightpage,psfixbb]{preview}

\PreviewEnvironment{pgfpicture}

\setlength\PreviewBorder{0pt}
\begin{document}

\begin{tikzpicture}[x=1pt,y=1pt]
\definecolor{fillColor}{RGB}{255,255,255}
\path[use as bounding box,fill=fillColor,fill opacity=0.00] (0,0) rectangle (411.94,224.04);
\begin{scope}
\path[clip] (  5.97, 37.91) rectangle (200.00,224.04);
\definecolor{drawColor}{RGB}{255,255,255}
\definecolor{fillColor}{RGB}{255,255,255}

\path[draw=drawColor,line width= 0.5pt,line join=round,line cap=round,fill=fillColor] (  5.97, 37.91) rectangle (200.00,224.04);
\end{scope}
\begin{scope}
\path[clip] ( 39.36, 63.40) rectangle (195.50,219.54);
\definecolor{fillColor}{RGB}{255,255,255}

\path[fill=fillColor] ( 39.36, 63.40) rectangle (195.50,219.54);
\definecolor{drawColor}{gray}{0.87}

\path[draw=drawColor,line width= 0.1pt,line join=round] ( 39.36, 70.50) --
	(195.50, 70.50);

\path[draw=drawColor,line width= 0.1pt,line join=round] ( 39.36,124.05) --
	(195.50,124.05);

\path[draw=drawColor,line width= 0.1pt,line join=round] ( 39.36,177.60) --
	(195.50,177.60);

\path[draw=drawColor,line width= 0.1pt,line join=round] ( 46.45, 63.40) --
	( 46.45,219.54);

\path[draw=drawColor,line width= 0.1pt,line join=round] (100.01, 63.40) --
	(100.01,219.54);

\path[draw=drawColor,line width= 0.1pt,line join=round] (153.56, 63.40) --
	(153.56,219.54);

\path[draw=drawColor,line width= 0.2pt,line join=round] ( 39.36, 97.27) --
	(195.50, 97.27);

\path[draw=drawColor,line width= 0.2pt,line join=round] ( 39.36,150.83) --
	(195.50,150.83);

\path[draw=drawColor,line width= 0.2pt,line join=round] ( 39.36,204.38) --
	(195.50,204.38);

\path[draw=drawColor,line width= 0.2pt,line join=round] ( 73.23, 63.40) --
	( 73.23,219.54);

\path[draw=drawColor,line width= 0.2pt,line join=round] (126.78, 63.40) --
	(126.78,219.54);

\path[draw=drawColor,line width= 0.2pt,line join=round] (180.34, 63.40) --
	(180.34,219.54);
\definecolor{drawColor}{RGB}{230,171,2}

\path[draw=drawColor,draw opacity=0.50,line width= 0.4pt,line join=round,line cap=round] (113.61,108.23) -- (116.46,111.08);

\path[draw=drawColor,draw opacity=0.50,line width= 0.4pt,line join=round,line cap=round] (113.61,111.08) -- (116.46,108.23);

\path[draw=drawColor,draw opacity=0.50,line width= 0.4pt,line join=round,line cap=round] (113.01,109.65) -- (117.05,109.65);

\path[draw=drawColor,draw opacity=0.50,line width= 0.4pt,line join=round,line cap=round] (115.03,107.64) -- (115.03,111.67);

\path[draw=drawColor,draw opacity=0.50,line width= 0.4pt,line join=round,line cap=round] (114.96,108.23) -- (117.81,111.08);

\path[draw=drawColor,draw opacity=0.50,line width= 0.4pt,line join=round,line cap=round] (114.96,111.08) -- (117.81,108.23);

\path[draw=drawColor,draw opacity=0.50,line width= 0.4pt,line join=round,line cap=round] (114.37,109.65) -- (118.40,109.65);

\path[draw=drawColor,draw opacity=0.50,line width= 0.4pt,line join=round,line cap=round] (116.39,107.64) -- (116.39,111.67);

\path[draw=drawColor,draw opacity=0.50,line width= 0.4pt,line join=round,line cap=round] (113.15,110.08) -- (116.00,112.93);

\path[draw=drawColor,draw opacity=0.50,line width= 0.4pt,line join=round,line cap=round] (113.15,112.93) -- (116.00,110.08);

\path[draw=drawColor,draw opacity=0.50,line width= 0.4pt,line join=round,line cap=round] (112.56,111.50) -- (116.59,111.50);

\path[draw=drawColor,draw opacity=0.50,line width= 0.4pt,line join=round,line cap=round] (114.58,109.49) -- (114.58,113.52);

\path[draw=drawColor,draw opacity=0.50,line width= 0.4pt,line join=round,line cap=round] (115.97,110.74) -- (118.82,113.59);

\path[draw=drawColor,draw opacity=0.50,line width= 0.4pt,line join=round,line cap=round] (115.97,113.59) -- (118.82,110.74);

\path[draw=drawColor,draw opacity=0.50,line width= 0.4pt,line join=round,line cap=round] (115.38,112.17) -- (119.41,112.17);

\path[draw=drawColor,draw opacity=0.50,line width= 0.4pt,line join=round,line cap=round] (117.39,110.15) -- (117.39,114.19);

\path[draw=drawColor,draw opacity=0.50,line width= 0.4pt,line join=round,line cap=round] (115.10,109.73) -- (117.96,112.58);

\path[draw=drawColor,draw opacity=0.50,line width= 0.4pt,line join=round,line cap=round] (115.10,112.58) -- (117.96,109.73);

\path[draw=drawColor,draw opacity=0.50,line width= 0.4pt,line join=round,line cap=round] (114.51,111.16) -- (118.55,111.16);

\path[draw=drawColor,draw opacity=0.50,line width= 0.4pt,line join=round,line cap=round] (116.53,109.14) -- (116.53,113.17);

\path[draw=drawColor,draw opacity=0.50,line width= 0.4pt,line join=round,line cap=round] (113.73,108.23) -- (116.59,111.08);

\path[draw=drawColor,draw opacity=0.50,line width= 0.4pt,line join=round,line cap=round] (113.73,111.08) -- (116.59,108.23);

\path[draw=drawColor,draw opacity=0.50,line width= 0.4pt,line join=round,line cap=round] (113.14,109.65) -- (117.18,109.65);

\path[draw=drawColor,draw opacity=0.50,line width= 0.4pt,line join=round,line cap=round] (115.16,107.64) -- (115.16,111.67);

\path[draw=drawColor,draw opacity=0.50,line width= 0.4pt,line join=round,line cap=round] (109.78,108.62) -- (112.63,111.47);

\path[draw=drawColor,draw opacity=0.50,line width= 0.4pt,line join=round,line cap=round] (109.78,111.47) -- (112.63,108.62);

\path[draw=drawColor,draw opacity=0.50,line width= 0.4pt,line join=round,line cap=round] (109.19,110.05) -- (113.23,110.05);

\path[draw=drawColor,draw opacity=0.50,line width= 0.4pt,line join=round,line cap=round] (111.21,108.03) -- (111.21,112.07);

\path[draw=drawColor,draw opacity=0.50,line width= 0.4pt,line join=round,line cap=round] (114.23,110.41) -- (117.08,113.27);

\path[draw=drawColor,draw opacity=0.50,line width= 0.4pt,line join=round,line cap=round] (114.23,113.27) -- (117.08,110.41);

\path[draw=drawColor,draw opacity=0.50,line width= 0.4pt,line join=round,line cap=round] (113.64,111.84) -- (117.67,111.84);

\path[draw=drawColor,draw opacity=0.50,line width= 0.4pt,line join=round,line cap=round] (115.65,109.82) -- (115.65,113.86);

\path[draw=drawColor,draw opacity=0.50,line width= 0.4pt,line join=round,line cap=round] (113.18,111.37) -- (116.04,114.22);

\path[draw=drawColor,draw opacity=0.50,line width= 0.4pt,line join=round,line cap=round] (113.18,114.22) -- (116.04,111.37);

\path[draw=drawColor,draw opacity=0.50,line width= 0.4pt,line join=round,line cap=round] (112.59,112.80) -- (116.63,112.80);

\path[draw=drawColor,draw opacity=0.50,line width= 0.4pt,line join=round,line cap=round] (114.61,110.78) -- (114.61,114.81);

\path[draw=drawColor,draw opacity=0.50,line width= 0.4pt,line join=round,line cap=round] (112.14,107.82) -- (115.00,110.67);

\path[draw=drawColor,draw opacity=0.50,line width= 0.4pt,line join=round,line cap=round] (112.14,110.67) -- (115.00,107.82);

\path[draw=drawColor,draw opacity=0.50,line width= 0.4pt,line join=round,line cap=round] (111.55,109.25) -- (115.59,109.25);

\path[draw=drawColor,draw opacity=0.50,line width= 0.4pt,line join=round,line cap=round] (113.57,107.23) -- (113.57,111.26);

\path[draw=drawColor,draw opacity=0.50,line width= 0.4pt,line join=round,line cap=round] (110.64,109.00) -- (113.49,111.86);

\path[draw=drawColor,draw opacity=0.50,line width= 0.4pt,line join=round,line cap=round] (110.64,111.86) -- (113.49,109.00);

\path[draw=drawColor,draw opacity=0.50,line width= 0.4pt,line join=round,line cap=round] (110.05,110.43) -- (114.08,110.43);

\path[draw=drawColor,draw opacity=0.50,line width= 0.4pt,line join=round,line cap=round] (112.06,108.41) -- (112.06,112.45);

\path[draw=drawColor,draw opacity=0.50,line width= 0.4pt,line join=round,line cap=round] (115.21,108.62) -- (118.07,111.47);

\path[draw=drawColor,draw opacity=0.50,line width= 0.4pt,line join=round,line cap=round] (115.21,111.47) -- (118.07,108.62);

\path[draw=drawColor,draw opacity=0.50,line width= 0.4pt,line join=round,line cap=round] (114.62,110.05) -- (118.66,110.05);

\path[draw=drawColor,draw opacity=0.50,line width= 0.4pt,line join=round,line cap=round] (116.64,108.03) -- (116.64,112.07);

\path[draw=drawColor,draw opacity=0.50,line width= 0.4pt,line join=round,line cap=round] (113.61,109.37) -- (116.46,112.23);

\path[draw=drawColor,draw opacity=0.50,line width= 0.4pt,line join=round,line cap=round] (113.61,112.23) -- (116.46,109.37);

\path[draw=drawColor,draw opacity=0.50,line width= 0.4pt,line join=round,line cap=round] (113.01,110.80) -- (117.05,110.80);

\path[draw=drawColor,draw opacity=0.50,line width= 0.4pt,line join=round,line cap=round] (115.03,108.78) -- (115.03,112.82);

\path[draw=drawColor,draw opacity=0.50,line width= 0.4pt,line join=round,line cap=round] (112.78,108.62) -- (115.63,111.47);

\path[draw=drawColor,draw opacity=0.50,line width= 0.4pt,line join=round,line cap=round] (112.78,111.47) -- (115.63,108.62);

\path[draw=drawColor,draw opacity=0.50,line width= 0.4pt,line join=round,line cap=round] (112.19,110.05) -- (116.22,110.05);

\path[draw=drawColor,draw opacity=0.50,line width= 0.4pt,line join=round,line cap=round] (114.20,108.03) -- (114.20,112.07);

\path[draw=drawColor,draw opacity=0.50,line width= 0.4pt,line join=round,line cap=round] (115.65,110.41) -- (118.50,113.27);

\path[draw=drawColor,draw opacity=0.50,line width= 0.4pt,line join=round,line cap=round] (115.65,113.27) -- (118.50,110.41);

\path[draw=drawColor,draw opacity=0.50,line width= 0.4pt,line join=round,line cap=round] (115.06,111.84) -- (119.09,111.84);

\path[draw=drawColor,draw opacity=0.50,line width= 0.4pt,line join=round,line cap=round] (117.08,109.82) -- (117.08,113.86);

\path[draw=drawColor,draw opacity=0.50,line width= 0.4pt,line join=round,line cap=round] (115.68,109.37) -- (118.53,112.23);

\path[draw=drawColor,draw opacity=0.50,line width= 0.4pt,line join=round,line cap=round] (115.68,112.23) -- (118.53,109.37);

\path[draw=drawColor,draw opacity=0.50,line width= 0.4pt,line join=round,line cap=round] (115.09,110.80) -- (119.12,110.80);

\path[draw=drawColor,draw opacity=0.50,line width= 0.4pt,line join=round,line cap=round] (117.10,108.78) -- (117.10,112.82);

\path[draw=drawColor,draw opacity=0.50,line width= 0.4pt,line join=round,line cap=round] (115.54,113.84) -- (118.40,116.70);

\path[draw=drawColor,draw opacity=0.50,line width= 0.4pt,line join=round,line cap=round] (115.54,116.70) -- (118.40,113.84);

\path[draw=drawColor,draw opacity=0.50,line width= 0.4pt,line join=round,line cap=round] (114.95,115.27) -- (118.99,115.27);

\path[draw=drawColor,draw opacity=0.50,line width= 0.4pt,line join=round,line cap=round] (116.97,113.25) -- (116.97,117.29);

\path[draw=drawColor,draw opacity=0.50,line width= 0.4pt,line join=round,line cap=round] (116.20,112.25) -- (119.05,115.11);

\path[draw=drawColor,draw opacity=0.50,line width= 0.4pt,line join=round,line cap=round] (116.20,115.11) -- (119.05,112.25);

\path[draw=drawColor,draw opacity=0.50,line width= 0.4pt,line join=round,line cap=round] (115.61,113.68) -- (119.64,113.68);

\path[draw=drawColor,draw opacity=0.50,line width= 0.4pt,line join=round,line cap=round] (117.63,111.66) -- (117.63,115.70);

\path[draw=drawColor,draw opacity=0.50,line width= 0.4pt,line join=round,line cap=round] (113.02,111.06) -- (115.87,113.91);

\path[draw=drawColor,draw opacity=0.50,line width= 0.4pt,line join=round,line cap=round] (113.02,113.91) -- (115.87,111.06);

\path[draw=drawColor,draw opacity=0.50,line width= 0.4pt,line join=round,line cap=round] (112.42,112.49) -- (116.46,112.49);

\path[draw=drawColor,draw opacity=0.50,line width= 0.4pt,line join=round,line cap=round] (114.44,110.47) -- (114.44,114.50);

\path[draw=drawColor,draw opacity=0.50,line width= 0.4pt,line join=round,line cap=round] (116.85,111.67) -- (119.70,114.53);

\path[draw=drawColor,draw opacity=0.50,line width= 0.4pt,line join=round,line cap=round] (116.85,114.53) -- (119.70,111.67);

\path[draw=drawColor,draw opacity=0.50,line width= 0.4pt,line join=round,line cap=round] (116.26,113.10) -- (120.29,113.10);

\path[draw=drawColor,draw opacity=0.50,line width= 0.4pt,line join=round,line cap=round] (118.27,111.08) -- (118.27,115.12);

\path[draw=drawColor,draw opacity=0.50,line width= 0.4pt,line join=round,line cap=round] (113.61,113.34) -- (116.46,116.19);

\path[draw=drawColor,draw opacity=0.50,line width= 0.4pt,line join=round,line cap=round] (113.61,116.19) -- (116.46,113.34);

\path[draw=drawColor,draw opacity=0.50,line width= 0.4pt,line join=round,line cap=round] (113.01,114.76) -- (117.05,114.76);

\path[draw=drawColor,draw opacity=0.50,line width= 0.4pt,line join=round,line cap=round] (115.03,112.75) -- (115.03,116.78);

\path[draw=drawColor,draw opacity=0.50,line width= 0.4pt,line join=round,line cap=round] (113.61,110.41) -- (116.46,113.27);

\path[draw=drawColor,draw opacity=0.50,line width= 0.4pt,line join=round,line cap=round] (113.61,113.27) -- (116.46,110.41);

\path[draw=drawColor,draw opacity=0.50,line width= 0.4pt,line join=round,line cap=round] (113.01,111.84) -- (117.05,111.84);

\path[draw=drawColor,draw opacity=0.50,line width= 0.4pt,line join=round,line cap=round] (115.03,109.82) -- (115.03,113.86);

\path[draw=drawColor,draw opacity=0.50,line width= 0.4pt,line join=round,line cap=round] (111.40,125.40) -- (114.25,128.26);

\path[draw=drawColor,draw opacity=0.50,line width= 0.4pt,line join=round,line cap=round] (111.40,128.26) -- (114.25,125.40);

\path[draw=drawColor,draw opacity=0.50,line width= 0.4pt,line join=round,line cap=round] (110.80,126.83) -- (114.84,126.83);

\path[draw=drawColor,draw opacity=0.50,line width= 0.4pt,line join=round,line cap=round] (112.82,124.81) -- (112.82,128.85);

\path[draw=drawColor,draw opacity=0.50,line width= 0.4pt,line join=round,line cap=round] (113.08,110.41) -- (115.94,113.27);

\path[draw=drawColor,draw opacity=0.50,line width= 0.4pt,line join=round,line cap=round] (113.08,113.27) -- (115.94,110.41);

\path[draw=drawColor,draw opacity=0.50,line width= 0.4pt,line join=round,line cap=round] (112.49,111.84) -- (116.53,111.84);

\path[draw=drawColor,draw opacity=0.50,line width= 0.4pt,line join=round,line cap=round] (114.51,109.82) -- (114.51,113.86);

\path[draw=drawColor,draw opacity=0.50,line width= 0.4pt,line join=round,line cap=round] (112.60,110.41) -- (115.46,113.27);

\path[draw=drawColor,draw opacity=0.50,line width= 0.4pt,line join=round,line cap=round] (112.60,113.27) -- (115.46,110.41);

\path[draw=drawColor,draw opacity=0.50,line width= 0.4pt,line join=round,line cap=round] (112.01,111.84) -- (116.05,111.84);

\path[draw=drawColor,draw opacity=0.50,line width= 0.4pt,line join=round,line cap=round] (114.03,109.82) -- (114.03,113.86);

\path[draw=drawColor,draw opacity=0.50,line width= 0.4pt,line join=round,line cap=round] (115.54,112.25) -- (118.40,115.11);

\path[draw=drawColor,draw opacity=0.50,line width= 0.4pt,line join=round,line cap=round] (115.54,115.11) -- (118.40,112.25);

\path[draw=drawColor,draw opacity=0.50,line width= 0.4pt,line join=round,line cap=round] (114.95,113.68) -- (118.99,113.68);

\path[draw=drawColor,draw opacity=0.50,line width= 0.4pt,line join=round,line cap=round] (116.97,111.66) -- (116.97,115.70);

\path[draw=drawColor,draw opacity=0.50,line width= 0.4pt,line join=round,line cap=round] (116.35,114.79) -- (119.21,117.65);

\path[draw=drawColor,draw opacity=0.50,line width= 0.4pt,line join=round,line cap=round] (116.35,117.65) -- (119.21,114.79);

\path[draw=drawColor,draw opacity=0.50,line width= 0.4pt,line join=round,line cap=round] (115.76,116.22) -- (119.80,116.22);

\path[draw=drawColor,draw opacity=0.50,line width= 0.4pt,line join=round,line cap=round] (117.78,114.20) -- (117.78,118.24);

\path[draw=drawColor,draw opacity=0.50,line width= 0.4pt,line join=round,line cap=round] (115.02,111.37) -- (117.87,114.22);

\path[draw=drawColor,draw opacity=0.50,line width= 0.4pt,line join=round,line cap=round] (115.02,114.22) -- (117.87,111.37);

\path[draw=drawColor,draw opacity=0.50,line width= 0.4pt,line join=round,line cap=round] (114.43,112.80) -- (118.46,112.80);

\path[draw=drawColor,draw opacity=0.50,line width= 0.4pt,line join=round,line cap=round] (116.44,110.78) -- (116.44,114.81);

\path[draw=drawColor,draw opacity=0.50,line width= 0.4pt,line join=round,line cap=round] (112.32,109.73) -- (115.18,112.58);

\path[draw=drawColor,draw opacity=0.50,line width= 0.4pt,line join=round,line cap=round] (112.32,112.58) -- (115.18,109.73);

\path[draw=drawColor,draw opacity=0.50,line width= 0.4pt,line join=round,line cap=round] (111.73,111.16) -- (115.77,111.16);

\path[draw=drawColor,draw opacity=0.50,line width= 0.4pt,line join=round,line cap=round] (113.75,109.14) -- (113.75,113.17);

\path[draw=drawColor,draw opacity=0.50,line width= 0.4pt,line join=round,line cap=round] (113.64,111.97) -- (116.49,114.82);

\path[draw=drawColor,draw opacity=0.50,line width= 0.4pt,line join=round,line cap=round] (113.64,114.82) -- (116.49,111.97);

\path[draw=drawColor,draw opacity=0.50,line width= 0.4pt,line join=round,line cap=round] (113.05,113.39) -- (117.08,113.39);

\path[draw=drawColor,draw opacity=0.50,line width= 0.4pt,line join=round,line cap=round] (115.06,111.38) -- (115.06,115.41);

\path[draw=drawColor,draw opacity=0.50,line width= 0.4pt,line join=round,line cap=round] (112.22,109.73) -- (115.07,112.58);

\path[draw=drawColor,draw opacity=0.50,line width= 0.4pt,line join=round,line cap=round] (112.22,112.58) -- (115.07,109.73);

\path[draw=drawColor,draw opacity=0.50,line width= 0.4pt,line join=round,line cap=round] (111.62,111.16) -- (115.66,111.16);

\path[draw=drawColor,draw opacity=0.50,line width= 0.4pt,line join=round,line cap=round] (113.64,109.14) -- (113.64,113.17);

\path[draw=drawColor,draw opacity=0.50,line width= 0.4pt,line join=round,line cap=round] (113.95,114.09) -- (116.80,116.94);

\path[draw=drawColor,draw opacity=0.50,line width= 0.4pt,line join=round,line cap=round] (113.95,116.94) -- (116.80,114.09);

\path[draw=drawColor,draw opacity=0.50,line width= 0.4pt,line join=round,line cap=round] (113.36,115.51) -- (117.40,115.51);

\path[draw=drawColor,draw opacity=0.50,line width= 0.4pt,line join=round,line cap=round] (115.38,113.50) -- (115.38,117.53);

\path[draw=drawColor,draw opacity=0.50,line width= 0.4pt,line join=round,line cap=round] (117.20,123.08) -- (120.06,125.93);

\path[draw=drawColor,draw opacity=0.50,line width= 0.4pt,line join=round,line cap=round] (117.20,125.93) -- (120.06,123.08);

\path[draw=drawColor,draw opacity=0.50,line width= 0.4pt,line join=round,line cap=round] (116.61,124.51) -- (120.65,124.51);

\path[draw=drawColor,draw opacity=0.50,line width= 0.4pt,line join=round,line cap=round] (118.63,122.49) -- (118.63,126.52);

\path[draw=drawColor,draw opacity=0.50,line width= 0.4pt,line join=round,line cap=round] (113.38,113.84) -- (116.23,116.70);

\path[draw=drawColor,draw opacity=0.50,line width= 0.4pt,line join=round,line cap=round] (113.38,116.70) -- (116.23,113.84);

\path[draw=drawColor,draw opacity=0.50,line width= 0.4pt,line join=round,line cap=round] (112.79,115.27) -- (116.82,115.27);

\path[draw=drawColor,draw opacity=0.50,line width= 0.4pt,line join=round,line cap=round] (114.81,113.25) -- (114.81,117.29);

\path[draw=drawColor,draw opacity=0.50,line width= 0.4pt,line join=round,line cap=round] (113.05,112.81) -- (115.90,115.66);

\path[draw=drawColor,draw opacity=0.50,line width= 0.4pt,line join=round,line cap=round] (113.05,115.66) -- (115.90,112.81);

\path[draw=drawColor,draw opacity=0.50,line width= 0.4pt,line join=round,line cap=round] (112.46,114.23) -- (116.49,114.23);

\path[draw=drawColor,draw opacity=0.50,line width= 0.4pt,line join=round,line cap=round] (114.48,112.22) -- (114.48,116.25);

\path[draw=drawColor,draw opacity=0.50,line width= 0.4pt,line join=round,line cap=round] (114.82,113.59) -- (117.67,116.45);

\path[draw=drawColor,draw opacity=0.50,line width= 0.4pt,line join=round,line cap=round] (114.82,116.45) -- (117.67,113.59);

\path[draw=drawColor,draw opacity=0.50,line width= 0.4pt,line join=round,line cap=round] (114.23,115.02) -- (118.26,115.02);

\path[draw=drawColor,draw opacity=0.50,line width= 0.4pt,line join=round,line cap=round] (116.24,113.00) -- (116.24,117.04);

\path[draw=drawColor,draw opacity=0.50,line width= 0.4pt,line join=round,line cap=round] (117.06,115.24) -- (119.92,118.09);

\path[draw=drawColor,draw opacity=0.50,line width= 0.4pt,line join=round,line cap=round] (117.06,118.09) -- (119.92,115.24);

\path[draw=drawColor,draw opacity=0.50,line width= 0.4pt,line join=round,line cap=round] (116.47,116.67) -- (120.51,116.67);

\path[draw=drawColor,draw opacity=0.50,line width= 0.4pt,line join=round,line cap=round] (118.49,114.65) -- (118.49,118.68);

\path[draw=drawColor,draw opacity=0.50,line width= 0.4pt,line join=round,line cap=round] (113.73,111.97) -- (116.59,114.82);

\path[draw=drawColor,draw opacity=0.50,line width= 0.4pt,line join=round,line cap=round] (113.73,114.82) -- (116.59,111.97);

\path[draw=drawColor,draw opacity=0.50,line width= 0.4pt,line join=round,line cap=round] (113.14,113.39) -- (117.18,113.39);

\path[draw=drawColor,draw opacity=0.50,line width= 0.4pt,line join=round,line cap=round] (115.16,111.38) -- (115.16,115.41);

\path[draw=drawColor,draw opacity=0.50,line width= 0.4pt,line join=round,line cap=round] (115.76,119.73) -- (118.61,122.59);

\path[draw=drawColor,draw opacity=0.50,line width= 0.4pt,line join=round,line cap=round] (115.76,122.59) -- (118.61,119.73);

\path[draw=drawColor,draw opacity=0.50,line width= 0.4pt,line join=round,line cap=round] (115.17,121.16) -- (119.20,121.16);

\path[draw=drawColor,draw opacity=0.50,line width= 0.4pt,line join=round,line cap=round] (117.18,119.14) -- (117.18,123.18);

\path[draw=drawColor,draw opacity=0.50,line width= 0.4pt,line join=round,line cap=round] (115.46,116.09) -- (118.31,118.94);

\path[draw=drawColor,draw opacity=0.50,line width= 0.4pt,line join=round,line cap=round] (115.46,118.94) -- (118.31,116.09);

\path[draw=drawColor,draw opacity=0.50,line width= 0.4pt,line join=round,line cap=round] (114.87,117.51) -- (118.91,117.51);

\path[draw=drawColor,draw opacity=0.50,line width= 0.4pt,line join=round,line cap=round] (116.89,115.49) -- (116.89,119.53);

\path[draw=drawColor,draw opacity=0.50,line width= 0.4pt,line join=round,line cap=round] (116.70,114.33) -- (119.55,117.18);

\path[draw=drawColor,draw opacity=0.50,line width= 0.4pt,line join=round,line cap=round] (116.70,117.18) -- (119.55,114.33);

\path[draw=drawColor,draw opacity=0.50,line width= 0.4pt,line join=round,line cap=round] (116.11,115.75) -- (120.14,115.75);

\path[draw=drawColor,draw opacity=0.50,line width= 0.4pt,line join=round,line cap=round] (118.13,113.74) -- (118.13,117.77);

\path[draw=drawColor,draw opacity=0.50,line width= 0.4pt,line join=round,line cap=round] (116.33,123.94) -- (119.18,126.79);

\path[draw=drawColor,draw opacity=0.50,line width= 0.4pt,line join=round,line cap=round] (116.33,126.79) -- (119.18,123.94);

\path[draw=drawColor,draw opacity=0.50,line width= 0.4pt,line join=round,line cap=round] (115.74,125.37) -- (119.77,125.37);

\path[draw=drawColor,draw opacity=0.50,line width= 0.4pt,line join=round,line cap=round] (117.75,123.35) -- (117.75,127.38);

\path[draw=drawColor,draw opacity=0.50,line width= 0.4pt,line join=round,line cap=round] (126.57,113.84) -- (129.42,116.70);

\path[draw=drawColor,draw opacity=0.50,line width= 0.4pt,line join=round,line cap=round] (126.57,116.70) -- (129.42,113.84);

\path[draw=drawColor,draw opacity=0.50,line width= 0.4pt,line join=round,line cap=round] (125.98,115.27) -- (130.02,115.27);

\path[draw=drawColor,draw opacity=0.50,line width= 0.4pt,line join=round,line cap=round] (128.00,113.25) -- (128.00,117.29);

\path[draw=drawColor,draw opacity=0.50,line width= 0.4pt,line join=round,line cap=round] (113.86,112.81) -- (116.71,115.66);

\path[draw=drawColor,draw opacity=0.50,line width= 0.4pt,line join=round,line cap=round] (113.86,115.66) -- (116.71,112.81);

\path[draw=drawColor,draw opacity=0.50,line width= 0.4pt,line join=round,line cap=round] (113.27,114.23) -- (117.30,114.23);

\path[draw=drawColor,draw opacity=0.50,line width= 0.4pt,line join=round,line cap=round] (115.28,112.22) -- (115.28,116.25);

\path[draw=drawColor,draw opacity=0.50,line width= 0.4pt,line join=round,line cap=round] (112.57,112.25) -- (115.42,115.11);

\path[draw=drawColor,draw opacity=0.50,line width= 0.4pt,line join=round,line cap=round] (112.57,115.11) -- (115.42,112.25);

\path[draw=drawColor,draw opacity=0.50,line width= 0.4pt,line join=round,line cap=round] (111.98,113.68) -- (116.01,113.68);

\path[draw=drawColor,draw opacity=0.50,line width= 0.4pt,line join=round,line cap=round] (114.00,111.66) -- (114.00,115.70);

\path[draw=drawColor,draw opacity=0.50,line width= 0.4pt,line join=round,line cap=round] (119.26,113.59) -- (122.11,116.45);

\path[draw=drawColor,draw opacity=0.50,line width= 0.4pt,line join=round,line cap=round] (119.26,116.45) -- (122.11,113.59);

\path[draw=drawColor,draw opacity=0.50,line width= 0.4pt,line join=round,line cap=round] (118.67,115.02) -- (122.71,115.02);

\path[draw=drawColor,draw opacity=0.50,line width= 0.4pt,line join=round,line cap=round] (120.69,113.00) -- (120.69,117.04);

\path[draw=drawColor,draw opacity=0.50,line width= 0.4pt,line join=round,line cap=round] (113.86,113.59) -- (116.71,116.45);

\path[draw=drawColor,draw opacity=0.50,line width= 0.4pt,line join=round,line cap=round] (113.86,116.45) -- (116.71,113.59);

\path[draw=drawColor,draw opacity=0.50,line width= 0.4pt,line join=round,line cap=round] (113.27,115.02) -- (117.30,115.02);

\path[draw=drawColor,draw opacity=0.50,line width= 0.4pt,line join=round,line cap=round] (115.28,113.00) -- (115.28,117.04);

\path[draw=drawColor,draw opacity=0.50,line width= 0.4pt,line join=round,line cap=round] (117.75,116.87) -- (120.61,119.73);

\path[draw=drawColor,draw opacity=0.50,line width= 0.4pt,line join=round,line cap=round] (117.75,119.73) -- (120.61,116.87);

\path[draw=drawColor,draw opacity=0.50,line width= 0.4pt,line join=round,line cap=round] (117.16,118.30) -- (121.20,118.30);

\path[draw=drawColor,draw opacity=0.50,line width= 0.4pt,line join=round,line cap=round] (119.18,116.28) -- (119.18,120.32);

\path[draw=drawColor,draw opacity=0.50,line width= 0.4pt,line join=round,line cap=round] (127.62,147.69) -- (130.48,150.54);

\path[draw=drawColor,draw opacity=0.50,line width= 0.4pt,line join=round,line cap=round] (127.62,150.54) -- (130.48,147.69);

\path[draw=drawColor,draw opacity=0.50,line width= 0.4pt,line join=round,line cap=round] (127.03,149.11) -- (131.07,149.11);

\path[draw=drawColor,draw opacity=0.50,line width= 0.4pt,line join=round,line cap=round] (129.05,147.09) -- (129.05,151.13);

\path[draw=drawColor,draw opacity=0.50,line width= 0.4pt,line join=round,line cap=round] (126.32,149.34) -- (129.17,152.19);

\path[draw=drawColor,draw opacity=0.50,line width= 0.4pt,line join=round,line cap=round] (126.32,152.19) -- (129.17,149.34);

\path[draw=drawColor,draw opacity=0.50,line width= 0.4pt,line join=round,line cap=round] (125.73,150.77) -- (129.76,150.77);

\path[draw=drawColor,draw opacity=0.50,line width= 0.4pt,line join=round,line cap=round] (127.74,148.75) -- (127.74,152.79);

\path[draw=drawColor,draw opacity=0.50,line width= 0.4pt,line join=round,line cap=round] (121.68,147.73) -- (124.53,150.58);

\path[draw=drawColor,draw opacity=0.50,line width= 0.4pt,line join=round,line cap=round] (121.68,150.58) -- (124.53,147.73);

\path[draw=drawColor,draw opacity=0.50,line width= 0.4pt,line join=round,line cap=round] (121.09,149.15) -- (125.13,149.15);

\path[draw=drawColor,draw opacity=0.50,line width= 0.4pt,line join=round,line cap=round] (123.11,147.14) -- (123.11,151.17);

\path[draw=drawColor,draw opacity=0.50,line width= 0.4pt,line join=round,line cap=round] (125.91,147.19) -- (128.77,150.04);

\path[draw=drawColor,draw opacity=0.50,line width= 0.4pt,line join=round,line cap=round] (125.91,150.04) -- (128.77,147.19);

\path[draw=drawColor,draw opacity=0.50,line width= 0.4pt,line join=round,line cap=round] (125.32,148.62) -- (129.36,148.62);

\path[draw=drawColor,draw opacity=0.50,line width= 0.4pt,line join=round,line cap=round] (127.34,146.60) -- (127.34,150.63);

\path[draw=drawColor,draw opacity=0.50,line width= 0.4pt,line join=round,line cap=round] (129.73,147.79) -- (132.58,150.65);

\path[draw=drawColor,draw opacity=0.50,line width= 0.4pt,line join=round,line cap=round] (129.73,150.65) -- (132.58,147.79);

\path[draw=drawColor,draw opacity=0.50,line width= 0.4pt,line join=round,line cap=round] (129.14,149.22) -- (133.17,149.22);

\path[draw=drawColor,draw opacity=0.50,line width= 0.4pt,line join=round,line cap=round] (131.15,147.20) -- (131.15,151.24);

\path[draw=drawColor,draw opacity=0.50,line width= 0.4pt,line join=round,line cap=round] (129.93,147.75) -- (132.79,150.61);

\path[draw=drawColor,draw opacity=0.50,line width= 0.4pt,line join=round,line cap=round] (129.93,150.61) -- (132.79,147.75);

\path[draw=drawColor,draw opacity=0.50,line width= 0.4pt,line join=round,line cap=round] (129.34,149.18) -- (133.38,149.18);

\path[draw=drawColor,draw opacity=0.50,line width= 0.4pt,line join=round,line cap=round] (131.36,147.16) -- (131.36,151.20);

\path[draw=drawColor,draw opacity=0.50,line width= 0.4pt,line join=round,line cap=round] (122.38,148.70) -- (125.23,151.56);

\path[draw=drawColor,draw opacity=0.50,line width= 0.4pt,line join=round,line cap=round] (122.38,151.56) -- (125.23,148.70);

\path[draw=drawColor,draw opacity=0.50,line width= 0.4pt,line join=round,line cap=round] (121.79,150.13) -- (125.82,150.13);

\path[draw=drawColor,draw opacity=0.50,line width= 0.4pt,line join=round,line cap=round] (123.80,148.11) -- (123.80,152.15);

\path[draw=drawColor,draw opacity=0.50,line width= 0.4pt,line join=round,line cap=round] (123.39,147.78) -- (126.24,150.63);

\path[draw=drawColor,draw opacity=0.50,line width= 0.4pt,line join=round,line cap=round] (123.39,150.63) -- (126.24,147.78);

\path[draw=drawColor,draw opacity=0.50,line width= 0.4pt,line join=round,line cap=round] (122.79,149.21) -- (126.83,149.21);

\path[draw=drawColor,draw opacity=0.50,line width= 0.4pt,line join=round,line cap=round] (124.81,147.19) -- (124.81,151.22);

\path[draw=drawColor,draw opacity=0.50,line width= 0.4pt,line join=round,line cap=round] (123.37,147.78) -- (126.22,150.63);

\path[draw=drawColor,draw opacity=0.50,line width= 0.4pt,line join=round,line cap=round] (123.37,150.63) -- (126.22,147.78);

\path[draw=drawColor,draw opacity=0.50,line width= 0.4pt,line join=round,line cap=round] (122.78,149.21) -- (126.82,149.21);

\path[draw=drawColor,draw opacity=0.50,line width= 0.4pt,line join=round,line cap=round] (124.80,147.19) -- (124.80,151.22);

\path[draw=drawColor,draw opacity=0.50,line width= 0.4pt,line join=round,line cap=round] (122.83,148.17) -- (125.69,151.03);

\path[draw=drawColor,draw opacity=0.50,line width= 0.4pt,line join=round,line cap=round] (122.83,151.03) -- (125.69,148.17);

\path[draw=drawColor,draw opacity=0.50,line width= 0.4pt,line join=round,line cap=round] (122.24,149.60) -- (126.28,149.60);

\path[draw=drawColor,draw opacity=0.50,line width= 0.4pt,line join=round,line cap=round] (124.26,147.58) -- (124.26,151.62);

\path[draw=drawColor,draw opacity=0.50,line width= 0.4pt,line join=round,line cap=round] (121.01,147.09) -- (123.86,149.94);

\path[draw=drawColor,draw opacity=0.50,line width= 0.4pt,line join=round,line cap=round] (121.01,149.94) -- (123.86,147.09);

\path[draw=drawColor,draw opacity=0.50,line width= 0.4pt,line join=round,line cap=round] (120.42,148.52) -- (124.45,148.52);

\path[draw=drawColor,draw opacity=0.50,line width= 0.4pt,line join=round,line cap=round] (122.44,146.50) -- (122.44,150.54);

\path[draw=drawColor,draw opacity=0.50,line width= 0.4pt,line join=round,line cap=round] (128.31,147.60) -- (131.16,150.46);

\path[draw=drawColor,draw opacity=0.50,line width= 0.4pt,line join=round,line cap=round] (128.31,150.46) -- (131.16,147.60);

\path[draw=drawColor,draw opacity=0.50,line width= 0.4pt,line join=round,line cap=round] (127.72,149.03) -- (131.75,149.03);

\path[draw=drawColor,draw opacity=0.50,line width= 0.4pt,line join=round,line cap=round] (129.74,147.01) -- (129.74,151.05);

\path[draw=drawColor,draw opacity=0.50,line width= 0.4pt,line join=round,line cap=round] (130.23,149.18) -- (133.08,152.03);

\path[draw=drawColor,draw opacity=0.50,line width= 0.4pt,line join=round,line cap=round] (130.23,152.03) -- (133.08,149.18);

\path[draw=drawColor,draw opacity=0.50,line width= 0.4pt,line join=round,line cap=round] (129.64,150.60) -- (133.67,150.60);

\path[draw=drawColor,draw opacity=0.50,line width= 0.4pt,line join=round,line cap=round] (131.65,148.59) -- (131.65,152.62);

\path[draw=drawColor,draw opacity=0.50,line width= 0.4pt,line join=round,line cap=round] (128.21,148.64) -- (131.06,151.50);

\path[draw=drawColor,draw opacity=0.50,line width= 0.4pt,line join=round,line cap=round] (128.21,151.50) -- (131.06,148.64);

\path[draw=drawColor,draw opacity=0.50,line width= 0.4pt,line join=round,line cap=round] (127.62,150.07) -- (131.65,150.07);

\path[draw=drawColor,draw opacity=0.50,line width= 0.4pt,line join=round,line cap=round] (129.64,148.05) -- (129.64,152.09);

\path[draw=drawColor,draw opacity=0.50,line width= 0.4pt,line join=round,line cap=round] (125.16,147.50) -- (128.01,150.35);

\path[draw=drawColor,draw opacity=0.50,line width= 0.4pt,line join=round,line cap=round] (125.16,150.35) -- (128.01,147.50);

\path[draw=drawColor,draw opacity=0.50,line width= 0.4pt,line join=round,line cap=round] (124.57,148.92) -- (128.60,148.92);

\path[draw=drawColor,draw opacity=0.50,line width= 0.4pt,line join=round,line cap=round] (126.58,146.90) -- (126.58,150.94);

\path[draw=drawColor,draw opacity=0.50,line width= 0.4pt,line join=round,line cap=round] (126.54,151.33) -- (129.39,154.19);

\path[draw=drawColor,draw opacity=0.50,line width= 0.4pt,line join=round,line cap=round] (126.54,154.19) -- (129.39,151.33);

\path[draw=drawColor,draw opacity=0.50,line width= 0.4pt,line join=round,line cap=round] (125.95,152.76) -- (129.98,152.76);

\path[draw=drawColor,draw opacity=0.50,line width= 0.4pt,line join=round,line cap=round] (127.97,150.74) -- (127.97,154.78);

\path[draw=drawColor,draw opacity=0.50,line width= 0.4pt,line join=round,line cap=round] (128.11,151.35) -- (130.96,154.21);

\path[draw=drawColor,draw opacity=0.50,line width= 0.4pt,line join=round,line cap=round] (128.11,154.21) -- (130.96,151.35);

\path[draw=drawColor,draw opacity=0.50,line width= 0.4pt,line join=round,line cap=round] (127.52,152.78) -- (131.55,152.78);

\path[draw=drawColor,draw opacity=0.50,line width= 0.4pt,line join=round,line cap=round] (129.54,150.76) -- (129.54,154.80);

\path[draw=drawColor,draw opacity=0.50,line width= 0.4pt,line join=round,line cap=round] (122.03,155.45) -- (124.88,158.31);

\path[draw=drawColor,draw opacity=0.50,line width= 0.4pt,line join=round,line cap=round] (122.03,158.31) -- (124.88,155.45);

\path[draw=drawColor,draw opacity=0.50,line width= 0.4pt,line join=round,line cap=round] (121.44,156.88) -- (125.47,156.88);

\path[draw=drawColor,draw opacity=0.50,line width= 0.4pt,line join=round,line cap=round] (123.45,154.86) -- (123.45,158.90);

\path[draw=drawColor,draw opacity=0.50,line width= 0.4pt,line join=round,line cap=round] (124.92,151.87) -- (127.77,154.73);

\path[draw=drawColor,draw opacity=0.50,line width= 0.4pt,line join=round,line cap=round] (124.92,154.73) -- (127.77,151.87);

\path[draw=drawColor,draw opacity=0.50,line width= 0.4pt,line join=round,line cap=round] (124.33,153.30) -- (128.36,153.30);

\path[draw=drawColor,draw opacity=0.50,line width= 0.4pt,line join=round,line cap=round] (126.35,151.28) -- (126.35,155.32);

\path[draw=drawColor,draw opacity=0.50,line width= 0.4pt,line join=round,line cap=round] (126.38,148.59) -- (129.23,151.45);

\path[draw=drawColor,draw opacity=0.50,line width= 0.4pt,line join=round,line cap=round] (126.38,151.45) -- (129.23,148.59);

\path[draw=drawColor,draw opacity=0.50,line width= 0.4pt,line join=round,line cap=round] (125.79,150.02) -- (129.83,150.02);

\path[draw=drawColor,draw opacity=0.50,line width= 0.4pt,line join=round,line cap=round] (127.81,148.00) -- (127.81,152.04);

\path[draw=drawColor,draw opacity=0.50,line width= 0.4pt,line join=round,line cap=round] (129.83,149.13) -- (132.68,151.98);

\path[draw=drawColor,draw opacity=0.50,line width= 0.4pt,line join=round,line cap=round] (129.83,151.98) -- (132.68,149.13);

\path[draw=drawColor,draw opacity=0.50,line width= 0.4pt,line join=round,line cap=round] (129.24,150.56) -- (133.27,150.56);

\path[draw=drawColor,draw opacity=0.50,line width= 0.4pt,line join=round,line cap=round] (131.26,148.54) -- (131.26,152.57);

\path[draw=drawColor,draw opacity=0.50,line width= 0.4pt,line join=round,line cap=round] (124.98,148.88) -- (127.83,151.73);

\path[draw=drawColor,draw opacity=0.50,line width= 0.4pt,line join=round,line cap=round] (124.98,151.73) -- (127.83,148.88);

\path[draw=drawColor,draw opacity=0.50,line width= 0.4pt,line join=round,line cap=round] (124.39,150.30) -- (128.42,150.30);

\path[draw=drawColor,draw opacity=0.50,line width= 0.4pt,line join=round,line cap=round] (126.41,148.28) -- (126.41,152.32);

\path[draw=drawColor,draw opacity=0.50,line width= 0.4pt,line join=round,line cap=round] (124.88,148.56) -- (127.74,151.41);

\path[draw=drawColor,draw opacity=0.50,line width= 0.4pt,line join=round,line cap=round] (124.88,151.41) -- (127.74,148.56);

\path[draw=drawColor,draw opacity=0.50,line width= 0.4pt,line join=round,line cap=round] (124.29,149.98) -- (128.33,149.98);

\path[draw=drawColor,draw opacity=0.50,line width= 0.4pt,line join=round,line cap=round] (126.31,147.96) -- (126.31,152.00);

\path[draw=drawColor,draw opacity=0.50,line width= 0.4pt,line join=round,line cap=round] (124.97,150.49) -- (127.82,153.34);

\path[draw=drawColor,draw opacity=0.50,line width= 0.4pt,line join=round,line cap=round] (124.97,153.34) -- (127.82,150.49);

\path[draw=drawColor,draw opacity=0.50,line width= 0.4pt,line join=round,line cap=round] (124.38,151.91) -- (128.41,151.91);

\path[draw=drawColor,draw opacity=0.50,line width= 0.4pt,line join=round,line cap=round] (126.39,149.90) -- (126.39,153.93);

\path[draw=drawColor,draw opacity=0.50,line width= 0.4pt,line join=round,line cap=round] (125.60,148.54) -- (128.45,151.40);

\path[draw=drawColor,draw opacity=0.50,line width= 0.4pt,line join=round,line cap=round] (125.60,151.40) -- (128.45,148.54);

\path[draw=drawColor,draw opacity=0.50,line width= 0.4pt,line join=round,line cap=round] (125.01,149.97) -- (129.04,149.97);

\path[draw=drawColor,draw opacity=0.50,line width= 0.4pt,line join=round,line cap=round] (127.03,147.95) -- (127.03,151.99);

\path[draw=drawColor,draw opacity=0.50,line width= 0.4pt,line join=round,line cap=round] (122.29,148.48) -- (125.14,151.33);

\path[draw=drawColor,draw opacity=0.50,line width= 0.4pt,line join=round,line cap=round] (122.29,151.33) -- (125.14,148.48);

\path[draw=drawColor,draw opacity=0.50,line width= 0.4pt,line join=round,line cap=round] (121.70,149.91) -- (125.73,149.91);

\path[draw=drawColor,draw opacity=0.50,line width= 0.4pt,line join=round,line cap=round] (123.71,147.89) -- (123.71,151.92);

\path[draw=drawColor,draw opacity=0.50,line width= 0.4pt,line join=round,line cap=round] (123.43,150.57) -- (126.28,153.42);

\path[draw=drawColor,draw opacity=0.50,line width= 0.4pt,line join=round,line cap=round] (123.43,153.42) -- (126.28,150.57);

\path[draw=drawColor,draw opacity=0.50,line width= 0.4pt,line join=round,line cap=round] (122.84,152.00) -- (126.87,152.00);

\path[draw=drawColor,draw opacity=0.50,line width= 0.4pt,line join=round,line cap=round] (124.85,149.98) -- (124.85,154.01);

\path[draw=drawColor,draw opacity=0.50,line width= 0.4pt,line join=round,line cap=round] (130.30,148.35) -- (133.15,151.21);

\path[draw=drawColor,draw opacity=0.50,line width= 0.4pt,line join=round,line cap=round] (130.30,151.21) -- (133.15,148.35);

\path[draw=drawColor,draw opacity=0.50,line width= 0.4pt,line join=round,line cap=round] (129.70,149.78) -- (133.74,149.78);

\path[draw=drawColor,draw opacity=0.50,line width= 0.4pt,line join=round,line cap=round] (131.72,147.76) -- (131.72,151.80);

\path[draw=drawColor,draw opacity=0.50,line width= 0.4pt,line join=round,line cap=round] (126.20,151.83) -- (129.05,154.68);

\path[draw=drawColor,draw opacity=0.50,line width= 0.4pt,line join=round,line cap=round] (126.20,154.68) -- (129.05,151.83);

\path[draw=drawColor,draw opacity=0.50,line width= 0.4pt,line join=round,line cap=round] (125.61,153.25) -- (129.64,153.25);

\path[draw=drawColor,draw opacity=0.50,line width= 0.4pt,line join=round,line cap=round] (127.63,151.23) -- (127.63,155.27);

\path[draw=drawColor,draw opacity=0.50,line width= 0.4pt,line join=round,line cap=round] (126.17,149.51) -- (129.02,152.37);

\path[draw=drawColor,draw opacity=0.50,line width= 0.4pt,line join=round,line cap=round] (126.17,152.37) -- (129.02,149.51);

\path[draw=drawColor,draw opacity=0.50,line width= 0.4pt,line join=round,line cap=round] (125.57,150.94) -- (129.61,150.94);

\path[draw=drawColor,draw opacity=0.50,line width= 0.4pt,line join=round,line cap=round] (127.59,148.92) -- (127.59,152.96);

\path[draw=drawColor,draw opacity=0.50,line width= 0.4pt,line join=round,line cap=round] (129.26,149.63) -- (132.12,152.48);

\path[draw=drawColor,draw opacity=0.50,line width= 0.4pt,line join=round,line cap=round] (129.26,152.48) -- (132.12,149.63);

\path[draw=drawColor,draw opacity=0.50,line width= 0.4pt,line join=round,line cap=round] (128.67,151.06) -- (132.71,151.06);

\path[draw=drawColor,draw opacity=0.50,line width= 0.4pt,line join=round,line cap=round] (130.69,149.04) -- (130.69,153.07);

\path[draw=drawColor,draw opacity=0.50,line width= 0.4pt,line join=round,line cap=round] (129.06,151.02) -- (131.91,153.88);

\path[draw=drawColor,draw opacity=0.50,line width= 0.4pt,line join=round,line cap=round] (129.06,153.88) -- (131.91,151.02);

\path[draw=drawColor,draw opacity=0.50,line width= 0.4pt,line join=round,line cap=round] (128.47,152.45) -- (132.50,152.45);

\path[draw=drawColor,draw opacity=0.50,line width= 0.4pt,line join=round,line cap=round] (130.49,150.43) -- (130.49,154.47);

\path[draw=drawColor,draw opacity=0.50,line width= 0.4pt,line join=round,line cap=round] (128.31,152.07) -- (131.16,154.92);

\path[draw=drawColor,draw opacity=0.50,line width= 0.4pt,line join=round,line cap=round] (128.31,154.92) -- (131.16,152.07);

\path[draw=drawColor,draw opacity=0.50,line width= 0.4pt,line join=round,line cap=round] (127.72,153.49) -- (131.75,153.49);

\path[draw=drawColor,draw opacity=0.50,line width= 0.4pt,line join=round,line cap=round] (129.74,151.48) -- (129.74,155.51);

\path[draw=drawColor,draw opacity=0.50,line width= 0.4pt,line join=round,line cap=round] (127.57,156.27) -- (130.43,159.12);

\path[draw=drawColor,draw opacity=0.50,line width= 0.4pt,line join=round,line cap=round] (127.57,159.12) -- (130.43,156.27);

\path[draw=drawColor,draw opacity=0.50,line width= 0.4pt,line join=round,line cap=round] (126.98,157.69) -- (131.02,157.69);

\path[draw=drawColor,draw opacity=0.50,line width= 0.4pt,line join=round,line cap=round] (129.00,155.68) -- (129.00,159.71);

\path[draw=drawColor,draw opacity=0.50,line width= 0.4pt,line join=round,line cap=round] (130.06,153.65) -- (132.91,156.50);

\path[draw=drawColor,draw opacity=0.50,line width= 0.4pt,line join=round,line cap=round] (130.06,156.50) -- (132.91,153.65);

\path[draw=drawColor,draw opacity=0.50,line width= 0.4pt,line join=round,line cap=round] (129.47,155.07) -- (133.50,155.07);

\path[draw=drawColor,draw opacity=0.50,line width= 0.4pt,line join=round,line cap=round] (131.48,153.06) -- (131.48,157.09);

\path[draw=drawColor,draw opacity=0.50,line width= 0.4pt,line join=round,line cap=round] (129.48,152.48) -- (132.34,155.33);

\path[draw=drawColor,draw opacity=0.50,line width= 0.4pt,line join=round,line cap=round] (129.48,155.33) -- (132.34,152.48);

\path[draw=drawColor,draw opacity=0.50,line width= 0.4pt,line join=round,line cap=round] (128.89,153.90) -- (132.93,153.90);

\path[draw=drawColor,draw opacity=0.50,line width= 0.4pt,line join=round,line cap=round] (130.91,151.89) -- (130.91,155.92);

\path[draw=drawColor,draw opacity=0.50,line width= 0.4pt,line join=round,line cap=round] (128.62,153.77) -- (131.47,156.62);

\path[draw=drawColor,draw opacity=0.50,line width= 0.4pt,line join=round,line cap=round] (128.62,156.62) -- (131.47,153.77);

\path[draw=drawColor,draw opacity=0.50,line width= 0.4pt,line join=round,line cap=round] (128.03,155.19) -- (132.07,155.19);

\path[draw=drawColor,draw opacity=0.50,line width= 0.4pt,line join=round,line cap=round] (130.05,153.18) -- (130.05,157.21);

\path[draw=drawColor,draw opacity=0.50,line width= 0.4pt,line join=round,line cap=round] (127.59,153.94) -- (130.45,156.80);

\path[draw=drawColor,draw opacity=0.50,line width= 0.4pt,line join=round,line cap=round] (127.59,156.80) -- (130.45,153.94);

\path[draw=drawColor,draw opacity=0.50,line width= 0.4pt,line join=round,line cap=round] (127.00,155.37) -- (131.04,155.37);

\path[draw=drawColor,draw opacity=0.50,line width= 0.4pt,line join=round,line cap=round] (129.02,153.35) -- (129.02,157.39);

\path[draw=drawColor,draw opacity=0.50,line width= 0.4pt,line join=round,line cap=round] (129.37,155.06) -- (132.22,157.91);

\path[draw=drawColor,draw opacity=0.50,line width= 0.4pt,line join=round,line cap=round] (129.37,157.91) -- (132.22,155.06);

\path[draw=drawColor,draw opacity=0.50,line width= 0.4pt,line join=round,line cap=round] (128.78,156.49) -- (132.81,156.49);

\path[draw=drawColor,draw opacity=0.50,line width= 0.4pt,line join=round,line cap=round] (130.80,154.47) -- (130.80,158.50);

\path[draw=drawColor,draw opacity=0.50,line width= 0.4pt,line join=round,line cap=round] (124.89,153.63) -- (127.75,156.48);

\path[draw=drawColor,draw opacity=0.50,line width= 0.4pt,line join=round,line cap=round] (124.89,156.48) -- (127.75,153.63);

\path[draw=drawColor,draw opacity=0.50,line width= 0.4pt,line join=round,line cap=round] (124.30,155.06) -- (128.34,155.06);

\path[draw=drawColor,draw opacity=0.50,line width= 0.4pt,line join=round,line cap=round] (126.32,153.04) -- (126.32,157.08);

\path[draw=drawColor,draw opacity=0.50,line width= 0.4pt,line join=round,line cap=round] (125.09,154.88) -- (127.94,157.73);

\path[draw=drawColor,draw opacity=0.50,line width= 0.4pt,line join=round,line cap=round] (125.09,157.73) -- (127.94,154.88);

\path[draw=drawColor,draw opacity=0.50,line width= 0.4pt,line join=round,line cap=round] (124.50,156.31) -- (128.53,156.31);

\path[draw=drawColor,draw opacity=0.50,line width= 0.4pt,line join=round,line cap=round] (126.51,154.29) -- (126.51,158.32);

\path[draw=drawColor,draw opacity=0.50,line width= 0.4pt,line join=round,line cap=round] (126.59,152.69) -- (129.45,155.54);

\path[draw=drawColor,draw opacity=0.50,line width= 0.4pt,line join=round,line cap=round] (126.59,155.54) -- (129.45,152.69);

\path[draw=drawColor,draw opacity=0.50,line width= 0.4pt,line join=round,line cap=round] (126.00,154.12) -- (130.04,154.12);

\path[draw=drawColor,draw opacity=0.50,line width= 0.4pt,line join=round,line cap=round] (128.02,152.10) -- (128.02,156.13);

\path[draw=drawColor,draw opacity=0.50,line width= 0.4pt,line join=round,line cap=round] (123.02,150.26) -- (125.87,153.12);

\path[draw=drawColor,draw opacity=0.50,line width= 0.4pt,line join=round,line cap=round] (123.02,153.12) -- (125.87,150.26);

\path[draw=drawColor,draw opacity=0.50,line width= 0.4pt,line join=round,line cap=round] (122.43,151.69) -- (126.47,151.69);

\path[draw=drawColor,draw opacity=0.50,line width= 0.4pt,line join=round,line cap=round] (124.45,149.67) -- (124.45,153.71);

\path[draw=drawColor,draw opacity=0.50,line width= 0.4pt,line join=round,line cap=round] (129.21,153.91) -- (132.07,156.76);

\path[draw=drawColor,draw opacity=0.50,line width= 0.4pt,line join=round,line cap=round] (129.21,156.76) -- (132.07,153.91);

\path[draw=drawColor,draw opacity=0.50,line width= 0.4pt,line join=round,line cap=round] (128.62,155.34) -- (132.66,155.34);

\path[draw=drawColor,draw opacity=0.50,line width= 0.4pt,line join=round,line cap=round] (130.64,153.32) -- (130.64,157.36);

\path[draw=drawColor,draw opacity=0.50,line width= 0.4pt,line join=round,line cap=round] (128.57,153.81) -- (131.42,156.66);

\path[draw=drawColor,draw opacity=0.50,line width= 0.4pt,line join=round,line cap=round] (128.57,156.66) -- (131.42,153.81);

\path[draw=drawColor,draw opacity=0.50,line width= 0.4pt,line join=round,line cap=round] (127.98,155.23) -- (132.01,155.23);

\path[draw=drawColor,draw opacity=0.50,line width= 0.4pt,line join=round,line cap=round] (129.99,153.22) -- (129.99,157.25);

\path[draw=drawColor,draw opacity=0.50,line width= 0.4pt,line join=round,line cap=round] (128.45,152.01) -- (131.31,154.87);

\path[draw=drawColor,draw opacity=0.50,line width= 0.4pt,line join=round,line cap=round] (128.45,154.87) -- (131.31,152.01);

\path[draw=drawColor,draw opacity=0.50,line width= 0.4pt,line join=round,line cap=round] (127.86,153.44) -- (131.90,153.44);

\path[draw=drawColor,draw opacity=0.50,line width= 0.4pt,line join=round,line cap=round] (129.88,151.42) -- (129.88,155.46);

\path[draw=drawColor,draw opacity=0.50,line width= 0.4pt,line join=round,line cap=round] (127.83,155.57) -- (130.68,158.42);

\path[draw=drawColor,draw opacity=0.50,line width= 0.4pt,line join=round,line cap=round] (127.83,158.42) -- (130.68,155.57);

\path[draw=drawColor,draw opacity=0.50,line width= 0.4pt,line join=round,line cap=round] (127.24,157.00) -- (131.27,157.00);

\path[draw=drawColor,draw opacity=0.50,line width= 0.4pt,line join=round,line cap=round] (129.26,154.98) -- (129.26,159.01);

\path[draw=drawColor,draw opacity=0.50,line width= 0.4pt,line join=round,line cap=round] (130.11,157.99) -- (132.96,160.85);

\path[draw=drawColor,draw opacity=0.50,line width= 0.4pt,line join=round,line cap=round] (130.11,160.85) -- (132.96,157.99);

\path[draw=drawColor,draw opacity=0.50,line width= 0.4pt,line join=round,line cap=round] (129.52,159.42) -- (133.56,159.42);

\path[draw=drawColor,draw opacity=0.50,line width= 0.4pt,line join=round,line cap=round] (131.54,157.40) -- (131.54,161.44);

\path[draw=drawColor,draw opacity=0.50,line width= 0.4pt,line join=round,line cap=round] (134.23,169.28) -- (137.08,172.14);

\path[draw=drawColor,draw opacity=0.50,line width= 0.4pt,line join=round,line cap=round] (134.23,172.14) -- (137.08,169.28);

\path[draw=drawColor,draw opacity=0.50,line width= 0.4pt,line join=round,line cap=round] (133.64,170.71) -- (137.67,170.71);

\path[draw=drawColor,draw opacity=0.50,line width= 0.4pt,line join=round,line cap=round] (135.65,168.69) -- (135.65,172.73);

\path[draw=drawColor,draw opacity=0.50,line width= 0.4pt,line join=round,line cap=round] (134.21,170.22) -- (137.06,173.07);

\path[draw=drawColor,draw opacity=0.50,line width= 0.4pt,line join=round,line cap=round] (134.21,173.07) -- (137.06,170.22);

\path[draw=drawColor,draw opacity=0.50,line width= 0.4pt,line join=round,line cap=round] (133.62,171.64) -- (137.65,171.64);

\path[draw=drawColor,draw opacity=0.50,line width= 0.4pt,line join=round,line cap=round] (135.64,169.63) -- (135.64,173.66);

\path[draw=drawColor,draw opacity=0.50,line width= 0.4pt,line join=round,line cap=round] (135.66,170.47) -- (138.52,173.33);

\path[draw=drawColor,draw opacity=0.50,line width= 0.4pt,line join=round,line cap=round] (135.66,173.33) -- (138.52,170.47);

\path[draw=drawColor,draw opacity=0.50,line width= 0.4pt,line join=round,line cap=round] (135.07,171.90) -- (139.11,171.90);

\path[draw=drawColor,draw opacity=0.50,line width= 0.4pt,line join=round,line cap=round] (137.09,169.88) -- (137.09,173.92);

\path[draw=drawColor,draw opacity=0.50,line width= 0.4pt,line join=round,line cap=round] (135.47,169.73) -- (138.33,172.59);

\path[draw=drawColor,draw opacity=0.50,line width= 0.4pt,line join=round,line cap=round] (135.47,172.59) -- (138.33,169.73);

\path[draw=drawColor,draw opacity=0.50,line width= 0.4pt,line join=round,line cap=round] (134.88,171.16) -- (138.92,171.16);

\path[draw=drawColor,draw opacity=0.50,line width= 0.4pt,line join=round,line cap=round] (136.90,169.14) -- (136.90,173.18);

\path[draw=drawColor,draw opacity=0.50,line width= 0.4pt,line join=round,line cap=round] (134.38,172.96) -- (137.24,175.81);

\path[draw=drawColor,draw opacity=0.50,line width= 0.4pt,line join=round,line cap=round] (134.38,175.81) -- (137.24,172.96);

\path[draw=drawColor,draw opacity=0.50,line width= 0.4pt,line join=round,line cap=round] (133.79,174.38) -- (137.83,174.38);

\path[draw=drawColor,draw opacity=0.50,line width= 0.4pt,line join=round,line cap=round] (135.81,172.36) -- (135.81,176.40);

\path[draw=drawColor,draw opacity=0.50,line width= 0.4pt,line join=round,line cap=round] (135.45,171.53) -- (138.30,174.39);

\path[draw=drawColor,draw opacity=0.50,line width= 0.4pt,line join=round,line cap=round] (135.45,174.39) -- (138.30,171.53);

\path[draw=drawColor,draw opacity=0.50,line width= 0.4pt,line join=round,line cap=round] (134.86,172.96) -- (138.89,172.96);

\path[draw=drawColor,draw opacity=0.50,line width= 0.4pt,line join=round,line cap=round] (136.88,170.94) -- (136.88,174.98);

\path[draw=drawColor,draw opacity=0.50,line width= 0.4pt,line join=round,line cap=round] (131.51,171.07) -- (134.37,173.92);

\path[draw=drawColor,draw opacity=0.50,line width= 0.4pt,line join=round,line cap=round] (131.51,173.92) -- (134.37,171.07);

\path[draw=drawColor,draw opacity=0.50,line width= 0.4pt,line join=round,line cap=round] (130.92,172.49) -- (134.96,172.49);

\path[draw=drawColor,draw opacity=0.50,line width= 0.4pt,line join=round,line cap=round] (132.94,170.47) -- (132.94,174.51);

\path[draw=drawColor,draw opacity=0.50,line width= 0.4pt,line join=round,line cap=round] (130.25,172.10) -- (133.10,174.95);

\path[draw=drawColor,draw opacity=0.50,line width= 0.4pt,line join=round,line cap=round] (130.25,174.95) -- (133.10,172.10);

\path[draw=drawColor,draw opacity=0.50,line width= 0.4pt,line join=round,line cap=round] (129.66,173.53) -- (133.69,173.53);

\path[draw=drawColor,draw opacity=0.50,line width= 0.4pt,line join=round,line cap=round] (131.68,171.51) -- (131.68,175.54);

\path[draw=drawColor,draw opacity=0.50,line width= 0.4pt,line join=round,line cap=round] (131.58,171.53) -- (134.43,174.39);

\path[draw=drawColor,draw opacity=0.50,line width= 0.4pt,line join=round,line cap=round] (131.58,174.39) -- (134.43,171.53);

\path[draw=drawColor,draw opacity=0.50,line width= 0.4pt,line join=round,line cap=round] (130.98,172.96) -- (135.02,172.96);

\path[draw=drawColor,draw opacity=0.50,line width= 0.4pt,line join=round,line cap=round] (133.00,170.94) -- (133.00,174.98);

\path[draw=drawColor,draw opacity=0.50,line width= 0.4pt,line join=round,line cap=round] (133.94,172.18) -- (136.79,175.03);

\path[draw=drawColor,draw opacity=0.50,line width= 0.4pt,line join=round,line cap=round] (133.94,175.03) -- (136.79,172.18);

\path[draw=drawColor,draw opacity=0.50,line width= 0.4pt,line join=round,line cap=round] (133.35,173.60) -- (137.39,173.60);

\path[draw=drawColor,draw opacity=0.50,line width= 0.4pt,line join=round,line cap=round] (135.37,171.59) -- (135.37,175.62);

\path[draw=drawColor,draw opacity=0.50,line width= 0.4pt,line join=round,line cap=round] (134.18,173.15) -- (137.04,176.00);

\path[draw=drawColor,draw opacity=0.50,line width= 0.4pt,line join=round,line cap=round] (134.18,176.00) -- (137.04,173.15);

\path[draw=drawColor,draw opacity=0.50,line width= 0.4pt,line join=round,line cap=round] (133.59,174.58) -- (137.63,174.58);

\path[draw=drawColor,draw opacity=0.50,line width= 0.4pt,line join=round,line cap=round] (135.61,172.56) -- (135.61,176.59);

\path[draw=drawColor,draw opacity=0.50,line width= 0.4pt,line join=round,line cap=round] (136.86,172.68) -- (139.71,175.53);

\path[draw=drawColor,draw opacity=0.50,line width= 0.4pt,line join=round,line cap=round] (136.86,175.53) -- (139.71,172.68);

\path[draw=drawColor,draw opacity=0.50,line width= 0.4pt,line join=round,line cap=round] (136.27,174.10) -- (140.30,174.10);

\path[draw=drawColor,draw opacity=0.50,line width= 0.4pt,line join=round,line cap=round] (138.28,172.08) -- (138.28,176.12);

\path[draw=drawColor,draw opacity=0.50,line width= 0.4pt,line join=round,line cap=round] (134.78,169.35) -- (137.64,172.20);

\path[draw=drawColor,draw opacity=0.50,line width= 0.4pt,line join=round,line cap=round] (134.78,172.20) -- (137.64,169.35);

\path[draw=drawColor,draw opacity=0.50,line width= 0.4pt,line join=round,line cap=round] (134.19,170.77) -- (138.23,170.77);

\path[draw=drawColor,draw opacity=0.50,line width= 0.4pt,line join=round,line cap=round] (136.21,168.76) -- (136.21,172.79);

\path[draw=drawColor,draw opacity=0.50,line width= 0.4pt,line join=round,line cap=round] (137.09,172.82) -- (139.94,175.67);

\path[draw=drawColor,draw opacity=0.50,line width= 0.4pt,line join=round,line cap=round] (137.09,175.67) -- (139.94,172.82);

\path[draw=drawColor,draw opacity=0.50,line width= 0.4pt,line join=round,line cap=round] (136.50,174.25) -- (140.54,174.25);

\path[draw=drawColor,draw opacity=0.50,line width= 0.4pt,line join=round,line cap=round] (138.52,172.23) -- (138.52,176.26);

\path[draw=drawColor,draw opacity=0.50,line width= 0.4pt,line join=round,line cap=round] (132.13,169.60) -- (134.99,172.45);

\path[draw=drawColor,draw opacity=0.50,line width= 0.4pt,line join=round,line cap=round] (132.13,172.45) -- (134.99,169.60);

\path[draw=drawColor,draw opacity=0.50,line width= 0.4pt,line join=round,line cap=round] (131.54,171.02) -- (135.58,171.02);

\path[draw=drawColor,draw opacity=0.50,line width= 0.4pt,line join=round,line cap=round] (133.56,169.01) -- (133.56,173.04);

\path[draw=drawColor,draw opacity=0.50,line width= 0.4pt,line join=round,line cap=round] (136.06,171.47) -- (138.92,174.32);

\path[draw=drawColor,draw opacity=0.50,line width= 0.4pt,line join=round,line cap=round] (136.06,174.32) -- (138.92,171.47);

\path[draw=drawColor,draw opacity=0.50,line width= 0.4pt,line join=round,line cap=round] (135.47,172.89) -- (139.51,172.89);

\path[draw=drawColor,draw opacity=0.50,line width= 0.4pt,line join=round,line cap=round] (137.49,170.88) -- (137.49,174.91);

\path[draw=drawColor,draw opacity=0.50,line width= 0.4pt,line join=round,line cap=round] (134.68,171.07) -- (137.53,173.93);

\path[draw=drawColor,draw opacity=0.50,line width= 0.4pt,line join=round,line cap=round] (134.68,173.93) -- (137.53,171.07);

\path[draw=drawColor,draw opacity=0.50,line width= 0.4pt,line join=round,line cap=round] (134.09,172.50) -- (138.12,172.50);

\path[draw=drawColor,draw opacity=0.50,line width= 0.4pt,line join=round,line cap=round] (136.11,170.48) -- (136.11,174.52);

\path[draw=drawColor,draw opacity=0.50,line width= 0.4pt,line join=round,line cap=round] (137.45,171.69) -- (140.31,174.54);

\path[draw=drawColor,draw opacity=0.50,line width= 0.4pt,line join=round,line cap=round] (137.45,174.54) -- (140.31,171.69);

\path[draw=drawColor,draw opacity=0.50,line width= 0.4pt,line join=round,line cap=round] (136.86,173.12) -- (140.90,173.12);

\path[draw=drawColor,draw opacity=0.50,line width= 0.4pt,line join=round,line cap=round] (138.88,171.10) -- (138.88,175.13);

\path[draw=drawColor,draw opacity=0.50,line width= 0.4pt,line join=round,line cap=round] (137.90,175.00) -- (140.75,177.86);

\path[draw=drawColor,draw opacity=0.50,line width= 0.4pt,line join=round,line cap=round] (137.90,177.86) -- (140.75,175.00);

\path[draw=drawColor,draw opacity=0.50,line width= 0.4pt,line join=round,line cap=round] (137.31,176.43) -- (141.34,176.43);

\path[draw=drawColor,draw opacity=0.50,line width= 0.4pt,line join=round,line cap=round] (139.32,174.41) -- (139.32,178.45);

\path[draw=drawColor,draw opacity=0.50,line width= 0.4pt,line join=round,line cap=round] (136.79,171.92) -- (139.64,174.77);

\path[draw=drawColor,draw opacity=0.50,line width= 0.4pt,line join=round,line cap=round] (136.79,174.77) -- (139.64,171.92);

\path[draw=drawColor,draw opacity=0.50,line width= 0.4pt,line join=round,line cap=round] (136.20,173.34) -- (140.24,173.34);

\path[draw=drawColor,draw opacity=0.50,line width= 0.4pt,line join=round,line cap=round] (138.22,171.33) -- (138.22,175.36);

\path[draw=drawColor,draw opacity=0.50,line width= 0.4pt,line join=round,line cap=round] (134.98,171.02) -- (137.84,173.87);

\path[draw=drawColor,draw opacity=0.50,line width= 0.4pt,line join=round,line cap=round] (134.98,173.87) -- (137.84,171.02);

\path[draw=drawColor,draw opacity=0.50,line width= 0.4pt,line join=round,line cap=round] (134.39,172.45) -- (138.43,172.45);

\path[draw=drawColor,draw opacity=0.50,line width= 0.4pt,line join=round,line cap=round] (136.41,170.43) -- (136.41,174.46);

\path[draw=drawColor,draw opacity=0.50,line width= 0.4pt,line join=round,line cap=round] (136.73,172.66) -- (139.58,175.52);

\path[draw=drawColor,draw opacity=0.50,line width= 0.4pt,line join=round,line cap=round] (136.73,175.52) -- (139.58,172.66);

\path[draw=drawColor,draw opacity=0.50,line width= 0.4pt,line join=round,line cap=round] (136.14,174.09) -- (140.17,174.09);

\path[draw=drawColor,draw opacity=0.50,line width= 0.4pt,line join=round,line cap=round] (138.16,172.07) -- (138.16,176.11);

\path[draw=drawColor,draw opacity=0.50,line width= 0.4pt,line join=round,line cap=round] (131.97,171.39) -- (134.82,174.24);

\path[draw=drawColor,draw opacity=0.50,line width= 0.4pt,line join=round,line cap=round] (131.97,174.24) -- (134.82,171.39);

\path[draw=drawColor,draw opacity=0.50,line width= 0.4pt,line join=round,line cap=round] (131.38,172.81) -- (135.42,172.81);

\path[draw=drawColor,draw opacity=0.50,line width= 0.4pt,line join=round,line cap=round] (133.40,170.80) -- (133.40,174.83);

\path[draw=drawColor,draw opacity=0.50,line width= 0.4pt,line join=round,line cap=round] (133.48,172.06) -- (136.33,174.91);

\path[draw=drawColor,draw opacity=0.50,line width= 0.4pt,line join=round,line cap=round] (133.48,174.91) -- (136.33,172.06);

\path[draw=drawColor,draw opacity=0.50,line width= 0.4pt,line join=round,line cap=round] (132.89,173.49) -- (136.92,173.49);

\path[draw=drawColor,draw opacity=0.50,line width= 0.4pt,line join=round,line cap=round] (134.90,171.47) -- (134.90,175.50);

\path[draw=drawColor,draw opacity=0.50,line width= 0.4pt,line join=round,line cap=round] (130.87,171.03) -- (133.73,173.88);

\path[draw=drawColor,draw opacity=0.50,line width= 0.4pt,line join=round,line cap=round] (130.87,173.88) -- (133.73,171.03);

\path[draw=drawColor,draw opacity=0.50,line width= 0.4pt,line join=round,line cap=round] (130.28,172.46) -- (134.32,172.46);

\path[draw=drawColor,draw opacity=0.50,line width= 0.4pt,line join=round,line cap=round] (132.30,170.44) -- (132.30,174.47);

\path[draw=drawColor,draw opacity=0.50,line width= 0.4pt,line join=round,line cap=round] (132.59,173.33) -- (135.44,176.18);

\path[draw=drawColor,draw opacity=0.50,line width= 0.4pt,line join=round,line cap=round] (132.59,176.18) -- (135.44,173.33);

\path[draw=drawColor,draw opacity=0.50,line width= 0.4pt,line join=round,line cap=round] (132.00,174.75) -- (136.03,174.75);

\path[draw=drawColor,draw opacity=0.50,line width= 0.4pt,line join=round,line cap=round] (134.01,172.74) -- (134.01,176.77);

\path[draw=drawColor,draw opacity=0.50,line width= 0.4pt,line join=round,line cap=round] (132.52,171.06) -- (135.38,173.92);

\path[draw=drawColor,draw opacity=0.50,line width= 0.4pt,line join=round,line cap=round] (132.52,173.92) -- (135.38,171.06);

\path[draw=drawColor,draw opacity=0.50,line width= 0.4pt,line join=round,line cap=round] (131.93,172.49) -- (135.97,172.49);

\path[draw=drawColor,draw opacity=0.50,line width= 0.4pt,line join=round,line cap=round] (133.95,170.47) -- (133.95,174.51);

\path[draw=drawColor,draw opacity=0.50,line width= 0.4pt,line join=round,line cap=round] (134.69,173.06) -- (137.55,175.91);

\path[draw=drawColor,draw opacity=0.50,line width= 0.4pt,line join=round,line cap=round] (134.69,175.91) -- (137.55,173.06);

\path[draw=drawColor,draw opacity=0.50,line width= 0.4pt,line join=round,line cap=round] (134.10,174.48) -- (138.14,174.48);

\path[draw=drawColor,draw opacity=0.50,line width= 0.4pt,line join=round,line cap=round] (136.12,172.47) -- (136.12,176.50);

\path[draw=drawColor,draw opacity=0.50,line width= 0.4pt,line join=round,line cap=round] (136.15,175.89) -- (139.00,178.74);

\path[draw=drawColor,draw opacity=0.50,line width= 0.4pt,line join=round,line cap=round] (136.15,178.74) -- (139.00,175.89);

\path[draw=drawColor,draw opacity=0.50,line width= 0.4pt,line join=round,line cap=round] (135.56,177.32) -- (139.60,177.32);

\path[draw=drawColor,draw opacity=0.50,line width= 0.4pt,line join=round,line cap=round] (137.58,175.30) -- (137.58,179.34);

\path[draw=drawColor,draw opacity=0.50,line width= 0.4pt,line join=round,line cap=round] (136.14,174.89) -- (138.99,177.74);

\path[draw=drawColor,draw opacity=0.50,line width= 0.4pt,line join=round,line cap=round] (136.14,177.74) -- (138.99,174.89);

\path[draw=drawColor,draw opacity=0.50,line width= 0.4pt,line join=round,line cap=round] (135.55,176.31) -- (139.58,176.31);

\path[draw=drawColor,draw opacity=0.50,line width= 0.4pt,line join=round,line cap=round] (137.56,174.30) -- (137.56,178.33);

\path[draw=drawColor,draw opacity=0.50,line width= 0.4pt,line join=round,line cap=round] (134.78,173.28) -- (137.64,176.14);

\path[draw=drawColor,draw opacity=0.50,line width= 0.4pt,line join=round,line cap=round] (134.78,176.14) -- (137.64,173.28);

\path[draw=drawColor,draw opacity=0.50,line width= 0.4pt,line join=round,line cap=round] (134.19,174.71) -- (138.23,174.71);

\path[draw=drawColor,draw opacity=0.50,line width= 0.4pt,line join=round,line cap=round] (136.21,172.69) -- (136.21,176.73);

\path[draw=drawColor,draw opacity=0.50,line width= 0.4pt,line join=round,line cap=round] (135.83,174.29) -- (138.69,177.14);

\path[draw=drawColor,draw opacity=0.50,line width= 0.4pt,line join=round,line cap=round] (135.83,177.14) -- (138.69,174.29);

\path[draw=drawColor,draw opacity=0.50,line width= 0.4pt,line join=round,line cap=round] (135.24,175.72) -- (139.28,175.72);

\path[draw=drawColor,draw opacity=0.50,line width= 0.4pt,line join=round,line cap=round] (137.26,173.70) -- (137.26,177.73);

\path[draw=drawColor,draw opacity=0.50,line width= 0.4pt,line join=round,line cap=round] (134.60,173.80) -- (137.45,176.65);

\path[draw=drawColor,draw opacity=0.50,line width= 0.4pt,line join=round,line cap=round] (134.60,176.65) -- (137.45,173.80);

\path[draw=drawColor,draw opacity=0.50,line width= 0.4pt,line join=round,line cap=round] (134.01,175.23) -- (138.04,175.23);

\path[draw=drawColor,draw opacity=0.50,line width= 0.4pt,line join=round,line cap=round] (136.03,173.21) -- (136.03,177.24);

\path[draw=drawColor,draw opacity=0.50,line width= 0.4pt,line join=round,line cap=round] (135.98,175.79) -- (138.83,178.64);

\path[draw=drawColor,draw opacity=0.50,line width= 0.4pt,line join=round,line cap=round] (135.98,178.64) -- (138.83,175.79);

\path[draw=drawColor,draw opacity=0.50,line width= 0.4pt,line join=round,line cap=round] (135.39,177.21) -- (139.42,177.21);

\path[draw=drawColor,draw opacity=0.50,line width= 0.4pt,line join=round,line cap=round] (137.41,175.20) -- (137.41,179.23);

\path[draw=drawColor,draw opacity=0.50,line width= 0.4pt,line join=round,line cap=round] (136.19,172.88) -- (139.05,175.74);

\path[draw=drawColor,draw opacity=0.50,line width= 0.4pt,line join=round,line cap=round] (136.19,175.74) -- (139.05,172.88);

\path[draw=drawColor,draw opacity=0.50,line width= 0.4pt,line join=round,line cap=round] (135.60,174.31) -- (139.64,174.31);

\path[draw=drawColor,draw opacity=0.50,line width= 0.4pt,line join=round,line cap=round] (137.62,172.29) -- (137.62,176.33);

\path[draw=drawColor,draw opacity=0.50,line width= 0.4pt,line join=round,line cap=round] (137.21,174.45) -- (140.06,177.30);

\path[draw=drawColor,draw opacity=0.50,line width= 0.4pt,line join=round,line cap=round] (137.21,177.30) -- (140.06,174.45);

\path[draw=drawColor,draw opacity=0.50,line width= 0.4pt,line join=round,line cap=round] (136.62,175.87) -- (140.65,175.87);

\path[draw=drawColor,draw opacity=0.50,line width= 0.4pt,line join=round,line cap=round] (138.64,173.86) -- (138.64,177.89);

\path[draw=drawColor,draw opacity=0.50,line width= 0.4pt,line join=round,line cap=round] (134.68,173.85) -- (137.53,176.70);

\path[draw=drawColor,draw opacity=0.50,line width= 0.4pt,line join=round,line cap=round] (134.68,176.70) -- (137.53,173.85);

\path[draw=drawColor,draw opacity=0.50,line width= 0.4pt,line join=round,line cap=round] (134.09,175.28) -- (138.12,175.28);

\path[draw=drawColor,draw opacity=0.50,line width= 0.4pt,line join=round,line cap=round] (136.11,173.26) -- (136.11,177.30);

\path[draw=drawColor,draw opacity=0.50,line width= 0.4pt,line join=round,line cap=round] (139.10,179.57) -- (141.95,182.43);

\path[draw=drawColor,draw opacity=0.50,line width= 0.4pt,line join=round,line cap=round] (139.10,182.43) -- (141.95,179.57);

\path[draw=drawColor,draw opacity=0.50,line width= 0.4pt,line join=round,line cap=round] (138.51,181.00) -- (142.54,181.00);

\path[draw=drawColor,draw opacity=0.50,line width= 0.4pt,line join=round,line cap=round] (140.53,178.98) -- (140.53,183.02);

\path[draw=drawColor,draw opacity=0.50,line width= 0.4pt,line join=round,line cap=round] (136.53,174.69) -- (139.38,177.55);

\path[draw=drawColor,draw opacity=0.50,line width= 0.4pt,line join=round,line cap=round] (136.53,177.55) -- (139.38,174.69);

\path[draw=drawColor,draw opacity=0.50,line width= 0.4pt,line join=round,line cap=round] (135.94,176.12) -- (139.98,176.12);

\path[draw=drawColor,draw opacity=0.50,line width= 0.4pt,line join=round,line cap=round] (137.96,174.10) -- (137.96,178.14);

\path[draw=drawColor,draw opacity=0.50,line width= 0.4pt,line join=round,line cap=round] (130.54,174.00) -- (133.39,176.85);

\path[draw=drawColor,draw opacity=0.50,line width= 0.4pt,line join=round,line cap=round] (130.54,176.85) -- (133.39,174.00);

\path[draw=drawColor,draw opacity=0.50,line width= 0.4pt,line join=round,line cap=round] (129.95,175.42) -- (133.98,175.42);

\path[draw=drawColor,draw opacity=0.50,line width= 0.4pt,line join=round,line cap=round] (131.96,173.41) -- (131.96,177.44);

\path[draw=drawColor,draw opacity=0.50,line width= 0.4pt,line join=round,line cap=round] (131.31,172.39) -- (134.16,175.24);

\path[draw=drawColor,draw opacity=0.50,line width= 0.4pt,line join=round,line cap=round] (131.31,175.24) -- (134.16,172.39);

\path[draw=drawColor,draw opacity=0.50,line width= 0.4pt,line join=round,line cap=round] (130.72,173.81) -- (134.75,173.81);

\path[draw=drawColor,draw opacity=0.50,line width= 0.4pt,line join=round,line cap=round] (132.73,171.80) -- (132.73,175.83);

\path[draw=drawColor,draw opacity=0.50,line width= 0.4pt,line join=round,line cap=round] (132.43,172.57) -- (135.28,175.42);

\path[draw=drawColor,draw opacity=0.50,line width= 0.4pt,line join=round,line cap=round] (132.43,175.42) -- (135.28,172.57);

\path[draw=drawColor,draw opacity=0.50,line width= 0.4pt,line join=round,line cap=round] (131.84,173.99) -- (135.87,173.99);

\path[draw=drawColor,draw opacity=0.50,line width= 0.4pt,line join=round,line cap=round] (133.86,171.97) -- (133.86,176.01);

\path[draw=drawColor,draw opacity=0.50,line width= 0.4pt,line join=round,line cap=round] (133.41,173.82) -- (136.27,176.68);

\path[draw=drawColor,draw opacity=0.50,line width= 0.4pt,line join=round,line cap=round] (133.41,176.68) -- (136.27,173.82);

\path[draw=drawColor,draw opacity=0.50,line width= 0.4pt,line join=round,line cap=round] (132.82,175.25) -- (136.86,175.25);

\path[draw=drawColor,draw opacity=0.50,line width= 0.4pt,line join=round,line cap=round] (134.84,173.23) -- (134.84,177.27);

\path[draw=drawColor,draw opacity=0.50,line width= 0.4pt,line join=round,line cap=round] (138.00,172.60) -- (140.85,175.45);

\path[draw=drawColor,draw opacity=0.50,line width= 0.4pt,line join=round,line cap=round] (138.00,175.45) -- (140.85,172.60);

\path[draw=drawColor,draw opacity=0.50,line width= 0.4pt,line join=round,line cap=round] (137.41,174.02) -- (141.44,174.02);

\path[draw=drawColor,draw opacity=0.50,line width= 0.4pt,line join=round,line cap=round] (139.42,172.01) -- (139.42,176.04);

\path[draw=drawColor,draw opacity=0.50,line width= 0.4pt,line join=round,line cap=round] (138.36,175.53) -- (141.21,178.38);

\path[draw=drawColor,draw opacity=0.50,line width= 0.4pt,line join=round,line cap=round] (138.36,178.38) -- (141.21,175.53);

\path[draw=drawColor,draw opacity=0.50,line width= 0.4pt,line join=round,line cap=round] (137.77,176.95) -- (141.80,176.95);

\path[draw=drawColor,draw opacity=0.50,line width= 0.4pt,line join=round,line cap=round] (139.79,174.94) -- (139.79,178.97);

\path[draw=drawColor,draw opacity=0.50,line width= 0.4pt,line join=round,line cap=round] (140.84,175.69) -- (143.69,178.54);

\path[draw=drawColor,draw opacity=0.50,line width= 0.4pt,line join=round,line cap=round] (140.84,178.54) -- (143.69,175.69);

\path[draw=drawColor,draw opacity=0.50,line width= 0.4pt,line join=round,line cap=round] (140.25,177.11) -- (144.28,177.11);

\path[draw=drawColor,draw opacity=0.50,line width= 0.4pt,line join=round,line cap=round] (142.26,175.10) -- (142.26,179.13);

\path[draw=drawColor,draw opacity=0.50,line width= 0.4pt,line join=round,line cap=round] (134.05,177.43) -- (136.90,180.28);

\path[draw=drawColor,draw opacity=0.50,line width= 0.4pt,line join=round,line cap=round] (134.05,180.28) -- (136.90,177.43);

\path[draw=drawColor,draw opacity=0.50,line width= 0.4pt,line join=round,line cap=round] (133.46,178.85) -- (137.49,178.85);

\path[draw=drawColor,draw opacity=0.50,line width= 0.4pt,line join=round,line cap=round] (135.47,176.83) -- (135.47,180.87);

\path[draw=drawColor,draw opacity=0.50,line width= 0.4pt,line join=round,line cap=round] (136.96,173.56) -- (139.81,176.41);

\path[draw=drawColor,draw opacity=0.50,line width= 0.4pt,line join=round,line cap=round] (136.96,176.41) -- (139.81,173.56);

\path[draw=drawColor,draw opacity=0.50,line width= 0.4pt,line join=round,line cap=round] (136.36,174.99) -- (140.40,174.99);

\path[draw=drawColor,draw opacity=0.50,line width= 0.4pt,line join=round,line cap=round] (138.38,172.97) -- (138.38,177.00);

\path[draw=drawColor,draw opacity=0.50,line width= 0.4pt,line join=round,line cap=round] (147.97,192.68) -- (150.83,195.53);

\path[draw=drawColor,draw opacity=0.50,line width= 0.4pt,line join=round,line cap=round] (147.97,195.53) -- (150.83,192.68);

\path[draw=drawColor,draw opacity=0.50,line width= 0.4pt,line join=round,line cap=round] (147.38,194.10) -- (151.42,194.10);

\path[draw=drawColor,draw opacity=0.50,line width= 0.4pt,line join=round,line cap=round] (149.40,192.09) -- (149.40,196.12);

\path[draw=drawColor,draw opacity=0.50,line width= 0.4pt,line join=round,line cap=round] (146.70,192.05) -- (149.55,194.90);

\path[draw=drawColor,draw opacity=0.50,line width= 0.4pt,line join=round,line cap=round] (146.70,194.90) -- (149.55,192.05);

\path[draw=drawColor,draw opacity=0.50,line width= 0.4pt,line join=round,line cap=round] (146.11,193.48) -- (150.14,193.48);

\path[draw=drawColor,draw opacity=0.50,line width= 0.4pt,line join=round,line cap=round] (148.13,191.46) -- (148.13,195.49);

\path[draw=drawColor,draw opacity=0.50,line width= 0.4pt,line join=round,line cap=round] (148.16,194.64) -- (151.02,197.50);

\path[draw=drawColor,draw opacity=0.50,line width= 0.4pt,line join=round,line cap=round] (148.16,197.50) -- (151.02,194.64);

\path[draw=drawColor,draw opacity=0.50,line width= 0.4pt,line join=round,line cap=round] (147.57,196.07) -- (151.61,196.07);

\path[draw=drawColor,draw opacity=0.50,line width= 0.4pt,line join=round,line cap=round] (149.59,194.05) -- (149.59,198.09);

\path[draw=drawColor,draw opacity=0.50,line width= 0.4pt,line join=round,line cap=round] (146.42,190.24) -- (149.27,193.09);

\path[draw=drawColor,draw opacity=0.50,line width= 0.4pt,line join=round,line cap=round] (146.42,193.09) -- (149.27,190.24);

\path[draw=drawColor,draw opacity=0.50,line width= 0.4pt,line join=round,line cap=round] (145.83,191.67) -- (149.86,191.67);

\path[draw=drawColor,draw opacity=0.50,line width= 0.4pt,line join=round,line cap=round] (147.84,189.65) -- (147.84,193.68);

\path[draw=drawColor,draw opacity=0.50,line width= 0.4pt,line join=round,line cap=round] (146.81,191.71) -- (149.67,194.57);

\path[draw=drawColor,draw opacity=0.50,line width= 0.4pt,line join=round,line cap=round] (146.81,194.57) -- (149.67,191.71);

\path[draw=drawColor,draw opacity=0.50,line width= 0.4pt,line join=round,line cap=round] (146.22,193.14) -- (150.26,193.14);

\path[draw=drawColor,draw opacity=0.50,line width= 0.4pt,line join=round,line cap=round] (148.24,191.12) -- (148.24,195.16);

\path[draw=drawColor,draw opacity=0.50,line width= 0.4pt,line join=round,line cap=round] (146.81,191.56) -- (149.67,194.42);

\path[draw=drawColor,draw opacity=0.50,line width= 0.4pt,line join=round,line cap=round] (146.81,194.42) -- (149.67,191.56);

\path[draw=drawColor,draw opacity=0.50,line width= 0.4pt,line join=round,line cap=round] (146.22,192.99) -- (150.26,192.99);

\path[draw=drawColor,draw opacity=0.50,line width= 0.4pt,line join=round,line cap=round] (148.24,190.97) -- (148.24,195.01);

\path[draw=drawColor,draw opacity=0.50,line width= 0.4pt,line join=round,line cap=round] (141.15,191.25) -- (144.00,194.10);

\path[draw=drawColor,draw opacity=0.50,line width= 0.4pt,line join=round,line cap=round] (141.15,194.10) -- (144.00,191.25);

\path[draw=drawColor,draw opacity=0.50,line width= 0.4pt,line join=round,line cap=round] (140.56,192.67) -- (144.60,192.67);

\path[draw=drawColor,draw opacity=0.50,line width= 0.4pt,line join=round,line cap=round] (142.58,190.66) -- (142.58,194.69);

\path[draw=drawColor,draw opacity=0.50,line width= 0.4pt,line join=round,line cap=round] (146.15,194.93) -- (149.00,197.78);

\path[draw=drawColor,draw opacity=0.50,line width= 0.4pt,line join=round,line cap=round] (146.15,197.78) -- (149.00,194.93);

\path[draw=drawColor,draw opacity=0.50,line width= 0.4pt,line join=round,line cap=round] (145.56,196.36) -- (149.60,196.36);

\path[draw=drawColor,draw opacity=0.50,line width= 0.4pt,line join=round,line cap=round] (147.58,194.34) -- (147.58,198.37);

\path[draw=drawColor,draw opacity=0.50,line width= 0.4pt,line join=round,line cap=round] (143.45,192.37) -- (146.31,195.22);

\path[draw=drawColor,draw opacity=0.50,line width= 0.4pt,line join=round,line cap=round] (143.45,195.22) -- (146.31,192.37);

\path[draw=drawColor,draw opacity=0.50,line width= 0.4pt,line join=round,line cap=round] (142.86,193.79) -- (146.90,193.79);

\path[draw=drawColor,draw opacity=0.50,line width= 0.4pt,line join=round,line cap=round] (144.88,191.78) -- (144.88,195.81);

\path[draw=drawColor,draw opacity=0.50,line width= 0.4pt,line join=round,line cap=round] (140.14,188.99) -- (142.99,191.84);

\path[draw=drawColor,draw opacity=0.50,line width= 0.4pt,line join=round,line cap=round] (140.14,191.84) -- (142.99,188.99);

\path[draw=drawColor,draw opacity=0.50,line width= 0.4pt,line join=round,line cap=round] (139.55,190.42) -- (143.58,190.42);

\path[draw=drawColor,draw opacity=0.50,line width= 0.4pt,line join=round,line cap=round] (141.56,188.40) -- (141.56,192.43);

\path[draw=drawColor,draw opacity=0.50,line width= 0.4pt,line join=round,line cap=round] (141.29,193.99) -- (144.14,196.84);

\path[draw=drawColor,draw opacity=0.50,line width= 0.4pt,line join=round,line cap=round] (141.29,196.84) -- (144.14,193.99);

\path[draw=drawColor,draw opacity=0.50,line width= 0.4pt,line join=round,line cap=round] (140.70,195.41) -- (144.74,195.41);

\path[draw=drawColor,draw opacity=0.50,line width= 0.4pt,line join=round,line cap=round] (142.72,193.39) -- (142.72,197.43);

\path[draw=drawColor,draw opacity=0.50,line width= 0.4pt,line join=round,line cap=round] (145.51,191.67) -- (148.37,194.53);

\path[draw=drawColor,draw opacity=0.50,line width= 0.4pt,line join=round,line cap=round] (145.51,194.53) -- (148.37,191.67);

\path[draw=drawColor,draw opacity=0.50,line width= 0.4pt,line join=round,line cap=round] (144.92,193.10) -- (148.96,193.10);

\path[draw=drawColor,draw opacity=0.50,line width= 0.4pt,line join=round,line cap=round] (146.94,191.08) -- (146.94,195.12);

\path[draw=drawColor,draw opacity=0.50,line width= 0.4pt,line join=round,line cap=round] (149.65,191.66) -- (152.51,194.51);

\path[draw=drawColor,draw opacity=0.50,line width= 0.4pt,line join=round,line cap=round] (149.65,194.51) -- (152.51,191.66);

\path[draw=drawColor,draw opacity=0.50,line width= 0.4pt,line join=round,line cap=round] (149.06,193.08) -- (153.10,193.08);

\path[draw=drawColor,draw opacity=0.50,line width= 0.4pt,line join=round,line cap=round] (151.08,191.07) -- (151.08,195.10);

\path[draw=drawColor,draw opacity=0.50,line width= 0.4pt,line join=round,line cap=round] (148.58,191.21) -- (151.43,194.06);

\path[draw=drawColor,draw opacity=0.50,line width= 0.4pt,line join=round,line cap=round] (148.58,194.06) -- (151.43,191.21);

\path[draw=drawColor,draw opacity=0.50,line width= 0.4pt,line join=round,line cap=round] (147.99,192.63) -- (152.02,192.63);

\path[draw=drawColor,draw opacity=0.50,line width= 0.4pt,line join=round,line cap=round] (150.01,190.61) -- (150.01,194.65);

\path[draw=drawColor,draw opacity=0.50,line width= 0.4pt,line join=round,line cap=round] (148.88,192.74) -- (151.74,195.59);

\path[draw=drawColor,draw opacity=0.50,line width= 0.4pt,line join=round,line cap=round] (148.88,195.59) -- (151.74,192.74);

\path[draw=drawColor,draw opacity=0.50,line width= 0.4pt,line join=round,line cap=round] (148.29,194.16) -- (152.33,194.16);

\path[draw=drawColor,draw opacity=0.50,line width= 0.4pt,line join=round,line cap=round] (150.31,192.14) -- (150.31,196.18);

\path[draw=drawColor,draw opacity=0.50,line width= 0.4pt,line join=round,line cap=round] (145.11,193.94) -- (147.96,196.80);

\path[draw=drawColor,draw opacity=0.50,line width= 0.4pt,line join=round,line cap=round] (145.11,196.80) -- (147.96,193.94);

\path[draw=drawColor,draw opacity=0.50,line width= 0.4pt,line join=round,line cap=round] (144.52,195.37) -- (148.55,195.37);

\path[draw=drawColor,draw opacity=0.50,line width= 0.4pt,line join=round,line cap=round] (146.53,193.35) -- (146.53,197.39);

\path[draw=drawColor,draw opacity=0.50,line width= 0.4pt,line join=round,line cap=round] (150.40,193.81) -- (153.25,196.66);

\path[draw=drawColor,draw opacity=0.50,line width= 0.4pt,line join=round,line cap=round] (150.40,196.66) -- (153.25,193.81);

\path[draw=drawColor,draw opacity=0.50,line width= 0.4pt,line join=round,line cap=round] (149.81,195.24) -- (153.84,195.24);

\path[draw=drawColor,draw opacity=0.50,line width= 0.4pt,line join=round,line cap=round] (151.82,193.22) -- (151.82,197.25);

\path[draw=drawColor,draw opacity=0.50,line width= 0.4pt,line join=round,line cap=round] (148.83,190.82) -- (151.69,193.67);

\path[draw=drawColor,draw opacity=0.50,line width= 0.4pt,line join=round,line cap=round] (148.83,193.67) -- (151.69,190.82);

\path[draw=drawColor,draw opacity=0.50,line width= 0.4pt,line join=round,line cap=round] (148.24,192.25) -- (152.28,192.25);

\path[draw=drawColor,draw opacity=0.50,line width= 0.4pt,line join=round,line cap=round] (150.26,190.23) -- (150.26,194.26);

\path[draw=drawColor,draw opacity=0.50,line width= 0.4pt,line join=round,line cap=round] (146.62,193.71) -- (149.47,196.56);

\path[draw=drawColor,draw opacity=0.50,line width= 0.4pt,line join=round,line cap=round] (146.62,196.56) -- (149.47,193.71);

\path[draw=drawColor,draw opacity=0.50,line width= 0.4pt,line join=round,line cap=round] (146.03,195.14) -- (150.06,195.14);

\path[draw=drawColor,draw opacity=0.50,line width= 0.4pt,line join=round,line cap=round] (148.04,193.12) -- (148.04,197.16);

\path[draw=drawColor,draw opacity=0.50,line width= 0.4pt,line join=round,line cap=round] (149.13,192.70) -- (151.98,195.55);

\path[draw=drawColor,draw opacity=0.50,line width= 0.4pt,line join=round,line cap=round] (149.13,195.55) -- (151.98,192.70);

\path[draw=drawColor,draw opacity=0.50,line width= 0.4pt,line join=round,line cap=round] (148.54,194.12) -- (152.58,194.12);

\path[draw=drawColor,draw opacity=0.50,line width= 0.4pt,line join=round,line cap=round] (150.56,192.11) -- (150.56,196.14);

\path[draw=drawColor,draw opacity=0.50,line width= 0.4pt,line join=round,line cap=round] (149.82,197.59) -- (152.67,200.44);

\path[draw=drawColor,draw opacity=0.50,line width= 0.4pt,line join=round,line cap=round] (149.82,200.44) -- (152.67,197.59);

\path[draw=drawColor,draw opacity=0.50,line width= 0.4pt,line join=round,line cap=round] (149.23,199.01) -- (153.26,199.01);

\path[draw=drawColor,draw opacity=0.50,line width= 0.4pt,line join=round,line cap=round] (151.25,197.00) -- (151.25,201.03);

\path[draw=drawColor,draw opacity=0.50,line width= 0.4pt,line join=round,line cap=round] (148.62,193.73) -- (151.47,196.58);

\path[draw=drawColor,draw opacity=0.50,line width= 0.4pt,line join=round,line cap=round] (148.62,196.58) -- (151.47,193.73);

\path[draw=drawColor,draw opacity=0.50,line width= 0.4pt,line join=round,line cap=round] (148.02,195.16) -- (152.06,195.16);

\path[draw=drawColor,draw opacity=0.50,line width= 0.4pt,line join=round,line cap=round] (150.04,193.14) -- (150.04,197.18);

\path[draw=drawColor,draw opacity=0.50,line width= 0.4pt,line join=round,line cap=round] (145.81,192.89) -- (148.66,195.74);

\path[draw=drawColor,draw opacity=0.50,line width= 0.4pt,line join=round,line cap=round] (145.81,195.74) -- (148.66,192.89);

\path[draw=drawColor,draw opacity=0.50,line width= 0.4pt,line join=round,line cap=round] (145.22,194.32) -- (149.25,194.32);

\path[draw=drawColor,draw opacity=0.50,line width= 0.4pt,line join=round,line cap=round] (147.24,192.30) -- (147.24,196.34);

\path[draw=drawColor,draw opacity=0.50,line width= 0.4pt,line join=round,line cap=round] (141.12,193.14) -- (143.97,196.00);

\path[draw=drawColor,draw opacity=0.50,line width= 0.4pt,line join=round,line cap=round] (141.12,196.00) -- (143.97,193.14);

\path[draw=drawColor,draw opacity=0.50,line width= 0.4pt,line join=round,line cap=round] (140.53,194.57) -- (144.56,194.57);

\path[draw=drawColor,draw opacity=0.50,line width= 0.4pt,line join=round,line cap=round] (142.54,192.55) -- (142.54,196.59);

\path[draw=drawColor,draw opacity=0.50,line width= 0.4pt,line join=round,line cap=round] (141.79,193.42) -- (144.64,196.27);

\path[draw=drawColor,draw opacity=0.50,line width= 0.4pt,line join=round,line cap=round] (141.79,196.27) -- (144.64,193.42);

\path[draw=drawColor,draw opacity=0.50,line width= 0.4pt,line join=round,line cap=round] (141.20,194.84) -- (145.24,194.84);

\path[draw=drawColor,draw opacity=0.50,line width= 0.4pt,line join=round,line cap=round] (143.22,192.82) -- (143.22,196.86);

\path[draw=drawColor,draw opacity=0.50,line width= 0.4pt,line join=round,line cap=round] (142.22,190.72) -- (145.08,193.58);

\path[draw=drawColor,draw opacity=0.50,line width= 0.4pt,line join=round,line cap=round] (142.22,193.58) -- (145.08,190.72);

\path[draw=drawColor,draw opacity=0.50,line width= 0.4pt,line join=round,line cap=round] (141.63,192.15) -- (145.67,192.15);

\path[draw=drawColor,draw opacity=0.50,line width= 0.4pt,line join=round,line cap=round] (143.65,190.13) -- (143.65,194.17);

\path[draw=drawColor,draw opacity=0.50,line width= 0.4pt,line join=round,line cap=round] (142.34,193.78) -- (145.19,196.63);

\path[draw=drawColor,draw opacity=0.50,line width= 0.4pt,line join=round,line cap=round] (142.34,196.63) -- (145.19,193.78);

\path[draw=drawColor,draw opacity=0.50,line width= 0.4pt,line join=round,line cap=round] (141.75,195.21) -- (145.78,195.21);

\path[draw=drawColor,draw opacity=0.50,line width= 0.4pt,line join=round,line cap=round] (143.77,193.19) -- (143.77,197.22);

\path[draw=drawColor,draw opacity=0.50,line width= 0.4pt,line join=round,line cap=round] (147.41,192.10) -- (150.27,194.95);

\path[draw=drawColor,draw opacity=0.50,line width= 0.4pt,line join=round,line cap=round] (147.41,194.95) -- (150.27,192.10);

\path[draw=drawColor,draw opacity=0.50,line width= 0.4pt,line join=round,line cap=round] (146.82,193.52) -- (150.86,193.52);

\path[draw=drawColor,draw opacity=0.50,line width= 0.4pt,line join=round,line cap=round] (148.84,191.51) -- (148.84,195.54);

\path[draw=drawColor,draw opacity=0.50,line width= 0.4pt,line join=round,line cap=round] (148.59,192.64) -- (151.44,195.49);

\path[draw=drawColor,draw opacity=0.50,line width= 0.4pt,line join=round,line cap=round] (148.59,195.49) -- (151.44,192.64);

\path[draw=drawColor,draw opacity=0.50,line width= 0.4pt,line join=round,line cap=round] (148.00,194.07) -- (152.03,194.07);

\path[draw=drawColor,draw opacity=0.50,line width= 0.4pt,line join=round,line cap=round] (150.02,192.05) -- (150.02,196.08);

\path[draw=drawColor,draw opacity=0.50,line width= 0.4pt,line join=round,line cap=round] (147.46,191.70) -- (150.31,194.55);

\path[draw=drawColor,draw opacity=0.50,line width= 0.4pt,line join=round,line cap=round] (147.46,194.55) -- (150.31,191.70);

\path[draw=drawColor,draw opacity=0.50,line width= 0.4pt,line join=round,line cap=round] (146.87,193.13) -- (150.90,193.13);

\path[draw=drawColor,draw opacity=0.50,line width= 0.4pt,line join=round,line cap=round] (148.88,191.11) -- (148.88,195.15);

\path[draw=drawColor,draw opacity=0.50,line width= 0.4pt,line join=round,line cap=round] (146.00,192.97) -- (148.85,195.82);

\path[draw=drawColor,draw opacity=0.50,line width= 0.4pt,line join=round,line cap=round] (146.00,195.82) -- (148.85,192.97);

\path[draw=drawColor,draw opacity=0.50,line width= 0.4pt,line join=round,line cap=round] (145.41,194.39) -- (149.44,194.39);

\path[draw=drawColor,draw opacity=0.50,line width= 0.4pt,line join=round,line cap=round] (147.42,192.37) -- (147.42,196.41);

\path[draw=drawColor,draw opacity=0.50,line width= 0.4pt,line join=round,line cap=round] (145.99,194.55) -- (148.84,197.41);

\path[draw=drawColor,draw opacity=0.50,line width= 0.4pt,line join=round,line cap=round] (145.99,197.41) -- (148.84,194.55);

\path[draw=drawColor,draw opacity=0.50,line width= 0.4pt,line join=round,line cap=round] (145.40,195.98) -- (149.43,195.98);

\path[draw=drawColor,draw opacity=0.50,line width= 0.4pt,line join=round,line cap=round] (147.42,193.96) -- (147.42,198.00);

\path[draw=drawColor,draw opacity=0.50,line width= 0.4pt,line join=round,line cap=round] (149.12,196.86) -- (151.98,199.71);

\path[draw=drawColor,draw opacity=0.50,line width= 0.4pt,line join=round,line cap=round] (149.12,199.71) -- (151.98,196.86);

\path[draw=drawColor,draw opacity=0.50,line width= 0.4pt,line join=round,line cap=round] (148.53,198.29) -- (152.57,198.29);

\path[draw=drawColor,draw opacity=0.50,line width= 0.4pt,line join=round,line cap=round] (150.55,196.27) -- (150.55,200.31);

\path[draw=drawColor,draw opacity=0.50,line width= 0.4pt,line join=round,line cap=round] (149.82,196.17) -- (152.68,199.02);

\path[draw=drawColor,draw opacity=0.50,line width= 0.4pt,line join=round,line cap=round] (149.82,199.02) -- (152.68,196.17);

\path[draw=drawColor,draw opacity=0.50,line width= 0.4pt,line join=round,line cap=round] (149.23,197.60) -- (153.27,197.60);

\path[draw=drawColor,draw opacity=0.50,line width= 0.4pt,line join=round,line cap=round] (151.25,195.58) -- (151.25,199.61);

\path[draw=drawColor,draw opacity=0.50,line width= 0.4pt,line join=round,line cap=round] (148.61,196.85) -- (151.46,199.70);

\path[draw=drawColor,draw opacity=0.50,line width= 0.4pt,line join=round,line cap=round] (148.61,199.70) -- (151.46,196.85);

\path[draw=drawColor,draw opacity=0.50,line width= 0.4pt,line join=round,line cap=round] (148.02,198.28) -- (152.05,198.28);

\path[draw=drawColor,draw opacity=0.50,line width= 0.4pt,line join=round,line cap=round] (150.04,196.26) -- (150.04,200.29);

\path[draw=drawColor,draw opacity=0.50,line width= 0.4pt,line join=round,line cap=round] (147.95,193.45) -- (150.81,196.30);

\path[draw=drawColor,draw opacity=0.50,line width= 0.4pt,line join=round,line cap=round] (147.95,196.30) -- (150.81,193.45);

\path[draw=drawColor,draw opacity=0.50,line width= 0.4pt,line join=round,line cap=round] (147.36,194.87) -- (151.40,194.87);

\path[draw=drawColor,draw opacity=0.50,line width= 0.4pt,line join=round,line cap=round] (149.38,192.86) -- (149.38,196.89);

\path[draw=drawColor,draw opacity=0.50,line width= 0.4pt,line join=round,line cap=round] (148.82,198.27) -- (151.67,201.13);

\path[draw=drawColor,draw opacity=0.50,line width= 0.4pt,line join=round,line cap=round] (148.82,201.13) -- (151.67,198.27);

\path[draw=drawColor,draw opacity=0.50,line width= 0.4pt,line join=round,line cap=round] (148.23,199.70) -- (152.26,199.70);

\path[draw=drawColor,draw opacity=0.50,line width= 0.4pt,line join=round,line cap=round] (150.24,197.68) -- (150.24,201.72);

\path[draw=drawColor,draw opacity=0.50,line width= 0.4pt,line join=round,line cap=round] (149.78,194.34) -- (152.64,197.20);

\path[draw=drawColor,draw opacity=0.50,line width= 0.4pt,line join=round,line cap=round] (149.78,197.20) -- (152.64,194.34);

\path[draw=drawColor,draw opacity=0.50,line width= 0.4pt,line join=round,line cap=round] (149.19,195.77) -- (153.23,195.77);

\path[draw=drawColor,draw opacity=0.50,line width= 0.4pt,line join=round,line cap=round] (151.21,193.75) -- (151.21,197.79);

\path[draw=drawColor,draw opacity=0.50,line width= 0.4pt,line join=round,line cap=round] (140.90,191.31) -- (143.75,194.16);

\path[draw=drawColor,draw opacity=0.50,line width= 0.4pt,line join=round,line cap=round] (140.90,194.16) -- (143.75,191.31);

\path[draw=drawColor,draw opacity=0.50,line width= 0.4pt,line join=round,line cap=round] (140.31,192.73) -- (144.34,192.73);

\path[draw=drawColor,draw opacity=0.50,line width= 0.4pt,line join=round,line cap=round] (142.32,190.71) -- (142.32,194.75);

\path[draw=drawColor,draw opacity=0.50,line width= 0.4pt,line join=round,line cap=round] (139.54,195.59) -- (142.40,198.44);

\path[draw=drawColor,draw opacity=0.50,line width= 0.4pt,line join=round,line cap=round] (139.54,198.44) -- (142.40,195.59);

\path[draw=drawColor,draw opacity=0.50,line width= 0.4pt,line join=round,line cap=round] (138.95,197.02) -- (142.99,197.02);

\path[draw=drawColor,draw opacity=0.50,line width= 0.4pt,line join=round,line cap=round] (140.97,195.00) -- (140.97,199.04);

\path[draw=drawColor,draw opacity=0.50,line width= 0.4pt,line join=round,line cap=round] (140.29,196.82) -- (143.14,199.68);

\path[draw=drawColor,draw opacity=0.50,line width= 0.4pt,line join=round,line cap=round] (140.29,199.68) -- (143.14,196.82);

\path[draw=drawColor,draw opacity=0.50,line width= 0.4pt,line join=round,line cap=round] (139.70,198.25) -- (143.73,198.25);

\path[draw=drawColor,draw opacity=0.50,line width= 0.4pt,line join=round,line cap=round] (141.72,196.23) -- (141.72,200.27);

\path[draw=drawColor,draw opacity=0.50,line width= 0.4pt,line join=round,line cap=round] (142.50,192.61) -- (145.36,195.46);

\path[draw=drawColor,draw opacity=0.50,line width= 0.4pt,line join=round,line cap=round] (142.50,195.46) -- (145.36,192.61);

\path[draw=drawColor,draw opacity=0.50,line width= 0.4pt,line join=round,line cap=round] (141.91,194.03) -- (145.95,194.03);

\path[draw=drawColor,draw opacity=0.50,line width= 0.4pt,line join=round,line cap=round] (143.93,192.02) -- (143.93,196.05);

\path[draw=drawColor,draw opacity=0.50,line width= 0.4pt,line join=round,line cap=round] (143.78,193.19) -- (146.63,196.05);

\path[draw=drawColor,draw opacity=0.50,line width= 0.4pt,line join=round,line cap=round] (143.78,196.05) -- (146.63,193.19);

\path[draw=drawColor,draw opacity=0.50,line width= 0.4pt,line join=round,line cap=round] (143.19,194.62) -- (147.22,194.62);

\path[draw=drawColor,draw opacity=0.50,line width= 0.4pt,line join=round,line cap=round] (145.21,192.60) -- (145.21,196.64);

\path[draw=drawColor,draw opacity=0.50,line width= 0.4pt,line join=round,line cap=round] (148.08,193.33) -- (150.93,196.19);

\path[draw=drawColor,draw opacity=0.50,line width= 0.4pt,line join=round,line cap=round] (148.08,196.19) -- (150.93,193.33);

\path[draw=drawColor,draw opacity=0.50,line width= 0.4pt,line join=round,line cap=round] (147.49,194.76) -- (151.52,194.76);

\path[draw=drawColor,draw opacity=0.50,line width= 0.4pt,line join=round,line cap=round] (149.50,192.74) -- (149.50,196.78);

\path[draw=drawColor,draw opacity=0.50,line width= 0.4pt,line join=round,line cap=round] (148.26,193.38) -- (151.11,196.23);

\path[draw=drawColor,draw opacity=0.50,line width= 0.4pt,line join=round,line cap=round] (148.26,196.23) -- (151.11,193.38);

\path[draw=drawColor,draw opacity=0.50,line width= 0.4pt,line join=round,line cap=round] (147.67,194.81) -- (151.70,194.81);

\path[draw=drawColor,draw opacity=0.50,line width= 0.4pt,line join=round,line cap=round] (149.68,192.79) -- (149.68,196.82);

\path[draw=drawColor,draw opacity=0.50,line width= 0.4pt,line join=round,line cap=round] (147.35,195.52) -- (150.21,198.37);

\path[draw=drawColor,draw opacity=0.50,line width= 0.4pt,line join=round,line cap=round] (147.35,198.37) -- (150.21,195.52);

\path[draw=drawColor,draw opacity=0.50,line width= 0.4pt,line join=round,line cap=round] (146.76,196.94) -- (150.80,196.94);

\path[draw=drawColor,draw opacity=0.50,line width= 0.4pt,line join=round,line cap=round] (148.78,194.93) -- (148.78,198.96);

\path[draw=drawColor,draw opacity=0.50,line width= 0.4pt,line join=round,line cap=round] (146.41,195.64) -- (149.27,198.49);

\path[draw=drawColor,draw opacity=0.50,line width= 0.4pt,line join=round,line cap=round] (146.41,198.49) -- (149.27,195.64);

\path[draw=drawColor,draw opacity=0.50,line width= 0.4pt,line join=round,line cap=round] (145.82,197.07) -- (149.86,197.07);

\path[draw=drawColor,draw opacity=0.50,line width= 0.4pt,line join=round,line cap=round] (147.84,195.05) -- (147.84,199.08);

\path[draw=drawColor,draw opacity=0.50,line width= 0.4pt,line join=round,line cap=round] (149.43,199.49) -- (152.28,202.34);

\path[draw=drawColor,draw opacity=0.50,line width= 0.4pt,line join=round,line cap=round] (149.43,202.34) -- (152.28,199.49);

\path[draw=drawColor,draw opacity=0.50,line width= 0.4pt,line join=round,line cap=round] (148.84,200.91) -- (152.87,200.91);

\path[draw=drawColor,draw opacity=0.50,line width= 0.4pt,line join=round,line cap=round] (150.86,198.89) -- (150.86,202.93);

\path[draw=drawColor,draw opacity=0.50,line width= 0.4pt,line join=round,line cap=round] (152.75,204.00) -- (155.60,206.86);

\path[draw=drawColor,draw opacity=0.50,line width= 0.4pt,line join=round,line cap=round] (152.75,206.86) -- (155.60,204.00);

\path[draw=drawColor,draw opacity=0.50,line width= 0.4pt,line join=round,line cap=round] (152.16,205.43) -- (156.19,205.43);

\path[draw=drawColor,draw opacity=0.50,line width= 0.4pt,line join=round,line cap=round] (154.18,203.41) -- (154.18,207.45);

\path[draw=drawColor,draw opacity=0.50,line width= 0.4pt,line join=round,line cap=round] (152.55,211.01) -- (155.41,213.87);

\path[draw=drawColor,draw opacity=0.50,line width= 0.4pt,line join=round,line cap=round] (152.55,213.87) -- (155.41,211.01);

\path[draw=drawColor,draw opacity=0.50,line width= 0.4pt,line join=round,line cap=round] (151.96,212.44) -- (156.00,212.44);

\path[draw=drawColor,draw opacity=0.50,line width= 0.4pt,line join=round,line cap=round] (153.98,210.42) -- (153.98,214.46);

\path[draw=drawColor,draw opacity=0.50,line width= 0.4pt,line join=round,line cap=round] (152.47,204.24) -- (155.33,207.09);

\path[draw=drawColor,draw opacity=0.50,line width= 0.4pt,line join=round,line cap=round] (152.47,207.09) -- (155.33,204.24);

\path[draw=drawColor,draw opacity=0.50,line width= 0.4pt,line join=round,line cap=round] (151.88,205.66) -- (155.92,205.66);

\path[draw=drawColor,draw opacity=0.50,line width= 0.4pt,line join=round,line cap=round] (153.90,203.64) -- (153.90,207.68);

\path[draw=drawColor,draw opacity=0.50,line width= 0.4pt,line join=round,line cap=round] (153.41,199.83) -- (156.26,202.68);

\path[draw=drawColor,draw opacity=0.50,line width= 0.4pt,line join=round,line cap=round] (153.41,202.68) -- (156.26,199.83);

\path[draw=drawColor,draw opacity=0.50,line width= 0.4pt,line join=round,line cap=round] (152.82,201.26) -- (156.85,201.26);

\path[draw=drawColor,draw opacity=0.50,line width= 0.4pt,line join=round,line cap=round] (154.84,199.24) -- (154.84,203.28);

\path[draw=drawColor,draw opacity=0.50,line width= 0.4pt,line join=round,line cap=round] (154.21,201.48) -- (157.06,204.33);

\path[draw=drawColor,draw opacity=0.50,line width= 0.4pt,line join=round,line cap=round] (154.21,204.33) -- (157.06,201.48);

\path[draw=drawColor,draw opacity=0.50,line width= 0.4pt,line join=round,line cap=round] (153.61,202.91) -- (157.65,202.91);

\path[draw=drawColor,draw opacity=0.50,line width= 0.4pt,line join=round,line cap=round] (155.63,200.89) -- (155.63,204.92);

\path[draw=drawColor,draw opacity=0.50,line width= 0.4pt,line join=round,line cap=round] (154.13,201.69) -- (156.98,204.54);

\path[draw=drawColor,draw opacity=0.50,line width= 0.4pt,line join=round,line cap=round] (154.13,204.54) -- (156.98,201.69);

\path[draw=drawColor,draw opacity=0.50,line width= 0.4pt,line join=round,line cap=round] (153.54,203.11) -- (157.57,203.11);

\path[draw=drawColor,draw opacity=0.50,line width= 0.4pt,line join=round,line cap=round] (155.56,201.09) -- (155.56,205.13);

\path[draw=drawColor,draw opacity=0.50,line width= 0.4pt,line join=round,line cap=round] (143.54,199.25) -- (146.39,202.10);

\path[draw=drawColor,draw opacity=0.50,line width= 0.4pt,line join=round,line cap=round] (143.54,202.10) -- (146.39,199.25);

\path[draw=drawColor,draw opacity=0.50,line width= 0.4pt,line join=round,line cap=round] (142.95,200.68) -- (146.98,200.68);

\path[draw=drawColor,draw opacity=0.50,line width= 0.4pt,line join=round,line cap=round] (144.96,198.66) -- (144.96,202.70);

\path[draw=drawColor,draw opacity=0.50,line width= 0.4pt,line join=round,line cap=round] (151.99,197.37) -- (154.84,200.22);

\path[draw=drawColor,draw opacity=0.50,line width= 0.4pt,line join=round,line cap=round] (151.99,200.22) -- (154.84,197.37);

\path[draw=drawColor,draw opacity=0.50,line width= 0.4pt,line join=round,line cap=round] (151.40,198.80) -- (155.43,198.80);

\path[draw=drawColor,draw opacity=0.50,line width= 0.4pt,line join=round,line cap=round] (153.42,196.78) -- (153.42,200.81);

\path[draw=drawColor,draw opacity=0.50,line width= 0.4pt,line join=round,line cap=round] (142.93,197.28) -- (145.79,200.14);

\path[draw=drawColor,draw opacity=0.50,line width= 0.4pt,line join=round,line cap=round] (142.93,200.14) -- (145.79,197.28);

\path[draw=drawColor,draw opacity=0.50,line width= 0.4pt,line join=round,line cap=round] (142.34,198.71) -- (146.38,198.71);

\path[draw=drawColor,draw opacity=0.50,line width= 0.4pt,line join=round,line cap=round] (144.36,196.69) -- (144.36,200.73);

\path[draw=drawColor,draw opacity=0.50,line width= 0.4pt,line join=round,line cap=round] (143.52,202.36) -- (146.37,205.21);

\path[draw=drawColor,draw opacity=0.50,line width= 0.4pt,line join=round,line cap=round] (143.52,205.21) -- (146.37,202.36);

\path[draw=drawColor,draw opacity=0.50,line width= 0.4pt,line join=round,line cap=round] (142.93,203.78) -- (146.96,203.78);

\path[draw=drawColor,draw opacity=0.50,line width= 0.4pt,line join=round,line cap=round] (144.95,201.77) -- (144.95,205.80);

\path[draw=drawColor,draw opacity=0.50,line width= 0.4pt,line join=round,line cap=round] (144.32,200.25) -- (147.17,203.11);

\path[draw=drawColor,draw opacity=0.50,line width= 0.4pt,line join=round,line cap=round] (144.32,203.11) -- (147.17,200.25);

\path[draw=drawColor,draw opacity=0.50,line width= 0.4pt,line join=round,line cap=round] (143.73,201.68) -- (147.76,201.68);

\path[draw=drawColor,draw opacity=0.50,line width= 0.4pt,line join=round,line cap=round] (145.74,199.66) -- (145.74,203.70);

\path[draw=drawColor,draw opacity=0.50,line width= 0.4pt,line join=round,line cap=round] (152.27,198.02) -- (155.12,200.88);

\path[draw=drawColor,draw opacity=0.50,line width= 0.4pt,line join=round,line cap=round] (152.27,200.88) -- (155.12,198.02);

\path[draw=drawColor,draw opacity=0.50,line width= 0.4pt,line join=round,line cap=round] (151.68,199.45) -- (155.71,199.45);

\path[draw=drawColor,draw opacity=0.50,line width= 0.4pt,line join=round,line cap=round] (153.70,197.43) -- (153.70,201.47);

\path[draw=drawColor,draw opacity=0.50,line width= 0.4pt,line join=round,line cap=round] (152.36,204.14) -- (155.22,206.99);

\path[draw=drawColor,draw opacity=0.50,line width= 0.4pt,line join=round,line cap=round] (152.36,206.99) -- (155.22,204.14);

\path[draw=drawColor,draw opacity=0.50,line width= 0.4pt,line join=round,line cap=round] (151.77,205.56) -- (155.81,205.56);

\path[draw=drawColor,draw opacity=0.50,line width= 0.4pt,line join=round,line cap=round] (153.79,203.54) -- (153.79,207.58);

\path[draw=drawColor,draw opacity=0.50,line width= 0.4pt,line join=round,line cap=round] (151.05,199.46) -- (153.90,202.31);

\path[draw=drawColor,draw opacity=0.50,line width= 0.4pt,line join=round,line cap=round] (151.05,202.31) -- (153.90,199.46);

\path[draw=drawColor,draw opacity=0.50,line width= 0.4pt,line join=round,line cap=round] (150.46,200.89) -- (154.49,200.89);

\path[draw=drawColor,draw opacity=0.50,line width= 0.4pt,line join=round,line cap=round] (152.47,198.87) -- (152.47,202.90);

\path[draw=drawColor,draw opacity=0.50,line width= 0.4pt,line join=round,line cap=round] (151.10,203.22) -- (153.95,206.07);

\path[draw=drawColor,draw opacity=0.50,line width= 0.4pt,line join=round,line cap=round] (151.10,206.07) -- (153.95,203.22);

\path[draw=drawColor,draw opacity=0.50,line width= 0.4pt,line join=round,line cap=round] (150.51,204.64) -- (154.55,204.64);

\path[draw=drawColor,draw opacity=0.50,line width= 0.4pt,line join=round,line cap=round] (152.53,202.62) -- (152.53,206.66);

\path[draw=drawColor,draw opacity=0.50,line width= 0.4pt,line join=round,line cap=round] (152.96,203.79) -- (155.82,206.65);

\path[draw=drawColor,draw opacity=0.50,line width= 0.4pt,line join=round,line cap=round] (152.96,206.65) -- (155.82,203.79);

\path[draw=drawColor,draw opacity=0.50,line width= 0.4pt,line join=round,line cap=round] (152.37,205.22) -- (156.41,205.22);

\path[draw=drawColor,draw opacity=0.50,line width= 0.4pt,line join=round,line cap=round] (154.39,203.20) -- (154.39,207.24);

\path[draw=drawColor,draw opacity=0.50,line width= 0.4pt,line join=round,line cap=round] (156.54,203.73) -- (159.39,206.58);

\path[draw=drawColor,draw opacity=0.50,line width= 0.4pt,line join=round,line cap=round] (156.54,206.58) -- (159.39,203.73);

\path[draw=drawColor,draw opacity=0.50,line width= 0.4pt,line join=round,line cap=round] (155.95,205.15) -- (159.98,205.15);

\path[draw=drawColor,draw opacity=0.50,line width= 0.4pt,line join=round,line cap=round] (157.97,203.14) -- (157.97,207.17);

\path[draw=drawColor,draw opacity=0.50,line width= 0.4pt,line join=round,line cap=round] (155.49,211.01) -- (158.35,213.87);

\path[draw=drawColor,draw opacity=0.50,line width= 0.4pt,line join=round,line cap=round] (155.49,213.87) -- (158.35,211.01);

\path[draw=drawColor,draw opacity=0.50,line width= 0.4pt,line join=round,line cap=round] (154.90,212.44) -- (158.94,212.44);

\path[draw=drawColor,draw opacity=0.50,line width= 0.4pt,line join=round,line cap=round] (156.92,210.42) -- (156.92,214.46);

\path[draw=drawColor,draw opacity=0.50,line width= 0.4pt,line join=round,line cap=round] (156.11,211.01) -- (158.97,213.87);

\path[draw=drawColor,draw opacity=0.50,line width= 0.4pt,line join=round,line cap=round] (156.11,213.87) -- (158.97,211.01);

\path[draw=drawColor,draw opacity=0.50,line width= 0.4pt,line join=round,line cap=round] (155.52,212.44) -- (159.56,212.44);

\path[draw=drawColor,draw opacity=0.50,line width= 0.4pt,line join=round,line cap=round] (157.54,210.42) -- (157.54,214.46);

\path[draw=drawColor,draw opacity=0.50,line width= 0.4pt,line join=round,line cap=round] (155.24,211.01) -- (158.09,213.87);

\path[draw=drawColor,draw opacity=0.50,line width= 0.4pt,line join=round,line cap=round] (155.24,213.87) -- (158.09,211.01);

\path[draw=drawColor,draw opacity=0.50,line width= 0.4pt,line join=round,line cap=round] (154.64,212.44) -- (158.68,212.44);

\path[draw=drawColor,draw opacity=0.50,line width= 0.4pt,line join=round,line cap=round] (156.66,210.42) -- (156.66,214.46);

\path[draw=drawColor,draw opacity=0.50,line width= 0.4pt,line join=round,line cap=round] (158.55,211.01) -- (161.40,213.87);

\path[draw=drawColor,draw opacity=0.50,line width= 0.4pt,line join=round,line cap=round] (158.55,213.87) -- (161.40,211.01);

\path[draw=drawColor,draw opacity=0.50,line width= 0.4pt,line join=round,line cap=round] (157.95,212.44) -- (161.99,212.44);

\path[draw=drawColor,draw opacity=0.50,line width= 0.4pt,line join=round,line cap=round] (159.97,210.42) -- (159.97,214.46);

\path[draw=drawColor,draw opacity=0.50,line width= 0.4pt,line join=round,line cap=round] (155.16,211.01) -- (158.01,213.87);

\path[draw=drawColor,draw opacity=0.50,line width= 0.4pt,line join=round,line cap=round] (155.16,213.87) -- (158.01,211.01);

\path[draw=drawColor,draw opacity=0.50,line width= 0.4pt,line join=round,line cap=round] (154.57,212.44) -- (158.60,212.44);

\path[draw=drawColor,draw opacity=0.50,line width= 0.4pt,line join=round,line cap=round] (156.59,210.42) -- (156.59,214.46);

\path[draw=drawColor,draw opacity=0.50,line width= 0.4pt,line join=round,line cap=round] (144.70,202.52) -- (147.55,205.37);

\path[draw=drawColor,draw opacity=0.50,line width= 0.4pt,line join=round,line cap=round] (144.70,205.37) -- (147.55,202.52);

\path[draw=drawColor,draw opacity=0.50,line width= 0.4pt,line join=round,line cap=round] (144.11,203.94) -- (148.14,203.94);

\path[draw=drawColor,draw opacity=0.50,line width= 0.4pt,line join=round,line cap=round] (146.13,201.93) -- (146.13,205.96);

\path[draw=drawColor,draw opacity=0.50,line width= 0.4pt,line join=round,line cap=round] (145.73,198.34) -- (148.58,201.19);

\path[draw=drawColor,draw opacity=0.50,line width= 0.4pt,line join=round,line cap=round] (145.73,201.19) -- (148.58,198.34);

\path[draw=drawColor,draw opacity=0.50,line width= 0.4pt,line join=round,line cap=round] (145.14,199.77) -- (149.17,199.77);

\path[draw=drawColor,draw opacity=0.50,line width= 0.4pt,line join=round,line cap=round] (147.15,197.75) -- (147.15,201.79);

\path[draw=drawColor,draw opacity=0.50,line width= 0.4pt,line join=round,line cap=round] (143.94,201.36) -- (146.79,204.21);

\path[draw=drawColor,draw opacity=0.50,line width= 0.4pt,line join=round,line cap=round] (143.94,204.21) -- (146.79,201.36);

\path[draw=drawColor,draw opacity=0.50,line width= 0.4pt,line join=round,line cap=round] (143.34,202.78) -- (147.38,202.78);

\path[draw=drawColor,draw opacity=0.50,line width= 0.4pt,line join=round,line cap=round] (145.36,200.77) -- (145.36,204.80);

\path[draw=drawColor,draw opacity=0.50,line width= 0.4pt,line join=round,line cap=round] (144.67,198.80) -- (147.52,201.66);

\path[draw=drawColor,draw opacity=0.50,line width= 0.4pt,line join=round,line cap=round] (144.67,201.66) -- (147.52,198.80);

\path[draw=drawColor,draw opacity=0.50,line width= 0.4pt,line join=round,line cap=round] (144.08,200.23) -- (148.11,200.23);

\path[draw=drawColor,draw opacity=0.50,line width= 0.4pt,line join=round,line cap=round] (146.09,198.21) -- (146.09,202.25);

\path[draw=drawColor,draw opacity=0.50,line width= 0.4pt,line join=round,line cap=round] (146.15,201.58) -- (149.00,204.43);

\path[draw=drawColor,draw opacity=0.50,line width= 0.4pt,line join=round,line cap=round] (146.15,204.43) -- (149.00,201.58);

\path[draw=drawColor,draw opacity=0.50,line width= 0.4pt,line join=round,line cap=round] (145.56,203.01) -- (149.59,203.01);

\path[draw=drawColor,draw opacity=0.50,line width= 0.4pt,line join=round,line cap=round] (147.58,200.99) -- (147.58,205.03);

\path[draw=drawColor,draw opacity=0.50,line width= 0.4pt,line join=round,line cap=round] (158.01,203.58) -- (160.86,206.44);

\path[draw=drawColor,draw opacity=0.50,line width= 0.4pt,line join=round,line cap=round] (158.01,206.44) -- (160.86,203.58);

\path[draw=drawColor,draw opacity=0.50,line width= 0.4pt,line join=round,line cap=round] (157.42,205.01) -- (161.45,205.01);

\path[draw=drawColor,draw opacity=0.50,line width= 0.4pt,line join=round,line cap=round] (159.44,202.99) -- (159.44,207.03);

\path[draw=drawColor,draw opacity=0.50,line width= 0.4pt,line join=round,line cap=round] (154.08,211.01) -- (156.93,213.87);

\path[draw=drawColor,draw opacity=0.50,line width= 0.4pt,line join=round,line cap=round] (154.08,213.87) -- (156.93,211.01);

\path[draw=drawColor,draw opacity=0.50,line width= 0.4pt,line join=round,line cap=round] (153.49,212.44) -- (157.52,212.44);

\path[draw=drawColor,draw opacity=0.50,line width= 0.4pt,line join=round,line cap=round] (155.50,210.42) -- (155.50,214.46);

\path[draw=drawColor,draw opacity=0.50,line width= 0.4pt,line join=round,line cap=round] (154.87,204.45) -- (157.72,207.30);

\path[draw=drawColor,draw opacity=0.50,line width= 0.4pt,line join=round,line cap=round] (154.87,207.30) -- (157.72,204.45);

\path[draw=drawColor,draw opacity=0.50,line width= 0.4pt,line join=round,line cap=round] (154.28,205.88) -- (158.31,205.88);

\path[draw=drawColor,draw opacity=0.50,line width= 0.4pt,line join=round,line cap=round] (156.30,203.86) -- (156.30,207.89);

\path[draw=drawColor,draw opacity=0.50,line width= 0.4pt,line join=round,line cap=round] (153.06,211.01) -- (155.91,213.87);

\path[draw=drawColor,draw opacity=0.50,line width= 0.4pt,line join=round,line cap=round] (153.06,213.87) -- (155.91,211.01);

\path[draw=drawColor,draw opacity=0.50,line width= 0.4pt,line join=round,line cap=round] (152.47,212.44) -- (156.50,212.44);

\path[draw=drawColor,draw opacity=0.50,line width= 0.4pt,line join=round,line cap=round] (154.49,210.42) -- (154.49,214.46);

\path[draw=drawColor,draw opacity=0.50,line width= 0.4pt,line join=round,line cap=round] (156.64,203.92) -- (159.49,206.77);

\path[draw=drawColor,draw opacity=0.50,line width= 0.4pt,line join=round,line cap=round] (156.64,206.77) -- (159.49,203.92);

\path[draw=drawColor,draw opacity=0.50,line width= 0.4pt,line join=round,line cap=round] (156.05,205.34) -- (160.08,205.34);

\path[draw=drawColor,draw opacity=0.50,line width= 0.4pt,line join=round,line cap=round] (158.06,203.33) -- (158.06,207.36);

\path[draw=drawColor,draw opacity=0.50,line width= 0.4pt,line join=round,line cap=round] (157.41,211.01) -- (160.26,213.87);

\path[draw=drawColor,draw opacity=0.50,line width= 0.4pt,line join=round,line cap=round] (157.41,213.87) -- (160.26,211.01);

\path[draw=drawColor,draw opacity=0.50,line width= 0.4pt,line join=round,line cap=round] (156.82,212.44) -- (160.85,212.44);

\path[draw=drawColor,draw opacity=0.50,line width= 0.4pt,line join=round,line cap=round] (158.83,210.42) -- (158.83,214.46);

\path[draw=drawColor,draw opacity=0.50,line width= 0.4pt,line join=round,line cap=round] (163.23,203.40) -- (166.09,206.25);

\path[draw=drawColor,draw opacity=0.50,line width= 0.4pt,line join=round,line cap=round] (163.23,206.25) -- (166.09,203.40);

\path[draw=drawColor,draw opacity=0.50,line width= 0.4pt,line join=round,line cap=round] (162.64,204.82) -- (166.68,204.82);

\path[draw=drawColor,draw opacity=0.50,line width= 0.4pt,line join=round,line cap=round] (164.66,202.81) -- (164.66,206.84);

\path[draw=drawColor,draw opacity=0.50,line width= 0.4pt,line join=round,line cap=round] (161.38,202.64) -- (164.24,205.50);

\path[draw=drawColor,draw opacity=0.50,line width= 0.4pt,line join=round,line cap=round] (161.38,205.50) -- (164.24,202.64);

\path[draw=drawColor,draw opacity=0.50,line width= 0.4pt,line join=round,line cap=round] (160.79,204.07) -- (164.83,204.07);

\path[draw=drawColor,draw opacity=0.50,line width= 0.4pt,line join=round,line cap=round] (162.81,202.05) -- (162.81,206.09);

\path[draw=drawColor,draw opacity=0.50,line width= 0.4pt,line join=round,line cap=round] (160.79,202.53) -- (163.64,205.38);

\path[draw=drawColor,draw opacity=0.50,line width= 0.4pt,line join=round,line cap=round] (160.79,205.38) -- (163.64,202.53);

\path[draw=drawColor,draw opacity=0.50,line width= 0.4pt,line join=round,line cap=round] (160.20,203.96) -- (164.23,203.96);

\path[draw=drawColor,draw opacity=0.50,line width= 0.4pt,line join=round,line cap=round] (162.21,201.94) -- (162.21,205.97);

\path[draw=drawColor,draw opacity=0.50,line width= 0.4pt,line join=round,line cap=round] (158.99,204.60) -- (161.84,207.46);

\path[draw=drawColor,draw opacity=0.50,line width= 0.4pt,line join=round,line cap=round] (158.99,207.46) -- (161.84,204.60);

\path[draw=drawColor,draw opacity=0.50,line width= 0.4pt,line join=round,line cap=round] (158.40,206.03) -- (162.43,206.03);

\path[draw=drawColor,draw opacity=0.50,line width= 0.4pt,line join=round,line cap=round] (160.42,204.01) -- (160.42,208.05);

\path[draw=drawColor,draw opacity=0.50,line width= 0.4pt,line join=round,line cap=round] (162.93,204.42) -- (165.78,207.27);

\path[draw=drawColor,draw opacity=0.50,line width= 0.4pt,line join=round,line cap=round] (162.93,207.27) -- (165.78,204.42);

\path[draw=drawColor,draw opacity=0.50,line width= 0.4pt,line join=round,line cap=round] (162.34,205.85) -- (166.37,205.85);

\path[draw=drawColor,draw opacity=0.50,line width= 0.4pt,line join=round,line cap=round] (164.36,203.83) -- (164.36,207.87);

\path[draw=drawColor,draw opacity=0.50,line width= 0.4pt,line join=round,line cap=round] (145.26,202.81) -- (148.11,205.67);

\path[draw=drawColor,draw opacity=0.50,line width= 0.4pt,line join=round,line cap=round] (145.26,205.67) -- (148.11,202.81);

\path[draw=drawColor,draw opacity=0.50,line width= 0.4pt,line join=round,line cap=round] (144.66,204.24) -- (148.70,204.24);

\path[draw=drawColor,draw opacity=0.50,line width= 0.4pt,line join=round,line cap=round] (146.68,202.22) -- (146.68,206.26);

\path[draw=drawColor,draw opacity=0.50,line width= 0.4pt,line join=round,line cap=round] (148.04,199.25) -- (150.89,202.10);

\path[draw=drawColor,draw opacity=0.50,line width= 0.4pt,line join=round,line cap=round] (148.04,202.10) -- (150.89,199.25);

\path[draw=drawColor,draw opacity=0.50,line width= 0.4pt,line join=round,line cap=round] (147.45,200.67) -- (151.48,200.67);

\path[draw=drawColor,draw opacity=0.50,line width= 0.4pt,line join=round,line cap=round] (149.46,198.66) -- (149.46,202.69);

\path[draw=drawColor,draw opacity=0.50,line width= 0.4pt,line join=round,line cap=round] (146.35,203.50) -- (149.20,206.35);

\path[draw=drawColor,draw opacity=0.50,line width= 0.4pt,line join=round,line cap=round] (146.35,206.35) -- (149.20,203.50);

\path[draw=drawColor,draw opacity=0.50,line width= 0.4pt,line join=round,line cap=round] (145.76,204.92) -- (149.80,204.92);

\path[draw=drawColor,draw opacity=0.50,line width= 0.4pt,line join=round,line cap=round] (147.78,202.91) -- (147.78,206.94);

\path[draw=drawColor,draw opacity=0.50,line width= 0.4pt,line join=round,line cap=round] (146.43,198.48) -- (149.28,201.33);

\path[draw=drawColor,draw opacity=0.50,line width= 0.4pt,line join=round,line cap=round] (146.43,201.33) -- (149.28,198.48);

\path[draw=drawColor,draw opacity=0.50,line width= 0.4pt,line join=round,line cap=round] (145.84,199.91) -- (149.87,199.91);

\path[draw=drawColor,draw opacity=0.50,line width= 0.4pt,line join=round,line cap=round] (147.86,197.89) -- (147.86,201.92);

\path[draw=drawColor,draw opacity=0.50,line width= 0.4pt,line join=round,line cap=round] (146.05,202.35) -- (148.90,205.20);

\path[draw=drawColor,draw opacity=0.50,line width= 0.4pt,line join=round,line cap=round] (146.05,205.20) -- (148.90,202.35);

\path[draw=drawColor,draw opacity=0.50,line width= 0.4pt,line join=round,line cap=round] (145.46,203.77) -- (149.49,203.77);

\path[draw=drawColor,draw opacity=0.50,line width= 0.4pt,line join=round,line cap=round] (147.48,201.76) -- (147.48,205.79);

\path[draw=drawColor,draw opacity=0.50,line width= 0.4pt,line join=round,line cap=round] (153.66,200.05) -- (156.51,202.91);

\path[draw=drawColor,draw opacity=0.50,line width= 0.4pt,line join=round,line cap=round] (153.66,202.91) -- (156.51,200.05);

\path[draw=drawColor,draw opacity=0.50,line width= 0.4pt,line join=round,line cap=round] (153.06,201.48) -- (157.10,201.48);

\path[draw=drawColor,draw opacity=0.50,line width= 0.4pt,line join=round,line cap=round] (155.08,199.46) -- (155.08,203.50);

\path[draw=drawColor,draw opacity=0.50,line width= 0.4pt,line join=round,line cap=round] (155.45,211.01) -- (158.31,213.87);

\path[draw=drawColor,draw opacity=0.50,line width= 0.4pt,line join=round,line cap=round] (155.45,213.87) -- (158.31,211.01);

\path[draw=drawColor,draw opacity=0.50,line width= 0.4pt,line join=round,line cap=round] (154.86,212.44) -- (158.90,212.44);

\path[draw=drawColor,draw opacity=0.50,line width= 0.4pt,line join=round,line cap=round] (156.88,210.42) -- (156.88,214.46);

\path[draw=drawColor,draw opacity=0.50,line width= 0.4pt,line join=round,line cap=round] (153.44,202.96) -- (156.29,205.82);

\path[draw=drawColor,draw opacity=0.50,line width= 0.4pt,line join=round,line cap=round] (153.44,205.82) -- (156.29,202.96);

\path[draw=drawColor,draw opacity=0.50,line width= 0.4pt,line join=round,line cap=round] (152.85,204.39) -- (156.88,204.39);

\path[draw=drawColor,draw opacity=0.50,line width= 0.4pt,line join=round,line cap=round] (154.87,202.37) -- (154.87,206.41);

\path[draw=drawColor,draw opacity=0.50,line width= 0.4pt,line join=round,line cap=round] (158.45,206.06) -- (161.31,208.91);

\path[draw=drawColor,draw opacity=0.50,line width= 0.4pt,line join=round,line cap=round] (158.45,208.91) -- (161.31,206.06);

\path[draw=drawColor,draw opacity=0.50,line width= 0.4pt,line join=round,line cap=round] (157.86,207.49) -- (161.90,207.49);

\path[draw=drawColor,draw opacity=0.50,line width= 0.4pt,line join=round,line cap=round] (159.88,205.47) -- (159.88,209.50);

\path[draw=drawColor,draw opacity=0.50,line width= 0.4pt,line join=round,line cap=round] (153.47,205.30) -- (156.32,208.16);

\path[draw=drawColor,draw opacity=0.50,line width= 0.4pt,line join=round,line cap=round] (153.47,208.16) -- (156.32,205.30);

\path[draw=drawColor,draw opacity=0.50,line width= 0.4pt,line join=round,line cap=round] (152.88,206.73) -- (156.91,206.73);

\path[draw=drawColor,draw opacity=0.50,line width= 0.4pt,line join=round,line cap=round] (154.89,204.71) -- (154.89,208.75);
\definecolor{fillColor}{RGB}{117,112,179}

\path[fill=fillColor,fill opacity=0.50] (117.44,115.88) --
	(120.29,115.88) --
	(120.29,118.73) --
	(117.44,118.73) --
	cycle;

\path[fill=fillColor,fill opacity=0.50] (118.70,117.43) --
	(121.55,117.43) --
	(121.55,120.29) --
	(118.70,120.29) --
	cycle;

\path[fill=fillColor,fill opacity=0.50] (114.76,125.31) --
	(117.61,125.31) --
	(117.61,128.16) --
	(114.76,128.16) --
	cycle;

\path[fill=fillColor,fill opacity=0.50] (116.38,113.84) --
	(119.23,113.84) --
	(119.23,116.70) --
	(116.38,116.70) --
	cycle;

\path[fill=fillColor,fill opacity=0.50] (117.18,115.67) --
	(120.03,115.67) --
	(120.03,118.52) --
	(117.18,118.52) --
	cycle;

\path[fill=fillColor,fill opacity=0.50] (114.76,122.27) --
	(117.61,122.27) --
	(117.61,125.12) --
	(114.76,125.12) --
	cycle;

\path[fill=fillColor,fill opacity=0.50] (115.46,116.49) --
	(118.31,116.49) --
	(118.31,119.34) --
	(115.46,119.34) --
	cycle;

\path[fill=fillColor,fill opacity=0.50] (114.56,113.59) --
	(117.41,113.59) --
	(117.41,116.45) --
	(114.56,116.45) --
	cycle;

\path[fill=fillColor,fill opacity=0.50] (113.41,113.59) --
	(116.27,113.59) --
	(116.27,116.45) --
	(113.41,116.45) --
	cycle;

\path[fill=fillColor,fill opacity=0.50] (115.68,115.46) --
	(118.53,115.46) --
	(118.53,118.31) --
	(115.68,118.31) --
	cycle;

\path[fill=fillColor,fill opacity=0.50] (116.25,117.97) --
	(119.10,117.97) --
	(119.10,120.82) --
	(116.25,120.82) --
	cycle;

\path[fill=fillColor,fill opacity=0.50] (114.82,115.88) --
	(117.67,115.88) --
	(117.67,118.73) --
	(114.82,118.73) --
	cycle;

\path[fill=fillColor,fill opacity=0.50] (113.28,125.03) --
	(116.13,125.03) --
	(116.13,127.88) --
	(113.28,127.88) --
	cycle;

\path[fill=fillColor,fill opacity=0.50] (119.04,116.49) --
	(121.90,116.49) --
	(121.90,119.34) --
	(119.04,119.34) --
	cycle;

\path[fill=fillColor,fill opacity=0.50] (116.12,116.09) --
	(118.98,116.09) --
	(118.98,118.94) --
	(116.12,118.94) --
	cycle;

\path[fill=fillColor,fill opacity=0.50] (116.60,115.46) --
	(119.46,115.46) --
	(119.46,118.31) --
	(116.60,118.31) --
	cycle;

\path[fill=fillColor,fill opacity=0.50] (126.44,133.82) --
	(129.30,133.82) --
	(129.30,136.68) --
	(126.44,136.68) --
	cycle;

\path[fill=fillColor,fill opacity=0.50] (120.89,128.09) --
	(123.74,128.09) --
	(123.74,130.94) --
	(120.89,130.94) --
	cycle;

\path[fill=fillColor,fill opacity=0.50] (119.28,125.67) --
	(122.13,125.67) --
	(122.13,128.53) --
	(119.28,128.53) --
	cycle;

\path[fill=fillColor,fill opacity=0.50] (120.00,126.86) --
	(122.86,126.86) --
	(122.86,129.72) --
	(120.00,129.72) --
	cycle;

\path[fill=fillColor,fill opacity=0.50] (120.56,126.62) --
	(123.41,126.62) --
	(123.41,129.47) --
	(120.56,129.47) --
	cycle;

\path[fill=fillColor,fill opacity=0.50] (120.84,127.26) --
	(123.69,127.26) --
	(123.69,130.11) --
	(120.84,130.11) --
	cycle;

\path[fill=fillColor,fill opacity=0.50] (118.53,127.34) --
	(121.38,127.34) --
	(121.38,130.19) --
	(118.53,130.19) --
	cycle;

\path[fill=fillColor,fill opacity=0.50] (120.45,126.28) --
	(123.31,126.28) --
	(123.31,129.14) --
	(120.45,129.14) --
	cycle;

\path[fill=fillColor,fill opacity=0.50] (121.24,128.45) --
	(124.10,128.45) --
	(124.10,131.30) --
	(121.24,131.30) --
	cycle;

\path[fill=fillColor,fill opacity=0.50] (120.28,135.82) --
	(123.13,135.82) --
	(123.13,138.67) --
	(120.28,138.67) --
	cycle;

\path[fill=fillColor,fill opacity=0.50] (120.79,131.03) --
	(123.64,131.03) --
	(123.64,133.88) --
	(120.79,133.88) --
	cycle;

\path[fill=fillColor,fill opacity=0.50] (121.18,130.74) --
	(124.03,130.74) --
	(124.03,133.59) --
	(121.18,133.59) --
	cycle;

\path[fill=fillColor,fill opacity=0.50] (123.06,130.51) --
	(125.92,130.51) --
	(125.92,133.36) --
	(123.06,133.36) --
	cycle;

\path[fill=fillColor,fill opacity=0.50] (123.09,135.12) --
	(125.95,135.12) --
	(125.95,137.98) --
	(123.09,137.98) --
	cycle;

\path[fill=fillColor,fill opacity=0.50] (124.93,133.96) --
	(127.78,133.96) --
	(127.78,136.81) --
	(124.93,136.81) --
	cycle;

\path[fill=fillColor,fill opacity=0.50] (123.45,130.39) --
	(126.31,130.39) --
	(126.31,133.24) --
	(123.45,133.24) --
	cycle;

\path[fill=fillColor,fill opacity=0.50] (145.62,158.61) --
	(148.47,158.61) --
	(148.47,161.46) --
	(145.62,161.46) --
	cycle;

\path[fill=fillColor,fill opacity=0.50] (123.43,145.30) --
	(126.28,145.30) --
	(126.28,148.15) --
	(123.43,148.15) --
	cycle;

\path[fill=fillColor,fill opacity=0.50] (123.84,140.54) --
	(126.70,140.54) --
	(126.70,143.40) --
	(123.84,143.40) --
	cycle;

\path[fill=fillColor,fill opacity=0.50] (123.04,140.67) --
	(125.89,140.67) --
	(125.89,143.52) --
	(123.04,143.52) --
	cycle;

\path[fill=fillColor,fill opacity=0.50] (125.16,143.28) --
	(128.01,143.28) --
	(128.01,146.14) --
	(125.16,146.14) --
	cycle;

\path[fill=fillColor,fill opacity=0.50] (122.09,142.26) --
	(124.94,142.26) --
	(124.94,145.11) --
	(122.09,145.11) --
	cycle;

\path[fill=fillColor,fill opacity=0.50] (128.04,138.71) --
	(130.89,138.71) --
	(130.89,141.57) --
	(128.04,141.57) --
	cycle;

\path[fill=fillColor,fill opacity=0.50] (121.19,136.29) --
	(124.05,136.29) --
	(124.05,139.15) --
	(121.19,139.15) --
	cycle;

\path[fill=fillColor,fill opacity=0.50] (121.76,140.82) --
	(124.61,140.82) --
	(124.61,143.67) --
	(121.76,143.67) --
	cycle;

\path[fill=fillColor,fill opacity=0.50] (120.08,135.20) --
	(122.93,135.20) --
	(122.93,138.06) --
	(120.08,138.06) --
	cycle;

\path[fill=fillColor,fill opacity=0.50] (122.94,137.61) --
	(125.79,137.61) --
	(125.79,140.47) --
	(122.94,140.47) --
	cycle;

\path[fill=fillColor,fill opacity=0.50] (134.28,157.22) --
	(137.13,157.22) --
	(137.13,160.07) --
	(134.28,160.07) --
	cycle;

\path[fill=fillColor,fill opacity=0.50] (133.72,159.27) --
	(136.58,159.27) --
	(136.58,162.12) --
	(133.72,162.12) --
	cycle;

\path[fill=fillColor,fill opacity=0.50] (132.99,156.65) --
	(135.84,156.65) --
	(135.84,159.50) --
	(132.99,159.50) --
	cycle;

\path[fill=fillColor,fill opacity=0.50] (131.38,160.85) --
	(134.24,160.85) --
	(134.24,163.71) --
	(131.38,163.71) --
	cycle;

\path[fill=fillColor,fill opacity=0.50] (130.44,159.78) --
	(133.29,159.78) --
	(133.29,162.63) --
	(130.44,162.63) --
	cycle;

\path[fill=fillColor,fill opacity=0.50] (119.80,118.96) --
	(122.65,118.96) --
	(122.65,121.82) --
	(119.80,121.82) --
	cycle;

\path[fill=fillColor,fill opacity=0.50] (116.85,116.68) --
	(119.70,116.68) --
	(119.70,119.54) --
	(116.85,119.54) --
	cycle;

\path[fill=fillColor,fill opacity=0.50] (113.92,116.49) --
	(116.77,116.49) --
	(116.77,119.34) --
	(113.92,119.34) --
	cycle;

\path[fill=fillColor,fill opacity=0.50] (115.73,124.65) --
	(118.58,124.65) --
	(118.58,127.50) --
	(115.73,127.50) --
	cycle;

\path[fill=fillColor,fill opacity=0.50] (117.71,117.79) --
	(120.56,117.79) --
	(120.56,120.64) --
	(117.71,120.64) --
	cycle;

\path[fill=fillColor,fill opacity=0.50] (117.55,116.68) --
	(120.40,116.68) --
	(120.40,119.54) --
	(117.55,119.54) --
	cycle;

\path[fill=fillColor,fill opacity=0.50] (117.48,118.31) --
	(120.33,118.31) --
	(120.33,121.16) --
	(117.48,121.16) --
	cycle;

\path[fill=fillColor,fill opacity=0.50] (119.14,118.31) --
	(122.00,118.31) --
	(122.00,121.16) --
	(119.14,121.16) --
	cycle;

\path[fill=fillColor,fill opacity=0.50] (117.44,116.29) --
	(120.29,116.29) --
	(120.29,119.14) --
	(117.44,119.14) --
	cycle;

\path[fill=fillColor,fill opacity=0.50] (118.00,120.46) --
	(120.85,120.46) --
	(120.85,123.31) --
	(118.00,123.31) --
	cycle;

\path[fill=fillColor,fill opacity=0.50] (115.24,118.47) --
	(118.09,118.47) --
	(118.09,121.33) --
	(115.24,121.33) --
	cycle;

\path[fill=fillColor,fill opacity=0.50] (116.75,118.14) --
	(119.60,118.14) --
	(119.60,120.99) --
	(116.75,120.99) --
	cycle;

\path[fill=fillColor,fill opacity=0.50] (120.02,120.31) --
	(122.87,120.31) --
	(122.87,123.17) --
	(120.02,123.17) --
	cycle;

\path[fill=fillColor,fill opacity=0.50] (119.18,119.73) --
	(122.04,119.73) --
	(122.04,122.59) --
	(119.18,122.59) --
	cycle;

\path[fill=fillColor,fill opacity=0.50] (121.70,120.31) --
	(124.55,120.31) --
	(124.55,123.17) --
	(121.70,123.17) --
	cycle;

\path[fill=fillColor,fill opacity=0.50] (119.82,119.73) --
	(122.67,119.73) --
	(122.67,122.59) --
	(119.82,122.59) --
	cycle;

\path[fill=fillColor,fill opacity=0.50] (138.66,155.52) --
	(141.51,155.52) --
	(141.51,158.38) --
	(138.66,158.38) --
	cycle;

\path[fill=fillColor,fill opacity=0.50] (120.58,129.59) --
	(123.43,129.59) --
	(123.43,132.44) --
	(120.58,132.44) --
	cycle;

\path[fill=fillColor,fill opacity=0.50] (118.94,129.78) --
	(121.80,129.78) --
	(121.80,132.63) --
	(118.94,132.63) --
	cycle;

\path[fill=fillColor,fill opacity=0.50] (120.79,133.18) --
	(123.64,133.18) --
	(123.64,136.04) --
	(120.79,136.04) --
	cycle;

\path[fill=fillColor,fill opacity=0.50] (120.47,130.39) --
	(123.33,130.39) --
	(123.33,133.24) --
	(120.47,133.24) --
	cycle;

\path[fill=fillColor,fill opacity=0.50] (118.21,130.74) --
	(121.07,130.74) --
	(121.07,133.59) --
	(118.21,133.59) --
	cycle;

\path[fill=fillColor,fill opacity=0.50] (120.20,132.05) --
	(123.06,132.05) --
	(123.06,134.91) --
	(120.20,134.91) --
	cycle;

\path[fill=fillColor,fill opacity=0.50] (118.90,131.31) --
	(121.76,131.31) --
	(121.76,134.16) --
	(118.90,134.16) --
	cycle;

\path[fill=fillColor,fill opacity=0.50] (121.28,135.24) --
	(124.13,135.24) --
	(124.13,138.10) --
	(121.28,138.10) --
	cycle;

\path[fill=fillColor,fill opacity=0.50] (118.41,129.52) --
	(121.26,129.52) --
	(121.26,132.38) --
	(118.41,132.38) --
	cycle;

\path[fill=fillColor,fill opacity=0.50] (117.97,128.59) --
	(120.83,128.59) --
	(120.83,131.44) --
	(117.97,131.44) --
	cycle;

\path[fill=fillColor,fill opacity=0.50] (129.52,143.97) --
	(132.37,143.97) --
	(132.37,146.82) --
	(129.52,146.82) --
	cycle;

\path[fill=fillColor,fill opacity=0.50] (127.37,139.83) --
	(130.22,139.83) --
	(130.22,142.68) --
	(127.37,142.68) --
	cycle;

\path[fill=fillColor,fill opacity=0.50] (127.75,139.69) --
	(130.60,139.69) --
	(130.60,142.55) --
	(127.75,142.55) --
	cycle;

\path[fill=fillColor,fill opacity=0.50] (128.50,140.37) --
	(131.35,140.37) --
	(131.35,143.22) --
	(128.50,143.22) --
	cycle;

\path[fill=fillColor,fill opacity=0.50] (127.96,142.94) --
	(130.81,142.94) --
	(130.81,145.80) --
	(127.96,145.80) --
	cycle;

\path[fill=fillColor,fill opacity=0.50] (125.19,133.28) --
	(128.05,133.28) --
	(128.05,136.13) --
	(125.19,136.13) --
	cycle;

\path[fill=fillColor,fill opacity=0.50] (116.70,121.27) --
	(119.55,121.27) --
	(119.55,124.12) --
	(116.70,124.12) --
	cycle;

\path[fill=fillColor,fill opacity=0.50] (120.28,122.85) --
	(123.13,122.85) --
	(123.13,125.71) --
	(120.28,125.71) --
	cycle;

\path[fill=fillColor,fill opacity=0.50] (117.69,121.27) --
	(120.54,121.27) --
	(120.54,124.12) --
	(117.69,124.12) --
	cycle;

\path[fill=fillColor,fill opacity=0.50] (119.46,123.30) --
	(122.31,123.30) --
	(122.31,126.15) --
	(119.46,126.15) --
	cycle;

\path[fill=fillColor,fill opacity=0.50] (116.82,120.87) --
	(119.68,120.87) --
	(119.68,123.72) --
	(116.82,123.72) --
	cycle;

\path[fill=fillColor,fill opacity=0.50] (122.76,122.62) --
	(125.62,122.62) --
	(125.62,125.48) --
	(122.76,125.48) --
	cycle;

\path[fill=fillColor,fill opacity=0.50] (118.04,121.78) --
	(120.89,121.78) --
	(120.89,124.63) --
	(118.04,124.63) --
	cycle;

\path[fill=fillColor,fill opacity=0.50] (117.93,121.78) --
	(120.78,121.78) --
	(120.78,124.63) --
	(117.93,124.63) --
	cycle;

\path[fill=fillColor,fill opacity=0.50] (118.82,124.35) --
	(121.67,124.35) --
	(121.67,127.20) --
	(118.82,127.20) --
	cycle;

\path[fill=fillColor,fill opacity=0.50] (118.74,121.78) --
	(121.59,121.78) --
	(121.59,124.63) --
	(118.74,124.63) --
	cycle;

\path[fill=fillColor,fill opacity=0.50] (120.09,125.03) --
	(122.95,125.03) --
	(122.95,127.88) --
	(120.09,127.88) --
	cycle;

\path[fill=fillColor,fill opacity=0.50] (120.58,129.00) --
	(123.43,129.00) --
	(123.43,131.85) --
	(120.58,131.85) --
	cycle;

\path[fill=fillColor,fill opacity=0.50] (121.37,124.74) --
	(124.23,124.74) --
	(124.23,127.60) --
	(121.37,127.60) --
	cycle;

\path[fill=fillColor,fill opacity=0.50] (119.42,122.85) --
	(122.27,122.85) --
	(122.27,125.71) --
	(119.42,125.71) --
	cycle;

\path[fill=fillColor,fill opacity=0.50] (123.45,128.93) --
	(126.31,128.93) --
	(126.31,131.78) --
	(123.45,131.78) --
	cycle;

\path[fill=fillColor,fill opacity=0.50] (126.53,149.19) --
	(129.38,149.19) --
	(129.38,152.04) --
	(126.53,152.04) --
	cycle;

\path[fill=fillColor,fill opacity=0.50] (118.41,123.41) --
	(121.26,123.41) --
	(121.26,126.26) --
	(118.41,126.26) --
	cycle;

\path[fill=fillColor,fill opacity=0.50] (115.27,124.93) --
	(118.12,124.93) --
	(118.12,127.79) --
	(115.27,127.79) --
	cycle;

\path[fill=fillColor,fill opacity=0.50] (117.11,124.45) --
	(119.96,124.45) --
	(119.96,127.30) --
	(117.11,127.30) --
	cycle;

\path[fill=fillColor,fill opacity=0.50] (119.93,126.03) --
	(122.78,126.03) --
	(122.78,128.88) --
	(119.93,128.88) --
	cycle;

\path[fill=fillColor,fill opacity=0.50] (116.38,125.49) --
	(119.23,125.49) --
	(119.23,128.35) --
	(116.38,128.35) --
	cycle;

\path[fill=fillColor,fill opacity=0.50] (118.11,122.85) --
	(120.96,122.85) --
	(120.96,125.71) --
	(118.11,125.71) --
	cycle;

\path[fill=fillColor,fill opacity=0.50] (118.51,126.28) --
	(121.36,126.28) --
	(121.36,129.14) --
	(118.51,129.14) --
	cycle;

\path[fill=fillColor,fill opacity=0.50] (118.19,124.55) --
	(121.05,124.55) --
	(121.05,127.40) --
	(118.19,127.40) --
	cycle;

\path[fill=fillColor,fill opacity=0.50] (115.13,131.52) --
	(117.98,131.52) --
	(117.98,134.38) --
	(115.13,134.38) --
	cycle;

\path[fill=fillColor,fill opacity=0.50] (118.13,126.28) --
	(120.98,126.28) --
	(120.98,129.14) --
	(118.13,129.14) --
	cycle;

\path[fill=fillColor,fill opacity=0.50] (123.33,138.69) --
	(126.18,138.69) --
	(126.18,141.54) --
	(123.33,141.54) --
	cycle;

\path[fill=fillColor,fill opacity=0.50] (123.43,139.80) --
	(126.28,139.80) --
	(126.28,142.65) --
	(123.43,142.65) --
	cycle;

\path[fill=fillColor,fill opacity=0.50] (125.29,143.08) --
	(128.14,143.08) --
	(128.14,145.94) --
	(125.29,145.94) --
	cycle;

\path[fill=fillColor,fill opacity=0.50] (122.91,143.14) --
	(125.76,143.14) --
	(125.76,146.00) --
	(122.91,146.00) --
	cycle;

\path[fill=fillColor,fill opacity=0.50] (123.11,145.17) --
	(125.96,145.17) --
	(125.96,148.02) --
	(123.11,148.02) --
	cycle;

\path[fill=fillColor,fill opacity=0.50] (142.20,211.01) --
	(145.06,211.01) --
	(145.06,213.87) --
	(142.20,213.87) --
	cycle;

\path[fill=fillColor,fill opacity=0.50] (122.75,200.30) --
	(125.60,200.30) --
	(125.60,203.16) --
	(122.75,203.16) --
	cycle;

\path[fill=fillColor,fill opacity=0.50] (118.72,188.91) --
	(121.57,188.91) --
	(121.57,191.76) --
	(118.72,191.76) --
	cycle;

\path[fill=fillColor,fill opacity=0.50] (121.79,171.17) --
	(124.65,171.17) --
	(124.65,174.02) --
	(121.79,174.02) --
	cycle;

\path[fill=fillColor,fill opacity=0.50] (123.27,177.74) --
	(126.13,177.74) --
	(126.13,180.59) --
	(123.27,180.59) --
	cycle;

\path[fill=fillColor,fill opacity=0.50] (123.56,190.68) --
	(126.42,190.68) --
	(126.42,193.53) --
	(123.56,193.53) --
	cycle;

\path[fill=fillColor,fill opacity=0.50] (117.34,130.33) --
	(120.20,130.33) --
	(120.20,133.18) --
	(117.34,133.18) --
	cycle;

\path[fill=fillColor,fill opacity=0.50] (121.21,132.85) --
	(124.06,132.85) --
	(124.06,135.70) --
	(121.21,135.70) --
	cycle;

\path[fill=fillColor,fill opacity=0.50] (119.84,131.79) --
	(122.69,131.79) --
	(122.69,134.64) --
	(119.84,134.64) --
	cycle;

\path[fill=fillColor,fill opacity=0.50] (122.35,133.37) --
	(125.20,133.37) --
	(125.20,136.22) --
	(122.35,136.22) --
	cycle;

\path[fill=fillColor,fill opacity=0.50] (122.15,211.01) --
	(125.00,211.01) --
	(125.00,213.87) --
	(122.15,213.87) --
	cycle;

\path[fill=fillColor,fill opacity=0.50] (129.99,211.01) --
	(132.84,211.01) --
	(132.84,213.87) --
	(129.99,213.87) --
	cycle;

\path[fill=fillColor,fill opacity=0.50] (128.92,211.01) --
	(131.77,211.01) --
	(131.77,213.87) --
	(128.92,213.87) --
	cycle;

\path[fill=fillColor,fill opacity=0.50] (132.89,211.01) --
	(135.74,211.01) --
	(135.74,213.87) --
	(132.89,213.87) --
	cycle;

\path[fill=fillColor,fill opacity=0.50] (133.34,211.01) --
	(136.20,211.01) --
	(136.20,213.87) --
	(133.34,213.87) --
	cycle;

\path[fill=fillColor,fill opacity=0.50] (134.31,211.01) --
	(137.16,211.01) --
	(137.16,213.87) --
	(134.31,213.87) --
	cycle;
\definecolor{drawColor}{RGB}{231,41,138}

\path[draw=drawColor,draw opacity=0.50,line width= 0.4pt,line join=round,line cap=round] (104.95,103.44) -- (108.99,103.44);

\path[draw=drawColor,draw opacity=0.50,line width= 0.4pt,line join=round,line cap=round] (106.97,101.43) -- (106.97,105.46);

\path[draw=drawColor,draw opacity=0.50,line width= 0.4pt,line join=round,line cap=round] (104.83, 99.39) -- (108.86, 99.39);

\path[draw=drawColor,draw opacity=0.50,line width= 0.4pt,line join=round,line cap=round] (106.84, 97.37) -- (106.84,101.41);

\path[draw=drawColor,draw opacity=0.50,line width= 0.4pt,line join=round,line cap=round] (102.86, 98.38) -- (106.89, 98.38);

\path[draw=drawColor,draw opacity=0.50,line width= 0.4pt,line join=round,line cap=round] (104.88, 96.36) -- (104.88,100.40);

\path[draw=drawColor,draw opacity=0.50,line width= 0.4pt,line join=round,line cap=round] (105.39,100.32) -- (109.43,100.32);

\path[draw=drawColor,draw opacity=0.50,line width= 0.4pt,line join=round,line cap=round] (107.41, 98.31) -- (107.41,102.34);

\path[draw=drawColor,draw opacity=0.50,line width= 0.4pt,line join=round,line cap=round] (104.56,100.32) -- (108.60,100.32);

\path[draw=drawColor,draw opacity=0.50,line width= 0.4pt,line join=round,line cap=round] (106.58, 98.31) -- (106.58,102.34);

\path[draw=drawColor,draw opacity=0.50,line width= 0.4pt,line join=round,line cap=round] (105.27,101.19) -- (109.30,101.19);

\path[draw=drawColor,draw opacity=0.50,line width= 0.4pt,line join=round,line cap=round] (107.29, 99.17) -- (107.29,103.20);

\path[draw=drawColor,draw opacity=0.50,line width= 0.4pt,line join=round,line cap=round] (102.86, 99.39) -- (106.89, 99.39);

\path[draw=drawColor,draw opacity=0.50,line width= 0.4pt,line join=round,line cap=round] (104.88, 97.37) -- (104.88,101.41);

\path[draw=drawColor,draw opacity=0.50,line width= 0.4pt,line join=round,line cap=round] (105.76,100.32) -- (109.79,100.32);

\path[draw=drawColor,draw opacity=0.50,line width= 0.4pt,line join=round,line cap=round] (107.77, 98.31) -- (107.77,102.34);

\path[draw=drawColor,draw opacity=0.50,line width= 0.4pt,line join=round,line cap=round] (104.89,100.32) -- (108.92,100.32);

\path[draw=drawColor,draw opacity=0.50,line width= 0.4pt,line join=round,line cap=round] (106.91, 98.31) -- (106.91,102.34);

\path[draw=drawColor,draw opacity=0.50,line width= 0.4pt,line join=round,line cap=round] (106.11,100.32) -- (110.14,100.32);

\path[draw=drawColor,draw opacity=0.50,line width= 0.4pt,line join=round,line cap=round] (108.13, 98.31) -- (108.13,102.34);

\path[draw=drawColor,draw opacity=0.50,line width= 0.4pt,line join=round,line cap=round] (104.70, 99.39) -- (108.73, 99.39);

\path[draw=drawColor,draw opacity=0.50,line width= 0.4pt,line join=round,line cap=round] (106.71, 97.37) -- (106.71,101.41);

\path[draw=drawColor,draw opacity=0.50,line width= 0.4pt,line join=round,line cap=round] (105.02, 99.39) -- (109.05, 99.39);

\path[draw=drawColor,draw opacity=0.50,line width= 0.4pt,line join=round,line cap=round] (107.03, 97.37) -- (107.03,101.41);

\path[draw=drawColor,draw opacity=0.50,line width= 0.4pt,line join=round,line cap=round] (104.95,100.32) -- (108.99,100.32);

\path[draw=drawColor,draw opacity=0.50,line width= 0.4pt,line join=round,line cap=round] (106.97, 98.31) -- (106.97,102.34);

\path[draw=drawColor,draw opacity=0.50,line width= 0.4pt,line join=round,line cap=round] (103.53, 98.38) -- (107.56, 98.38);

\path[draw=drawColor,draw opacity=0.50,line width= 0.4pt,line join=round,line cap=round] (105.55, 96.36) -- (105.55,100.40);

\path[draw=drawColor,draw opacity=0.50,line width= 0.4pt,line join=round,line cap=round] (105.51,100.32) -- (109.55,100.32);

\path[draw=drawColor,draw opacity=0.50,line width= 0.4pt,line join=round,line cap=round] (107.53, 98.31) -- (107.53,102.34);

\path[draw=drawColor,draw opacity=0.50,line width= 0.4pt,line join=round,line cap=round] (105.21,101.19) -- (109.24,101.19);

\path[draw=drawColor,draw opacity=0.50,line width= 0.4pt,line join=round,line cap=round] (107.22, 99.17) -- (107.22,103.20);
\definecolor{drawColor}{RGB}{230,171,2}

\path[draw=drawColor,draw opacity=0.50,line width= 0.4pt,line join=round,line cap=round] (134.22,117.97) -- (137.07,120.82);

\path[draw=drawColor,draw opacity=0.50,line width= 0.4pt,line join=round,line cap=round] (134.22,120.82) -- (137.07,117.97);

\path[draw=drawColor,draw opacity=0.50,line width= 0.4pt,line join=round,line cap=round] (133.63,119.39) -- (137.67,119.39);

\path[draw=drawColor,draw opacity=0.50,line width= 0.4pt,line join=round,line cap=round] (135.65,117.37) -- (135.65,121.41);

\path[draw=drawColor,draw opacity=0.50,line width= 0.4pt,line join=round,line cap=round] (116.35,117.06) -- (119.21,119.92);

\path[draw=drawColor,draw opacity=0.50,line width= 0.4pt,line join=round,line cap=round] (116.35,119.92) -- (119.21,117.06);

\path[draw=drawColor,draw opacity=0.50,line width= 0.4pt,line join=round,line cap=round] (115.76,118.49) -- (119.80,118.49);

\path[draw=drawColor,draw opacity=0.50,line width= 0.4pt,line join=round,line cap=round] (117.78,116.47) -- (117.78,120.51);

\path[draw=drawColor,draw opacity=0.50,line width= 0.4pt,line join=round,line cap=round] (115.49,116.87) -- (118.34,119.73);

\path[draw=drawColor,draw opacity=0.50,line width= 0.4pt,line join=round,line cap=round] (115.49,119.73) -- (118.34,116.87);

\path[draw=drawColor,draw opacity=0.50,line width= 0.4pt,line join=round,line cap=round] (114.90,118.30) -- (118.93,118.30);

\path[draw=drawColor,draw opacity=0.50,line width= 0.4pt,line join=round,line cap=round] (116.92,116.28) -- (116.92,120.32);

\path[draw=drawColor,draw opacity=0.50,line width= 0.4pt,line join=round,line cap=round] (115.35,117.61) -- (118.21,120.47);

\path[draw=drawColor,draw opacity=0.50,line width= 0.4pt,line join=round,line cap=round] (115.35,120.47) -- (118.21,117.61);

\path[draw=drawColor,draw opacity=0.50,line width= 0.4pt,line join=round,line cap=round] (114.76,119.04) -- (118.80,119.04);

\path[draw=drawColor,draw opacity=0.50,line width= 0.4pt,line join=round,line cap=round] (116.78,117.02) -- (116.78,121.06);

\path[draw=drawColor,draw opacity=0.50,line width= 0.4pt,line join=round,line cap=round] (116.10,117.25) -- (118.95,120.10);

\path[draw=drawColor,draw opacity=0.50,line width= 0.4pt,line join=round,line cap=round] (116.10,120.10) -- (118.95,117.25);

\path[draw=drawColor,draw opacity=0.50,line width= 0.4pt,line join=round,line cap=round] (115.51,118.68) -- (119.54,118.68);

\path[draw=drawColor,draw opacity=0.50,line width= 0.4pt,line join=round,line cap=round] (117.52,116.66) -- (117.52,120.69);

\path[draw=drawColor,draw opacity=0.50,line width= 0.4pt,line join=round,line cap=round] (116.20,117.79) -- (119.05,120.64);

\path[draw=drawColor,draw opacity=0.50,line width= 0.4pt,line join=round,line cap=round] (116.20,120.64) -- (119.05,117.79);

\path[draw=drawColor,draw opacity=0.50,line width= 0.4pt,line join=round,line cap=round] (115.61,119.22) -- (119.64,119.22);

\path[draw=drawColor,draw opacity=0.50,line width= 0.4pt,line join=round,line cap=round] (117.63,117.20) -- (117.63,121.23);

\path[draw=drawColor,draw opacity=0.50,line width= 0.4pt,line join=round,line cap=round] (114.01,115.46) -- (116.87,118.31);

\path[draw=drawColor,draw opacity=0.50,line width= 0.4pt,line join=round,line cap=round] (114.01,118.31) -- (116.87,115.46);

\path[draw=drawColor,draw opacity=0.50,line width= 0.4pt,line join=round,line cap=round] (113.42,116.88) -- (117.46,116.88);

\path[draw=drawColor,draw opacity=0.50,line width= 0.4pt,line join=round,line cap=round] (115.44,114.87) -- (115.44,118.90);

\path[draw=drawColor,draw opacity=0.50,line width= 0.4pt,line join=round,line cap=round] (114.85,116.09) -- (117.70,118.94);

\path[draw=drawColor,draw opacity=0.50,line width= 0.4pt,line join=round,line cap=round] (114.85,118.94) -- (117.70,116.09);

\path[draw=drawColor,draw opacity=0.50,line width= 0.4pt,line join=round,line cap=round] (114.26,117.51) -- (118.29,117.51);

\path[draw=drawColor,draw opacity=0.50,line width= 0.4pt,line join=round,line cap=round] (116.27,115.49) -- (116.27,119.53);

\path[draw=drawColor,draw opacity=0.50,line width= 0.4pt,line join=round,line cap=round] (116.72,117.43) -- (119.58,120.29);

\path[draw=drawColor,draw opacity=0.50,line width= 0.4pt,line join=round,line cap=round] (116.72,120.29) -- (119.58,117.43);

\path[draw=drawColor,draw opacity=0.50,line width= 0.4pt,line join=round,line cap=round] (116.13,118.86) -- (120.17,118.86);

\path[draw=drawColor,draw opacity=0.50,line width= 0.4pt,line join=round,line cap=round] (118.15,116.84) -- (118.15,120.88);

\path[draw=drawColor,draw opacity=0.50,line width= 0.4pt,line join=round,line cap=round] (114.76,116.49) -- (117.61,119.34);

\path[draw=drawColor,draw opacity=0.50,line width= 0.4pt,line join=round,line cap=round] (114.76,119.34) -- (117.61,116.49);

\path[draw=drawColor,draw opacity=0.50,line width= 0.4pt,line join=round,line cap=round] (114.17,117.91) -- (118.20,117.91);

\path[draw=drawColor,draw opacity=0.50,line width= 0.4pt,line join=round,line cap=round] (116.19,115.90) -- (116.19,119.93);

\path[draw=drawColor,draw opacity=0.50,line width= 0.4pt,line join=round,line cap=round] (114.85,116.68) -- (117.70,119.54);

\path[draw=drawColor,draw opacity=0.50,line width= 0.4pt,line join=round,line cap=round] (114.85,119.54) -- (117.70,116.68);

\path[draw=drawColor,draw opacity=0.50,line width= 0.4pt,line join=round,line cap=round] (114.26,118.11) -- (118.29,118.11);

\path[draw=drawColor,draw opacity=0.50,line width= 0.4pt,line join=round,line cap=round] (116.27,116.09) -- (116.27,120.13);

\path[draw=drawColor,draw opacity=0.50,line width= 0.4pt,line join=round,line cap=round] (115.49,116.87) -- (118.34,119.73);

\path[draw=drawColor,draw opacity=0.50,line width= 0.4pt,line join=round,line cap=round] (115.49,119.73) -- (118.34,116.87);

\path[draw=drawColor,draw opacity=0.50,line width= 0.4pt,line join=round,line cap=round] (114.90,118.30) -- (118.93,118.30);

\path[draw=drawColor,draw opacity=0.50,line width= 0.4pt,line join=round,line cap=round] (116.92,116.28) -- (116.92,120.32);

\path[draw=drawColor,draw opacity=0.50,line width= 0.4pt,line join=round,line cap=round] (115.05,116.68) -- (117.90,119.54);

\path[draw=drawColor,draw opacity=0.50,line width= 0.4pt,line join=round,line cap=round] (115.05,119.54) -- (117.90,116.68);

\path[draw=drawColor,draw opacity=0.50,line width= 0.4pt,line join=round,line cap=round] (114.45,118.11) -- (118.49,118.11);

\path[draw=drawColor,draw opacity=0.50,line width= 0.4pt,line join=round,line cap=round] (116.47,116.09) -- (116.47,120.13);

\path[draw=drawColor,draw opacity=0.50,line width= 0.4pt,line join=round,line cap=round] (114.90,117.06) -- (117.76,119.92);

\path[draw=drawColor,draw opacity=0.50,line width= 0.4pt,line join=round,line cap=round] (114.90,119.92) -- (117.76,117.06);

\path[draw=drawColor,draw opacity=0.50,line width= 0.4pt,line join=round,line cap=round] (114.31,118.49) -- (118.35,118.49);

\path[draw=drawColor,draw opacity=0.50,line width= 0.4pt,line join=round,line cap=round] (116.33,116.47) -- (116.33,120.51);

\path[draw=drawColor,draw opacity=0.50,line width= 0.4pt,line join=round,line cap=round] (115.19,117.06) -- (118.04,119.92);

\path[draw=drawColor,draw opacity=0.50,line width= 0.4pt,line join=round,line cap=round] (115.19,119.92) -- (118.04,117.06);

\path[draw=drawColor,draw opacity=0.50,line width= 0.4pt,line join=round,line cap=round] (114.59,118.49) -- (118.63,118.49);

\path[draw=drawColor,draw opacity=0.50,line width= 0.4pt,line join=round,line cap=round] (116.61,116.47) -- (116.61,120.51);

\path[draw=drawColor,draw opacity=0.50,line width= 0.4pt,line join=round,line cap=round] (115.81,117.25) -- (118.66,120.10);

\path[draw=drawColor,draw opacity=0.50,line width= 0.4pt,line join=round,line cap=round] (115.81,120.10) -- (118.66,117.25);

\path[draw=drawColor,draw opacity=0.50,line width= 0.4pt,line join=round,line cap=round] (115.22,118.68) -- (119.25,118.68);

\path[draw=drawColor,draw opacity=0.50,line width= 0.4pt,line join=round,line cap=round] (117.24,116.66) -- (117.24,120.69);
\definecolor{drawColor}{RGB}{231,41,138}

\path[draw=drawColor,draw opacity=0.50,line width= 0.4pt,line join=round,line cap=round] (186.38,174.26) -- (190.42,174.26);

\path[draw=drawColor,draw opacity=0.50,line width= 0.4pt,line join=round,line cap=round] (188.40,172.24) -- (188.40,176.27);

\path[draw=drawColor,draw opacity=0.50,line width= 0.4pt,line join=round,line cap=round] (186.38,173.60) -- (190.42,173.60);

\path[draw=drawColor,draw opacity=0.50,line width= 0.4pt,line join=round,line cap=round] (188.40,171.58) -- (188.40,175.61);

\path[draw=drawColor,draw opacity=0.50,line width= 0.4pt,line join=round,line cap=round] (186.38,174.39) -- (190.42,174.39);

\path[draw=drawColor,draw opacity=0.50,line width= 0.4pt,line join=round,line cap=round] (188.40,172.37) -- (188.40,176.40);

\path[draw=drawColor,draw opacity=0.50,line width= 0.4pt,line join=round,line cap=round] (186.38,175.19) -- (190.42,175.19);

\path[draw=drawColor,draw opacity=0.50,line width= 0.4pt,line join=round,line cap=round] (188.40,173.18) -- (188.40,177.21);

\path[draw=drawColor,draw opacity=0.50,line width= 0.4pt,line join=round,line cap=round] (186.38,175.17) -- (190.42,175.17);

\path[draw=drawColor,draw opacity=0.50,line width= 0.4pt,line join=round,line cap=round] (188.40,173.15) -- (188.40,177.19);

\path[draw=drawColor,draw opacity=0.50,line width= 0.4pt,line join=round,line cap=round] (186.38,175.61) -- (190.42,175.61);

\path[draw=drawColor,draw opacity=0.50,line width= 0.4pt,line join=round,line cap=round] (188.40,173.60) -- (188.40,177.63);

\path[draw=drawColor,draw opacity=0.50,line width= 0.4pt,line join=round,line cap=round] (186.38,174.99) -- (190.42,174.99);

\path[draw=drawColor,draw opacity=0.50,line width= 0.4pt,line join=round,line cap=round] (188.40,172.97) -- (188.40,177.01);

\path[draw=drawColor,draw opacity=0.50,line width= 0.4pt,line join=round,line cap=round] (186.38,175.14) -- (190.42,175.14);

\path[draw=drawColor,draw opacity=0.50,line width= 0.4pt,line join=round,line cap=round] (188.40,173.12) -- (188.40,177.16);

\path[draw=drawColor,draw opacity=0.50,line width= 0.4pt,line join=round,line cap=round] (174.85,175.35) -- (178.89,175.35);

\path[draw=drawColor,draw opacity=0.50,line width= 0.4pt,line join=round,line cap=round] (176.87,173.33) -- (176.87,177.37);

\path[draw=drawColor,draw opacity=0.50,line width= 0.4pt,line join=round,line cap=round] (186.38,173.49) -- (190.42,173.49);

\path[draw=drawColor,draw opacity=0.50,line width= 0.4pt,line join=round,line cap=round] (188.40,171.48) -- (188.40,175.51);

\path[draw=drawColor,draw opacity=0.50,line width= 0.4pt,line join=round,line cap=round] (186.38,173.11) -- (190.42,173.11);

\path[draw=drawColor,draw opacity=0.50,line width= 0.4pt,line join=round,line cap=round] (188.40,171.10) -- (188.40,175.13);

\path[draw=drawColor,draw opacity=0.50,line width= 0.4pt,line join=round,line cap=round] (186.38,174.23) -- (190.42,174.23);

\path[draw=drawColor,draw opacity=0.50,line width= 0.4pt,line join=round,line cap=round] (188.40,172.22) -- (188.40,176.25);

\path[draw=drawColor,draw opacity=0.50,line width= 0.4pt,line join=round,line cap=round] (176.70,174.61) -- (180.74,174.61);

\path[draw=drawColor,draw opacity=0.50,line width= 0.4pt,line join=round,line cap=round] (178.72,172.60) -- (178.72,176.63);

\path[draw=drawColor,draw opacity=0.50,line width= 0.4pt,line join=round,line cap=round] (172.20,176.23) -- (176.23,176.23);

\path[draw=drawColor,draw opacity=0.50,line width= 0.4pt,line join=round,line cap=round] (174.21,174.22) -- (174.21,178.25);

\path[draw=drawColor,draw opacity=0.50,line width= 0.4pt,line join=round,line cap=round] (172.61,176.42) -- (176.64,176.42);

\path[draw=drawColor,draw opacity=0.50,line width= 0.4pt,line join=round,line cap=round] (174.63,174.40) -- (174.63,178.44);

\path[draw=drawColor,draw opacity=0.50,line width= 0.4pt,line join=round,line cap=round] (174.04,176.38) -- (178.07,176.38);

\path[draw=drawColor,draw opacity=0.50,line width= 0.4pt,line join=round,line cap=round] (176.05,174.36) -- (176.05,178.40);

\path[draw=drawColor,draw opacity=0.50,line width= 0.4pt,line join=round,line cap=round] (107.98,112.80) -- (112.01,112.80);

\path[draw=drawColor,draw opacity=0.50,line width= 0.4pt,line join=round,line cap=round] (109.99,110.78) -- (109.99,114.81);

\path[draw=drawColor,draw opacity=0.50,line width= 0.4pt,line join=round,line cap=round] (186.38,113.39) -- (190.42,113.39);

\path[draw=drawColor,draw opacity=0.50,line width= 0.4pt,line join=round,line cap=round] (188.40,111.38) -- (188.40,115.41);

\path[draw=drawColor,draw opacity=0.50,line width= 0.4pt,line join=round,line cap=round] (186.38,113.96) -- (190.42,113.96);

\path[draw=drawColor,draw opacity=0.50,line width= 0.4pt,line join=round,line cap=round] (188.40,111.94) -- (188.40,115.98);

\path[draw=drawColor,draw opacity=0.50,line width= 0.4pt,line join=round,line cap=round] (186.38,113.96) -- (190.42,113.96);

\path[draw=drawColor,draw opacity=0.50,line width= 0.4pt,line join=round,line cap=round] (188.40,111.94) -- (188.40,115.98);

\path[draw=drawColor,draw opacity=0.50,line width= 0.4pt,line join=round,line cap=round] (186.38,114.50) -- (190.42,114.50);

\path[draw=drawColor,draw opacity=0.50,line width= 0.4pt,line join=round,line cap=round] (188.40,112.48) -- (188.40,116.52);

\path[draw=drawColor,draw opacity=0.50,line width= 0.4pt,line join=round,line cap=round] (186.38,113.96) -- (190.42,113.96);

\path[draw=drawColor,draw opacity=0.50,line width= 0.4pt,line join=round,line cap=round] (188.40,111.94) -- (188.40,115.98);

\path[draw=drawColor,draw opacity=0.50,line width= 0.4pt,line join=round,line cap=round] (108.60,113.68) -- (112.63,113.68);

\path[draw=drawColor,draw opacity=0.50,line width= 0.4pt,line join=round,line cap=round] (110.62,111.66) -- (110.62,115.70);

\path[draw=drawColor,draw opacity=0.50,line width= 0.4pt,line join=round,line cap=round] (107.37,113.96) -- (111.40,113.96);

\path[draw=drawColor,draw opacity=0.50,line width= 0.4pt,line join=round,line cap=round] (109.39,111.94) -- (109.39,115.98);

\path[draw=drawColor,draw opacity=0.50,line width= 0.4pt,line join=round,line cap=round] (109.50,115.51) -- (113.53,115.51);

\path[draw=drawColor,draw opacity=0.50,line width= 0.4pt,line join=round,line cap=round] (111.51,113.50) -- (111.51,117.53);

\path[draw=drawColor,draw opacity=0.50,line width= 0.4pt,line join=round,line cap=round] (108.83,112.80) -- (112.86,112.80);

\path[draw=drawColor,draw opacity=0.50,line width= 0.4pt,line join=round,line cap=round] (110.85,110.78) -- (110.85,114.81);

\path[draw=drawColor,draw opacity=0.50,line width= 0.4pt,line join=round,line cap=round] (108.07,113.96) -- (112.11,113.96);

\path[draw=drawColor,draw opacity=0.50,line width= 0.4pt,line join=round,line cap=round] (110.09,111.94) -- (110.09,115.98);

\path[draw=drawColor,draw opacity=0.50,line width= 0.4pt,line join=round,line cap=round] (109.19,113.39) -- (113.23,113.39);

\path[draw=drawColor,draw opacity=0.50,line width= 0.4pt,line join=round,line cap=round] (111.21,111.38) -- (111.21,115.41);

\path[draw=drawColor,draw opacity=0.50,line width= 0.4pt,line join=round,line cap=round] (109.71,113.96) -- (113.75,113.96);

\path[draw=drawColor,draw opacity=0.50,line width= 0.4pt,line join=round,line cap=round] (111.73,111.94) -- (111.73,115.98);

\path[draw=drawColor,draw opacity=0.50,line width= 0.4pt,line join=round,line cap=round] (108.02,113.96) -- (112.06,113.96);

\path[draw=drawColor,draw opacity=0.50,line width= 0.4pt,line join=round,line cap=round] (110.04,111.94) -- (110.04,115.98);

\path[draw=drawColor,draw opacity=0.50,line width= 0.4pt,line join=round,line cap=round] (107.78,114.50) -- (111.81,114.50);

\path[draw=drawColor,draw opacity=0.50,line width= 0.4pt,line join=round,line cap=round] (109.79,112.48) -- (109.79,116.52);

\path[draw=drawColor,draw opacity=0.50,line width= 0.4pt,line join=round,line cap=round] (107.98,113.10) -- (112.01,113.10);

\path[draw=drawColor,draw opacity=0.50,line width= 0.4pt,line join=round,line cap=round] (109.99,111.08) -- (109.99,115.12);

\path[fill=fillColor,fill opacity=0.50] (119.14,116.29) --
	(122.00,116.29) --
	(122.00,119.14) --
	(119.14,119.14) --
	cycle;

\path[fill=fillColor,fill opacity=0.50] (117.01,117.97) --
	(119.87,117.97) --
	(119.87,120.82) --
	(117.01,120.82) --
	cycle;

\path[fill=fillColor,fill opacity=0.50] (115.32,116.87) --
	(118.18,116.87) --
	(118.18,119.73) --
	(115.32,119.73) --
	cycle;

\path[fill=fillColor,fill opacity=0.50] (115.70,118.14) --
	(118.56,118.14) --
	(118.56,120.99) --
	(115.70,120.99) --
	cycle;

\path[fill=fillColor,fill opacity=0.50] (116.75,117.79) --
	(119.60,117.79) --
	(119.60,120.64) --
	(116.75,120.64) --
	cycle;

\path[fill=fillColor,fill opacity=0.50] (116.85,117.79) --
	(119.70,117.79) --
	(119.70,120.64) --
	(116.85,120.64) --
	cycle;

\path[fill=fillColor,fill opacity=0.50] (116.58,118.31) --
	(119.43,118.31) --
	(119.43,121.16) --
	(116.58,121.16) --
	cycle;

\path[fill=fillColor,fill opacity=0.50] (115.54,118.96) --
	(118.40,118.96) --
	(118.40,121.82) --
	(115.54,121.82) --
	cycle;

\path[fill=fillColor,fill opacity=0.50] (115.84,118.96) --
	(118.69,118.96) --
	(118.69,121.82) --
	(115.84,121.82) --
	cycle;

\path[fill=fillColor,fill opacity=0.50] (115.62,116.29) --
	(118.48,116.29) --
	(118.48,119.14) --
	(115.62,119.14) --
	cycle;

\path[fill=fillColor,fill opacity=0.50] (114.79,117.97) --
	(117.64,117.97) --
	(117.64,120.82) --
	(114.79,120.82) --
	cycle;

\path[fill=fillColor,fill opacity=0.50] (116.58,118.80) --
	(119.43,118.80) --
	(119.43,121.66) --
	(116.58,121.66) --
	cycle;

\path[fill=fillColor,fill opacity=0.50] (116.53,116.68) --
	(119.38,116.68) --
	(119.38,119.54) --
	(116.53,119.54) --
	cycle;

\path[fill=fillColor,fill opacity=0.50] (117.18,119.28) --
	(120.03,119.28) --
	(120.03,122.13) --
	(117.18,122.13) --
	cycle;

\path[fill=fillColor,fill opacity=0.50] (116.38,118.96) --
	(119.23,118.96) --
	(119.23,121.82) --
	(116.38,121.82) --
	cycle;

\path[fill=fillColor,fill opacity=0.50] (117.01,119.12) --
	(119.87,119.12) --
	(119.87,121.97) --
	(117.01,121.97) --
	cycle;

\path[draw=drawColor,draw opacity=0.50,line width= 0.4pt,line join=round,line cap=round] (147.75,192.66) -- (151.79,192.66);

\path[draw=drawColor,draw opacity=0.50,line width= 0.4pt,line join=round,line cap=round] (149.77,190.65) -- (149.77,194.68);

\path[draw=drawColor,draw opacity=0.50,line width= 0.4pt,line join=round,line cap=round] (144.77,192.11) -- (148.81,192.11);

\path[draw=drawColor,draw opacity=0.50,line width= 0.4pt,line join=round,line cap=round] (146.79,190.09) -- (146.79,194.13);

\path[draw=drawColor,draw opacity=0.50,line width= 0.4pt,line join=round,line cap=round] (151.09,192.03) -- (155.13,192.03);

\path[draw=drawColor,draw opacity=0.50,line width= 0.4pt,line join=round,line cap=round] (153.11,190.01) -- (153.11,194.05);

\path[draw=drawColor,draw opacity=0.50,line width= 0.4pt,line join=round,line cap=round] (145.72,191.33) -- (149.75,191.33);

\path[draw=drawColor,draw opacity=0.50,line width= 0.4pt,line join=round,line cap=round] (147.74,189.32) -- (147.74,193.35);

\path[draw=drawColor,draw opacity=0.50,line width= 0.4pt,line join=round,line cap=round] (146.65,191.21) -- (150.69,191.21);

\path[draw=drawColor,draw opacity=0.50,line width= 0.4pt,line join=round,line cap=round] (148.67,189.19) -- (148.67,193.23);

\path[draw=drawColor,draw opacity=0.50,line width= 0.4pt,line join=round,line cap=round] (145.03,190.81) -- (149.07,190.81);

\path[draw=drawColor,draw opacity=0.50,line width= 0.4pt,line join=round,line cap=round] (147.05,188.79) -- (147.05,192.83);

\path[draw=drawColor,draw opacity=0.50,line width= 0.4pt,line join=round,line cap=round] (137.80,191.99) -- (141.83,191.99);

\path[draw=drawColor,draw opacity=0.50,line width= 0.4pt,line join=round,line cap=round] (139.82,189.97) -- (139.82,194.01);

\path[draw=drawColor,draw opacity=0.50,line width= 0.4pt,line join=round,line cap=round] (186.38,192.57) -- (190.42,192.57);

\path[draw=drawColor,draw opacity=0.50,line width= 0.4pt,line join=round,line cap=round] (188.40,190.55) -- (188.40,194.58);

\path[draw=drawColor,draw opacity=0.50,line width= 0.4pt,line join=round,line cap=round] (186.38,191.56) -- (190.42,191.56);

\path[draw=drawColor,draw opacity=0.50,line width= 0.4pt,line join=round,line cap=round] (188.40,189.54) -- (188.40,193.57);

\path[draw=drawColor,draw opacity=0.50,line width= 0.4pt,line join=round,line cap=round] (117.16,191.33) -- (121.20,191.33);

\path[draw=drawColor,draw opacity=0.50,line width= 0.4pt,line join=round,line cap=round] (119.18,189.31) -- (119.18,193.35);

\path[draw=drawColor,draw opacity=0.50,line width= 0.4pt,line join=round,line cap=round] (114.62,191.89) -- (118.66,191.89);

\path[draw=drawColor,draw opacity=0.50,line width= 0.4pt,line join=round,line cap=round] (116.64,189.87) -- (116.64,193.91);

\path[draw=drawColor,draw opacity=0.50,line width= 0.4pt,line join=round,line cap=round] (161.05,190.99) -- (165.08,190.99);

\path[draw=drawColor,draw opacity=0.50,line width= 0.4pt,line join=round,line cap=round] (163.06,188.98) -- (163.06,193.01);

\path[draw=drawColor,draw opacity=0.50,line width= 0.4pt,line join=round,line cap=round] (165.97,190.94) -- (170.01,190.94);

\path[draw=drawColor,draw opacity=0.50,line width= 0.4pt,line join=round,line cap=round] (167.99,188.92) -- (167.99,192.96);

\path[draw=drawColor,draw opacity=0.50,line width= 0.4pt,line join=round,line cap=round] (166.11,192.02) -- (170.14,192.02);

\path[draw=drawColor,draw opacity=0.50,line width= 0.4pt,line join=round,line cap=round] (168.12,190.00) -- (168.12,194.04);

\path[draw=drawColor,draw opacity=0.50,line width= 0.4pt,line join=round,line cap=round] (162.72,190.05) -- (166.75,190.05);

\path[draw=drawColor,draw opacity=0.50,line width= 0.4pt,line join=round,line cap=round] (164.73,188.03) -- (164.73,192.06);

\path[draw=drawColor,draw opacity=0.50,line width= 0.4pt,line join=round,line cap=round] (150.26,192.27) -- (154.30,192.27);

\path[draw=drawColor,draw opacity=0.50,line width= 0.4pt,line join=round,line cap=round] (152.28,190.26) -- (152.28,194.29);

\path[draw=drawColor,draw opacity=0.50,line width= 0.4pt,line join=round,line cap=round] (161.51,164.83) -- (165.55,164.83);

\path[draw=drawColor,draw opacity=0.50,line width= 0.4pt,line join=round,line cap=round] (163.53,162.81) -- (163.53,166.84);

\path[draw=drawColor,draw opacity=0.50,line width= 0.4pt,line join=round,line cap=round] (113.39,140.54) -- (117.43,140.54);

\path[draw=drawColor,draw opacity=0.50,line width= 0.4pt,line join=round,line cap=round] (115.41,138.52) -- (115.41,142.56);

\path[draw=drawColor,draw opacity=0.50,line width= 0.4pt,line join=round,line cap=round] (112.85,129.80) -- (116.89,129.80);

\path[draw=drawColor,draw opacity=0.50,line width= 0.4pt,line join=round,line cap=round] (114.87,127.78) -- (114.87,131.82);

\path[draw=drawColor,draw opacity=0.50,line width= 0.4pt,line join=round,line cap=round] (112.12,129.07) -- (116.15,129.07);

\path[draw=drawColor,draw opacity=0.50,line width= 0.4pt,line join=round,line cap=round] (114.14,127.05) -- (114.14,131.09);

\path[draw=drawColor,draw opacity=0.50,line width= 0.4pt,line join=round,line cap=round] (110.61,137.02) -- (114.64,137.02);

\path[draw=drawColor,draw opacity=0.50,line width= 0.4pt,line join=round,line cap=round] (112.63,135.00) -- (112.63,139.03);

\path[draw=drawColor,draw opacity=0.50,line width= 0.4pt,line join=round,line cap=round] (112.89,131.70) -- (116.92,131.70);

\path[draw=drawColor,draw opacity=0.50,line width= 0.4pt,line join=round,line cap=round] (114.90,129.68) -- (114.90,133.71);

\path[draw=drawColor,draw opacity=0.50,line width= 0.4pt,line join=round,line cap=round] (115.32,141.07) -- (119.36,141.07);

\path[draw=drawColor,draw opacity=0.50,line width= 0.4pt,line join=round,line cap=round] (117.34,139.05) -- (117.34,143.08);

\path[draw=drawColor,draw opacity=0.50,line width= 0.4pt,line join=round,line cap=round] (186.38,134.33) -- (190.42,134.33);

\path[draw=drawColor,draw opacity=0.50,line width= 0.4pt,line join=round,line cap=round] (188.40,132.31) -- (188.40,136.34);

\path[draw=drawColor,draw opacity=0.50,line width= 0.4pt,line join=round,line cap=round] (111.15,127.19) -- (115.18,127.19);

\path[draw=drawColor,draw opacity=0.50,line width= 0.4pt,line join=round,line cap=round] (113.16,125.17) -- (113.16,129.21);

\path[draw=drawColor,draw opacity=0.50,line width= 0.4pt,line join=round,line cap=round] (113.33,129.80) -- (117.36,129.80);

\path[draw=drawColor,draw opacity=0.50,line width= 0.4pt,line join=round,line cap=round] (115.35,127.78) -- (115.35,131.82);

\path[draw=drawColor,draw opacity=0.50,line width= 0.4pt,line join=round,line cap=round] (115.30,148.77) -- (119.33,148.77);

\path[draw=drawColor,draw opacity=0.50,line width= 0.4pt,line join=round,line cap=round] (117.32,146.75) -- (117.32,150.79);

\path[draw=drawColor,draw opacity=0.50,line width= 0.4pt,line join=round,line cap=round] (133.80,164.26) -- (137.84,164.26);

\path[draw=drawColor,draw opacity=0.50,line width= 0.4pt,line join=round,line cap=round] (135.82,162.24) -- (135.82,166.28);

\path[draw=drawColor,draw opacity=0.50,line width= 0.4pt,line join=round,line cap=round] (161.65,176.33) -- (165.69,176.33);

\path[draw=drawColor,draw opacity=0.50,line width= 0.4pt,line join=round,line cap=round] (163.67,174.31) -- (163.67,178.34);

\path[draw=drawColor,draw opacity=0.50,line width= 0.4pt,line join=round,line cap=round] (162.84,175.54) -- (166.87,175.54);

\path[draw=drawColor,draw opacity=0.50,line width= 0.4pt,line join=round,line cap=round] (164.85,173.53) -- (164.85,177.56);

\path[draw=drawColor,draw opacity=0.50,line width= 0.4pt,line join=round,line cap=round] (162.32,177.73) -- (166.35,177.73);

\path[draw=drawColor,draw opacity=0.50,line width= 0.4pt,line join=round,line cap=round] (164.34,175.71) -- (164.34,179.75);

\path[draw=drawColor,draw opacity=0.50,line width= 0.4pt,line join=round,line cap=round] (161.30,174.46) -- (165.34,174.46);

\path[draw=drawColor,draw opacity=0.50,line width= 0.4pt,line join=round,line cap=round] (163.32,172.44) -- (163.32,176.48);

\path[draw=drawColor,draw opacity=0.50,line width= 0.4pt,line join=round,line cap=round] (151.29,150.60) -- (155.33,150.60);

\path[draw=drawColor,draw opacity=0.50,line width= 0.4pt,line join=round,line cap=round] (153.31,148.59) -- (153.31,152.62);

\path[draw=drawColor,draw opacity=0.50,line width= 0.4pt,line join=round,line cap=round] (128.41,149.34) -- (132.45,149.34);

\path[draw=drawColor,draw opacity=0.50,line width= 0.4pt,line join=round,line cap=round] (130.43,147.32) -- (130.43,151.36);

\path[draw=drawColor,draw opacity=0.50,line width= 0.4pt,line join=round,line cap=round] (120.85,149.07) -- (124.88,149.07);

\path[draw=drawColor,draw opacity=0.50,line width= 0.4pt,line join=round,line cap=round] (122.87,147.05) -- (122.87,151.09);

\path[draw=drawColor,draw opacity=0.50,line width= 0.4pt,line join=round,line cap=round] (121.20,148.07) -- (125.24,148.07);

\path[draw=drawColor,draw opacity=0.50,line width= 0.4pt,line join=round,line cap=round] (123.22,146.05) -- (123.22,150.09);

\path[draw=drawColor,draw opacity=0.50,line width= 0.4pt,line join=round,line cap=round] (119.11,147.42) -- (123.15,147.42);

\path[draw=drawColor,draw opacity=0.50,line width= 0.4pt,line join=round,line cap=round] (121.13,145.40) -- (121.13,149.44);

\path[draw=drawColor,draw opacity=0.50,line width= 0.4pt,line join=round,line cap=round] (120.50,148.29) -- (124.54,148.29);

\path[draw=drawColor,draw opacity=0.50,line width= 0.4pt,line join=round,line cap=round] (122.52,146.27) -- (122.52,150.31);

\path[draw=drawColor,draw opacity=0.50,line width= 0.4pt,line join=round,line cap=round] (117.79,147.68) -- (121.83,147.68);

\path[draw=drawColor,draw opacity=0.50,line width= 0.4pt,line join=round,line cap=round] (119.81,145.66) -- (119.81,149.70);

\path[draw=drawColor,draw opacity=0.50,line width= 0.4pt,line join=round,line cap=round] (118.45,147.39) -- (122.49,147.39);

\path[draw=drawColor,draw opacity=0.50,line width= 0.4pt,line join=round,line cap=round] (120.47,145.37) -- (120.47,149.40);

\path[draw=drawColor,draw opacity=0.50,line width= 0.4pt,line join=round,line cap=round] (120.23,148.64) -- (124.27,148.64);

\path[draw=drawColor,draw opacity=0.50,line width= 0.4pt,line join=round,line cap=round] (122.25,146.63) -- (122.25,150.66);

\path[draw=drawColor,draw opacity=0.50,line width= 0.4pt,line join=round,line cap=round] (118.17,147.98) -- (122.20,147.98);

\path[draw=drawColor,draw opacity=0.50,line width= 0.4pt,line join=round,line cap=round] (120.19,145.96) -- (120.19,150.00);

\path[draw=drawColor,draw opacity=0.50,line width= 0.4pt,line join=round,line cap=round] (118.88,148.56) -- (122.92,148.56);

\path[draw=drawColor,draw opacity=0.50,line width= 0.4pt,line join=round,line cap=round] (120.90,146.54) -- (120.90,150.58);

\path[draw=drawColor,draw opacity=0.50,line width= 0.4pt,line join=round,line cap=round] (186.38,151.50) -- (190.42,151.50);

\path[draw=drawColor,draw opacity=0.50,line width= 0.4pt,line join=round,line cap=round] (188.40,149.49) -- (188.40,153.52);

\path[draw=drawColor,draw opacity=0.50,line width= 0.4pt,line join=round,line cap=round] (186.38,153.59) -- (190.42,153.59);

\path[draw=drawColor,draw opacity=0.50,line width= 0.4pt,line join=round,line cap=round] (188.40,151.57) -- (188.40,155.60);

\path[draw=drawColor,draw opacity=0.50,line width= 0.4pt,line join=round,line cap=round] (186.38,151.69) -- (190.42,151.69);

\path[draw=drawColor,draw opacity=0.50,line width= 0.4pt,line join=round,line cap=round] (188.40,149.67) -- (188.40,153.71);

\path[draw=drawColor,draw opacity=0.50,line width= 0.4pt,line join=round,line cap=round] (186.38,151.66) -- (190.42,151.66);

\path[draw=drawColor,draw opacity=0.50,line width= 0.4pt,line join=round,line cap=round] (188.40,149.64) -- (188.40,153.67);

\path[draw=drawColor,draw opacity=0.50,line width= 0.4pt,line join=round,line cap=round] (186.38,151.33) -- (190.42,151.33);

\path[draw=drawColor,draw opacity=0.50,line width= 0.4pt,line join=round,line cap=round] (188.40,149.31) -- (188.40,153.34);

\path[draw=drawColor,draw opacity=0.50,line width= 0.4pt,line join=round,line cap=round] (151.88,149.02) -- (155.92,149.02);

\path[draw=drawColor,draw opacity=0.50,line width= 0.4pt,line join=round,line cap=round] (153.90,147.00) -- (153.90,151.04);

\path[draw=drawColor,draw opacity=0.50,line width= 0.4pt,line join=round,line cap=round] (120.14,149.51) -- (124.18,149.51);

\path[draw=drawColor,draw opacity=0.50,line width= 0.4pt,line join=round,line cap=round] (122.16,147.49) -- (122.16,151.53);

\path[draw=drawColor,draw opacity=0.50,line width= 0.4pt,line join=round,line cap=round] (121.51,149.43) -- (125.55,149.43);

\path[draw=drawColor,draw opacity=0.50,line width= 0.4pt,line join=round,line cap=round] (123.53,147.41) -- (123.53,151.45);

\path[draw=drawColor,draw opacity=0.50,line width= 0.4pt,line join=round,line cap=round] (123.12,149.65) -- (127.16,149.65);

\path[draw=drawColor,draw opacity=0.50,line width= 0.4pt,line join=round,line cap=round] (125.14,147.63) -- (125.14,151.67);

\path[draw=drawColor,draw opacity=0.50,line width= 0.4pt,line join=round,line cap=round] (121.64,149.21) -- (125.67,149.21);

\path[draw=drawColor,draw opacity=0.50,line width= 0.4pt,line join=round,line cap=round] (123.65,147.19) -- (123.65,151.22);

\path[draw=drawColor,draw opacity=0.50,line width= 0.4pt,line join=round,line cap=round] (123.37,147.13) -- (127.41,147.13);

\path[draw=drawColor,draw opacity=0.50,line width= 0.4pt,line join=round,line cap=round] (125.39,145.12) -- (125.39,149.15);

\path[draw=drawColor,draw opacity=0.50,line width= 0.4pt,line join=round,line cap=round] (132.25,147.25) -- (136.29,147.25);

\path[draw=drawColor,draw opacity=0.50,line width= 0.4pt,line join=round,line cap=round] (134.27,145.23) -- (134.27,149.26);

\path[draw=drawColor,draw opacity=0.50,line width= 0.4pt,line join=round,line cap=round] (120.87,148.53) -- (124.90,148.53);

\path[draw=drawColor,draw opacity=0.50,line width= 0.4pt,line join=round,line cap=round] (122.88,146.51) -- (122.88,150.55);

\path[draw=drawColor,draw opacity=0.50,line width= 0.4pt,line join=round,line cap=round] (118.90,148.16) -- (122.94,148.16);

\path[draw=drawColor,draw opacity=0.50,line width= 0.4pt,line join=round,line cap=round] (120.92,146.14) -- (120.92,150.18);

\path[draw=drawColor,draw opacity=0.50,line width= 0.4pt,line join=round,line cap=round] (117.27,146.48) -- (121.31,146.48);

\path[draw=drawColor,draw opacity=0.50,line width= 0.4pt,line join=round,line cap=round] (119.29,144.46) -- (119.29,148.49);

\path[draw=drawColor,draw opacity=0.50,line width= 0.4pt,line join=round,line cap=round] (119.23,148.67) -- (123.26,148.67);

\path[draw=drawColor,draw opacity=0.50,line width= 0.4pt,line join=round,line cap=round] (121.24,146.66) -- (121.24,150.69);

\path[draw=drawColor,draw opacity=0.50,line width= 0.4pt,line join=round,line cap=round] (156.68,150.65) -- (160.72,150.65);

\path[draw=drawColor,draw opacity=0.50,line width= 0.4pt,line join=round,line cap=round] (158.70,148.63) -- (158.70,152.67);

\path[draw=drawColor,draw opacity=0.50,line width= 0.4pt,line join=round,line cap=round] (149.46,150.50) -- (153.49,150.50);

\path[draw=drawColor,draw opacity=0.50,line width= 0.4pt,line join=round,line cap=round] (151.48,148.48) -- (151.48,152.51);

\path[draw=drawColor,draw opacity=0.50,line width= 0.4pt,line join=round,line cap=round] (151.14,149.94) -- (155.18,149.94);

\path[draw=drawColor,draw opacity=0.50,line width= 0.4pt,line join=round,line cap=round] (153.16,147.93) -- (153.16,151.96);

\path[draw=drawColor,draw opacity=0.50,line width= 0.4pt,line join=round,line cap=round] (138.51,149.04) -- (142.54,149.04);

\path[draw=drawColor,draw opacity=0.50,line width= 0.4pt,line join=round,line cap=round] (140.53,147.03) -- (140.53,151.06);

\path[draw=drawColor,draw opacity=0.50,line width= 0.4pt,line join=round,line cap=round] (156.02,149.47) -- (160.06,149.47);

\path[draw=drawColor,draw opacity=0.50,line width= 0.4pt,line join=round,line cap=round] (158.04,147.45) -- (158.04,151.49);

\path[draw=drawColor,draw opacity=0.50,line width= 0.4pt,line join=round,line cap=round] (152.82,154.76) -- (156.85,154.76);

\path[draw=drawColor,draw opacity=0.50,line width= 0.4pt,line join=round,line cap=round] (154.83,152.75) -- (154.83,156.78);

\path[draw=drawColor,draw opacity=0.50,line width= 0.4pt,line join=round,line cap=round] (121.79,154.07) -- (125.82,154.07);

\path[draw=drawColor,draw opacity=0.50,line width= 0.4pt,line join=round,line cap=round] (123.80,152.05) -- (123.80,156.09);

\path[draw=drawColor,draw opacity=0.50,line width= 0.4pt,line join=round,line cap=round] (125.74,152.62) -- (129.77,152.62);

\path[draw=drawColor,draw opacity=0.50,line width= 0.4pt,line join=round,line cap=round] (127.75,150.60) -- (127.75,154.64);

\path[draw=drawColor,draw opacity=0.50,line width= 0.4pt,line join=round,line cap=round] (122.78,152.82) -- (126.82,152.82);

\path[draw=drawColor,draw opacity=0.50,line width= 0.4pt,line join=round,line cap=round] (124.80,150.80) -- (124.80,154.84);

\path[draw=drawColor,draw opacity=0.50,line width= 0.4pt,line join=round,line cap=round] (126.99,151.93) -- (131.03,151.93);

\path[draw=drawColor,draw opacity=0.50,line width= 0.4pt,line join=round,line cap=round] (129.01,149.92) -- (129.01,153.95);

\path[draw=drawColor,draw opacity=0.50,line width= 0.4pt,line join=round,line cap=round] (121.85,154.19) -- (125.88,154.19);

\path[draw=drawColor,draw opacity=0.50,line width= 0.4pt,line join=round,line cap=round] (123.86,152.17) -- (123.86,156.20);

\path[draw=drawColor,draw opacity=0.50,line width= 0.4pt,line join=round,line cap=round] (156.64,154.78) -- (160.68,154.78);

\path[draw=drawColor,draw opacity=0.50,line width= 0.4pt,line join=round,line cap=round] (158.66,152.76) -- (158.66,156.80);

\path[draw=drawColor,draw opacity=0.50,line width= 0.4pt,line join=round,line cap=round] (186.38,151.49) -- (190.42,151.49);

\path[draw=drawColor,draw opacity=0.50,line width= 0.4pt,line join=round,line cap=round] (188.40,149.47) -- (188.40,153.51);

\path[draw=drawColor,draw opacity=0.50,line width= 0.4pt,line join=round,line cap=round] (118.69,149.75) -- (122.72,149.75);

\path[draw=drawColor,draw opacity=0.50,line width= 0.4pt,line join=round,line cap=round] (120.71,147.74) -- (120.71,151.77);

\path[draw=drawColor,draw opacity=0.50,line width= 0.4pt,line join=round,line cap=round] (119.50,149.33) -- (123.54,149.33);

\path[draw=drawColor,draw opacity=0.50,line width= 0.4pt,line join=round,line cap=round] (121.52,147.31) -- (121.52,151.34);

\path[draw=drawColor,draw opacity=0.50,line width= 0.4pt,line join=round,line cap=round] (186.38,150.73) -- (190.42,150.73);

\path[draw=drawColor,draw opacity=0.50,line width= 0.4pt,line join=round,line cap=round] (188.40,148.71) -- (188.40,152.75);

\path[draw=drawColor,draw opacity=0.50,line width= 0.4pt,line join=round,line cap=round] (186.38,158.41) -- (190.42,158.41);

\path[draw=drawColor,draw opacity=0.50,line width= 0.4pt,line join=round,line cap=round] (188.40,156.39) -- (188.40,160.42);

\path[draw=drawColor,draw opacity=0.50,line width= 0.4pt,line join=round,line cap=round] (186.38,155.91) -- (190.42,155.91);

\path[draw=drawColor,draw opacity=0.50,line width= 0.4pt,line join=round,line cap=round] (188.40,153.89) -- (188.40,157.92);

\path[draw=drawColor,draw opacity=0.50,line width= 0.4pt,line join=round,line cap=round] (186.38,154.48) -- (190.42,154.48);

\path[draw=drawColor,draw opacity=0.50,line width= 0.4pt,line join=round,line cap=round] (188.40,152.46) -- (188.40,156.50);

\path[draw=drawColor,draw opacity=0.50,line width= 0.4pt,line join=round,line cap=round] (186.38,157.00) -- (190.42,157.00);

\path[draw=drawColor,draw opacity=0.50,line width= 0.4pt,line join=round,line cap=round] (188.40,154.99) -- (188.40,159.02);

\path[draw=drawColor,draw opacity=0.50,line width= 0.4pt,line join=round,line cap=round] (186.38,157.68) -- (190.42,157.68);

\path[draw=drawColor,draw opacity=0.50,line width= 0.4pt,line join=round,line cap=round] (188.40,155.66) -- (188.40,159.70);

\path[draw=drawColor,draw opacity=0.50,line width= 0.4pt,line join=round,line cap=round] (119.09,145.83) -- (123.13,145.83);

\path[draw=drawColor,draw opacity=0.50,line width= 0.4pt,line join=round,line cap=round] (121.11,143.82) -- (121.11,147.85);

\path[draw=drawColor,draw opacity=0.50,line width= 0.4pt,line join=round,line cap=round] (111.00,145.49) -- (115.03,145.49);

\path[draw=drawColor,draw opacity=0.50,line width= 0.4pt,line join=round,line cap=round] (113.01,143.47) -- (113.01,147.51);

\path[draw=drawColor,draw opacity=0.50,line width= 0.4pt,line join=round,line cap=round] (113.76,145.76) -- (117.79,145.76);

\path[draw=drawColor,draw opacity=0.50,line width= 0.4pt,line join=round,line cap=round] (115.77,143.74) -- (115.77,147.78);

\path[draw=drawColor,draw opacity=0.50,line width= 0.4pt,line join=round,line cap=round] (111.00,145.85) -- (115.03,145.85);

\path[draw=drawColor,draw opacity=0.50,line width= 0.4pt,line join=round,line cap=round] (113.01,143.83) -- (113.01,147.87);

\path[draw=drawColor,draw opacity=0.50,line width= 0.4pt,line join=round,line cap=round] (111.07,143.29) -- (115.11,143.29);

\path[draw=drawColor,draw opacity=0.50,line width= 0.4pt,line join=round,line cap=round] (113.09,141.27) -- (113.09,145.31);

\path[draw=drawColor,draw opacity=0.50,line width= 0.4pt,line join=round,line cap=round] (109.54,143.44) -- (113.58,143.44);

\path[draw=drawColor,draw opacity=0.50,line width= 0.4pt,line join=round,line cap=round] (111.56,141.43) -- (111.56,145.46);

\path[draw=drawColor,draw opacity=0.50,line width= 0.4pt,line join=round,line cap=round] (116.35,145.29) -- (120.39,145.29);

\path[draw=drawColor,draw opacity=0.50,line width= 0.4pt,line join=round,line cap=round] (118.37,143.27) -- (118.37,147.30);

\path[draw=drawColor,draw opacity=0.50,line width= 0.4pt,line join=round,line cap=round] (115.32,145.80) -- (119.36,145.80);

\path[draw=drawColor,draw opacity=0.50,line width= 0.4pt,line join=round,line cap=round] (117.34,143.78) -- (117.34,147.82);

\path[draw=drawColor,draw opacity=0.50,line width= 0.4pt,line join=round,line cap=round] (109.96,148.62) -- (114.00,148.62);

\path[draw=drawColor,draw opacity=0.50,line width= 0.4pt,line join=round,line cap=round] (111.98,146.60) -- (111.98,150.63);

\path[draw=drawColor,draw opacity=0.50,line width= 0.4pt,line join=round,line cap=round] (115.22,145.89) -- (119.25,145.89);

\path[draw=drawColor,draw opacity=0.50,line width= 0.4pt,line join=round,line cap=round] (117.24,143.87) -- (117.24,147.91);

\path[draw=drawColor,draw opacity=0.50,line width= 0.4pt,line join=round,line cap=round] (117.60,145.10) -- (121.64,145.10);

\path[draw=drawColor,draw opacity=0.50,line width= 0.4pt,line join=round,line cap=round] (119.62,143.08) -- (119.62,147.11);

\path[draw=drawColor,draw opacity=0.50,line width= 0.4pt,line join=round,line cap=round] (114.62,145.76) -- (118.66,145.76);

\path[draw=drawColor,draw opacity=0.50,line width= 0.4pt,line join=round,line cap=round] (116.64,143.74) -- (116.64,147.78);

\path[draw=drawColor,draw opacity=0.50,line width= 0.4pt,line join=round,line cap=round] (113.61,145.32) -- (117.64,145.32);

\path[draw=drawColor,draw opacity=0.50,line width= 0.4pt,line join=round,line cap=round] (115.62,143.31) -- (115.62,147.34);

\path[draw=drawColor,draw opacity=0.50,line width= 0.4pt,line join=round,line cap=round] (115.22,145.58) -- (119.25,145.58);

\path[draw=drawColor,draw opacity=0.50,line width= 0.4pt,line join=round,line cap=round] (117.24,143.56) -- (117.24,147.60);

\path[draw=drawColor,draw opacity=0.50,line width= 0.4pt,line join=round,line cap=round] (113.54,144.47) -- (117.58,144.47);

\path[draw=drawColor,draw opacity=0.50,line width= 0.4pt,line join=round,line cap=round] (115.56,142.45) -- (115.56,146.49);

\path[draw=drawColor,draw opacity=0.50,line width= 0.4pt,line join=round,line cap=round] (114.79,145.45) -- (118.82,145.45);

\path[draw=drawColor,draw opacity=0.50,line width= 0.4pt,line join=round,line cap=round] (116.81,143.44) -- (116.81,147.47);

\path[draw=drawColor,draw opacity=0.50,line width= 0.4pt,line join=round,line cap=round] (130.20,137.90) -- (134.23,137.90);

\path[draw=drawColor,draw opacity=0.50,line width= 0.4pt,line join=round,line cap=round] (132.21,135.88) -- (132.21,139.92);

\path[draw=drawColor,draw opacity=0.50,line width= 0.4pt,line join=round,line cap=round] (186.38,121.88) -- (190.42,121.88);

\path[draw=drawColor,draw opacity=0.50,line width= 0.4pt,line join=round,line cap=round] (188.40,119.86) -- (188.40,123.90);

\path[draw=drawColor,draw opacity=0.50,line width= 0.4pt,line join=round,line cap=round] (111.11,120.70) -- (115.14,120.70);

\path[draw=drawColor,draw opacity=0.50,line width= 0.4pt,line join=round,line cap=round] (113.13,118.69) -- (113.13,122.72);

\path[draw=drawColor,draw opacity=0.50,line width= 0.4pt,line join=round,line cap=round] (186.38,116.44) -- (190.42,116.44);

\path[draw=drawColor,draw opacity=0.50,line width= 0.4pt,line join=round,line cap=round] (188.40,114.43) -- (188.40,118.46);

\path[draw=drawColor,draw opacity=0.50,line width= 0.4pt,line join=round,line cap=round] (186.38,116.67) -- (190.42,116.67);

\path[draw=drawColor,draw opacity=0.50,line width= 0.4pt,line join=round,line cap=round] (188.40,114.65) -- (188.40,118.68);

\path[draw=drawColor,draw opacity=0.50,line width= 0.4pt,line join=round,line cap=round] (112.66,122.69) -- (116.69,122.69);

\path[draw=drawColor,draw opacity=0.50,line width= 0.4pt,line join=round,line cap=round] (114.67,120.68) -- (114.67,124.71);

\path[draw=drawColor,draw opacity=0.50,line width= 0.4pt,line join=round,line cap=round] (186.38,116.88) -- (190.42,116.88);

\path[draw=drawColor,draw opacity=0.50,line width= 0.4pt,line join=round,line cap=round] (188.40,114.87) -- (188.40,118.90);

\path[draw=drawColor,draw opacity=0.50,line width= 0.4pt,line join=round,line cap=round] (111.70,117.31) -- (115.73,117.31);

\path[draw=drawColor,draw opacity=0.50,line width= 0.4pt,line join=round,line cap=round] (113.71,115.29) -- (113.71,119.32);

\path[draw=drawColor,draw opacity=0.50,line width= 0.4pt,line join=round,line cap=round] (186.38,117.10) -- (190.42,117.10);

\path[draw=drawColor,draw opacity=0.50,line width= 0.4pt,line join=round,line cap=round] (188.40,115.08) -- (188.40,119.11);

\path[draw=drawColor,draw opacity=0.50,line width= 0.4pt,line join=round,line cap=round] (112.59,119.90) -- (116.63,119.90);

\path[draw=drawColor,draw opacity=0.50,line width= 0.4pt,line join=round,line cap=round] (114.61,117.88) -- (114.61,121.92);

\path[draw=drawColor,draw opacity=0.50,line width= 0.4pt,line join=round,line cap=round] (110.80,116.22) -- (114.84,116.22);

\path[draw=drawColor,draw opacity=0.50,line width= 0.4pt,line join=round,line cap=round] (112.82,114.20) -- (112.82,118.24);

\path[draw=drawColor,draw opacity=0.50,line width= 0.4pt,line join=round,line cap=round] (114.90,124.73) -- (118.93,124.73);

\path[draw=drawColor,draw opacity=0.50,line width= 0.4pt,line join=round,line cap=round] (116.92,122.71) -- (116.92,126.74);

\path[draw=drawColor,draw opacity=0.50,line width= 0.4pt,line join=round,line cap=round] (116.94,124.05) -- (120.97,124.05);

\path[draw=drawColor,draw opacity=0.50,line width= 0.4pt,line join=round,line cap=round] (118.95,122.03) -- (118.95,126.07);

\path[draw=drawColor,draw opacity=0.50,line width= 0.4pt,line join=round,line cap=round] (116.52,124.94) -- (120.55,124.94);

\path[draw=drawColor,draw opacity=0.50,line width= 0.4pt,line join=round,line cap=round] (118.54,122.93) -- (118.54,126.96);

\path[draw=drawColor,draw opacity=0.50,line width= 0.4pt,line join=round,line cap=round] (114.34,116.88) -- (118.38,116.88);

\path[draw=drawColor,draw opacity=0.50,line width= 0.4pt,line join=round,line cap=round] (116.36,114.87) -- (116.36,118.90);

\path[draw=drawColor,draw opacity=0.50,line width= 0.4pt,line join=round,line cap=round] (117.47,125.26) -- (121.51,125.26);

\path[draw=drawColor,draw opacity=0.50,line width= 0.4pt,line join=round,line cap=round] (119.49,123.25) -- (119.49,127.28);
\definecolor{drawColor}{RGB}{102,166,30}

\path[draw=drawColor,draw opacity=0.50,line width= 0.4pt,line join=round,line cap=round] (112.95,104.47) rectangle (115.80,107.33);

\path[draw=drawColor,draw opacity=0.50,line width= 0.4pt,line join=round,line cap=round] (112.95,104.47) -- (115.80,107.33);

\path[draw=drawColor,draw opacity=0.50,line width= 0.4pt,line join=round,line cap=round] (112.95,107.33) -- (115.80,104.47);

\path[draw=drawColor,draw opacity=0.50,line width= 0.4pt,line join=round,line cap=round] (112.03,106.50) rectangle (114.89,109.35);

\path[draw=drawColor,draw opacity=0.50,line width= 0.4pt,line join=round,line cap=round] (112.03,106.50) -- (114.89,109.35);

\path[draw=drawColor,draw opacity=0.50,line width= 0.4pt,line join=round,line cap=round] (112.03,109.35) -- (114.89,106.50);

\path[draw=drawColor,draw opacity=0.50,line width= 0.4pt,line join=round,line cap=round] (112.22,105.53) rectangle (115.07,108.38);

\path[draw=drawColor,draw opacity=0.50,line width= 0.4pt,line join=round,line cap=round] (112.22,105.53) -- (115.07,108.38);

\path[draw=drawColor,draw opacity=0.50,line width= 0.4pt,line join=round,line cap=round] (112.22,108.38) -- (115.07,105.53);

\path[draw=drawColor,draw opacity=0.50,line width= 0.4pt,line join=round,line cap=round] (112.36,111.37) rectangle (115.21,114.22);

\path[draw=drawColor,draw opacity=0.50,line width= 0.4pt,line join=round,line cap=round] (112.36,111.37) -- (115.21,114.22);

\path[draw=drawColor,draw opacity=0.50,line width= 0.4pt,line join=round,line cap=round] (112.36,114.22) -- (115.21,111.37);

\path[draw=drawColor,draw opacity=0.50,line width= 0.4pt,line join=round,line cap=round] (112.50,105.01) rectangle (115.35,107.87);

\path[draw=drawColor,draw opacity=0.50,line width= 0.4pt,line join=round,line cap=round] (112.50,105.01) -- (115.35,107.87);

\path[draw=drawColor,draw opacity=0.50,line width= 0.4pt,line join=round,line cap=round] (112.50,107.87) -- (115.35,105.01);

\path[draw=drawColor,draw opacity=0.50,line width= 0.4pt,line join=round,line cap=round] (112.57,105.01) rectangle (115.42,107.87);

\path[draw=drawColor,draw opacity=0.50,line width= 0.4pt,line join=round,line cap=round] (112.57,105.01) -- (115.42,107.87);

\path[draw=drawColor,draw opacity=0.50,line width= 0.4pt,line join=round,line cap=round] (112.57,107.87) -- (115.42,105.01);

\path[draw=drawColor,draw opacity=0.50,line width= 0.4pt,line join=round,line cap=round] (111.20,104.47) rectangle (114.05,107.33);

\path[draw=drawColor,draw opacity=0.50,line width= 0.4pt,line join=round,line cap=round] (111.20,104.47) -- (114.05,107.33);

\path[draw=drawColor,draw opacity=0.50,line width= 0.4pt,line join=round,line cap=round] (111.20,107.33) -- (114.05,104.47);

\path[draw=drawColor,draw opacity=0.50,line width= 0.4pt,line join=round,line cap=round] (111.40,103.91) rectangle (114.25,106.76);

\path[draw=drawColor,draw opacity=0.50,line width= 0.4pt,line join=round,line cap=round] (111.40,103.91) -- (114.25,106.76);

\path[draw=drawColor,draw opacity=0.50,line width= 0.4pt,line join=round,line cap=round] (111.40,106.76) -- (114.25,103.91);

\path[draw=drawColor,draw opacity=0.50,line width= 0.4pt,line join=round,line cap=round] (110.84,105.01) rectangle (113.69,107.87);

\path[draw=drawColor,draw opacity=0.50,line width= 0.4pt,line join=round,line cap=round] (110.84,105.01) -- (113.69,107.87);

\path[draw=drawColor,draw opacity=0.50,line width= 0.4pt,line join=round,line cap=round] (110.84,107.87) -- (113.69,105.01);

\path[draw=drawColor,draw opacity=0.50,line width= 0.4pt,line join=round,line cap=round] (110.47,105.53) rectangle (113.32,108.38);

\path[draw=drawColor,draw opacity=0.50,line width= 0.4pt,line join=round,line cap=round] (110.47,105.53) -- (113.32,108.38);

\path[draw=drawColor,draw opacity=0.50,line width= 0.4pt,line join=round,line cap=round] (110.47,108.38) -- (113.32,105.53);

\path[draw=drawColor,draw opacity=0.50,line width= 0.4pt,line join=round,line cap=round] (110.72,106.03) rectangle (113.57,108.88);

\path[draw=drawColor,draw opacity=0.50,line width= 0.4pt,line join=round,line cap=round] (110.72,106.03) -- (113.57,108.88);

\path[draw=drawColor,draw opacity=0.50,line width= 0.4pt,line join=round,line cap=round] (110.72,108.88) -- (113.57,106.03);

\path[draw=drawColor,draw opacity=0.50,line width= 0.4pt,line join=round,line cap=round] (112.29,106.03) rectangle (115.14,108.88);

\path[draw=drawColor,draw opacity=0.50,line width= 0.4pt,line join=round,line cap=round] (112.29,106.03) -- (115.14,108.88);

\path[draw=drawColor,draw opacity=0.50,line width= 0.4pt,line join=round,line cap=round] (112.29,108.88) -- (115.14,106.03);

\path[draw=drawColor,draw opacity=0.50,line width= 0.4pt,line join=round,line cap=round] (111.66,108.23) rectangle (114.52,111.08);

\path[draw=drawColor,draw opacity=0.50,line width= 0.4pt,line join=round,line cap=round] (111.66,108.23) -- (114.52,111.08);

\path[draw=drawColor,draw opacity=0.50,line width= 0.4pt,line join=round,line cap=round] (111.66,111.08) -- (114.52,108.23);

\path[draw=drawColor,draw opacity=0.50,line width= 0.4pt,line join=round,line cap=round] (112.14,106.96) rectangle (115.00,109.81);

\path[draw=drawColor,draw opacity=0.50,line width= 0.4pt,line join=round,line cap=round] (112.14,106.96) -- (115.00,109.81);

\path[draw=drawColor,draw opacity=0.50,line width= 0.4pt,line join=round,line cap=round] (112.14,109.81) -- (115.00,106.96);

\path[draw=drawColor,draw opacity=0.50,line width= 0.4pt,line join=round,line cap=round] (112.03,105.53) rectangle (114.89,108.38);

\path[draw=drawColor,draw opacity=0.50,line width= 0.4pt,line join=round,line cap=round] (112.03,105.53) -- (114.89,108.38);

\path[draw=drawColor,draw opacity=0.50,line width= 0.4pt,line join=round,line cap=round] (112.03,108.38) -- (114.89,105.53);

\path[draw=drawColor,draw opacity=0.50,line width= 0.4pt,line join=round,line cap=round] (110.88,106.96) rectangle (113.73,109.81);

\path[draw=drawColor,draw opacity=0.50,line width= 0.4pt,line join=round,line cap=round] (110.88,106.96) -- (113.73,109.81);

\path[draw=drawColor,draw opacity=0.50,line width= 0.4pt,line join=round,line cap=round] (110.88,109.81) -- (113.73,106.96);

\path[draw=drawColor,draw opacity=0.50,line width= 0.4pt,line join=round,line cap=round] (110.88,110.08) rectangle (113.73,112.93);

\path[draw=drawColor,draw opacity=0.50,line width= 0.4pt,line join=round,line cap=round] (110.88,110.08) -- (113.73,112.93);

\path[draw=drawColor,draw opacity=0.50,line width= 0.4pt,line join=round,line cap=round] (110.88,112.93) -- (113.73,110.08);

\path[draw=drawColor,draw opacity=0.50,line width= 0.4pt,line join=round,line cap=round] (108.06,110.08) rectangle (110.92,112.93);

\path[draw=drawColor,draw opacity=0.50,line width= 0.4pt,line join=round,line cap=round] (108.06,110.08) -- (110.92,112.93);

\path[draw=drawColor,draw opacity=0.50,line width= 0.4pt,line join=round,line cap=round] (108.06,112.93) -- (110.92,110.08);

\path[draw=drawColor,draw opacity=0.50,line width= 0.4pt,line join=round,line cap=round] (107.64,110.74) rectangle (110.50,113.59);

\path[draw=drawColor,draw opacity=0.50,line width= 0.4pt,line join=round,line cap=round] (107.64,110.74) -- (110.50,113.59);

\path[draw=drawColor,draw opacity=0.50,line width= 0.4pt,line join=round,line cap=round] (107.64,113.59) -- (110.50,110.74);

\path[draw=drawColor,draw opacity=0.50,line width= 0.4pt,line join=round,line cap=round] (109.10,110.08) rectangle (111.95,112.93);

\path[draw=drawColor,draw opacity=0.50,line width= 0.4pt,line join=round,line cap=round] (109.10,110.08) -- (111.95,112.93);

\path[draw=drawColor,draw opacity=0.50,line width= 0.4pt,line join=round,line cap=round] (109.10,112.93) -- (111.95,110.08);

\path[draw=drawColor,draw opacity=0.50,line width= 0.4pt,line join=round,line cap=round] (107.59,109.73) rectangle (110.44,112.58);

\path[draw=drawColor,draw opacity=0.50,line width= 0.4pt,line join=round,line cap=round] (107.59,109.73) -- (110.44,112.58);

\path[draw=drawColor,draw opacity=0.50,line width= 0.4pt,line join=round,line cap=round] (107.59,112.58) -- (110.44,109.73);

\path[draw=drawColor,draw opacity=0.50,line width= 0.4pt,line join=round,line cap=round] (107.04,109.73) rectangle (109.89,112.58);

\path[draw=drawColor,draw opacity=0.50,line width= 0.4pt,line join=round,line cap=round] (107.04,109.73) -- (109.89,112.58);

\path[draw=drawColor,draw opacity=0.50,line width= 0.4pt,line join=round,line cap=round] (107.04,112.58) -- (109.89,109.73);

\path[draw=drawColor,draw opacity=0.50,line width= 0.4pt,line join=round,line cap=round] (108.32,109.37) rectangle (111.17,112.23);

\path[draw=drawColor,draw opacity=0.50,line width= 0.4pt,line join=round,line cap=round] (108.32,109.37) -- (111.17,112.23);

\path[draw=drawColor,draw opacity=0.50,line width= 0.4pt,line join=round,line cap=round] (108.32,112.23) -- (111.17,109.37);

\path[draw=drawColor,draw opacity=0.50,line width= 0.4pt,line join=round,line cap=round] (109.96,109.37) rectangle (112.81,112.23);

\path[draw=drawColor,draw opacity=0.50,line width= 0.4pt,line join=round,line cap=round] (109.96,109.37) -- (112.81,112.23);

\path[draw=drawColor,draw opacity=0.50,line width= 0.4pt,line join=round,line cap=round] (109.96,112.23) -- (112.81,109.37);

\path[draw=drawColor,draw opacity=0.50,line width= 0.4pt,line join=round,line cap=round] (108.06,108.23) rectangle (110.92,111.08);

\path[draw=drawColor,draw opacity=0.50,line width= 0.4pt,line join=round,line cap=round] (108.06,108.23) -- (110.92,111.08);

\path[draw=drawColor,draw opacity=0.50,line width= 0.4pt,line join=round,line cap=round] (108.06,111.08) -- (110.92,108.23);

\path[draw=drawColor,draw opacity=0.50,line width= 0.4pt,line join=round,line cap=round] (109.60,106.96) rectangle (112.46,109.81);

\path[draw=drawColor,draw opacity=0.50,line width= 0.4pt,line join=round,line cap=round] (109.60,106.96) -- (112.46,109.81);

\path[draw=drawColor,draw opacity=0.50,line width= 0.4pt,line join=round,line cap=round] (109.60,109.81) -- (112.46,106.96);

\path[draw=drawColor,draw opacity=0.50,line width= 0.4pt,line join=round,line cap=round] (109.28,109.37) rectangle (112.14,112.23);

\path[draw=drawColor,draw opacity=0.50,line width= 0.4pt,line join=round,line cap=round] (109.28,109.37) -- (112.14,112.23);

\path[draw=drawColor,draw opacity=0.50,line width= 0.4pt,line join=round,line cap=round] (109.28,112.23) -- (112.14,109.37);

\path[draw=drawColor,draw opacity=0.50,line width= 0.4pt,line join=round,line cap=round] (110.55,110.08) rectangle (113.41,112.93);

\path[draw=drawColor,draw opacity=0.50,line width= 0.4pt,line join=round,line cap=round] (110.55,110.08) -- (113.41,112.93);

\path[draw=drawColor,draw opacity=0.50,line width= 0.4pt,line join=round,line cap=round] (110.55,112.93) -- (113.41,110.08);

\path[draw=drawColor,draw opacity=0.50,line width= 0.4pt,line join=round,line cap=round] (108.86,109.37) rectangle (111.71,112.23);

\path[draw=drawColor,draw opacity=0.50,line width= 0.4pt,line join=round,line cap=round] (108.86,109.37) -- (111.71,112.23);

\path[draw=drawColor,draw opacity=0.50,line width= 0.4pt,line join=round,line cap=round] (108.86,112.23) -- (111.71,109.37);

\path[draw=drawColor,draw opacity=0.50,line width= 0.4pt,line join=round,line cap=round] (110.80,117.97) rectangle (113.65,120.82);

\path[draw=drawColor,draw opacity=0.50,line width= 0.4pt,line join=round,line cap=round] (110.80,117.97) -- (113.65,120.82);

\path[draw=drawColor,draw opacity=0.50,line width= 0.4pt,line join=round,line cap=round] (110.80,120.82) -- (113.65,117.97);

\path[draw=drawColor,draw opacity=0.50,line width= 0.4pt,line join=round,line cap=round] (109.51,108.62) rectangle (112.37,111.47);

\path[draw=drawColor,draw opacity=0.50,line width= 0.4pt,line join=round,line cap=round] (109.51,108.62) -- (112.37,111.47);

\path[draw=drawColor,draw opacity=0.50,line width= 0.4pt,line join=round,line cap=round] (109.51,111.47) -- (112.37,108.62);

\path[draw=drawColor,draw opacity=0.50,line width= 0.4pt,line join=round,line cap=round] (110.43,109.73) rectangle (113.28,112.58);

\path[draw=drawColor,draw opacity=0.50,line width= 0.4pt,line join=round,line cap=round] (110.43,109.73) -- (113.28,112.58);

\path[draw=drawColor,draw opacity=0.50,line width= 0.4pt,line join=round,line cap=round] (110.43,112.58) -- (113.28,109.73);
\definecolor{drawColor}{RGB}{231,41,138}

\path[draw=drawColor,draw opacity=0.50,line width= 0.4pt,line join=round,line cap=round] (111.91,125.47) -- (115.94,125.47);

\path[draw=drawColor,draw opacity=0.50,line width= 0.4pt,line join=round,line cap=round] (113.93,123.45) -- (113.93,127.49);

\path[draw=drawColor,draw opacity=0.50,line width= 0.4pt,line join=round,line cap=round] (112.98,126.27) -- (117.02,126.27);

\path[draw=drawColor,draw opacity=0.50,line width= 0.4pt,line join=round,line cap=round] (115.00,124.25) -- (115.00,128.28);

\path[draw=drawColor,draw opacity=0.50,line width= 0.4pt,line join=round,line cap=round] (112.69,127.88) -- (116.73,127.88);

\path[draw=drawColor,draw opacity=0.50,line width= 0.4pt,line join=round,line cap=round] (114.71,125.86) -- (114.71,129.90);

\path[draw=drawColor,draw opacity=0.50,line width= 0.4pt,line join=round,line cap=round] (112.56,126.36) -- (116.59,126.36);

\path[draw=drawColor,draw opacity=0.50,line width= 0.4pt,line join=round,line cap=round] (114.58,124.34) -- (114.58,128.38);

\path[draw=drawColor,draw opacity=0.50,line width= 0.4pt,line join=round,line cap=round] (112.79,127.28) -- (116.82,127.28);

\path[draw=drawColor,draw opacity=0.50,line width= 0.4pt,line join=round,line cap=round] (114.81,125.26) -- (114.81,129.30);

\path[draw=drawColor,draw opacity=0.50,line width= 0.4pt,line join=round,line cap=round] (112.59,125.78) -- (116.63,125.78);

\path[draw=drawColor,draw opacity=0.50,line width= 0.4pt,line join=round,line cap=round] (114.61,123.76) -- (114.61,127.79);

\path[draw=drawColor,draw opacity=0.50,line width= 0.4pt,line join=round,line cap=round] (106.62,124.84) -- (110.65,124.84);

\path[draw=drawColor,draw opacity=0.50,line width= 0.4pt,line join=round,line cap=round] (108.64,122.82) -- (108.64,126.85);

\path[draw=drawColor,draw opacity=0.50,line width= 0.4pt,line join=round,line cap=round] (110.09,124.62) -- (114.12,124.62);

\path[draw=drawColor,draw opacity=0.50,line width= 0.4pt,line join=round,line cap=round] (112.10,122.60) -- (112.10,126.63);

\path[draw=drawColor,draw opacity=0.50,line width= 0.4pt,line join=round,line cap=round] (110.49,126.46) -- (114.53,126.46);

\path[draw=drawColor,draw opacity=0.50,line width= 0.4pt,line join=round,line cap=round] (112.51,124.44) -- (112.51,128.47);

\path[draw=drawColor,draw opacity=0.50,line width= 0.4pt,line join=round,line cap=round] (109.01,124.51) -- (113.05,124.51);

\path[draw=drawColor,draw opacity=0.50,line width= 0.4pt,line join=round,line cap=round] (111.03,122.49) -- (111.03,126.52);

\path[draw=drawColor,draw opacity=0.50,line width= 0.4pt,line join=round,line cap=round] (108.27,125.26) -- (112.30,125.26);

\path[draw=drawColor,draw opacity=0.50,line width= 0.4pt,line join=round,line cap=round] (110.28,123.25) -- (110.28,127.28);

\path[draw=drawColor,draw opacity=0.50,line width= 0.4pt,line join=round,line cap=round] (111.84,128.45) -- (115.87,128.45);

\path[draw=drawColor,draw opacity=0.50,line width= 0.4pt,line join=round,line cap=round] (113.86,126.43) -- (113.86,130.47);

\path[draw=drawColor,draw opacity=0.50,line width= 0.4pt,line join=round,line cap=round] (110.88,127.01) -- (114.92,127.01);

\path[draw=drawColor,draw opacity=0.50,line width= 0.4pt,line join=round,line cap=round] (112.90,124.99) -- (112.90,129.03);

\path[draw=drawColor,draw opacity=0.50,line width= 0.4pt,line join=round,line cap=round] (112.42,131.87) -- (116.46,131.87);

\path[draw=drawColor,draw opacity=0.50,line width= 0.4pt,line join=round,line cap=round] (114.44,129.86) -- (114.44,133.89);

\path[draw=drawColor,draw opacity=0.50,line width= 0.4pt,line join=round,line cap=round] (111.41,128.92) -- (115.44,128.92);

\path[draw=drawColor,draw opacity=0.50,line width= 0.4pt,line join=round,line cap=round] (113.42,126.90) -- (113.42,130.94);

\path[draw=drawColor,draw opacity=0.50,line width= 0.4pt,line join=round,line cap=round] (112.52,126.46) -- (116.56,126.46);

\path[draw=drawColor,draw opacity=0.50,line width= 0.4pt,line join=round,line cap=round] (114.54,124.44) -- (114.54,128.47);
\definecolor{drawColor}{RGB}{102,166,30}

\path[draw=drawColor,draw opacity=0.50,line width= 0.4pt,line join=round,line cap=round] (103.30, 95.85) rectangle (106.15, 98.70);

\path[draw=drawColor,draw opacity=0.50,line width= 0.4pt,line join=round,line cap=round] (103.30, 95.85) -- (106.15, 98.70);

\path[draw=drawColor,draw opacity=0.50,line width= 0.4pt,line join=round,line cap=round] (103.30, 98.70) -- (106.15, 95.85);

\path[draw=drawColor,draw opacity=0.50,line width= 0.4pt,line join=round,line cap=round] (111.92,100.56) rectangle (114.78,103.41);

\path[draw=drawColor,draw opacity=0.50,line width= 0.4pt,line join=round,line cap=round] (111.92,100.56) -- (114.78,103.41);

\path[draw=drawColor,draw opacity=0.50,line width= 0.4pt,line join=round,line cap=round] (111.92,103.41) -- (114.78,100.56);

\path[draw=drawColor,draw opacity=0.50,line width= 0.4pt,line join=round,line cap=round] (119.51,116.09) rectangle (122.37,118.94);

\path[draw=drawColor,draw opacity=0.50,line width= 0.4pt,line join=round,line cap=round] (119.51,116.09) -- (122.37,118.94);

\path[draw=drawColor,draw opacity=0.50,line width= 0.4pt,line join=round,line cap=round] (119.51,118.94) -- (122.37,116.09);

\path[draw=drawColor,draw opacity=0.50,line width= 0.4pt,line join=round,line cap=round] (125.81,135.12) rectangle (128.67,137.98);

\path[draw=drawColor,draw opacity=0.50,line width= 0.4pt,line join=round,line cap=round] (125.81,135.12) -- (128.67,137.98);

\path[draw=drawColor,draw opacity=0.50,line width= 0.4pt,line join=round,line cap=round] (125.81,137.98) -- (128.67,135.12);

\path[draw=drawColor,draw opacity=0.50,line width= 0.4pt,line join=round,line cap=round] (130.80,148.33) rectangle (133.65,151.18);

\path[draw=drawColor,draw opacity=0.50,line width= 0.4pt,line join=round,line cap=round] (130.80,148.33) -- (133.65,151.18);

\path[draw=drawColor,draw opacity=0.50,line width= 0.4pt,line join=round,line cap=round] (130.80,151.18) -- (133.65,148.33);

\path[draw=drawColor,draw opacity=0.50,line width= 0.4pt,line join=round,line cap=round] (137.57,162.40) rectangle (140.43,165.25);

\path[draw=drawColor,draw opacity=0.50,line width= 0.4pt,line join=round,line cap=round] (137.57,162.40) -- (140.43,165.25);

\path[draw=drawColor,draw opacity=0.50,line width= 0.4pt,line join=round,line cap=round] (137.57,165.25) -- (140.43,162.40);

\path[draw=drawColor,draw opacity=0.50,line width= 0.4pt,line join=round,line cap=round] (138.15,172.10) rectangle (141.00,174.95);

\path[draw=drawColor,draw opacity=0.50,line width= 0.4pt,line join=round,line cap=round] (138.15,172.10) -- (141.00,174.95);

\path[draw=drawColor,draw opacity=0.50,line width= 0.4pt,line join=round,line cap=round] (138.15,174.95) -- (141.00,172.10);

\path[draw=drawColor,draw opacity=0.50,line width= 0.4pt,line join=round,line cap=round] (146.01,184.58) rectangle (148.87,187.43);

\path[draw=drawColor,draw opacity=0.50,line width= 0.4pt,line join=round,line cap=round] (146.01,184.58) -- (148.87,187.43);

\path[draw=drawColor,draw opacity=0.50,line width= 0.4pt,line join=round,line cap=round] (146.01,187.43) -- (148.87,184.58);

\path[draw=drawColor,draw opacity=0.50,line width= 0.4pt,line join=round,line cap=round] (150.79,194.70) rectangle (153.64,197.55);

\path[draw=drawColor,draw opacity=0.50,line width= 0.4pt,line join=round,line cap=round] (150.79,194.70) -- (153.64,197.55);

\path[draw=drawColor,draw opacity=0.50,line width= 0.4pt,line join=round,line cap=round] (150.79,197.55) -- (153.64,194.70);

\path[draw=drawColor,draw opacity=0.50,line width= 0.4pt,line join=round,line cap=round] (154.37,202.26) rectangle (157.23,205.11);

\path[draw=drawColor,draw opacity=0.50,line width= 0.4pt,line join=round,line cap=round] (154.37,202.26) -- (157.23,205.11);

\path[draw=drawColor,draw opacity=0.50,line width= 0.4pt,line join=round,line cap=round] (154.37,205.11) -- (157.23,202.26);

\path[draw=drawColor,draw opacity=0.50,line width= 0.4pt,line join=round,line cap=round] (158.47,203.31) rectangle (161.33,206.16);

\path[draw=drawColor,draw opacity=0.50,line width= 0.4pt,line join=round,line cap=round] (158.47,203.31) -- (161.33,206.16);

\path[draw=drawColor,draw opacity=0.50,line width= 0.4pt,line join=round,line cap=round] (158.47,206.16) -- (161.33,203.31);
\definecolor{fillColor}{RGB}{27,158,119}

\path[fill=fillColor,fill opacity=0.50] (188.40,170.98) circle (  1.43);

\path[fill=fillColor,fill opacity=0.50] (188.40,152.93) circle (  1.43);

\path[fill=fillColor,fill opacity=0.50] (188.40,155.45) circle (  1.43);

\path[fill=fillColor,fill opacity=0.50] (188.40,175.70) circle (  1.43);

\path[fill=fillColor,fill opacity=0.50] (188.40,178.18) circle (  1.43);

\path[fill=fillColor,fill opacity=0.50] (188.40,170.52) circle (  1.43);

\path[fill=fillColor,fill opacity=0.50] (188.40,174.79) circle (  1.43);

\path[fill=fillColor,fill opacity=0.50] (188.40,157.29) circle (  1.43);

\path[fill=fillColor,fill opacity=0.50] (188.40,156.58) circle (  1.43);

\path[fill=fillColor,fill opacity=0.50] (188.40,165.25) circle (  1.43);

\path[fill=fillColor,fill opacity=0.50] (188.40,171.09) circle (  1.43);

\path[fill=fillColor,fill opacity=0.50] (188.40,188.10) circle (  1.43);

\path[fill=fillColor,fill opacity=0.50] (188.40,172.46) circle (  1.43);

\path[fill=fillColor,fill opacity=0.50] (188.40,177.05) circle (  1.43);

\path[fill=fillColor,fill opacity=0.50] (188.40,172.08) circle (  1.43);

\path[fill=fillColor,fill opacity=0.50] (188.40,179.03) circle (  1.43);

\path[fill=fillColor,fill opacity=0.50] (188.40,212.44) circle (  1.43);

\path[fill=fillColor,fill opacity=0.50] (188.40,212.44) circle (  1.43);

\path[fill=fillColor,fill opacity=0.50] (188.40,212.44) circle (  1.43);

\path[fill=fillColor,fill opacity=0.50] (188.40,196.61) circle (  1.43);

\path[fill=fillColor,fill opacity=0.50] (188.40,167.05) circle (  1.43);

\path[fill=fillColor,fill opacity=0.50] (188.40,138.07) circle (  1.43);

\path[fill=fillColor,fill opacity=0.50] (188.40,156.84) circle (  1.43);

\path[fill=fillColor,fill opacity=0.50] (188.40,135.64) circle (  1.43);

\path[fill=fillColor,fill opacity=0.50] (188.40,157.83) circle (  1.43);

\path[fill=fillColor,fill opacity=0.50] (188.40,149.56) circle (  1.43);

\path[fill=fillColor,fill opacity=0.50] (188.40,172.08) circle (  1.43);

\path[fill=fillColor,fill opacity=0.50] (188.40,139.82) circle (  1.43);

\path[fill=fillColor,fill opacity=0.50] (188.40,150.41) circle (  1.43);

\path[fill=fillColor,fill opacity=0.50] (188.40,146.22) circle (  1.43);

\path[fill=fillColor,fill opacity=0.50] (116.10,102.74) circle (  1.43);

\path[fill=fillColor,fill opacity=0.50] (106.11, 98.38) circle (  1.43);

\path[fill=fillColor,fill opacity=0.50] (105.25, 98.38) circle (  1.43);

\path[fill=fillColor,fill opacity=0.50] (107.35, 98.38) circle (  1.43);

\path[fill=fillColor,fill opacity=0.50] (107.47, 98.38) circle (  1.43);

\path[fill=fillColor,fill opacity=0.50] (106.31, 99.39) circle (  1.43);

\path[fill=fillColor,fill opacity=0.50] (107.16,102.74) circle (  1.43);

\path[fill=fillColor,fill opacity=0.50] (105.83, 99.39) circle (  1.43);

\path[fill=fillColor,fill opacity=0.50] (107.29, 98.38) circle (  1.43);

\path[fill=fillColor,fill opacity=0.50] (106.91, 98.38) circle (  1.43);

\path[fill=fillColor,fill opacity=0.50] (108.64,100.32) circle (  1.43);

\path[fill=fillColor,fill opacity=0.50] (108.64,100.32) circle (  1.43);

\path[fill=fillColor,fill opacity=0.50] (111.34,101.19) circle (  1.43);

\path[fill=fillColor,fill opacity=0.50] (109.99, 98.38) circle (  1.43);

\path[fill=fillColor,fill opacity=0.50] (106.97, 99.39) circle (  1.43);

\path[fill=fillColor,fill opacity=0.50] (106.71, 98.38) circle (  1.43);

\path[fill=fillColor,fill opacity=0.50] (111.94,101.19) circle (  1.43);

\path[fill=fillColor,fill opacity=0.50] (107.89, 98.38) circle (  1.43);

\path[fill=fillColor,fill opacity=0.50] (106.65, 98.38) circle (  1.43);

\path[fill=fillColor,fill opacity=0.50] (110.52,101.19) circle (  1.43);

\path[fill=fillColor,fill opacity=0.50] (110.94, 98.38) circle (  1.43);

\path[fill=fillColor,fill opacity=0.50] (109.54,100.32) circle (  1.43);

\path[fill=fillColor,fill opacity=0.50] (107.77, 98.38) circle (  1.43);

\path[fill=fillColor,fill opacity=0.50] (106.84,101.19) circle (  1.43);

\path[fill=fillColor,fill opacity=0.50] (113.64,100.32) circle (  1.43);

\path[fill=fillColor,fill opacity=0.50] (111.90, 98.38) circle (  1.43);

\path[fill=fillColor,fill opacity=0.50] (110.98,101.99) circle (  1.43);

\path[fill=fillColor,fill opacity=0.50] (108.64, 99.39) circle (  1.43);

\path[fill=fillColor,fill opacity=0.50] (132.54,101.19) circle (  1.43);

\path[fill=fillColor,fill opacity=0.50] (107.89, 98.38) circle (  1.43);

\path[fill=fillColor,fill opacity=0.50] (107.41, 99.39) circle (  1.43);

\path[fill=fillColor,fill opacity=0.50] (107.47, 99.39) circle (  1.43);

\path[fill=fillColor,fill opacity=0.50] (107.47, 98.38) circle (  1.43);

\path[fill=fillColor,fill opacity=0.50] (122.16, 99.39) circle (  1.43);

\path[fill=fillColor,fill opacity=0.50] (108.47,101.19) circle (  1.43);

\path[fill=fillColor,fill opacity=0.50] (108.30,101.99) circle (  1.43);

\path[fill=fillColor,fill opacity=0.50] (113.89,103.44) circle (  1.43);

\path[fill=fillColor,fill opacity=0.50] (116.19,101.99) circle (  1.43);

\path[fill=fillColor,fill opacity=0.50] (105.97, 98.38) circle (  1.43);

\path[fill=fillColor,fill opacity=0.50] (105.90, 98.38) circle (  1.43);

\path[fill=fillColor,fill opacity=0.50] (132.70,101.19) circle (  1.43);

\path[fill=fillColor,fill opacity=0.50] (112.86, 98.38) circle (  1.43);

\path[fill=fillColor,fill opacity=0.50] (109.99,101.19) circle (  1.43);

\path[fill=fillColor,fill opacity=0.50] (106.11, 98.38) circle (  1.43);

\path[fill=fillColor,fill opacity=0.50] (115.19,102.74) circle (  1.43);

\path[fill=fillColor,fill opacity=0.50] (111.16,112.17) circle (  1.43);

\path[fill=fillColor,fill opacity=0.50] (108.75, 98.38) circle (  1.43);

\path[fill=fillColor,fill opacity=0.50] (108.85,101.19) circle (  1.43);

\path[fill=fillColor,fill opacity=0.50] (135.46,104.11) circle (  1.43);

\path[fill=fillColor,fill opacity=0.50] (114.51,100.32) circle (  1.43);

\path[fill=fillColor,fill opacity=0.50] (107.77, 97.27) circle (  1.43);

\path[fill=fillColor,fill opacity=0.50] (106.78, 98.38) circle (  1.43);

\path[fill=fillColor,fill opacity=0.50] (124.81,107.45) circle (  1.43);

\path[fill=fillColor,fill opacity=0.50] (108.36, 98.38) circle (  1.43);

\path[fill=fillColor,fill opacity=0.50] (120.51, 98.38) circle (  1.43);

\path[fill=fillColor,fill opacity=0.50] (110.76,101.19) circle (  1.43);

\path[fill=fillColor,fill opacity=0.50] (110.19,100.32) circle (  1.43);

\path[fill=fillColor,fill opacity=0.50] (108.41, 99.39) circle (  1.43);

\path[fill=fillColor,fill opacity=0.50] (110.48, 98.38) circle (  1.43);

\path[fill=fillColor,fill opacity=0.50] (107.71, 98.38) circle (  1.43);

\path[fill=fillColor,fill opacity=0.50] (114.74, 99.39) circle (  1.43);

\path[fill=fillColor,fill opacity=0.50] (111.86,108.38) circle (  1.43);

\path[fill=fillColor,fill opacity=0.50] (107.41, 99.39) circle (  1.43);

\path[fill=fillColor,fill opacity=0.50] (106.91, 98.38) circle (  1.43);

\path[fill=fillColor,fill opacity=0.50] (114.44,101.99) circle (  1.43);

\path[fill=fillColor,fill opacity=0.50] (111.56,108.82) circle (  1.43);

\path[fill=fillColor,fill opacity=0.50] (106.78, 97.27) circle (  1.43);

\path[fill=fillColor,fill opacity=0.50] (105.97, 98.38) circle (  1.43);

\path[fill=fillColor,fill opacity=0.50] (113.09, 99.39) circle (  1.43);

\path[fill=fillColor,fill opacity=0.50] (110.66,101.99) circle (  1.43);

\path[fill=fillColor,fill opacity=0.50] (106.52,111.50) circle (  1.43);

\path[fill=fillColor,fill opacity=0.50] (107.53, 99.39) circle (  1.43);

\path[fill=fillColor,fill opacity=0.50] (109.49, 98.38) circle (  1.43);

\path[fill=fillColor,fill opacity=0.50] (108.85,101.19) circle (  1.43);

\path[fill=fillColor,fill opacity=0.50] (106.58, 98.38) circle (  1.43);

\path[fill=fillColor,fill opacity=0.50] (106.97, 99.39) circle (  1.43);

\path[fill=fillColor,fill opacity=0.50] (132.45,111.16) circle (  1.43);

\path[fill=fillColor,fill opacity=0.50] (110.57,101.99) circle (  1.43);

\path[fill=fillColor,fill opacity=0.50] (109.39,101.19) circle (  1.43);

\path[fill=fillColor,fill opacity=0.50] (108.96,101.19) circle (  1.43);

\path[fill=fillColor,fill opacity=0.50] (111.12,101.99) circle (  1.43);

\path[fill=fillColor,fill opacity=0.50] (106.25, 98.38) circle (  1.43);

\path[fill=fillColor,fill opacity=0.50] (105.83, 99.39) circle (  1.43);

\path[fill=fillColor,fill opacity=0.50] (107.10, 99.39) circle (  1.43);

\path[fill=fillColor,fill opacity=0.50] (108.64, 98.38) circle (  1.43);

\path[fill=fillColor,fill opacity=0.50] (108.52,101.19) circle (  1.43);

\path[fill=fillColor,fill opacity=0.50] (105.69, 98.38) circle (  1.43);

\path[fill=fillColor,fill opacity=0.50] (106.25, 99.39) circle (  1.43);

\path[fill=fillColor,fill opacity=0.50] (126.44,104.74) circle (  1.43);

\path[fill=fillColor,fill opacity=0.50] (109.39, 98.38) circle (  1.43);

\path[fill=fillColor,fill opacity=0.50] (106.58, 98.38) circle (  1.43);

\path[fill=fillColor,fill opacity=0.50] (107.59, 99.39) circle (  1.43);

\path[fill=fillColor,fill opacity=0.50] (106.58, 98.38) circle (  1.43);

\path[fill=fillColor,fill opacity=0.50] (105.55, 98.38) circle (  1.43);

\path[fill=fillColor,fill opacity=0.50] (106.58, 98.38) circle (  1.43);

\path[fill=fillColor,fill opacity=0.50] (119.18, 98.38) circle (  1.43);

\path[fill=fillColor,fill opacity=0.50] (120.35,100.32) circle (  1.43);

\path[fill=fillColor,fill opacity=0.50] (114.41,105.33) circle (  1.43);

\path[fill=fillColor,fill opacity=0.50] (108.41, 98.38) circle (  1.43);

\path[fill=fillColor,fill opacity=0.50] (108.18,101.99) circle (  1.43);

\path[fill=fillColor,fill opacity=0.50] (126.18,101.19) circle (  1.43);

\path[fill=fillColor,fill opacity=0.50] (107.59,111.84) circle (  1.43);

\path[fill=fillColor,fill opacity=0.50] (108.85,101.19) circle (  1.43);

\path[fill=fillColor,fill opacity=0.50] (108.18,100.32) circle (  1.43);

\path[fill=fillColor,fill opacity=0.50] (107.71,100.32) circle (  1.43);

\path[fill=fillColor,fill opacity=0.50] (107.41, 98.38) circle (  1.43);

\path[fill=fillColor,fill opacity=0.50] (108.41,101.19) circle (  1.43);

\path[fill=fillColor,fill opacity=0.50] (105.25, 98.38) circle (  1.43);

\path[fill=fillColor,fill opacity=0.50] (120.51,101.99) circle (  1.43);

\path[fill=fillColor,fill opacity=0.50] (106.71,112.49) circle (  1.43);

\path[fill=fillColor,fill opacity=0.50] (105.62, 98.38) circle (  1.43);

\path[fill=fillColor,fill opacity=0.50] (106.65,100.32) circle (  1.43);

\path[fill=fillColor,fill opacity=0.50] (113.24,100.32) circle (  1.43);

\path[fill=fillColor,fill opacity=0.50] (110.89,101.19) circle (  1.43);

\path[fill=fillColor,fill opacity=0.50] (106.65, 98.38) circle (  1.43);

\path[fill=fillColor,fill opacity=0.50] (106.18, 97.27) circle (  1.43);

\path[fill=fillColor,fill opacity=0.50] (112.63,117.31) circle (  1.43);

\path[fill=fillColor,fill opacity=0.50] (107.83,101.19) circle (  1.43);

\path[fill=fillColor,fill opacity=0.50] (106.25, 99.39) circle (  1.43);

\path[fill=fillColor,fill opacity=0.50] (108.52, 99.39) circle (  1.43);

\path[fill=fillColor,fill opacity=0.50] (135.95,111.16) circle (  1.43);

\path[fill=fillColor,fill opacity=0.50] (117.73,107.45) circle (  1.43);

\path[fill=fillColor,fill opacity=0.50] (111.43,101.19) circle (  1.43);

\path[fill=fillColor,fill opacity=0.50] (110.14,101.99) circle (  1.43);

\path[fill=fillColor,fill opacity=0.50] (132.68,108.38) circle (  1.43);

\path[fill=fillColor,fill opacity=0.50] (109.59,101.19) circle (  1.43);

\path[fill=fillColor,fill opacity=0.50] (114.90,100.32) circle (  1.43);

\path[fill=fillColor,fill opacity=0.50] (108.85,101.99) circle (  1.43);

\path[fill=fillColor,fill opacity=0.50] (121.61,101.99) circle (  1.43);

\path[fill=fillColor,fill opacity=0.50] (112.15,100.32) circle (  1.43);

\path[fill=fillColor,fill opacity=0.50] (107.10,101.19) circle (  1.43);

\path[fill=fillColor,fill opacity=0.50] (109.59,101.99) circle (  1.43);

\path[fill=fillColor,fill opacity=0.50] (152.06,111.84) circle (  1.43);

\path[fill=fillColor,fill opacity=0.50] (115.06,103.44) circle (  1.43);

\path[fill=fillColor,fill opacity=0.50] (108.58, 99.39) circle (  1.43);

\path[fill=fillColor,fill opacity=0.50] (107.22, 99.39) circle (  1.43);

\path[fill=fillColor,fill opacity=0.50] (113.82,106.96) circle (  1.43);

\path[fill=fillColor,fill opacity=0.50] (106.25,101.99) circle (  1.43);

\path[fill=fillColor,fill opacity=0.50] (112.71,101.19) circle (  1.43);

\path[fill=fillColor,fill opacity=0.50] (105.76, 98.38) circle (  1.43);

\path[fill=fillColor,fill opacity=0.50] (111.60,100.32) circle (  1.43);

\path[fill=fillColor,fill opacity=0.50] (109.74,101.99) circle (  1.43);

\path[fill=fillColor,fill opacity=0.50] (108.30,101.19) circle (  1.43);

\path[fill=fillColor,fill opacity=0.50] (106.78, 99.39) circle (  1.43);

\path[fill=fillColor,fill opacity=0.50] (146.49,109.25) circle (  1.43);

\path[fill=fillColor,fill opacity=0.50] (106.84, 98.38) circle (  1.43);

\path[fill=fillColor,fill opacity=0.50] (105.62, 98.38) circle (  1.43);

\path[fill=fillColor,fill opacity=0.50] (108.91,101.99) circle (  1.43);

\path[fill=fillColor,fill opacity=0.50] (112.27,100.32) circle (  1.43);

\path[fill=fillColor,fill opacity=0.50] (111.34, 99.39) circle (  1.43);

\path[fill=fillColor,fill opacity=0.50] (107.95, 98.38) circle (  1.43);

\path[fill=fillColor,fill opacity=0.50] (109.64,101.99) circle (  1.43);

\path[fill=fillColor,fill opacity=0.50] (110.28,101.99) circle (  1.43);

\path[fill=fillColor,fill opacity=0.50] (108.13, 99.39) circle (  1.43);

\path[fill=fillColor,fill opacity=0.50] (106.91, 98.38) circle (  1.43);

\path[fill=fillColor,fill opacity=0.50] (106.45, 99.39) circle (  1.43);

\path[fill=fillColor,fill opacity=0.50] (111.21,101.19) circle (  1.43);

\path[fill=fillColor,fill opacity=0.50] (107.22,101.19) circle (  1.43);

\path[fill=fillColor,fill opacity=0.50] (106.25, 99.39) circle (  1.43);

\path[fill=fillColor,fill opacity=0.50] (106.25, 98.38) circle (  1.43);

\path[fill=fillColor,fill opacity=0.50] (152.67,115.75) circle (  1.43);

\path[fill=fillColor,fill opacity=0.50] (118.35, 98.38) circle (  1.43);

\path[fill=fillColor,fill opacity=0.50] (109.44,101.19) circle (  1.43);

\path[fill=fillColor,fill opacity=0.50] (105.83, 98.38) circle (  1.43);

\path[fill=fillColor,fill opacity=0.50] (130.99,102.74) circle (  1.43);

\path[fill=fillColor,fill opacity=0.50] (112.23, 99.39) circle (  1.43);

\path[fill=fillColor,fill opacity=0.50] (110.52,101.19) circle (  1.43);

\path[fill=fillColor,fill opacity=0.50] (108.58,100.32) circle (  1.43);

\path[fill=fillColor,fill opacity=0.50] (141.23,114.23) circle (  1.43);

\path[fill=fillColor,fill opacity=0.50] (109.64,101.99) circle (  1.43);

\path[fill=fillColor,fill opacity=0.50] (106.38, 99.39) circle (  1.43);

\path[fill=fillColor,fill opacity=0.50] (106.91, 99.39) circle (  1.43);

\path[fill=fillColor,fill opacity=0.50] (110.38,101.99) circle (  1.43);

\path[fill=fillColor,fill opacity=0.50] (107.53, 99.39) circle (  1.43);

\path[fill=fillColor,fill opacity=0.50] (107.95, 99.39) circle (  1.43);

\path[fill=fillColor,fill opacity=0.50] (111.25,101.19) circle (  1.43);

\path[fill=fillColor,fill opacity=0.50] (148.23,116.67) circle (  1.43);

\path[fill=fillColor,fill opacity=0.50] (132.14,101.99) circle (  1.43);

\path[fill=fillColor,fill opacity=0.50] (113.64,102.74) circle (  1.43);

\path[fill=fillColor,fill opacity=0.50] (114.07,101.99) circle (  1.43);

\path[fill=fillColor,fill opacity=0.50] (157.05,120.07) circle (  1.43);

\path[fill=fillColor,fill opacity=0.50] (155.55,131.08) circle (  1.43);

\path[fill=fillColor,fill opacity=0.50] (120.92,107.93) circle (  1.43);

\path[fill=fillColor,fill opacity=0.50] (109.94,101.99) circle (  1.43);

\path[fill=fillColor,fill opacity=0.50] (168.45,135.60) circle (  1.43);

\path[fill=fillColor,fill opacity=0.50] (131.73,115.51) circle (  1.43);

\path[fill=fillColor,fill opacity=0.50] (114.48,105.90) circle (  1.43);

\path[fill=fillColor,fill opacity=0.50] (116.44,102.74) circle (  1.43);

\path[fill=fillColor,fill opacity=0.50] (137.95,114.23) circle (  1.43);

\path[fill=fillColor,fill opacity=0.50] (141.21,118.30) circle (  1.43);

\path[fill=fillColor,fill opacity=0.50] (112.71,101.19) circle (  1.43);

\path[fill=fillColor,fill opacity=0.50] (105.83, 98.38) circle (  1.43);

\path[fill=fillColor,fill opacity=0.50] (126.91,103.44) circle (  1.43);

\path[fill=fillColor,fill opacity=0.50] (106.18, 98.38) circle (  1.43);

\path[fill=fillColor,fill opacity=0.50] (106.31, 99.39) circle (  1.43);

\path[fill=fillColor,fill opacity=0.50] (105.40, 98.38) circle (  1.43);

\path[fill=fillColor,fill opacity=0.50] (118.65,111.16) circle (  1.43);

\path[fill=fillColor,fill opacity=0.50] (108.96,101.19) circle (  1.43);

\path[fill=fillColor,fill opacity=0.50] (123.25,100.32) circle (  1.43);

\path[fill=fillColor,fill opacity=0.50] (106.52, 99.39) circle (  1.43);

\path[fill=fillColor,fill opacity=0.50] (108.47, 97.27) circle (  1.43);

\path[fill=fillColor,fill opacity=0.50] (109.84,101.99) circle (  1.43);

\path[fill=fillColor,fill opacity=0.50] (105.97, 98.38) circle (  1.43);

\path[fill=fillColor,fill opacity=0.50] (108.30,101.99) circle (  1.43);

\path[fill=fillColor,fill opacity=0.50] (112.43,104.74) circle (  1.43);

\path[fill=fillColor,fill opacity=0.50] (110.62,101.19) circle (  1.43);

\path[fill=fillColor,fill opacity=0.50] (109.07,101.19) circle (  1.43);

\path[fill=fillColor,fill opacity=0.50] (109.12,101.99) circle (  1.43);

\path[fill=fillColor,fill opacity=0.50] (113.42,107.45) circle (  1.43);

\path[fill=fillColor,fill opacity=0.50] (110.28, 99.39) circle (  1.43);

\path[fill=fillColor,fill opacity=0.50] (106.25, 98.38) circle (  1.43);

\path[fill=fillColor,fill opacity=0.50] (105.76, 98.38) circle (  1.43);

\path[fill=fillColor,fill opacity=0.50] (133.86,109.65) circle (  1.43);

\path[fill=fillColor,fill opacity=0.50] (110.57,100.32) circle (  1.43);

\path[fill=fillColor,fill opacity=0.50] (106.31, 98.38) circle (  1.43);

\path[fill=fillColor,fill opacity=0.50] (105.55, 98.38) circle (  1.43);

\path[fill=fillColor,fill opacity=0.50] (108.75, 99.39) circle (  1.43);

\path[fill=fillColor,fill opacity=0.50] (108.07,101.19) circle (  1.43);

\path[fill=fillColor,fill opacity=0.50] (108.18,101.99) circle (  1.43);

\path[fill=fillColor,fill opacity=0.50] (107.22,101.99) circle (  1.43);

\path[fill=fillColor,fill opacity=0.50] (113.93,121.74) circle (  1.43);

\path[fill=fillColor,fill opacity=0.50] (106.52,101.19) circle (  1.43);

\path[fill=fillColor,fill opacity=0.50] (110.28,101.19) circle (  1.43);

\path[fill=fillColor,fill opacity=0.50] (109.02,101.99) circle (  1.43);

\path[fill=fillColor,fill opacity=0.50] (114.03, 99.39) circle (  1.43);

\path[fill=fillColor,fill opacity=0.50] (109.18,101.99) circle (  1.43);

\path[fill=fillColor,fill opacity=0.50] (108.47, 99.39) circle (  1.43);

\path[fill=fillColor,fill opacity=0.50] (108.47, 98.38) circle (  1.43);

\path[fill=fillColor,fill opacity=0.50] (108.30, 99.39) circle (  1.43);

\path[fill=fillColor,fill opacity=0.50] (107.29, 98.38) circle (  1.43);

\path[fill=fillColor,fill opacity=0.50] (107.03,103.44) circle (  1.43);

\path[fill=fillColor,fill opacity=0.50] (106.71, 99.39) circle (  1.43);

\path[fill=fillColor,fill opacity=0.50] (128.97,108.82) circle (  1.43);

\path[fill=fillColor,fill opacity=0.50] (109.07, 99.39) circle (  1.43);

\path[fill=fillColor,fill opacity=0.50] (108.47,101.99) circle (  1.43);

\path[fill=fillColor,fill opacity=0.50] (106.31, 98.38) circle (  1.43);

\path[fill=fillColor,fill opacity=0.50] (111.60, 99.39) circle (  1.43);

\path[fill=fillColor,fill opacity=0.50] (110.33,101.99) circle (  1.43);

\path[fill=fillColor,fill opacity=0.50] (106.04, 99.39) circle (  1.43);

\path[fill=fillColor,fill opacity=0.50] (107.71,101.19) circle (  1.43);

\path[fill=fillColor,fill opacity=0.50] (112.43, 98.38) circle (  1.43);

\path[fill=fillColor,fill opacity=0.50] (109.79,101.99) circle (  1.43);

\path[fill=fillColor,fill opacity=0.50] (108.91,101.99) circle (  1.43);

\path[fill=fillColor,fill opacity=0.50] (105.69, 98.38) circle (  1.43);

\path[fill=fillColor,fill opacity=0.50] (146.57,105.33) circle (  1.43);

\path[fill=fillColor,fill opacity=0.50] (111.30,101.19) circle (  1.43);

\path[fill=fillColor,fill opacity=0.50] (111.03,101.19) circle (  1.43);

\path[fill=fillColor,fill opacity=0.50] (108.91,101.99) circle (  1.43);

\path[fill=fillColor,fill opacity=0.50] (157.59,120.70) circle (  1.43);

\path[fill=fillColor,fill opacity=0.50] (111.90,102.74) circle (  1.43);

\path[fill=fillColor,fill opacity=0.50] (106.97,101.19) circle (  1.43);

\path[fill=fillColor,fill opacity=0.50] (107.77,101.19) circle (  1.43);

\path[fill=fillColor,fill opacity=0.50] (112.63,101.19) circle (  1.43);

\path[fill=fillColor,fill opacity=0.50] (106.38, 99.39) circle (  1.43);

\path[fill=fillColor,fill opacity=0.50] (105.69,101.99) circle (  1.43);

\path[fill=fillColor,fill opacity=0.50] (114.41,100.32) circle (  1.43);

\path[fill=fillColor,fill opacity=0.50] (161.50,118.68) circle (  1.43);

\path[fill=fillColor,fill opacity=0.50] (133.23,105.33) circle (  1.43);

\path[fill=fillColor,fill opacity=0.50] (121.50,104.11) circle (  1.43);

\path[fill=fillColor,fill opacity=0.50] (106.58, 99.39) circle (  1.43);

\path[fill=fillColor,fill opacity=0.50] (141.33,101.99) circle (  1.43);

\path[fill=fillColor,fill opacity=0.50] (110.28,101.19) circle (  1.43);

\path[fill=fillColor,fill opacity=0.50] (106.52, 98.38) circle (  1.43);

\path[fill=fillColor,fill opacity=0.50] (106.91, 98.38) circle (  1.43);

\path[fill=fillColor,fill opacity=0.50] (129.66,105.33) circle (  1.43);

\path[fill=fillColor,fill opacity=0.50] (109.74, 99.39) circle (  1.43);

\path[fill=fillColor,fill opacity=0.50] (108.36, 99.39) circle (  1.43);

\path[fill=fillColor,fill opacity=0.50] (106.52, 98.38) circle (  1.43);

\path[fill=fillColor,fill opacity=0.50] (149.68,111.50) circle (  1.43);

\path[fill=fillColor,fill opacity=0.50] (118.13,103.44) circle (  1.43);

\path[fill=fillColor,fill opacity=0.50] (108.01, 99.39) circle (  1.43);

\path[fill=fillColor,fill opacity=0.50] (107.41,101.19) circle (  1.43);

\path[fill=fillColor,fill opacity=0.50] (115.00,101.19) circle (  1.43);

\path[fill=fillColor,fill opacity=0.50] (105.55, 99.39) circle (  1.43);

\path[fill=fillColor,fill opacity=0.50] (106.65, 99.39) circle (  1.43);

\path[fill=fillColor,fill opacity=0.50] (108.47,101.19) circle (  1.43);

\path[fill=fillColor,fill opacity=0.50] (112.02,101.99) circle (  1.43);

\path[fill=fillColor,fill opacity=0.50] (108.64,102.74) circle (  1.43);

\path[fill=fillColor,fill opacity=0.50] (106.38, 98.38) circle (  1.43);

\path[fill=fillColor,fill opacity=0.50] (106.04,101.19) circle (  1.43);

\path[fill=fillColor,fill opacity=0.50] (118.27,102.74) circle (  1.43);

\path[fill=fillColor,fill opacity=0.50] (110.48,102.74) circle (  1.43);

\path[fill=fillColor,fill opacity=0.50] (110.80,101.19) circle (  1.43);

\path[fill=fillColor,fill opacity=0.50] (109.69,102.74) circle (  1.43);

\path[fill=fillColor,fill opacity=0.50] (137.69,114.23) circle (  1.43);

\path[fill=fillColor,fill opacity=0.50] (170.80,133.68) circle (  1.43);

\path[fill=fillColor,fill opacity=0.50] (130.26,113.39) circle (  1.43);

\path[fill=fillColor,fill opacity=0.50] (108.96, 99.39) circle (  1.43);

\path[fill=fillColor,fill opacity=0.50] (138.26,114.76) circle (  1.43);

\path[fill=fillColor,fill opacity=0.50] (119.13,111.16) circle (  1.43);

\path[fill=fillColor,fill opacity=0.50] (113.31,106.96) circle (  1.43);

\path[fill=fillColor,fill opacity=0.50] (108.13,100.32) circle (  1.43);

\path[fill=fillColor,fill opacity=0.50] (188.40,142.31) circle (  1.43);

\path[fill=fillColor,fill opacity=0.50] (128.93,113.10) circle (  1.43);

\path[fill=fillColor,fill opacity=0.50] (119.27,113.10) circle (  1.43);

\path[fill=fillColor,fill opacity=0.50] (108.85,101.19) circle (  1.43);

\path[fill=fillColor,fill opacity=0.50] (170.57,127.19) circle (  1.43);

\path[fill=fillColor,fill opacity=0.50] (140.72,113.68) circle (  1.43);

\path[fill=fillColor,fill opacity=0.50] (105.76,101.99) circle (  1.43);

\path[fill=fillColor,fill opacity=0.50] (107.95,101.99) circle (  1.43);

\path[fill=fillColor,fill opacity=0.50] (170.47,126.83) circle (  1.43);

\path[fill=fillColor,fill opacity=0.50] (133.82,129.80) circle (  1.43);

\path[fill=fillColor,fill opacity=0.50] (113.20,117.31) circle (  1.43);

\path[fill=fillColor,fill opacity=0.50] (109.59,102.74) circle (  1.43);

\path[fill=fillColor,fill opacity=0.50] (147.88, 99.39) circle (  1.43);

\path[fill=fillColor,fill opacity=0.50] (112.43,100.32) circle (  1.43);

\path[fill=fillColor,fill opacity=0.50] (106.38,101.99) circle (  1.43);

\path[fill=fillColor,fill opacity=0.50] (106.18, 99.39) circle (  1.43);

\path[fill=fillColor,fill opacity=0.50] (142.27,132.05) circle (  1.43);

\path[fill=fillColor,fill opacity=0.50] (117.37,116.67) circle (  1.43);

\path[fill=fillColor,fill opacity=0.50] (108.18,103.44) circle (  1.43);

\path[fill=fillColor,fill opacity=0.50] (108.69,101.99) circle (  1.43);

\path[fill=fillColor,fill opacity=0.50] (169.44,128.61) circle (  1.43);

\path[fill=fillColor,fill opacity=0.50] (154.24,125.05) circle (  1.43);

\path[fill=fillColor,fill opacity=0.50] (111.21,103.44) circle (  1.43);

\path[fill=fillColor,fill opacity=0.50] (109.49,101.99) circle (  1.43);

\path[fill=fillColor,fill opacity=0.50] (110.38,101.19) circle (  1.43);

\path[fill=fillColor,fill opacity=0.50] (107.22, 99.39) circle (  1.43);

\path[fill=fillColor,fill opacity=0.50] (108.18,101.99) circle (  1.43);

\path[fill=fillColor,fill opacity=0.50] (105.76, 98.38) circle (  1.43);

\path[fill=fillColor,fill opacity=0.50] (118.75,102.74) circle (  1.43);

\path[fill=fillColor,fill opacity=0.50] (106.65, 98.38) circle (  1.43);

\path[fill=fillColor,fill opacity=0.50] (109.12,101.99) circle (  1.43);

\path[fill=fillColor,fill opacity=0.50] (107.71,101.19) circle (  1.43);

\path[fill=fillColor,fill opacity=0.50] (118.79,100.32) circle (  1.43);

\path[fill=fillColor,fill opacity=0.50] (108.52, 98.38) circle (  1.43);

\path[fill=fillColor,fill opacity=0.50] (109.12, 98.38) circle (  1.43);

\path[fill=fillColor,fill opacity=0.50] (107.47, 99.39) circle (  1.43);

\path[fill=fillColor,fill opacity=0.50] (118.68,101.99) circle (  1.43);

\path[fill=fillColor,fill opacity=0.50] (109.54,100.32) circle (  1.43);

\path[fill=fillColor,fill opacity=0.50] (106.45, 98.38) circle (  1.43);

\path[fill=fillColor,fill opacity=0.50] (107.83,101.19) circle (  1.43);

\path[fill=fillColor,fill opacity=0.50] (106.97, 99.39) circle (  1.43);

\path[fill=fillColor,fill opacity=0.50] (107.71, 99.39) circle (  1.43);

\path[fill=fillColor,fill opacity=0.50] (106.84,100.32) circle (  1.43);

\path[fill=fillColor,fill opacity=0.50] (107.71,101.19) circle (  1.43);

\path[fill=fillColor,fill opacity=0.50] (142.42,105.90) circle (  1.43);

\path[fill=fillColor,fill opacity=0.50] (109.02,117.51) circle (  1.43);

\path[fill=fillColor,fill opacity=0.50] (109.18,101.99) circle (  1.43);

\path[fill=fillColor,fill opacity=0.50] (106.91,107.45) circle (  1.43);

\path[fill=fillColor,fill opacity=0.50] (115.16,100.32) circle (  1.43);

\path[fill=fillColor,fill opacity=0.50] (109.54, 98.38) circle (  1.43);

\path[fill=fillColor,fill opacity=0.50] (105.18, 98.38) circle (  1.43);

\path[fill=fillColor,fill opacity=0.50] (106.71,101.99) circle (  1.43);

\path[fill=fillColor,fill opacity=0.50] (165.45,106.44) circle (  1.43);

\path[fill=fillColor,fill opacity=0.50] (109.12, 99.39) circle (  1.43);

\path[fill=fillColor,fill opacity=0.50] (106.38, 99.39) circle (  1.43);

\path[fill=fillColor,fill opacity=0.50] (106.65, 99.39) circle (  1.43);

\path[fill=fillColor,fill opacity=0.50] (129.55,103.44) circle (  1.43);

\path[fill=fillColor,fill opacity=0.50] (110.80, 99.39) circle (  1.43);

\path[fill=fillColor,fill opacity=0.50] (108.80,101.19) circle (  1.43);

\path[fill=fillColor,fill opacity=0.50] (107.53, 98.38) circle (  1.43);

\path[fill=fillColor,fill opacity=0.50] (119.49,100.32) circle (  1.43);

\path[fill=fillColor,fill opacity=0.50] (110.89,101.99) circle (  1.43);

\path[fill=fillColor,fill opacity=0.50] (106.38, 98.38) circle (  1.43);

\path[fill=fillColor,fill opacity=0.50] (106.84, 98.38) circle (  1.43);

\path[fill=fillColor,fill opacity=0.50] (107.83,102.74) circle (  1.43);

\path[fill=fillColor,fill opacity=0.50] (108.91,101.19) circle (  1.43);

\path[fill=fillColor,fill opacity=0.50] (107.10,101.19) circle (  1.43);

\path[fill=fillColor,fill opacity=0.50] (105.90,100.32) circle (  1.43);
\definecolor{fillColor}{RGB}{217,95,2}

\path[fill=fillColor,fill opacity=0.50] (109.79,100.60) --
	(111.72, 97.27) --
	(107.87, 97.27) --
	cycle;

\path[fill=fillColor,fill opacity=0.50] (113.42,104.21) --
	(115.35,100.88) --
	(111.50,100.88) --
	cycle;

\path[fill=fillColor,fill opacity=0.50] (108.69,100.60) --
	(110.61, 97.27) --
	(106.77, 97.27) --
	cycle;

\path[fill=fillColor,fill opacity=0.50] (113.24,102.54) --
	(115.16, 99.21) --
	(111.32, 99.21) --
	cycle;

\path[fill=fillColor,fill opacity=0.50] (110.48,102.54) --
	(112.40, 99.21) --
	(108.55, 99.21) --
	cycle;

\path[fill=fillColor,fill opacity=0.50] (107.29, 99.49) --
	(109.21, 96.16) --
	(105.36, 96.16) --
	cycle;

\path[fill=fillColor,fill opacity=0.50] (108.13, 99.49) --
	(110.05, 96.16) --
	(106.20, 96.16) --
	cycle;

\path[fill=fillColor,fill opacity=0.50] (111.30,103.40) --
	(113.22,100.08) --
	(109.38,100.08) --
	cycle;

\path[fill=fillColor,fill opacity=0.50] (109.33,101.61) --
	(111.26, 98.28) --
	(107.41, 98.28) --
	cycle;

\path[fill=fillColor,fill opacity=0.50] (111.73,104.21) --
	(113.65,100.88) --
	(109.81,100.88) --
	cycle;

\path[fill=fillColor,fill opacity=0.50] (115.86,105.66) --
	(117.79,102.33) --
	(113.94,102.33) --
	cycle;

\path[fill=fillColor,fill opacity=0.50] (115.13,101.61) --
	(117.05, 98.28) --
	(113.21, 98.28) --
	cycle;

\path[fill=fillColor,fill opacity=0.50] (113.86,104.96) --
	(115.78,101.63) --
	(111.93,101.63) --
	cycle;

\path[fill=fillColor,fill opacity=0.50] (119.29,105.66) --
	(121.21,102.33) --
	(117.37,102.33) --
	cycle;

\path[fill=fillColor,fill opacity=0.50] (116.81,106.96) --
	(118.73,103.63) --
	(114.88,103.63) --
	cycle;

\path[fill=fillColor,fill opacity=0.50] (113.79,102.54) --
	(115.71, 99.21) --
	(111.86, 99.21) --
	cycle;

\path[fill=fillColor,fill opacity=0.50] (113.68,103.40) --
	(115.60,100.08) --
	(111.76,100.08) --
	cycle;

\path[fill=fillColor,fill opacity=0.50] (117.70,106.96) --
	(119.62,103.63) --
	(115.78,103.63) --
	cycle;

\path[fill=fillColor,fill opacity=0.50] (116.13,117.97) --
	(118.05,114.64) --
	(114.21,114.64) --
	cycle;

\path[fill=fillColor,fill opacity=0.50] (115.77,104.96) --
	(117.70,101.63) --
	(113.85,101.63) --
	cycle;

\path[fill=fillColor,fill opacity=0.50] (123.04,107.55) --
	(124.97,104.22) --
	(121.12,104.22) --
	cycle;

\path[fill=fillColor,fill opacity=0.50] (125.49,113.72) --
	(127.42,110.39) --
	(123.57,110.39) --
	cycle;

\path[fill=fillColor,fill opacity=0.50] (123.77,119.93) --
	(125.70,116.61) --
	(121.85,116.61) --
	cycle;

\path[fill=fillColor,fill opacity=0.50] (122.59,127.69) --
	(124.51,124.36) --
	(120.66,124.36) --
	cycle;

\path[fill=fillColor,fill opacity=0.50] (116.24,104.21) --
	(118.17,100.88) --
	(114.32,100.88) --
	cycle;

\path[fill=fillColor,fill opacity=0.50] (126.99,112.27) --
	(128.91,108.94) --
	(125.07,108.94) --
	cycle;

\path[fill=fillColor,fill opacity=0.50] (133.10,111.04) --
	(135.03,107.71) --
	(131.18,107.71) --
	cycle;

\path[fill=fillColor,fill opacity=0.50] (124.95,113.72) --
	(126.87,110.39) --
	(123.03,110.39) --
	cycle;

\path[fill=fillColor,fill opacity=0.50] (117.45,106.96) --
	(119.37,103.63) --
	(115.53,103.63) --
	cycle;

\path[fill=fillColor,fill opacity=0.50] (123.86,111.87) --
	(125.79,108.54) --
	(121.94,108.54) --
	cycle;

\path[fill=fillColor,fill opacity=0.50] (139.13,117.24) --
	(141.05,113.91) --
	(137.20,113.91) --
	cycle;

\path[fill=fillColor,fill opacity=0.50] (124.52,117.24) --
	(126.44,113.91) --
	(122.60,113.91) --
	cycle;

\path[fill=fillColor,fill opacity=0.50] (120.55,115.32) --
	(122.47,111.99) --
	(118.63,111.99) --
	cycle;

\path[fill=fillColor,fill opacity=0.50] (130.10,126.95) --
	(132.02,123.62) --
	(128.18,123.62) --
	cycle;

\path[fill=fillColor,fill opacity=0.50] (120.53,105.66) --
	(122.45,102.33) --
	(118.61,102.33) --
	cycle;

\path[fill=fillColor,fill opacity=0.50] (137.24,115.90) --
	(139.16,112.57) --
	(135.32,112.57) --
	cycle;

\path[fill=fillColor,fill opacity=0.50] (125.75,117.97) --
	(127.67,114.64) --
	(123.83,114.64) --
	cycle;

\path[fill=fillColor,fill opacity=0.50] (128.54,132.51) --
	(130.46,129.18) --
	(126.62,129.18) --
	cycle;

\path[fill=fillColor,fill opacity=0.50] (134.25,135.85) --
	(136.17,132.52) --
	(132.33,132.52) --
	cycle;

\path[fill=fillColor,fill opacity=0.50] (127.43,119.93) --
	(129.35,116.61) --
	(125.51,116.61) --
	cycle;

\path[fill=fillColor,fill opacity=0.50] (144.88,120.71) --
	(146.80,117.38) --
	(142.96,117.38) --
	cycle;

\path[fill=fillColor,fill opacity=0.50] (144.17,139.08) --
	(146.10,135.75) --
	(142.25,135.75) --
	cycle;

\path[fill=fillColor,fill opacity=0.50] (157.99,139.16) --
	(159.91,135.83) --
	(156.06,135.83) --
	cycle;

\path[fill=fillColor,fill opacity=0.50] (130.42,129.14) --
	(132.34,125.81) --
	(128.50,125.81) --
	cycle;

\path[fill=fillColor,fill opacity=0.50] (138.34,156.48) --
	(140.26,153.15) --
	(136.42,153.15) --
	cycle;

\path[fill=fillColor,fill opacity=0.50] (143.37,139.24) --
	(145.29,135.91) --
	(141.45,135.91) --
	cycle;

\path[fill=fillColor,fill opacity=0.50] (129.18,125.79) --
	(131.10,122.47) --
	(127.26,122.47) --
	cycle;

\path[fill=fillColor,fill opacity=0.50] (125.02,114.06) --
	(126.94,110.73) --
	(123.10,110.73) --
	cycle;

\path[fill=fillColor,fill opacity=0.50] (130.55,127.16) --
	(132.47,123.83) --
	(128.62,123.83) --
	cycle;

\path[fill=fillColor,fill opacity=0.50] (136.34,128.68) --
	(138.26,125.35) --
	(134.42,125.35) --
	cycle;

\path[fill=fillColor,fill opacity=0.50] (158.24,141.70) --
	(160.17,138.37) --
	(156.32,138.37) --
	cycle;

\path[fill=fillColor,fill opacity=0.50] (154.44,138.28) --
	(156.36,134.95) --
	(152.52,134.95) --
	cycle;

\path[fill=fillColor,fill opacity=0.50] (149.75,126.38) --
	(151.67,123.06) --
	(147.83,123.06) --
	cycle;

\path[fill=fillColor,fill opacity=0.50] (142.17,146.95) --
	(144.09,143.62) --
	(140.25,143.62) --
	cycle;

\path[fill=fillColor,fill opacity=0.50] (151.39,119.73) --
	(153.31,116.40) --
	(149.47,116.40) --
	cycle;

\path[fill=fillColor,fill opacity=0.50] (157.11,139.68) --
	(159.03,136.36) --
	(155.19,136.36) --
	cycle;

\path[fill=fillColor,fill opacity=0.50] (133.28,131.44) --
	(135.20,128.11) --
	(131.36,128.11) --
	cycle;

\path[fill=fillColor,fill opacity=0.50] (149.74,141.26) --
	(151.67,137.93) --
	(147.82,137.93) --
	cycle;

\path[fill=fillColor,fill opacity=0.50] (143.83,141.54) --
	(145.75,138.22) --
	(141.91,138.22) --
	cycle;

\path[fill=fillColor,fill opacity=0.50] (130.73,144.04) --
	(132.65,140.71) --
	(128.81,140.71) --
	cycle;

\path[fill=fillColor,fill opacity=0.50] (173.51,164.83) --
	(175.43,161.50) --
	(171.59,161.50) --
	cycle;

\path[fill=fillColor,fill opacity=0.50] (163.82,144.06) --
	(165.74,140.74) --
	(161.90,140.74) --
	cycle;

\path[fill=fillColor,fill opacity=0.50] (154.14,137.60) --
	(156.06,134.27) --
	(152.22,134.27) --
	cycle;

\path[fill=fillColor,fill opacity=0.50] (134.67,135.12) --
	(136.59,131.79) --
	(132.75,131.79) --
	cycle;

\path[fill=fillColor,fill opacity=0.50] (164.63,144.65) --
	(166.55,141.33) --
	(162.71,141.33) --
	cycle;

\path[fill=fillColor,fill opacity=0.50] (153.65,143.01) --
	(155.57,139.68) --
	(151.73,139.68) --
	cycle;

\path[fill=fillColor,fill opacity=0.50] (154.23,140.63) --
	(156.15,137.31) --
	(152.31,137.31) --
	cycle;

\path[fill=fillColor,fill opacity=0.50] (155.51,141.76) --
	(157.44,138.43) --
	(153.59,138.43) --
	cycle;

\path[fill=fillColor,fill opacity=0.50] (154.31,134.33) --
	(156.23,131.00) --
	(152.38,131.00) --
	cycle;

\path[fill=fillColor,fill opacity=0.50] (157.37,147.28) --
	(159.29,143.95) --
	(155.44,143.95) --
	cycle;

\path[fill=fillColor,fill opacity=0.50] (160.98,140.87) --
	(162.90,137.54) --
	(159.06,137.54) --
	cycle;

\path[fill=fillColor,fill opacity=0.50] (180.84,148.16) --
	(182.76,144.83) --
	(178.92,144.83) --
	cycle;

\path[fill=fillColor,fill opacity=0.50] (171.44,151.83) --
	(173.36,148.50) --
	(169.52,148.50) --
	cycle;

\path[fill=fillColor,fill opacity=0.50] (163.41,141.07) --
	(165.33,137.74) --
	(161.49,137.74) --
	cycle;

\path[fill=fillColor,fill opacity=0.50] (174.56,163.37) --
	(176.48,160.05) --
	(172.64,160.05) --
	cycle;

\path[fill=fillColor,fill opacity=0.50] (180.63,154.11) --
	(182.55,150.78) --
	(178.71,150.78) --
	cycle;

\path[fill=fillColor,fill opacity=0.50] (161.05,148.42) --
	(162.97,145.09) --
	(159.13,145.09) --
	cycle;

\path[fill=fillColor,fill opacity=0.50] (150.56,156.27) --
	(152.48,152.95) --
	(148.63,152.95) --
	cycle;

\path[fill=fillColor,fill opacity=0.50] (167.26,136.50) --
	(169.19,133.17) --
	(165.34,133.17) --
	cycle;

\path[fill=fillColor,fill opacity=0.50] (144.58,148.90) --
	(146.50,145.57) --
	(142.65,145.57) --
	cycle;

\path[fill=fillColor,fill opacity=0.50] (188.40,161.73) --
	(190.32,158.40) --
	(186.48,158.40) --
	cycle;

\path[fill=fillColor,fill opacity=0.50] (170.15,157.24) --
	(172.07,153.92) --
	(168.23,153.92) --
	cycle;

\path[fill=fillColor,fill opacity=0.50] (188.40,164.34) --
	(190.32,161.01) --
	(186.48,161.01) --
	cycle;

\path[fill=fillColor,fill opacity=0.50] (178.77,152.50) --
	(180.69,149.17) --
	(176.85,149.17) --
	cycle;

\path[fill=fillColor,fill opacity=0.50] (173.62,170.80) --
	(175.54,167.47) --
	(171.70,167.47) --
	cycle;

\path[fill=fillColor,fill opacity=0.50] (188.40,179.36) --
	(190.32,176.04) --
	(186.48,176.04) --
	cycle;

\path[fill=fillColor,fill opacity=0.50] (156.21,152.63) --
	(158.14,149.30) --
	(154.29,149.30) --
	cycle;

\path[fill=fillColor,fill opacity=0.50] (188.40,163.82) --
	(190.32,160.50) --
	(186.48,160.50) --
	cycle;

\path[fill=fillColor,fill opacity=0.50] (177.61,154.08) --
	(179.53,150.75) --
	(175.68,150.75) --
	cycle;

\path[fill=fillColor,fill opacity=0.50] (178.47,143.71) --
	(180.39,140.38) --
	(176.54,140.38) --
	cycle;

\path[fill=fillColor,fill opacity=0.50] (109.69,101.61) --
	(111.61, 98.28) --
	(107.77, 98.28) --
	cycle;

\path[fill=fillColor,fill opacity=0.50] (107.41,100.60) --
	(109.33, 97.27) --
	(105.49, 97.27) --
	cycle;

\path[fill=fillColor,fill opacity=0.50] (110.89,104.21) --
	(112.81,100.88) --
	(108.97,100.88) --
	cycle;

\path[fill=fillColor,fill opacity=0.50] (188.40,100.60) --
	(190.32, 97.27) --
	(186.48, 97.27) --
	cycle;

\path[fill=fillColor,fill opacity=0.50] (111.21,106.96) --
	(113.13,103.63) --
	(109.29,103.63) --
	cycle;

\path[fill=fillColor,fill opacity=0.50] (107.47,100.60) --
	(109.39, 97.27) --
	(105.55, 97.27) --
	cycle;

\path[fill=fillColor,fill opacity=0.50] (110.14,104.21) --
	(112.06,100.88) --
	(108.22,100.88) --
	cycle;

\path[fill=fillColor,fill opacity=0.50] (109.89,103.40) --
	(111.82,100.08) --
	(107.97,100.08) --
	cycle;

\path[fill=fillColor,fill opacity=0.50] (107.83,101.61) --
	(109.75, 98.28) --
	(105.91, 98.28) --
	cycle;

\path[fill=fillColor,fill opacity=0.50] (111.03,101.61) --
	(112.95, 98.28) --
	(109.11, 98.28) --
	cycle;

\path[fill=fillColor,fill opacity=0.50] (112.10,104.96) --
	(114.03,101.63) --
	(110.18,101.63) --
	cycle;

\path[fill=fillColor,fill opacity=0.50] (107.77,102.54) --
	(109.69, 99.21) --
	(105.85, 99.21) --
	cycle;

\path[fill=fillColor,fill opacity=0.50] (111.86,104.96) --
	(113.78,101.63) --
	(109.93,101.63) --
	cycle;

\path[fill=fillColor,fill opacity=0.50] (110.57,107.55) --
	(112.49,104.22) --
	(108.65,104.22) --
	cycle;

\path[fill=fillColor,fill opacity=0.50] (110.04,104.96) --
	(111.96,101.63) --
	(108.12,101.63) --
	cycle;

\path[fill=fillColor,fill opacity=0.50] (109.02,101.61) --
	(110.94, 98.28) --
	(107.10, 98.28) --
	cycle;

\path[fill=fillColor,fill opacity=0.50] (109.94,104.21) --
	(111.86,100.88) --
	(108.02,100.88) --
	cycle;

\path[fill=fillColor,fill opacity=0.50] (108.07,102.54) --
	(109.99, 99.21) --
	(106.15, 99.21) --
	cycle;

\path[fill=fillColor,fill opacity=0.50] (111.56,104.21) --
	(113.48,100.88) --
	(109.64,100.88) --
	cycle;

\path[fill=fillColor,fill opacity=0.50] (110.62,104.96) --
	(112.54,101.63) --
	(108.69,101.63) --
	cycle;

\path[fill=fillColor,fill opacity=0.50] (108.96,103.40) --
	(110.88,100.08) --
	(107.04,100.08) --
	cycle;

\path[fill=fillColor,fill opacity=0.50] (110.52,104.96) --
	(112.44,101.63) --
	(108.60,101.63) --
	cycle;

\path[fill=fillColor,fill opacity=0.50] (110.85,103.40) --
	(112.77,100.08) --
	(108.93,100.08) --
	cycle;

\path[fill=fillColor,fill opacity=0.50] (115.95,106.33) --
	(117.87,103.00) --
	(114.03,103.00) --
	cycle;

\path[fill=fillColor,fill opacity=0.50] (115.59,106.96) --
	(117.51,103.63) --
	(113.67,103.63) --
	cycle;

\path[fill=fillColor,fill opacity=0.50] (112.86,107.55) --
	(114.78,104.22) --
	(110.94,104.22) --
	cycle;

\path[fill=fillColor,fill opacity=0.50] (112.15,110.15) --
	(114.07,106.82) --
	(110.22,106.82) --
	cycle;

\path[fill=fillColor,fill opacity=0.50] (113.01,117.97) --
	(114.93,114.64) --
	(111.09,114.64) --
	cycle;

\path[fill=fillColor,fill opacity=0.50] (115.50,108.66) --
	(117.42,105.33) --
	(113.58,105.33) --
	cycle;

\path[fill=fillColor,fill opacity=0.50] (115.95,112.65) --
	(117.87,109.32) --
	(114.03,109.32) --
	cycle;

\path[fill=fillColor,fill opacity=0.50] (114.34,108.12) --
	(116.26,104.79) --
	(112.42,104.79) --
	cycle;

\path[fill=fillColor,fill opacity=0.50] (113.42,104.96) --
	(115.35,101.63) --
	(111.50,101.63) --
	cycle;

\path[fill=fillColor,fill opacity=0.50] (116.67,105.66) --
	(118.59,102.33) --
	(114.75,102.33) --
	cycle;

\path[fill=fillColor,fill opacity=0.50] (114.41,103.40) --
	(116.33,100.08) --
	(112.49,100.08) --
	cycle;

\path[fill=fillColor,fill opacity=0.50] (115.74,104.21) --
	(117.67,100.88) --
	(113.82,100.88) --
	cycle;

\path[fill=fillColor,fill opacity=0.50] (188.40,108.12) --
	(190.32,104.79) --
	(186.48,104.79) --
	cycle;

\path[fill=fillColor,fill opacity=0.50] (114.71,105.66) --
	(116.63,102.33) --
	(112.79,102.33) --
	cycle;

\path[fill=fillColor,fill opacity=0.50] (114.44,110.15) --
	(116.36,106.82) --
	(112.52,106.82) --
	cycle;

\path[fill=fillColor,fill opacity=0.50] (119.38,107.55) --
	(121.30,104.22) --
	(117.46,104.22) --
	cycle;

\path[fill=fillColor,fill opacity=0.50] (114.27,107.55) --
	(116.19,104.22) --
	(112.35,104.22) --
	cycle;

\path[fill=fillColor,fill opacity=0.50] (114.07,106.33) --
	(115.99,103.00) --
	(112.14,103.00) --
	cycle;

\path[fill=fillColor,fill opacity=0.50] (116.64,106.96) --
	(118.56,103.63) --
	(114.72,103.63) --
	cycle;

\path[fill=fillColor,fill opacity=0.50] (119.33,108.66) --
	(121.26,105.33) --
	(117.41,105.33) --
	cycle;

\path[fill=fillColor,fill opacity=0.50] (117.50,114.39) --
	(119.42,111.06) --
	(115.58,111.06) --
	cycle;

\path[fill=fillColor,fill opacity=0.50] (122.52,109.18) --
	(124.44,105.85) --
	(120.60,105.85) --
	cycle;

\path[fill=fillColor,fill opacity=0.50] (115.74,105.66) --
	(117.67,102.33) --
	(113.82,102.33) --
	cycle;

\path[fill=fillColor,fill opacity=0.50] (118.32,109.67) --
	(120.24,106.34) --
	(116.40,106.34) --
	cycle;

\path[fill=fillColor,fill opacity=0.50] (114.54,109.18) --
	(116.46,105.85) --
	(112.62,105.85) --
	cycle;

\path[fill=fillColor,fill opacity=0.50] (117.21,113.72) --
	(119.13,110.39) --
	(115.29,110.39) --
	cycle;

\path[fill=fillColor,fill opacity=0.50] (122.04,109.18) --
	(123.96,105.85) --
	(120.12,105.85) --
	cycle;

\path[fill=fillColor,fill opacity=0.50] (116.56,109.18) --
	(118.48,105.85) --
	(114.64,105.85) --
	cycle;

\path[fill=fillColor,fill opacity=0.50] (116.94,110.15) --
	(118.86,106.82) --
	(115.02,106.82) --
	cycle;

\path[fill=fillColor,fill opacity=0.50] (117.21,112.27) --
	(119.13,108.94) --
	(115.29,108.94) --
	cycle;

\path[fill=fillColor,fill opacity=0.50] (127.44,111.87) --
	(129.36,108.54) --
	(125.52,108.54) --
	cycle;

\path[fill=fillColor,fill opacity=0.50] (114.03,106.96) --
	(115.95,103.63) --
	(112.11,103.63) --
	cycle;

\path[fill=fillColor,fill opacity=0.50] (113.86,106.33) --
	(115.78,103.00) --
	(111.93,103.00) --
	cycle;

\path[fill=fillColor,fill opacity=0.50] (118.49,110.15) --
	(120.41,106.82) --
	(116.57,106.82) --
	cycle;

\path[fill=fillColor,fill opacity=0.50] (120.84,111.04) --
	(122.76,107.71) --
	(118.92,107.71) --
	cycle;

\path[fill=fillColor,fill opacity=0.50] (123.28,110.15) --
	(125.20,106.82) --
	(121.36,106.82) --
	cycle;

\path[fill=fillColor,fill opacity=0.50] (121.11,115.61) --
	(123.03,112.28) --
	(119.19,112.28) --
	cycle;

\path[fill=fillColor,fill opacity=0.50] (124.42,116.72) --
	(126.34,113.39) --
	(122.50,113.39) --
	cycle;

\path[fill=fillColor,fill opacity=0.50] (120.51,112.27) --
	(122.43,108.94) --
	(118.59,108.94) --
	cycle;

\path[fill=fillColor,fill opacity=0.50] (123.38,117.49) --
	(125.30,114.16) --
	(121.45,114.16) --
	cycle;

\path[fill=fillColor,fill opacity=0.50] (132.31,114.71) --
	(134.24,111.38) --
	(130.39,111.38) --
	cycle;

\path[fill=fillColor,fill opacity=0.50] (115.98,110.15) --
	(117.90,106.82) --
	(114.06,106.82) --
	cycle;

\path[fill=fillColor,fill opacity=0.50] (117.21,110.15) --
	(119.13,106.82) --
	(115.29,106.82) --
	cycle;

\path[fill=fillColor,fill opacity=0.50] (120.73,124.91) --
	(122.65,121.58) --
	(118.81,121.58) --
	cycle;

\path[fill=fillColor,fill opacity=0.50] (117.65,111.87) --
	(119.57,108.54) --
	(115.73,108.54) --
	cycle;

\path[fill=fillColor,fill opacity=0.50] (125.98,120.13) --
	(127.90,116.80) --
	(124.06,116.80) --
	cycle;

\path[fill=fillColor,fill opacity=0.50] (118.93,111.04) --
	(120.85,107.71) --
	(117.01,107.71) --
	cycle;

\path[fill=fillColor,fill opacity=0.50] (130.58,131.36) --
	(132.50,128.04) --
	(128.66,128.04) --
	cycle;

\path[fill=fillColor,fill opacity=0.50] (127.06,124.91) --
	(128.98,121.58) --
	(125.14,121.58) --
	cycle;

\path[fill=fillColor,fill opacity=0.50] (141.66,118.88) --
	(143.59,115.56) --
	(139.74,115.56) --
	cycle;

\path[fill=fillColor,fill opacity=0.50] (121.22,123.23) --
	(123.15,119.90) --
	(119.30,119.90) --
	cycle;

\path[fill=fillColor,fill opacity=0.50] (120.41,117.24) --
	(122.33,113.91) --
	(118.49,113.91) --
	cycle;

\path[fill=fillColor,fill opacity=0.50] (118.42,111.87) --
	(120.34,108.54) --
	(116.50,108.54) --
	cycle;

\path[fill=fillColor,fill opacity=0.50] (125.03,117.49) --
	(126.95,114.16) --
	(123.11,114.16) --
	cycle;

\path[fill=fillColor,fill opacity=0.50] (128.60,127.27) --
	(130.52,123.94) --
	(126.68,123.94) --
	cycle;

\path[fill=fillColor,fill opacity=0.50] (131.34,125.91) --
	(133.26,122.59) --
	(129.41,122.59) --
	cycle;

\path[fill=fillColor,fill opacity=0.50] (123.67,111.87) --
	(125.59,108.54) --
	(121.75,108.54) --
	cycle;

\path[fill=fillColor,fill opacity=0.50] (144.43,141.58) --
	(146.35,138.25) --
	(142.51,138.25) --
	cycle;

\path[fill=fillColor,fill opacity=0.50] (126.57,139.65) --
	(128.49,136.32) --
	(124.65,136.32) --
	cycle;

\path[fill=fillColor,fill opacity=0.50] (140.66,141.29) --
	(142.58,137.96) --
	(138.74,137.96) --
	cycle;

\path[fill=fillColor,fill opacity=0.50] (124.22,113.38) --
	(126.14,110.05) --
	(122.30,110.05) --
	cycle;

\path[fill=fillColor,fill opacity=0.50] (143.59,114.71) --
	(145.51,111.38) --
	(141.66,111.38) --
	cycle;

\path[fill=fillColor,fill opacity=0.50] (126.58,121.44) --
	(128.51,118.11) --
	(124.66,118.11) --
	cycle;

\path[fill=fillColor,fill opacity=0.50] (123.91,124.24) --
	(125.83,120.91) --
	(121.99,120.91) --
	cycle;

\path[fill=fillColor,fill opacity=0.50] (135.03,119.10) --
	(136.95,115.77) --
	(133.11,115.77) --
	cycle;

\path[fill=fillColor,fill opacity=0.50] (129.23,120.52) --
	(131.15,117.19) --
	(127.31,117.19) --
	cycle;

\path[fill=fillColor,fill opacity=0.50] (128.15,127.05) --
	(130.08,123.73) --
	(126.23,123.73) --
	cycle;

\path[fill=fillColor,fill opacity=0.50] (131.73,127.05) --
	(133.65,123.73) --
	(129.81,123.73) --
	cycle;

\path[fill=fillColor,fill opacity=0.50] (134.06,118.44) --
	(135.98,115.11) --
	(132.14,115.11) --
	cycle;

\path[fill=fillColor,fill opacity=0.50] (129.71,144.65) --
	(131.63,141.33) --
	(127.79,141.33) --
	cycle;

\path[fill=fillColor,fill opacity=0.50] (133.75,127.59) --
	(135.67,124.26) --
	(131.83,124.26) --
	cycle;

\path[fill=fillColor,fill opacity=0.50] (123.50,124.78) --
	(125.42,121.45) --
	(121.58,121.45) --
	cycle;

\path[fill=fillColor,fill opacity=0.50] (137.65,124.78) --
	(139.57,121.45) --
	(135.73,121.45) --
	cycle;

\path[fill=fillColor,fill opacity=0.50] (139.88,148.28) --
	(141.80,144.95) --
	(137.96,144.95) --
	cycle;

\path[fill=fillColor,fill opacity=0.50] (137.77,142.33) --
	(139.69,139.00) --
	(135.85,139.00) --
	cycle;

\path[fill=fillColor,fill opacity=0.50] (138.37,133.85) --
	(140.29,130.53) --
	(136.45,130.53) --
	cycle;

\path[fill=fillColor,fill opacity=0.50] (188.40,143.31) --
	(190.32,139.98) --
	(186.48,139.98) --
	cycle;

\path[fill=fillColor,fill opacity=0.50] (151.28,139.08) --
	(153.20,135.75) --
	(149.36,135.75) --
	cycle;

\path[fill=fillColor,fill opacity=0.50] (134.95,138.93) --
	(136.87,135.60) --
	(133.03,135.60) --
	cycle;

\path[fill=fillColor,fill opacity=0.50] (153.16,150.00) --
	(155.09,146.68) --
	(151.24,146.68) --
	cycle;

\path[fill=fillColor,fill opacity=0.50] (137.58,146.91) --
	(139.50,143.58) --
	(135.66,143.58) --
	cycle;

\path[fill=fillColor,fill opacity=0.50] (151.48,159.09) --
	(153.40,155.76) --
	(149.56,155.76) --
	cycle;

\path[fill=fillColor,fill opacity=0.50] (151.03,138.24) --
	(152.95,134.91) --
	(149.11,134.91) --
	cycle;

\path[fill=fillColor,fill opacity=0.50] (161.85,151.21) --
	(163.77,147.88) --
	(159.93,147.88) --
	cycle;

\path[fill=fillColor,fill opacity=0.50] (147.17,140.12) --
	(149.10,136.79) --
	(145.25,136.79) --
	cycle;

\path[fill=fillColor,fill opacity=0.50] (139.60,135.44) --
	(141.52,132.11) --
	(137.68,132.11) --
	cycle;

\path[fill=fillColor,fill opacity=0.50] (133.91,135.22) --
	(135.83,131.90) --
	(131.99,131.90) --
	cycle;

\path[fill=fillColor,fill opacity=0.50] (130.16,138.49) --
	(132.08,135.16) --
	(128.24,135.16) --
	cycle;

\path[fill=fillColor,fill opacity=0.50] (137.38,142.30) --
	(139.30,138.97) --
	(135.46,138.97) --
	cycle;

\path[fill=fillColor,fill opacity=0.50] (140.82,131.21) --
	(142.74,127.89) --
	(138.90,127.89) --
	cycle;

\path[fill=fillColor,fill opacity=0.50] (151.70,154.70) --
	(153.62,151.37) --
	(149.78,151.37) --
	cycle;

\path[fill=fillColor,fill opacity=0.50] (172.99,141.29) --
	(174.91,137.96) --
	(171.06,137.96) --
	cycle;

\path[fill=fillColor,fill opacity=0.50] (156.56,149.71) --
	(158.49,146.39) --
	(154.64,146.39) --
	cycle;

\path[fill=fillColor,fill opacity=0.50] (136.34,137.47) --
	(138.26,134.14) --
	(134.42,134.14) --
	cycle;

\path[fill=fillColor,fill opacity=0.50] (164.34,161.98) --
	(166.26,158.66) --
	(162.42,158.66) --
	cycle;

\path[fill=fillColor,fill opacity=0.50] (154.23,145.94) --
	(156.15,142.62) --
	(152.31,142.62) --
	cycle;

\path[fill=fillColor,fill opacity=0.50] (133.29,130.10) --
	(135.21,126.77) --
	(131.36,126.77) --
	cycle;

\path[fill=fillColor,fill opacity=0.50] (188.40,160.01) --
	(190.32,156.68) --
	(186.48,156.68) --
	cycle;

\path[fill=fillColor,fill opacity=0.50] (172.88,176.16) --
	(174.80,172.83) --
	(170.96,172.83) --
	cycle;

\path[fill=fillColor,fill opacity=0.50] (188.40,131.81) --
	(190.32,128.48) --
	(186.48,128.48) --
	cycle;

\path[fill=fillColor,fill opacity=0.50] (153.74,149.09) --
	(155.66,145.77) --
	(151.82,145.77) --
	cycle;

\path[fill=fillColor,fill opacity=0.50] (142.31,143.39) --
	(144.23,140.06) --
	(140.39,140.06) --
	cycle;

\path[fill=fillColor,fill opacity=0.50] (188.40,173.13) --
	(190.32,169.80) --
	(186.48,169.80) --
	cycle;

\path[fill=fillColor,fill opacity=0.50] (144.43,174.33) --
	(146.35,171.01) --
	(142.51,171.01) --
	cycle;

\path[fill=fillColor,fill opacity=0.50] (157.86,137.33) --
	(159.78,134.01) --
	(155.94,134.01) --
	cycle;

\path[fill=fillColor,fill opacity=0.50] (132.04,146.43) --
	(133.96,143.10) --
	(130.12,143.10) --
	cycle;

\path[fill=fillColor,fill opacity=0.50] (152.35,133.55) --
	(154.27,130.22) --
	(150.43,130.22) --
	cycle;

\path[fill=fillColor,fill opacity=0.50] (159.42,138.89) --
	(161.34,135.56) --
	(157.50,135.56) --
	cycle;

\path[fill=fillColor,fill opacity=0.50] (188.40,171.04) --
	(190.32,167.71) --
	(186.48,167.71) --
	cycle;

\path[fill=fillColor,fill opacity=0.50] (146.93,145.35) --
	(148.85,142.02) --
	(145.01,142.02) --
	cycle;

\path[fill=fillColor,fill opacity=0.50] (178.60,155.11) --
	(180.52,151.78) --
	(176.68,151.78) --
	cycle;

\path[fill=fillColor,fill opacity=0.50] (134.28,144.46) --
	(136.20,141.13) --
	(132.36,141.13) --
	cycle;

\path[fill=fillColor,fill opacity=0.50] (188.40,175.94) --
	(190.32,172.62) --
	(186.48,172.62) --
	cycle;

\path[fill=fillColor,fill opacity=0.50] (188.40,186.05) --
	(190.32,182.72) --
	(186.48,182.72) --
	cycle;

\path[fill=fillColor,fill opacity=0.50] (169.36,161.77) --
	(171.28,158.44) --
	(167.44,158.44) --
	cycle;

\path[fill=fillColor,fill opacity=0.50] (188.40,186.08) --
	(190.32,182.75) --
	(186.48,182.75) --
	cycle;

\path[fill=fillColor,fill opacity=0.50] (139.18,154.49) --
	(141.10,151.16) --
	(137.26,151.16) --
	cycle;

\path[fill=fillColor,fill opacity=0.50] (149.35,152.63) --
	(151.28,149.30) --
	(147.43,149.30) --
	cycle;

\path[fill=fillColor,fill opacity=0.50] (150.86,168.92) --
	(152.78,165.59) --
	(148.94,165.59) --
	cycle;

\path[fill=fillColor,fill opacity=0.50] (159.51,174.37) --
	(161.43,171.04) --
	(157.58,171.04) --
	cycle;

\path[fill=fillColor,fill opacity=0.50] (177.08,171.27) --
	(179.00,167.94) --
	(175.16,167.94) --
	cycle;

\path[fill=fillColor,fill opacity=0.50] (158.73,152.44) --
	(160.65,149.11) --
	(156.81,149.11) --
	cycle;

\path[fill=fillColor,fill opacity=0.50] (172.73,169.79) --
	(174.65,166.46) --
	(170.81,166.46) --
	cycle;

\path[fill=fillColor,fill opacity=0.50] (176.69,182.26) --
	(178.61,178.93) --
	(174.77,178.93) --
	cycle;

\path[fill=fillColor,fill opacity=0.50] (152.81,142.39) --
	(154.73,139.06) --
	(150.89,139.06) --
	cycle;

\path[fill=fillColor,fill opacity=0.50] (179.08,156.76) --
	(181.00,153.44) --
	(177.16,153.44) --
	cycle;

\path[fill=fillColor,fill opacity=0.50] (188.40,166.10) --
	(190.32,162.77) --
	(186.48,162.77) --
	cycle;

\path[fill=fillColor,fill opacity=0.50] (108.80,103.40) --
	(110.72,100.08) --
	(106.88,100.08) --
	cycle;

\path[fill=fillColor,fill opacity=0.50] (108.07,102.54) --
	(109.99, 99.21) --
	(106.15, 99.21) --
	cycle;

\path[fill=fillColor,fill opacity=0.50] (188.40,102.54) --
	(190.32, 99.21) --
	(186.48, 99.21) --
	cycle;

\path[fill=fillColor,fill opacity=0.50] (109.54,109.67) --
	(111.46,106.34) --
	(107.62,106.34) --
	cycle;

\path[fill=fillColor,fill opacity=0.50] (108.30,102.54) --
	(110.22, 99.21) --
	(106.38, 99.21) --
	cycle;

\path[fill=fillColor,fill opacity=0.50] (107.35,102.54) --
	(109.27, 99.21) --
	(105.43, 99.21) --
	cycle;

\path[fill=fillColor,fill opacity=0.50] (188.40,102.54) --
	(190.32, 99.21) --
	(186.48, 99.21) --
	cycle;

\path[fill=fillColor,fill opacity=0.50] (140.26,101.61) --
	(142.18, 98.28) --
	(138.34, 98.28) --
	cycle;

\path[fill=fillColor,fill opacity=0.50] (188.40,102.54) --
	(190.32, 99.21) --
	(186.48, 99.21) --
	cycle;

\path[fill=fillColor,fill opacity=0.50] (108.01,102.54) --
	(109.93, 99.21) --
	(106.09, 99.21) --
	cycle;

\path[fill=fillColor,fill opacity=0.50] (109.33,104.21) --
	(111.26,100.88) --
	(107.41,100.88) --
	cycle;

\path[fill=fillColor,fill opacity=0.50] (108.80,103.40) --
	(110.72,100.08) --
	(106.88,100.08) --
	cycle;

\path[fill=fillColor,fill opacity=0.50] (108.75,103.40) --
	(110.67,100.08) --
	(106.82,100.08) --
	cycle;

\path[fill=fillColor,fill opacity=0.50] (110.94,111.46) --
	(112.86,108.14) --
	(109.02,108.14) --
	cycle;

\path[fill=fillColor,fill opacity=0.50] (108.36,115.90) --
	(110.28,112.57) --
	(106.43,112.57) --
	cycle;

\path[fill=fillColor,fill opacity=0.50] (109.89,103.40) --
	(111.82,100.08) --
	(107.97,100.08) --
	cycle;

\path[fill=fillColor,fill opacity=0.50] (110.09,103.40) --
	(112.01,100.08) --
	(108.17,100.08) --
	cycle;

\path[fill=fillColor,fill opacity=0.50] (108.69,103.40) --
	(110.61,100.08) --
	(106.77,100.08) --
	cycle;

\path[fill=fillColor,fill opacity=0.50] (110.94,104.21) --
	(112.86,100.88) --
	(109.02,100.88) --
	cycle;

\path[fill=fillColor,fill opacity=0.50] (188.40,104.21) --
	(190.32,100.88) --
	(186.48,100.88) --
	cycle;

\path[fill=fillColor,fill opacity=0.50] (110.76,105.66) --
	(112.68,102.33) --
	(108.83,102.33) --
	cycle;

\path[fill=fillColor,fill opacity=0.50] (109.99,110.15) --
	(111.91,106.82) --
	(108.07,106.82) --
	cycle;

\path[fill=fillColor,fill opacity=0.50] (110.38,104.96) --
	(112.30,101.63) --
	(108.46,101.63) --
	cycle;

\path[fill=fillColor,fill opacity=0.50] (110.04,105.66) --
	(111.96,102.33) --
	(108.12,102.33) --
	cycle;

\path[fill=fillColor,fill opacity=0.50] (188.40,104.21) --
	(190.32,100.88) --
	(186.48,100.88) --
	cycle;

\path[fill=fillColor,fill opacity=0.50] (110.98,107.55) --
	(112.91,104.22) --
	(109.06,104.22) --
	cycle;

\path[fill=fillColor,fill opacity=0.50] (110.62,104.96) --
	(112.54,101.63) --
	(108.69,101.63) --
	cycle;

\path[fill=fillColor,fill opacity=0.50] (110.43,105.66) --
	(112.35,102.33) --
	(108.51,102.33) --
	cycle;

\path[fill=fillColor,fill opacity=0.50] (111.64,105.66) --
	(113.56,102.33) --
	(109.72,102.33) --
	cycle;

\path[fill=fillColor,fill opacity=0.50] (110.80,104.21) --
	(112.72,100.88) --
	(108.88,100.88) --
	cycle;

\path[fill=fillColor,fill opacity=0.50] (111.90,104.96) --
	(113.82,101.63) --
	(109.98,101.63) --
	cycle;

\path[fill=fillColor,fill opacity=0.50] (112.31,108.66) --
	(114.23,105.33) --
	(110.39,105.33) --
	cycle;

\path[fill=fillColor,fill opacity=0.50] (112.02,106.33) --
	(113.94,103.00) --
	(110.10,103.00) --
	cycle;

\path[fill=fillColor,fill opacity=0.50] (112.74,106.33) --
	(114.67,103.00) --
	(110.82,103.00) --
	cycle;

\path[fill=fillColor,fill opacity=0.50] (110.94,105.66) --
	(112.86,102.33) --
	(109.02,102.33) --
	cycle;

\path[fill=fillColor,fill opacity=0.50] (111.38,105.66) --
	(113.31,102.33) --
	(109.46,102.33) --
	cycle;

\path[fill=fillColor,fill opacity=0.50] (111.47,106.96) --
	(113.39,103.63) --
	(109.55,103.63) --
	cycle;

\path[fill=fillColor,fill opacity=0.50] (112.35,108.66) --
	(114.27,105.33) --
	(110.43,105.33) --
	cycle;

\path[fill=fillColor,fill opacity=0.50] (111.51,108.66) --
	(113.44,105.33) --
	(109.59,105.33) --
	cycle;

\path[fill=fillColor,fill opacity=0.50] (188.40,104.96) --
	(190.32,101.63) --
	(186.48,101.63) --
	cycle;

\path[fill=fillColor,fill opacity=0.50] (114.97,109.18) --
	(116.89,105.85) --
	(113.05,105.85) --
	cycle;

\path[fill=fillColor,fill opacity=0.50] (113.42,106.33) --
	(115.35,103.00) --
	(111.50,103.00) --
	cycle;

\path[fill=fillColor,fill opacity=0.50] (111.90,106.96) --
	(113.82,103.63) --
	(109.98,103.63) --
	cycle;

\path[fill=fillColor,fill opacity=0.50] (112.15,111.46) --
	(114.07,108.14) --
	(110.22,108.14) --
	cycle;

\path[fill=fillColor,fill opacity=0.50] (113.20,105.66) --
	(115.12,102.33) --
	(111.28,102.33) --
	cycle;

\path[fill=fillColor,fill opacity=0.50] (112.02,106.96) --
	(113.94,103.63) --
	(110.10,103.63) --
	cycle;

\path[fill=fillColor,fill opacity=0.50] (111.07,105.66) --
	(113.00,102.33) --
	(109.15,102.33) --
	cycle;

\path[fill=fillColor,fill opacity=0.50] (112.59,117.49) --
	(114.51,114.16) --
	(110.67,114.16) --
	cycle;

\path[fill=fillColor,fill opacity=0.50] (188.40,105.66) --
	(190.32,102.33) --
	(186.48,102.33) --
	cycle;

\path[fill=fillColor,fill opacity=0.50] (113.46,105.66) --
	(115.38,102.33) --
	(111.54,102.33) --
	cycle;

\path[fill=fillColor,fill opacity=0.50] (114.90,115.61) --
	(116.82,112.28) --
	(112.98,112.28) --
	cycle;

\path[fill=fillColor,fill opacity=0.50] (112.82,106.96) --
	(114.74,103.63) --
	(110.90,103.63) --
	cycle;

\path[fill=fillColor,fill opacity=0.50] (188.40,106.96) --
	(190.32,103.63) --
	(186.48,103.63) --
	cycle;

\path[fill=fillColor,fill opacity=0.50] (113.28,111.04) --
	(115.20,107.71) --
	(111.35,107.71) --
	cycle;

\path[fill=fillColor,fill opacity=0.50] (113.35,111.04) --
	(115.27,107.71) --
	(111.43,107.71) --
	cycle;

\path[fill=fillColor,fill opacity=0.50] (111.51,106.96) --
	(113.44,103.63) --
	(109.59,103.63) --
	cycle;

\path[fill=fillColor,fill opacity=0.50] (113.01,107.55) --
	(114.93,104.22) --
	(111.09,104.22) --
	cycle;

\path[fill=fillColor,fill opacity=0.50] (188.40,108.12) --
	(190.32,104.79) --
	(186.48,104.79) --
	cycle;

\path[fill=fillColor,fill opacity=0.50] (112.63,106.96) --
	(114.55,103.63) --
	(110.71,103.63) --
	cycle;

\path[fill=fillColor,fill opacity=0.50] (113.71,108.12) --
	(115.64,104.79) --
	(111.79,104.79) --
	cycle;

\path[fill=fillColor,fill opacity=0.50] (115.32,108.12) --
	(117.24,104.79) --
	(113.39,104.79) --
	cycle;

\path[fill=fillColor,fill opacity=0.50] (114.41,108.66) --
	(116.33,105.33) --
	(112.49,105.33) --
	cycle;

\path[fill=fillColor,fill opacity=0.50] (114.24,107.55) --
	(116.16,104.22) --
	(112.32,104.22) --
	cycle;

\path[fill=fillColor,fill opacity=0.50] (113.71,121.61) --
	(115.64,118.28) --
	(111.79,118.28) --
	cycle;

\path[fill=fillColor,fill opacity=0.50] (111.12,110.15) --
	(113.04,106.82) --
	(109.20,106.82) --
	cycle;

\path[fill=fillColor,fill opacity=0.50] (113.13,108.66) --
	(115.05,105.33) --
	(111.21,105.33) --
	cycle;

\path[fill=fillColor,fill opacity=0.50] (114.07,108.12) --
	(115.99,104.79) --
	(112.14,104.79) --
	cycle;

\path[fill=fillColor,fill opacity=0.50] (114.07,108.66) --
	(115.99,105.33) --
	(112.14,105.33) --
	cycle;

\path[fill=fillColor,fill opacity=0.50] (112.74,108.12) --
	(114.67,104.79) --
	(110.82,104.79) --
	cycle;

\path[fill=fillColor,fill opacity=0.50] (113.96,108.12) --
	(115.88,104.79) --
	(112.04,104.79) --
	cycle;

\path[fill=fillColor,fill opacity=0.50] (116.07,109.67) --
	(117.99,106.34) --
	(114.15,106.34) --
	cycle;

\path[fill=fillColor,fill opacity=0.50] (115.44,109.67) --
	(117.36,106.34) --
	(113.52,106.34) --
	cycle;

\path[fill=fillColor,fill opacity=0.50] (113.71,110.15) --
	(115.64,106.82) --
	(111.79,106.82) --
	cycle;

\path[fill=fillColor,fill opacity=0.50] (188.40,108.12) --
	(190.32,104.79) --
	(186.48,104.79) --
	cycle;

\path[fill=fillColor,fill opacity=0.50] (114.77,108.66) --
	(116.69,105.33) --
	(112.85,105.33) --
	cycle;

\path[fill=fillColor,fill opacity=0.50] (188.40,109.67) --
	(190.32,106.34) --
	(186.48,106.34) --
	cycle;

\path[fill=fillColor,fill opacity=0.50] (113.24,109.18) --
	(115.16,105.85) --
	(111.32,105.85) --
	cycle;

\path[fill=fillColor,fill opacity=0.50] (114.61,109.18) --
	(116.53,105.85) --
	(112.69,105.85) --
	cycle;

\path[fill=fillColor,fill opacity=0.50] (114.20,110.15) --
	(116.13,106.82) --
	(112.28,106.82) --
	cycle;

\path[fill=fillColor,fill opacity=0.50] (188.40,110.60) --
	(190.32,107.27) --
	(186.48,107.27) --
	cycle;

\path[fill=fillColor,fill opacity=0.50] (114.64,109.67) --
	(116.56,106.34) --
	(112.72,106.34) --
	cycle;

\path[fill=fillColor,fill opacity=0.50] (115.19,117.24) --
	(117.11,113.91) --
	(113.27,113.91) --
	cycle;

\path[fill=fillColor,fill opacity=0.50] (115.32,120.89) --
	(117.24,117.57) --
	(113.39,117.57) --
	cycle;

\path[fill=fillColor,fill opacity=0.50] (115.86,109.67) --
	(117.79,106.34) --
	(113.94,106.34) --
	cycle;

\path[fill=fillColor,fill opacity=0.50] (115.38,110.60) --
	(117.30,107.27) --
	(113.46,107.27) --
	cycle;

\path[fill=fillColor,fill opacity=0.50] (116.10,111.04) --
	(118.02,107.71) --
	(114.18,107.71) --
	cycle;

\path[fill=fillColor,fill opacity=0.50] (114.87,109.67) --
	(116.79,106.34) --
	(112.95,106.34) --
	cycle;

\path[fill=fillColor,fill opacity=0.50] (116.42,109.67) --
	(118.34,106.34) --
	(114.49,106.34) --
	cycle;

\path[fill=fillColor,fill opacity=0.50] (115.77,110.60) --
	(117.70,107.27) --
	(113.85,107.27) --
	cycle;

\path[fill=fillColor,fill opacity=0.50] (115.35,110.60) --
	(117.27,107.27) --
	(113.43,107.27) --
	cycle;

\path[fill=fillColor,fill opacity=0.50] (115.74,111.46) --
	(117.67,108.14) --
	(113.82,108.14) --
	cycle;

\path[fill=fillColor,fill opacity=0.50] (118.56,111.87) --
	(120.48,108.54) --
	(116.64,108.54) --
	cycle;

\path[fill=fillColor,fill opacity=0.50] (116.75,111.46) --
	(118.67,108.14) --
	(114.83,108.14) --
	cycle;

\path[fill=fillColor,fill opacity=0.50] (115.38,111.04) --
	(117.30,107.71) --
	(113.46,107.71) --
	cycle;

\path[fill=fillColor,fill opacity=0.50] (188.40,110.15) --
	(190.32,106.82) --
	(186.48,106.82) --
	cycle;

\path[fill=fillColor,fill opacity=0.50] (116.22,119.32) --
	(118.14,115.99) --
	(114.29,115.99) --
	cycle;

\path[fill=fillColor,fill opacity=0.50] (188.40,110.15) --
	(190.32,106.82) --
	(186.48,106.82) --
	cycle;

\path[fill=fillColor,fill opacity=0.50] (115.92,111.46) --
	(117.84,108.14) --
	(114.00,108.14) --
	cycle;

\path[fill=fillColor,fill opacity=0.50] (188.40,109.67) --
	(190.32,106.34) --
	(186.48,106.34) --
	cycle;

\path[fill=fillColor,fill opacity=0.50] (117.58,110.60) --
	(119.50,107.27) --
	(115.65,107.27) --
	cycle;

\path[fill=fillColor,fill opacity=0.50] (116.89,112.27) --
	(118.81,108.94) --
	(114.97,108.94) --
	cycle;

\path[fill=fillColor,fill opacity=0.50] (117.37,111.46) --
	(119.29,108.14) --
	(115.45,108.14) --
	cycle;

\path[fill=fillColor,fill opacity=0.50] (115.74,112.27) --
	(117.67,108.94) --
	(113.82,108.94) --
	cycle;

\path[fill=fillColor,fill opacity=0.50] (116.89,112.65) --
	(118.81,109.32) --
	(114.97,109.32) --
	cycle;

\path[fill=fillColor,fill opacity=0.50] (117.18,111.04) --
	(119.11,107.71) --
	(115.26,107.71) --
	cycle;

\path[fill=fillColor,fill opacity=0.50] (119.20,113.02) --
	(121.12,109.69) --
	(117.28,109.69) --
	cycle;

\path[fill=fillColor,fill opacity=0.50] (115.77,110.60) --
	(117.70,107.27) --
	(113.85,107.27) --
	cycle;

\path[fill=fillColor,fill opacity=0.50] (116.22,114.06) --
	(118.14,110.73) --
	(114.29,110.73) --
	cycle;

\path[fill=fillColor,fill opacity=0.50] (188.40,111.46) --
	(190.32,108.14) --
	(186.48,108.14) --
	cycle;

\path[fill=fillColor,fill opacity=0.50] (118.56,111.46) --
	(120.48,108.14) --
	(116.64,108.14) --
	cycle;

\path[fill=fillColor,fill opacity=0.50] (116.44,113.38) --
	(118.37,110.05) --
	(114.52,110.05) --
	cycle;

\path[fill=fillColor,fill opacity=0.50] (117.16,113.72) --
	(119.08,110.39) --
	(115.24,110.39) --
	cycle;

\path[fill=fillColor,fill opacity=0.50] (123.91,122.77) --
	(125.83,119.44) --
	(121.99,119.44) --
	cycle;

\path[fill=fillColor,fill opacity=0.50] (117.52,111.87) --
	(119.45,108.54) --
	(115.60,108.54) --
	cycle;

\path[fill=fillColor,fill opacity=0.50] (115.10,112.27) --
	(117.02,108.94) --
	(113.17,108.94) --
	cycle;

\path[fill=fillColor,fill opacity=0.50] (116.50,113.38) --
	(118.42,110.05) --
	(114.58,110.05) --
	cycle;

\path[fill=fillColor,fill opacity=0.50] (116.70,115.02) --
	(118.62,111.69) --
	(114.77,111.69) --
	cycle;

\path[fill=fillColor,fill opacity=0.50] (118.03,112.65) --
	(119.95,109.32) --
	(116.11,109.32) --
	cycle;

\path[fill=fillColor,fill opacity=0.50] (116.56,112.27) --
	(118.48,108.94) --
	(114.64,108.94) --
	cycle;

\path[fill=fillColor,fill opacity=0.50] (188.40,117.24) --
	(190.32,113.91) --
	(186.48,113.91) --
	cycle;

\path[fill=fillColor,fill opacity=0.50] (188.40,112.27) --
	(190.32,108.94) --
	(186.48,108.94) --
	cycle;

\path[fill=fillColor,fill opacity=0.50] (116.53,113.38) --
	(118.45,110.05) --
	(114.61,110.05) --
	cycle;

\path[fill=fillColor,fill opacity=0.50] (117.58,113.72) --
	(119.50,110.39) --
	(115.65,110.39) --
	cycle;

\path[fill=fillColor,fill opacity=0.50] (118.91,114.06) --
	(120.83,110.73) --
	(116.99,110.73) --
	cycle;

\path[fill=fillColor,fill opacity=0.50] (117.73,116.72) --
	(119.65,113.39) --
	(115.81,113.39) --
	cycle;

\path[fill=fillColor,fill opacity=0.50] (117.39,114.39) --
	(119.32,111.06) --
	(115.47,111.06) --
	cycle;

\path[fill=fillColor,fill opacity=0.50] (188.40,115.90) --
	(190.32,112.57) --
	(186.48,112.57) --
	cycle;

\path[fill=fillColor,fill opacity=0.50] (117.68,112.27) --
	(119.60,108.94) --
	(115.76,108.94) --
	cycle;

\path[fill=fillColor,fill opacity=0.50] (117.98,114.39) --
	(119.90,111.06) --
	(116.06,111.06) --
	cycle;

\path[fill=fillColor,fill opacity=0.50] (118.51,113.72) --
	(120.43,110.39) --
	(116.59,110.39) --
	cycle;

\path[fill=fillColor,fill opacity=0.50] (123.20,114.39) --
	(125.12,111.06) --
	(121.28,111.06) --
	cycle;

\path[fill=fillColor,fill opacity=0.50] (117.78,115.90) --
	(119.70,112.57) --
	(115.86,112.57) --
	cycle;

\path[fill=fillColor,fill opacity=0.50] (118.32,113.72) --
	(120.24,110.39) --
	(116.40,110.39) --
	cycle;

\path[fill=fillColor,fill opacity=0.50] (118.15,117.24) --
	(120.07,113.91) --
	(116.23,113.91) --
	cycle;

\path[fill=fillColor,fill opacity=0.50] (119.96,115.90) --
	(121.88,112.57) --
	(118.04,112.57) --
	cycle;

\path[fill=fillColor,fill opacity=0.50] (119.38,120.89) --
	(121.30,117.57) --
	(117.46,117.57) --
	cycle;

\path[fill=fillColor,fill opacity=0.50] (120.12,114.39) --
	(122.05,111.06) --
	(118.20,111.06) --
	cycle;

\path[fill=fillColor,fill opacity=0.50] (122.83,134.56) --
	(124.76,131.23) --
	(120.91,131.23) --
	cycle;

\path[fill=fillColor,fill opacity=0.50] (123.68,118.88) --
	(125.61,115.56) --
	(121.76,115.56) --
	cycle;

\path[fill=fillColor,fill opacity=0.50] (123.61,148.19) --
	(125.53,144.87) --
	(121.69,144.87) --
	cycle;

\path[fill=fillColor,fill opacity=0.50] (188.40,116.18) --
	(190.32,112.85) --
	(186.48,112.85) --
	cycle;

\path[fill=fillColor,fill opacity=0.50] (122.44,125.91) --
	(124.36,122.59) --
	(120.51,122.59) --
	cycle;

\path[fill=fillColor,fill opacity=0.50] (188.40,116.18) --
	(190.32,112.85) --
	(186.48,112.85) --
	cycle;

\path[fill=fillColor,fill opacity=0.50] (130.89,118.88) --
	(132.82,115.56) --
	(128.97,115.56) --
	cycle;

\path[fill=fillColor,fill opacity=0.50] (122.77,120.52) --
	(124.69,117.19) --
	(120.85,117.19) --
	cycle;

\path[fill=fillColor,fill opacity=0.50] (137.87,118.21) --
	(139.79,114.88) --
	(135.95,114.88) --
	cycle;

\path[fill=fillColor,fill opacity=0.50] (130.41,124.91) --
	(132.33,121.58) --
	(128.49,121.58) --
	cycle;

\path[fill=fillColor,fill opacity=0.50] (188.40,116.18) --
	(190.32,112.85) --
	(186.48,112.85) --
	cycle;

\path[fill=fillColor,fill opacity=0.50] (125.92,124.38) --
	(127.84,121.05) --
	(123.99,121.05) --
	cycle;

\path[fill=fillColor,fill opacity=0.50] (188.40,116.72) --
	(190.32,113.39) --
	(186.48,113.39) --
	cycle;

\path[fill=fillColor,fill opacity=0.50] (126.92,133.23) --
	(128.84,129.90) --
	(125.00,129.90) --
	cycle;

\path[fill=fillColor,fill opacity=0.50] (132.49,129.32) --
	(134.41,125.99) --
	(130.57,125.99) --
	cycle;

\path[fill=fillColor,fill opacity=0.50] (122.88,124.10) --
	(124.80,120.77) --
	(120.96,120.77) --
	cycle;

\path[fill=fillColor,fill opacity=0.50] (123.62,140.29) --
	(125.54,136.96) --
	(121.70,136.96) --
	cycle;

\path[fill=fillColor,fill opacity=0.50] (119.49,122.28) --
	(121.41,118.96) --
	(117.57,118.96) --
	cycle;

\path[fill=fillColor,fill opacity=0.50] (121.24,156.19) --
	(123.17,152.87) --
	(119.32,152.87) --
	cycle;

\path[fill=fillColor,fill opacity=0.50] (126.53,126.03) --
	(128.45,122.70) --
	(124.60,122.70) --
	cycle;

\path[fill=fillColor,fill opacity=0.50] (123.53,122.77) --
	(125.45,119.44) --
	(121.61,119.44) --
	cycle;

\path[fill=fillColor,fill opacity=0.50] (158.79,154.57) --
	(160.71,151.24) --
	(156.87,151.24) --
	cycle;

\path[fill=fillColor,fill opacity=0.50] (125.22,149.45) --
	(127.14,146.12) --
	(123.30,146.12) --
	cycle;

\path[fill=fillColor,fill opacity=0.50] (144.70,160.10) --
	(146.62,156.78) --
	(142.78,156.78) --
	cycle;

\path[fill=fillColor,fill opacity=0.50] (146.74,128.86) --
	(148.66,125.53) --
	(144.82,125.53) --
	cycle;

\path[fill=fillColor,fill opacity=0.50] (128.09,139.61) --
	(130.01,136.28) --
	(126.17,136.28) --
	cycle;

\path[fill=fillColor,fill opacity=0.50] (128.88,127.05) --
	(130.81,123.73) --
	(126.96,123.73) --
	cycle;

\path[fill=fillColor,fill opacity=0.50] (130.32,125.55) --
	(132.24,122.22) --
	(128.40,122.22) --
	cycle;

\path[fill=fillColor,fill opacity=0.50] (129.18,135.95) --
	(131.10,132.63) --
	(127.26,132.63) --
	cycle;

\path[fill=fillColor,fill opacity=0.50] (138.24,149.11) --
	(140.16,145.78) --
	(136.31,145.78) --
	cycle;

\path[fill=fillColor,fill opacity=0.50] (149.80,167.74) --
	(151.72,164.41) --
	(147.88,164.41) --
	cycle;

\path[fill=fillColor,fill opacity=0.50] (188.40,135.33) --
	(190.32,132.00) --
	(186.48,132.00) --
	cycle;

\path[fill=fillColor,fill opacity=0.50] (144.01,131.66) --
	(145.93,128.33) --
	(142.08,128.33) --
	cycle;

\path[fill=fillColor,fill opacity=0.50] (165.99,164.33) --
	(167.91,161.00) --
	(164.07,161.00) --
	cycle;

\path[fill=fillColor,fill opacity=0.50] (158.90,177.24) --
	(160.82,173.91) --
	(156.97,173.91) --
	cycle;

\path[fill=fillColor,fill opacity=0.50] (141.76,141.61) --
	(143.68,138.28) --
	(139.84,138.28) --
	cycle;

\path[fill=fillColor,fill opacity=0.50] (160.68,170.28) --
	(162.60,166.96) --
	(158.76,166.96) --
	cycle;

\path[fill=fillColor,fill opacity=0.50] (134.66,148.71) --
	(136.58,145.38) --
	(132.74,145.38) --
	cycle;

\path[fill=fillColor,fill opacity=0.50] (150.32,158.14) --
	(152.24,154.81) --
	(148.40,154.81) --
	cycle;

\path[fill=fillColor,fill opacity=0.50] (177.66,199.36) --
	(179.58,196.03) --
	(175.74,196.03) --
	cycle;

\path[fill=fillColor,fill opacity=0.50] (188.40,158.68) --
	(190.32,155.35) --
	(186.48,155.35) --
	cycle;

\path[fill=fillColor,fill opacity=0.50] (164.83,160.64) --
	(166.75,157.31) --
	(162.91,157.31) --
	cycle;

\path[fill=fillColor,fill opacity=0.50] (135.71,134.56) --
	(137.63,131.23) --
	(133.79,131.23) --
	cycle;

\path[fill=fillColor,fill opacity=0.50] (165.81,183.14) --
	(167.73,179.81) --
	(163.89,179.81) --
	cycle;

\path[fill=fillColor,fill opacity=0.50] (188.40,197.84) --
	(190.32,194.52) --
	(186.48,194.52) --
	cycle;

\path[fill=fillColor,fill opacity=0.50] (153.14,144.41) --
	(155.06,141.08) --
	(151.22,141.08) --
	cycle;

\path[fill=fillColor,fill opacity=0.50] (156.96,149.53) --
	(158.88,146.20) --
	(155.04,146.20) --
	cycle;

\path[fill=fillColor,fill opacity=0.50] (167.23,214.66) --
	(169.15,211.33) --
	(165.31,211.33) --
	cycle;

\path[fill=fillColor,fill opacity=0.50] (169.89,166.98) --
	(171.81,163.65) --
	(167.97,163.65) --
	cycle;

\path[fill=fillColor,fill opacity=0.50] (165.18,162.89) --
	(167.10,159.56) --
	(163.26,159.56) --
	cycle;

\path[fill=fillColor,fill opacity=0.50] (179.51,161.38) --
	(181.43,158.05) --
	(177.59,158.05) --
	cycle;

\path[fill=fillColor,fill opacity=0.50] (188.40,214.66) --
	(190.32,211.33) --
	(186.48,211.33) --
	cycle;

\path[fill=fillColor,fill opacity=0.50] (188.40,186.67) --
	(190.32,183.35) --
	(186.48,183.35) --
	cycle;

\path[fill=fillColor,fill opacity=0.50] (153.59,193.84) --
	(155.51,190.51) --
	(151.67,190.51) --
	cycle;

\path[fill=fillColor,fill opacity=0.50] (188.40,155.63) --
	(190.32,152.30) --
	(186.48,152.30) --
	cycle;

\path[fill=fillColor,fill opacity=0.50] (188.40,158.42) --
	(190.32,155.09) --
	(186.48,155.09) --
	cycle;

\path[fill=fillColor,fill opacity=0.50] (169.82,192.42) --
	(171.74,189.09) --
	(167.90,189.09) --
	cycle;

\path[fill=fillColor,fill opacity=0.50] (170.28,180.15) --
	(172.20,176.82) --
	(168.36,176.82) --
	cycle;

\path[fill=fillColor,fill opacity=0.50] (188.40,214.66) --
	(190.32,211.33) --
	(186.48,211.33) --
	cycle;

\path[fill=fillColor,fill opacity=0.50] (188.40,198.28) --
	(190.32,194.95) --
	(186.48,194.95) --
	cycle;

\path[fill=fillColor,fill opacity=0.50] (188.40,214.66) --
	(190.32,211.33) --
	(186.48,211.33) --
	cycle;

\path[fill=fillColor,fill opacity=0.50] (188.40,188.41) --
	(190.32,185.08) --
	(186.48,185.08) --
	cycle;

\path[fill=fillColor,fill opacity=0.50] (166.10,214.66) --
	(168.02,211.33) --
	(164.17,211.33) --
	cycle;

\path[fill=fillColor,fill opacity=0.50] (188.40,214.66) --
	(190.32,211.33) --
	(186.48,211.33) --
	cycle;

\path[fill=fillColor,fill opacity=0.50] (188.40,205.26) --
	(190.32,201.93) --
	(186.48,201.93) --
	cycle;

\path[fill=fillColor,fill opacity=0.50] (188.40,214.66) --
	(190.32,211.33) --
	(186.48,211.33) --
	cycle;

\path[fill=fillColor,fill opacity=0.50] (188.40,196.18) --
	(190.32,192.85) --
	(186.48,192.85) --
	cycle;

\path[fill=fillColor,fill opacity=0.50] (167.43,214.66) --
	(169.35,211.33) --
	(165.51,211.33) --
	cycle;

\path[fill=fillColor,fill opacity=0.50] (188.40,200.75) --
	(190.32,197.42) --
	(186.48,197.42) --
	cycle;

\path[fill=fillColor,fill opacity=0.50] (171.10,214.66) --
	(173.02,211.33) --
	(169.17,211.33) --
	cycle;

\path[fill=fillColor,fill opacity=0.50] (188.40,175.72) --
	(190.32,172.40) --
	(186.48,172.40) --
	cycle;

\path[fill=fillColor,fill opacity=0.50] (167.42,214.66) --
	(169.34,211.33) --
	(165.50,211.33) --
	cycle;

\path[fill=fillColor,fill opacity=0.50] (188.40,214.66) --
	(190.32,211.33) --
	(186.48,211.33) --
	cycle;

\path[draw=drawColor,draw opacity=0.50,line width= 0.4pt,line join=round,line cap=round] (117.11,111.06) rectangle (119.96,113.91);

\path[draw=drawColor,draw opacity=0.50,line width= 0.4pt,line join=round,line cap=round] (117.11,111.06) -- (119.96,113.91);

\path[draw=drawColor,draw opacity=0.50,line width= 0.4pt,line join=round,line cap=round] (117.11,113.91) -- (119.96,111.06);

\path[draw=drawColor,draw opacity=0.50,line width= 0.4pt,line join=round,line cap=round] (119.47,112.25) rectangle (122.33,115.11);

\path[draw=drawColor,draw opacity=0.50,line width= 0.4pt,line join=round,line cap=round] (119.47,112.25) -- (122.33,115.11);

\path[draw=drawColor,draw opacity=0.50,line width= 0.4pt,line join=round,line cap=round] (119.47,115.11) -- (122.33,112.25);

\path[draw=drawColor,draw opacity=0.50,line width= 0.4pt,line join=round,line cap=round] (186.97,167.88) rectangle (189.82,170.73);

\path[draw=drawColor,draw opacity=0.50,line width= 0.4pt,line join=round,line cap=round] (186.97,167.88) -- (189.82,170.73);

\path[draw=drawColor,draw opacity=0.50,line width= 0.4pt,line join=round,line cap=round] (186.97,170.73) -- (189.82,167.88);

\path[draw=drawColor,draw opacity=0.50,line width= 0.4pt,line join=round,line cap=round] (123.23,135.93) rectangle (126.09,138.78);

\path[draw=drawColor,draw opacity=0.50,line width= 0.4pt,line join=round,line cap=round] (123.23,135.93) -- (126.09,138.78);

\path[draw=drawColor,draw opacity=0.50,line width= 0.4pt,line join=round,line cap=round] (123.23,138.78) -- (126.09,135.93);

\path[draw=drawColor,draw opacity=0.50,line width= 0.4pt,line join=round,line cap=round] (130.88,178.59) rectangle (133.73,181.44);

\path[draw=drawColor,draw opacity=0.50,line width= 0.4pt,line join=round,line cap=round] (130.88,178.59) -- (133.73,181.44);

\path[draw=drawColor,draw opacity=0.50,line width= 0.4pt,line join=round,line cap=round] (130.88,181.44) -- (133.73,178.59);

\path[draw=drawColor,draw opacity=0.50,line width= 0.4pt,line join=round,line cap=round] (186.97,132.00) rectangle (189.82,134.85);

\path[draw=drawColor,draw opacity=0.50,line width= 0.4pt,line join=round,line cap=round] (186.97,132.00) -- (189.82,134.85);

\path[draw=drawColor,draw opacity=0.50,line width= 0.4pt,line join=round,line cap=round] (186.97,134.85) -- (189.82,132.00);

\path[draw=drawColor,draw opacity=0.50,line width= 0.4pt,line join=round,line cap=round] (186.97,139.34) rectangle (189.82,142.19);

\path[draw=drawColor,draw opacity=0.50,line width= 0.4pt,line join=round,line cap=round] (186.97,139.34) -- (189.82,142.19);

\path[draw=drawColor,draw opacity=0.50,line width= 0.4pt,line join=round,line cap=round] (186.97,142.19) -- (189.82,139.34);
\definecolor{drawColor}{RGB}{152,152,152}

\path[draw=drawColor,line width= 0.6pt,line join=round] (-116.78,-92.74) -- (351.63,375.68);
\definecolor{drawColor}{gray}{0.70}

\path[draw=drawColor,line width= 0.6pt,line join=round,line cap=round] ( 40.51, 63.40) -- ( 40.51, 66.24);

\path[draw=drawColor,line width= 0.6pt,line join=round,line cap=round] ( 42.31, 63.40) -- ( 42.31, 66.24);

\path[draw=drawColor,line width= 0.6pt,line join=round,line cap=round] ( 43.86, 63.40) -- ( 43.86, 66.24);

\path[draw=drawColor,line width= 0.6pt,line join=round,line cap=round] ( 45.23, 63.40) -- ( 45.23, 66.24);

\path[draw=drawColor,line width= 0.6pt,line join=round,line cap=round] ( 46.45, 63.40) -- ( 46.45, 71.93);

\path[draw=drawColor,line width= 0.6pt,line join=round,line cap=round] ( 54.51, 63.40) -- ( 54.51, 66.24);

\path[draw=drawColor,line width= 0.6pt,line join=round,line cap=round] ( 59.23, 63.40) -- ( 59.23, 66.24);

\path[draw=drawColor,line width= 0.6pt,line join=round,line cap=round] ( 62.58, 63.40) -- ( 62.58, 66.24);

\path[draw=drawColor,line width= 0.6pt,line join=round,line cap=round] ( 65.17, 63.40) -- ( 65.17, 69.09);

\path[draw=drawColor,line width= 0.6pt,line join=round,line cap=round] ( 67.29, 63.40) -- ( 67.29, 66.24);

\path[draw=drawColor,line width= 0.6pt,line join=round,line cap=round] ( 69.08, 63.40) -- ( 69.08, 66.24);

\path[draw=drawColor,line width= 0.6pt,line join=round,line cap=round] ( 70.64, 63.40) -- ( 70.64, 66.24);

\path[draw=drawColor,line width= 0.6pt,line join=round,line cap=round] ( 72.01, 63.40) -- ( 72.01, 66.24);

\path[draw=drawColor,line width= 0.6pt,line join=round,line cap=round] ( 73.23, 63.40) -- ( 73.23, 71.93);

\path[draw=drawColor,line width= 0.6pt,line join=round,line cap=round] ( 81.29, 63.40) -- ( 81.29, 66.24);

\path[draw=drawColor,line width= 0.6pt,line join=round,line cap=round] ( 86.01, 63.40) -- ( 86.01, 66.24);

\path[draw=drawColor,line width= 0.6pt,line join=round,line cap=round] ( 89.35, 63.40) -- ( 89.35, 66.24);

\path[draw=drawColor,line width= 0.6pt,line join=round,line cap=round] ( 91.95, 63.40) -- ( 91.95, 69.09);

\path[draw=drawColor,line width= 0.6pt,line join=round,line cap=round] ( 94.07, 63.40) -- ( 94.07, 66.24);

\path[draw=drawColor,line width= 0.6pt,line join=round,line cap=round] ( 95.86, 63.40) -- ( 95.86, 66.24);

\path[draw=drawColor,line width= 0.6pt,line join=round,line cap=round] ( 97.41, 63.40) -- ( 97.41, 66.24);

\path[draw=drawColor,line width= 0.6pt,line join=round,line cap=round] ( 98.78, 63.40) -- ( 98.78, 66.24);

\path[draw=drawColor,line width= 0.6pt,line join=round,line cap=round] (100.01, 63.40) -- (100.01, 71.93);

\path[draw=drawColor,line width= 0.6pt,line join=round,line cap=round] (108.07, 63.40) -- (108.07, 66.24);

\path[draw=drawColor,line width= 0.6pt,line join=round,line cap=round] (112.78, 63.40) -- (112.78, 66.24);

\path[draw=drawColor,line width= 0.6pt,line join=round,line cap=round] (116.13, 63.40) -- (116.13, 66.24);

\path[draw=drawColor,line width= 0.6pt,line join=round,line cap=round] (118.72, 63.40) -- (118.72, 69.09);

\path[draw=drawColor,line width= 0.6pt,line join=round,line cap=round] (120.84, 63.40) -- (120.84, 66.24);

\path[draw=drawColor,line width= 0.6pt,line join=round,line cap=round] (122.64, 63.40) -- (122.64, 66.24);

\path[draw=drawColor,line width= 0.6pt,line join=round,line cap=round] (124.19, 63.40) -- (124.19, 66.24);

\path[draw=drawColor,line width= 0.6pt,line join=round,line cap=round] (125.56, 63.40) -- (125.56, 66.24);

\path[draw=drawColor,line width= 0.6pt,line join=round,line cap=round] (126.78, 63.40) -- (126.78, 71.93);

\path[draw=drawColor,line width= 0.6pt,line join=round,line cap=round] (134.84, 63.40) -- (134.84, 66.24);

\path[draw=drawColor,line width= 0.6pt,line join=round,line cap=round] (139.56, 63.40) -- (139.56, 66.24);

\path[draw=drawColor,line width= 0.6pt,line join=round,line cap=round] (142.91, 63.40) -- (142.91, 66.24);

\path[draw=drawColor,line width= 0.6pt,line join=round,line cap=round] (145.50, 63.40) -- (145.50, 69.09);

\path[draw=drawColor,line width= 0.6pt,line join=round,line cap=round] (147.62, 63.40) -- (147.62, 66.24);

\path[draw=drawColor,line width= 0.6pt,line join=round,line cap=round] (149.41, 63.40) -- (149.41, 66.24);

\path[draw=drawColor,line width= 0.6pt,line join=round,line cap=round] (150.97, 63.40) -- (150.97, 66.24);

\path[draw=drawColor,line width= 0.6pt,line join=round,line cap=round] (152.34, 63.40) -- (152.34, 66.24);

\path[draw=drawColor,line width= 0.6pt,line join=round,line cap=round] (153.56, 63.40) -- (153.56, 71.93);

\path[draw=drawColor,line width= 0.6pt,line join=round,line cap=round] (161.62, 63.40) -- (161.62, 66.24);

\path[draw=drawColor,line width= 0.6pt,line join=round,line cap=round] (166.34, 63.40) -- (166.34, 66.24);

\path[draw=drawColor,line width= 0.6pt,line join=round,line cap=round] (169.68, 63.40) -- (169.68, 66.24);

\path[draw=drawColor,line width= 0.6pt,line join=round,line cap=round] (172.28, 63.40) -- (172.28, 69.09);

\path[draw=drawColor,line width= 0.6pt,line join=round,line cap=round] (174.40, 63.40) -- (174.40, 66.24);

\path[draw=drawColor,line width= 0.6pt,line join=round,line cap=round] (176.19, 63.40) -- (176.19, 66.24);

\path[draw=drawColor,line width= 0.6pt,line join=round,line cap=round] (177.74, 63.40) -- (177.74, 66.24);

\path[draw=drawColor,line width= 0.6pt,line join=round,line cap=round] (179.11, 63.40) -- (179.11, 66.24);

\path[draw=drawColor,line width= 0.6pt,line join=round,line cap=round] (180.34, 63.40) -- (180.34, 71.93);

\path[draw=drawColor,line width= 0.6pt,line join=round,line cap=round] (188.40, 63.40) -- (188.40, 66.24);

\path[draw=drawColor,line width= 0.6pt,line join=round,line cap=round] (193.11, 63.40) -- (193.11, 66.24);

\path[draw=drawColor,line width= 0.6pt,line join=round,line cap=round] ( 39.36, 64.56) -- ( 42.20, 64.56);

\path[draw=drawColor,line width= 0.6pt,line join=round,line cap=round] ( 39.36, 66.35) -- ( 42.20, 66.35);

\path[draw=drawColor,line width= 0.6pt,line join=round,line cap=round] ( 39.36, 67.90) -- ( 42.20, 67.90);

\path[draw=drawColor,line width= 0.6pt,line join=round,line cap=round] ( 39.36, 69.27) -- ( 42.20, 69.27);

\path[draw=drawColor,line width= 0.6pt,line join=round,line cap=round] ( 39.36, 70.50) -- ( 47.89, 70.50);

\path[draw=drawColor,line width= 0.6pt,line join=round,line cap=round] ( 39.36, 78.56) -- ( 42.20, 78.56);

\path[draw=drawColor,line width= 0.6pt,line join=round,line cap=round] ( 39.36, 83.27) -- ( 42.20, 83.27);

\path[draw=drawColor,line width= 0.6pt,line join=round,line cap=round] ( 39.36, 86.62) -- ( 42.20, 86.62);

\path[draw=drawColor,line width= 0.6pt,line join=round,line cap=round] ( 39.36, 89.21) -- ( 45.05, 89.21);

\path[draw=drawColor,line width= 0.6pt,line join=round,line cap=round] ( 39.36, 91.33) -- ( 42.20, 91.33);

\path[draw=drawColor,line width= 0.6pt,line join=round,line cap=round] ( 39.36, 93.12) -- ( 42.20, 93.12);

\path[draw=drawColor,line width= 0.6pt,line join=round,line cap=round] ( 39.36, 94.68) -- ( 42.20, 94.68);

\path[draw=drawColor,line width= 0.6pt,line join=round,line cap=round] ( 39.36, 96.05) -- ( 42.20, 96.05);

\path[draw=drawColor,line width= 0.6pt,line join=round,line cap=round] ( 39.36, 97.27) -- ( 47.89, 97.27);

\path[draw=drawColor,line width= 0.6pt,line join=round,line cap=round] ( 39.36,105.33) -- ( 42.20,105.33);

\path[draw=drawColor,line width= 0.6pt,line join=round,line cap=round] ( 39.36,110.05) -- ( 42.20,110.05);

\path[draw=drawColor,line width= 0.6pt,line join=round,line cap=round] ( 39.36,113.39) -- ( 42.20,113.39);

\path[draw=drawColor,line width= 0.6pt,line join=round,line cap=round] ( 39.36,115.99) -- ( 45.05,115.99);

\path[draw=drawColor,line width= 0.6pt,line join=round,line cap=round] ( 39.36,118.11) -- ( 42.20,118.11);

\path[draw=drawColor,line width= 0.6pt,line join=round,line cap=round] ( 39.36,119.90) -- ( 42.20,119.90);

\path[draw=drawColor,line width= 0.6pt,line join=round,line cap=round] ( 39.36,121.45) -- ( 42.20,121.45);

\path[draw=drawColor,line width= 0.6pt,line join=round,line cap=round] ( 39.36,122.82) -- ( 42.20,122.82);

\path[draw=drawColor,line width= 0.6pt,line join=round,line cap=round] ( 39.36,124.05) -- ( 47.89,124.05);

\path[draw=drawColor,line width= 0.6pt,line join=round,line cap=round] ( 39.36,132.11) -- ( 42.20,132.11);

\path[draw=drawColor,line width= 0.6pt,line join=round,line cap=round] ( 39.36,136.82) -- ( 42.20,136.82);

\path[draw=drawColor,line width= 0.6pt,line join=round,line cap=round] ( 39.36,140.17) -- ( 42.20,140.17);

\path[draw=drawColor,line width= 0.6pt,line join=round,line cap=round] ( 39.36,142.77) -- ( 45.05,142.77);

\path[draw=drawColor,line width= 0.6pt,line join=round,line cap=round] ( 39.36,144.89) -- ( 42.20,144.89);

\path[draw=drawColor,line width= 0.6pt,line join=round,line cap=round] ( 39.36,146.68) -- ( 42.20,146.68);

\path[draw=drawColor,line width= 0.6pt,line join=round,line cap=round] ( 39.36,148.23) -- ( 42.20,148.23);

\path[draw=drawColor,line width= 0.6pt,line join=round,line cap=round] ( 39.36,149.60) -- ( 42.20,149.60);

\path[draw=drawColor,line width= 0.6pt,line join=round,line cap=round] ( 39.36,150.83) -- ( 47.89,150.83);

\path[draw=drawColor,line width= 0.6pt,line join=round,line cap=round] ( 39.36,158.89) -- ( 42.20,158.89);

\path[draw=drawColor,line width= 0.6pt,line join=round,line cap=round] ( 39.36,163.60) -- ( 42.20,163.60);

\path[draw=drawColor,line width= 0.6pt,line join=round,line cap=round] ( 39.36,166.95) -- ( 42.20,166.95);

\path[draw=drawColor,line width= 0.6pt,line join=round,line cap=round] ( 39.36,169.54) -- ( 45.05,169.54);

\path[draw=drawColor,line width= 0.6pt,line join=round,line cap=round] ( 39.36,171.66) -- ( 42.20,171.66);

\path[draw=drawColor,line width= 0.6pt,line join=round,line cap=round] ( 39.36,173.45) -- ( 42.20,173.45);

\path[draw=drawColor,line width= 0.6pt,line join=round,line cap=round] ( 39.36,175.01) -- ( 42.20,175.01);

\path[draw=drawColor,line width= 0.6pt,line join=round,line cap=round] ( 39.36,176.38) -- ( 42.20,176.38);

\path[draw=drawColor,line width= 0.6pt,line join=round,line cap=round] ( 39.36,177.60) -- ( 47.89,177.60);

\path[draw=drawColor,line width= 0.6pt,line join=round,line cap=round] ( 39.36,185.66) -- ( 42.20,185.66);

\path[draw=drawColor,line width= 0.6pt,line join=round,line cap=round] ( 39.36,190.38) -- ( 42.20,190.38);

\path[draw=drawColor,line width= 0.6pt,line join=round,line cap=round] ( 39.36,193.72) -- ( 42.20,193.72);

\path[draw=drawColor,line width= 0.6pt,line join=round,line cap=round] ( 39.36,196.32) -- ( 45.05,196.32);

\path[draw=drawColor,line width= 0.6pt,line join=round,line cap=round] ( 39.36,198.44) -- ( 42.20,198.44);

\path[draw=drawColor,line width= 0.6pt,line join=round,line cap=round] ( 39.36,200.23) -- ( 42.20,200.23);

\path[draw=drawColor,line width= 0.6pt,line join=round,line cap=round] ( 39.36,201.78) -- ( 42.20,201.78);

\path[draw=drawColor,line width= 0.6pt,line join=round,line cap=round] ( 39.36,203.15) -- ( 42.20,203.15);

\path[draw=drawColor,line width= 0.6pt,line join=round,line cap=round] ( 39.36,204.38) -- ( 47.89,204.38);

\path[draw=drawColor,line width= 0.6pt,line join=round,line cap=round] ( 39.36,212.44) -- ( 42.20,212.44);

\path[draw=drawColor,line width= 0.6pt,line join=round,line cap=round] ( 39.36,217.15) -- ( 42.20,217.15);

\path[draw=drawColor,line width= 0.5pt,line join=round,line cap=round] ( 39.36, 63.40) rectangle (195.50,219.54);
\end{scope}
\begin{scope}
\path[clip] (  0.00,  0.00) rectangle (411.94,224.04);
\definecolor{drawColor}{gray}{0.30}

\node[text=drawColor,anchor=base east,inner sep=0pt, outer sep=0pt, scale=  0.72] at ( 35.31, 94.79) {0.1};

\node[text=drawColor,anchor=base east,inner sep=0pt, outer sep=0pt, scale=  0.72] at ( 35.31,148.35) {10};

\node[text=drawColor,anchor=base east,inner sep=0pt, outer sep=0pt, scale=  0.72] at ( 35.31,201.90) {1000};
\end{scope}
\begin{scope}
\path[clip] (  0.00,  0.00) rectangle (411.94,224.04);
\definecolor{drawColor}{gray}{0.70}

\path[draw=drawColor,line width= 0.2pt,line join=round] ( 37.11, 97.27) --
	( 39.36, 97.27);

\path[draw=drawColor,line width= 0.2pt,line join=round] ( 37.11,150.83) --
	( 39.36,150.83);

\path[draw=drawColor,line width= 0.2pt,line join=round] ( 37.11,204.38) --
	( 39.36,204.38);
\end{scope}
\begin{scope}
\path[clip] (  0.00,  0.00) rectangle (411.94,224.04);
\definecolor{drawColor}{gray}{0.70}

\path[draw=drawColor,line width= 0.2pt,line join=round] ( 73.23, 61.15) --
	( 73.23, 63.40);

\path[draw=drawColor,line width= 0.2pt,line join=round] (126.78, 61.15) --
	(126.78, 63.40);

\path[draw=drawColor,line width= 0.2pt,line join=round] (180.34, 61.15) --
	(180.34, 63.40);
\end{scope}
\begin{scope}
\path[clip] (  0.00,  0.00) rectangle (411.94,224.04);
\definecolor{drawColor}{gray}{0.30}

\node[text=drawColor,anchor=base,inner sep=0pt, outer sep=0pt, scale=  0.72] at ( 73.23, 54.39) {0.1};

\node[text=drawColor,anchor=base,inner sep=0pt, outer sep=0pt, scale=  0.72] at (126.78, 54.39) {10};

\node[text=drawColor,anchor=base,inner sep=0pt, outer sep=0pt, scale=  0.72] at (180.34, 54.39) {1000};
\end{scope}
\begin{scope}
\path[clip] (  0.00,  0.00) rectangle (411.94,224.04);
\definecolor{drawColor}{RGB}{0,0,0}

\node[text=drawColor,anchor=base,inner sep=0pt, outer sep=0pt, scale=  0.90] at (117.43, 44.40) {\textsc{Ace} + \texttt{cd06} time (s)};
\end{scope}
\begin{scope}
\path[clip] (  0.00,  0.00) rectangle (411.94,224.04);
\definecolor{drawColor}{RGB}{0,0,0}

\node[text=drawColor,rotate= 90.00,anchor=base,inner sep=0pt, outer sep=0pt, scale=  0.90] at ( 16.67,141.47) {\textsc{DPMC} + \texttt{bklm16++} time (s)};
\end{scope}
\begin{scope}
\path[clip] (211.94, 37.91) rectangle (405.96,224.04);
\definecolor{drawColor}{RGB}{255,255,255}
\definecolor{fillColor}{RGB}{255,255,255}

\path[draw=drawColor,line width= 0.5pt,line join=round,line cap=round,fill=fillColor] (211.94, 37.91) rectangle (405.96,224.04);
\end{scope}
\begin{scope}
\path[clip] (245.33, 63.40) rectangle (401.46,219.54);
\definecolor{fillColor}{RGB}{255,255,255}

\path[fill=fillColor] (245.33, 63.40) rectangle (401.46,219.54);
\definecolor{drawColor}{gray}{0.87}

\path[draw=drawColor,line width= 0.1pt,line join=round] (245.33, 70.50) --
	(401.46, 70.50);

\path[draw=drawColor,line width= 0.1pt,line join=round] (245.33,124.05) --
	(401.46,124.05);

\path[draw=drawColor,line width= 0.1pt,line join=round] (245.33,177.60) --
	(401.46,177.60);

\path[draw=drawColor,line width= 0.1pt,line join=round] (252.42, 63.40) --
	(252.42,219.54);

\path[draw=drawColor,line width= 0.1pt,line join=round] (305.98, 63.40) --
	(305.98,219.54);

\path[draw=drawColor,line width= 0.1pt,line join=round] (359.53, 63.40) --
	(359.53,219.54);

\path[draw=drawColor,line width= 0.2pt,line join=round] (245.33, 97.27) --
	(401.46, 97.27);

\path[draw=drawColor,line width= 0.2pt,line join=round] (245.33,150.83) --
	(401.46,150.83);

\path[draw=drawColor,line width= 0.2pt,line join=round] (245.33,204.38) --
	(401.46,204.38);

\path[draw=drawColor,line width= 0.2pt,line join=round] (279.20, 63.40) --
	(279.20,219.54);

\path[draw=drawColor,line width= 0.2pt,line join=round] (332.75, 63.40) --
	(332.75,219.54);

\path[draw=drawColor,line width= 0.2pt,line join=round] (386.31, 63.40) --
	(386.31,219.54);
\definecolor{drawColor}{RGB}{230,171,2}

\path[draw=drawColor,draw opacity=0.50,line width= 0.4pt,line join=round,line cap=round] (298.22,108.23) -- (301.07,111.08);

\path[draw=drawColor,draw opacity=0.50,line width= 0.4pt,line join=round,line cap=round] (298.22,111.08) -- (301.07,108.23);

\path[draw=drawColor,draw opacity=0.50,line width= 0.4pt,line join=round,line cap=round] (297.62,109.65) -- (301.66,109.65);

\path[draw=drawColor,draw opacity=0.50,line width= 0.4pt,line join=round,line cap=round] (299.64,107.64) -- (299.64,111.67);

\path[draw=drawColor,draw opacity=0.50,line width= 0.4pt,line join=round,line cap=round] (295.52,108.23) -- (298.37,111.08);

\path[draw=drawColor,draw opacity=0.50,line width= 0.4pt,line join=round,line cap=round] (295.52,111.08) -- (298.37,108.23);

\path[draw=drawColor,draw opacity=0.50,line width= 0.4pt,line join=round,line cap=round] (294.93,109.65) -- (298.96,109.65);

\path[draw=drawColor,draw opacity=0.50,line width= 0.4pt,line join=round,line cap=round] (296.95,107.64) -- (296.95,111.67);

\path[draw=drawColor,draw opacity=0.50,line width= 0.4pt,line join=round,line cap=round] (295.52,110.08) -- (298.37,112.93);

\path[draw=drawColor,draw opacity=0.50,line width= 0.4pt,line join=round,line cap=round] (295.52,112.93) -- (298.37,110.08);

\path[draw=drawColor,draw opacity=0.50,line width= 0.4pt,line join=round,line cap=round] (294.93,111.50) -- (298.96,111.50);

\path[draw=drawColor,draw opacity=0.50,line width= 0.4pt,line join=round,line cap=round] (296.95,109.49) -- (296.95,113.52);

\path[draw=drawColor,draw opacity=0.50,line width= 0.4pt,line join=round,line cap=round] (306.07,110.74) -- (308.93,113.59);

\path[draw=drawColor,draw opacity=0.50,line width= 0.4pt,line join=round,line cap=round] (306.07,113.59) -- (308.93,110.74);

\path[draw=drawColor,draw opacity=0.50,line width= 0.4pt,line join=round,line cap=round] (305.48,112.17) -- (309.52,112.17);

\path[draw=drawColor,draw opacity=0.50,line width= 0.4pt,line join=round,line cap=round] (307.50,110.15) -- (307.50,114.19);

\path[draw=drawColor,draw opacity=0.50,line width= 0.4pt,line join=round,line cap=round] (297.81,109.73) -- (300.66,112.58);

\path[draw=drawColor,draw opacity=0.50,line width= 0.4pt,line join=round,line cap=round] (297.81,112.58) -- (300.66,109.73);

\path[draw=drawColor,draw opacity=0.50,line width= 0.4pt,line join=round,line cap=round] (297.22,111.16) -- (301.25,111.16);

\path[draw=drawColor,draw opacity=0.50,line width= 0.4pt,line join=round,line cap=round] (299.23,109.14) -- (299.23,113.17);

\path[draw=drawColor,draw opacity=0.50,line width= 0.4pt,line join=round,line cap=round] (295.77,108.23) -- (298.62,111.08);

\path[draw=drawColor,draw opacity=0.50,line width= 0.4pt,line join=round,line cap=round] (295.77,111.08) -- (298.62,108.23);

\path[draw=drawColor,draw opacity=0.50,line width= 0.4pt,line join=round,line cap=round] (295.18,109.65) -- (299.21,109.65);

\path[draw=drawColor,draw opacity=0.50,line width= 0.4pt,line join=round,line cap=round] (297.20,107.64) -- (297.20,111.67);

\path[draw=drawColor,draw opacity=0.50,line width= 0.4pt,line join=round,line cap=round] (296.01,108.62) -- (298.87,111.47);

\path[draw=drawColor,draw opacity=0.50,line width= 0.4pt,line join=round,line cap=round] (296.01,111.47) -- (298.87,108.62);

\path[draw=drawColor,draw opacity=0.50,line width= 0.4pt,line join=round,line cap=round] (295.42,110.05) -- (299.46,110.05);

\path[draw=drawColor,draw opacity=0.50,line width= 0.4pt,line join=round,line cap=round] (297.44,108.03) -- (297.44,112.07);

\path[draw=drawColor,draw opacity=0.50,line width= 0.4pt,line join=round,line cap=round] (296.72,110.41) -- (299.57,113.27);

\path[draw=drawColor,draw opacity=0.50,line width= 0.4pt,line join=round,line cap=round] (296.72,113.27) -- (299.57,110.41);

\path[draw=drawColor,draw opacity=0.50,line width= 0.4pt,line join=round,line cap=round] (296.13,111.84) -- (300.16,111.84);

\path[draw=drawColor,draw opacity=0.50,line width= 0.4pt,line join=round,line cap=round] (298.15,109.82) -- (298.15,113.86);

\path[draw=drawColor,draw opacity=0.50,line width= 0.4pt,line join=round,line cap=round] (297.38,111.37) -- (300.24,114.22);

\path[draw=drawColor,draw opacity=0.50,line width= 0.4pt,line join=round,line cap=round] (297.38,114.22) -- (300.24,111.37);

\path[draw=drawColor,draw opacity=0.50,line width= 0.4pt,line join=round,line cap=round] (296.79,112.80) -- (300.83,112.80);

\path[draw=drawColor,draw opacity=0.50,line width= 0.4pt,line join=round,line cap=round] (298.81,110.78) -- (298.81,114.81);

\path[draw=drawColor,draw opacity=0.50,line width= 0.4pt,line join=round,line cap=round] (296.95,107.82) -- (299.80,110.67);

\path[draw=drawColor,draw opacity=0.50,line width= 0.4pt,line join=round,line cap=round] (296.95,110.67) -- (299.80,107.82);

\path[draw=drawColor,draw opacity=0.50,line width= 0.4pt,line join=round,line cap=round] (296.35,109.25) -- (300.39,109.25);

\path[draw=drawColor,draw opacity=0.50,line width= 0.4pt,line join=round,line cap=round] (298.37,107.23) -- (298.37,111.26);

\path[draw=drawColor,draw opacity=0.50,line width= 0.4pt,line join=round,line cap=round] (297.60,109.00) -- (300.45,111.86);

\path[draw=drawColor,draw opacity=0.50,line width= 0.4pt,line join=round,line cap=round] (297.60,111.86) -- (300.45,109.00);

\path[draw=drawColor,draw opacity=0.50,line width= 0.4pt,line join=round,line cap=round] (297.01,110.43) -- (301.04,110.43);

\path[draw=drawColor,draw opacity=0.50,line width= 0.4pt,line join=round,line cap=round] (299.02,108.41) -- (299.02,112.45);

\path[draw=drawColor,draw opacity=0.50,line width= 0.4pt,line join=round,line cap=round] (296.25,108.62) -- (299.11,111.47);

\path[draw=drawColor,draw opacity=0.50,line width= 0.4pt,line join=round,line cap=round] (296.25,111.47) -- (299.11,108.62);

\path[draw=drawColor,draw opacity=0.50,line width= 0.4pt,line join=round,line cap=round] (295.66,110.05) -- (299.70,110.05);

\path[draw=drawColor,draw opacity=0.50,line width= 0.4pt,line join=round,line cap=round] (297.68,108.03) -- (297.68,112.07);

\path[draw=drawColor,draw opacity=0.50,line width= 0.4pt,line join=round,line cap=round] (296.95,109.37) -- (299.80,112.23);

\path[draw=drawColor,draw opacity=0.50,line width= 0.4pt,line join=round,line cap=round] (296.95,112.23) -- (299.80,109.37);

\path[draw=drawColor,draw opacity=0.50,line width= 0.4pt,line join=round,line cap=round] (296.35,110.80) -- (300.39,110.80);

\path[draw=drawColor,draw opacity=0.50,line width= 0.4pt,line join=round,line cap=round] (298.37,108.78) -- (298.37,112.82);

\path[draw=drawColor,draw opacity=0.50,line width= 0.4pt,line join=round,line cap=round] (295.26,108.62) -- (298.12,111.47);

\path[draw=drawColor,draw opacity=0.50,line width= 0.4pt,line join=round,line cap=round] (295.26,111.47) -- (298.12,108.62);

\path[draw=drawColor,draw opacity=0.50,line width= 0.4pt,line join=round,line cap=round] (294.67,110.05) -- (298.71,110.05);

\path[draw=drawColor,draw opacity=0.50,line width= 0.4pt,line join=round,line cap=round] (296.69,108.03) -- (296.69,112.07);

\path[draw=drawColor,draw opacity=0.50,line width= 0.4pt,line join=round,line cap=round] (298.41,110.41) -- (301.27,113.27);

\path[draw=drawColor,draw opacity=0.50,line width= 0.4pt,line join=round,line cap=round] (298.41,113.27) -- (301.27,110.41);

\path[draw=drawColor,draw opacity=0.50,line width= 0.4pt,line join=round,line cap=round] (297.82,111.84) -- (301.86,111.84);

\path[draw=drawColor,draw opacity=0.50,line width= 0.4pt,line join=round,line cap=round] (299.84,109.82) -- (299.84,113.86);

\path[draw=drawColor,draw opacity=0.50,line width= 0.4pt,line join=round,line cap=round] (298.99,109.37) -- (301.84,112.23);

\path[draw=drawColor,draw opacity=0.50,line width= 0.4pt,line join=round,line cap=round] (298.99,112.23) -- (301.84,109.37);

\path[draw=drawColor,draw opacity=0.50,line width= 0.4pt,line join=round,line cap=round] (298.40,110.80) -- (302.44,110.80);

\path[draw=drawColor,draw opacity=0.50,line width= 0.4pt,line join=round,line cap=round] (300.42,108.78) -- (300.42,112.82);

\path[draw=drawColor,draw opacity=0.50,line width= 0.4pt,line join=round,line cap=round] (300.89,113.84) -- (303.74,116.70);

\path[draw=drawColor,draw opacity=0.50,line width= 0.4pt,line join=round,line cap=round] (300.89,116.70) -- (303.74,113.84);

\path[draw=drawColor,draw opacity=0.50,line width= 0.4pt,line join=round,line cap=round] (300.30,115.27) -- (304.33,115.27);

\path[draw=drawColor,draw opacity=0.50,line width= 0.4pt,line join=round,line cap=round] (302.32,113.25) -- (302.32,117.29);

\path[draw=drawColor,draw opacity=0.50,line width= 0.4pt,line join=round,line cap=round] (300.40,112.25) -- (303.26,115.11);

\path[draw=drawColor,draw opacity=0.50,line width= 0.4pt,line join=round,line cap=round] (300.40,115.11) -- (303.26,112.25);

\path[draw=drawColor,draw opacity=0.50,line width= 0.4pt,line join=round,line cap=round] (299.81,113.68) -- (303.85,113.68);

\path[draw=drawColor,draw opacity=0.50,line width= 0.4pt,line join=round,line cap=round] (301.83,111.66) -- (301.83,115.70);

\path[draw=drawColor,draw opacity=0.50,line width= 0.4pt,line join=round,line cap=round] (300.24,111.06) -- (303.09,113.91);

\path[draw=drawColor,draw opacity=0.50,line width= 0.4pt,line join=round,line cap=round] (300.24,113.91) -- (303.09,111.06);

\path[draw=drawColor,draw opacity=0.50,line width= 0.4pt,line join=round,line cap=round] (299.64,112.49) -- (303.68,112.49);

\path[draw=drawColor,draw opacity=0.50,line width= 0.4pt,line join=round,line cap=round] (301.66,110.47) -- (301.66,114.50);

\path[draw=drawColor,draw opacity=0.50,line width= 0.4pt,line join=round,line cap=round] (299.18,111.67) -- (302.03,114.53);

\path[draw=drawColor,draw opacity=0.50,line width= 0.4pt,line join=round,line cap=round] (299.18,114.53) -- (302.03,111.67);

\path[draw=drawColor,draw opacity=0.50,line width= 0.4pt,line join=round,line cap=round] (298.59,113.10) -- (302.62,113.10);

\path[draw=drawColor,draw opacity=0.50,line width= 0.4pt,line join=round,line cap=round] (300.60,111.08) -- (300.60,115.12);

\path[draw=drawColor,draw opacity=0.50,line width= 0.4pt,line join=round,line cap=round] (302.24,113.34) -- (305.10,116.19);

\path[draw=drawColor,draw opacity=0.50,line width= 0.4pt,line join=round,line cap=round] (302.24,116.19) -- (305.10,113.34);

\path[draw=drawColor,draw opacity=0.50,line width= 0.4pt,line join=round,line cap=round] (301.65,114.76) -- (305.69,114.76);

\path[draw=drawColor,draw opacity=0.50,line width= 0.4pt,line join=round,line cap=round] (303.67,112.75) -- (303.67,116.78);

\path[draw=drawColor,draw opacity=0.50,line width= 0.4pt,line join=round,line cap=round] (298.61,110.41) -- (301.46,113.27);

\path[draw=drawColor,draw opacity=0.50,line width= 0.4pt,line join=round,line cap=round] (298.61,113.27) -- (301.46,110.41);

\path[draw=drawColor,draw opacity=0.50,line width= 0.4pt,line join=round,line cap=round] (298.02,111.84) -- (302.05,111.84);

\path[draw=drawColor,draw opacity=0.50,line width= 0.4pt,line join=round,line cap=round] (300.04,109.82) -- (300.04,113.86);

\path[draw=drawColor,draw opacity=0.50,line width= 0.4pt,line join=round,line cap=round] (305.55,125.40) -- (308.41,128.26);

\path[draw=drawColor,draw opacity=0.50,line width= 0.4pt,line join=round,line cap=round] (305.55,128.26) -- (308.41,125.40);

\path[draw=drawColor,draw opacity=0.50,line width= 0.4pt,line join=round,line cap=round] (304.96,126.83) -- (309.00,126.83);

\path[draw=drawColor,draw opacity=0.50,line width= 0.4pt,line join=round,line cap=round] (306.98,124.81) -- (306.98,128.85);

\path[draw=drawColor,draw opacity=0.50,line width= 0.4pt,line join=round,line cap=round] (298.80,110.41) -- (301.66,113.27);

\path[draw=drawColor,draw opacity=0.50,line width= 0.4pt,line join=round,line cap=round] (298.80,113.27) -- (301.66,110.41);

\path[draw=drawColor,draw opacity=0.50,line width= 0.4pt,line join=round,line cap=round] (298.21,111.84) -- (302.25,111.84);

\path[draw=drawColor,draw opacity=0.50,line width= 0.4pt,line join=round,line cap=round] (300.23,109.82) -- (300.23,113.86);

\path[draw=drawColor,draw opacity=0.50,line width= 0.4pt,line join=round,line cap=round] (299.36,110.41) -- (302.21,113.27);

\path[draw=drawColor,draw opacity=0.50,line width= 0.4pt,line join=round,line cap=round] (299.36,113.27) -- (302.21,110.41);

\path[draw=drawColor,draw opacity=0.50,line width= 0.4pt,line join=round,line cap=round] (298.77,111.84) -- (302.80,111.84);

\path[draw=drawColor,draw opacity=0.50,line width= 0.4pt,line join=round,line cap=round] (300.79,109.82) -- (300.79,113.86);

\path[draw=drawColor,draw opacity=0.50,line width= 0.4pt,line join=round,line cap=round] (298.61,112.25) -- (301.46,115.11);

\path[draw=drawColor,draw opacity=0.50,line width= 0.4pt,line join=round,line cap=round] (298.61,115.11) -- (301.46,112.25);

\path[draw=drawColor,draw opacity=0.50,line width= 0.4pt,line join=round,line cap=round] (298.02,113.68) -- (302.05,113.68);

\path[draw=drawColor,draw opacity=0.50,line width= 0.4pt,line join=round,line cap=round] (300.04,111.66) -- (300.04,115.70);

\path[draw=drawColor,draw opacity=0.50,line width= 0.4pt,line join=round,line cap=round] (303.45,114.79) -- (306.31,117.65);

\path[draw=drawColor,draw opacity=0.50,line width= 0.4pt,line join=round,line cap=round] (303.45,117.65) -- (306.31,114.79);

\path[draw=drawColor,draw opacity=0.50,line width= 0.4pt,line join=round,line cap=round] (302.86,116.22) -- (306.90,116.22);

\path[draw=drawColor,draw opacity=0.50,line width= 0.4pt,line join=round,line cap=round] (304.88,114.20) -- (304.88,118.24);

\path[draw=drawColor,draw opacity=0.50,line width= 0.4pt,line join=round,line cap=round] (300.89,111.37) -- (303.74,114.22);

\path[draw=drawColor,draw opacity=0.50,line width= 0.4pt,line join=round,line cap=round] (300.89,114.22) -- (303.74,111.37);

\path[draw=drawColor,draw opacity=0.50,line width= 0.4pt,line join=round,line cap=round] (300.30,112.80) -- (304.33,112.80);

\path[draw=drawColor,draw opacity=0.50,line width= 0.4pt,line join=round,line cap=round] (302.32,110.78) -- (302.32,114.81);

\path[draw=drawColor,draw opacity=0.50,line width= 0.4pt,line join=round,line cap=round] (298.61,109.73) -- (301.46,112.58);

\path[draw=drawColor,draw opacity=0.50,line width= 0.4pt,line join=round,line cap=round] (298.61,112.58) -- (301.46,109.73);

\path[draw=drawColor,draw opacity=0.50,line width= 0.4pt,line join=round,line cap=round] (298.02,111.16) -- (302.05,111.16);

\path[draw=drawColor,draw opacity=0.50,line width= 0.4pt,line join=round,line cap=round] (300.04,109.14) -- (300.04,113.17);

\path[draw=drawColor,draw opacity=0.50,line width= 0.4pt,line join=round,line cap=round] (301.66,111.97) -- (304.51,114.82);

\path[draw=drawColor,draw opacity=0.50,line width= 0.4pt,line join=round,line cap=round] (301.66,114.82) -- (304.51,111.97);

\path[draw=drawColor,draw opacity=0.50,line width= 0.4pt,line join=round,line cap=round] (301.07,113.39) -- (305.11,113.39);

\path[draw=drawColor,draw opacity=0.50,line width= 0.4pt,line join=round,line cap=round] (303.09,111.38) -- (303.09,115.41);

\path[draw=drawColor,draw opacity=0.50,line width= 0.4pt,line join=round,line cap=round] (300.07,109.73) -- (302.92,112.58);

\path[draw=drawColor,draw opacity=0.50,line width= 0.4pt,line join=round,line cap=round] (300.07,112.58) -- (302.92,109.73);

\path[draw=drawColor,draw opacity=0.50,line width= 0.4pt,line join=round,line cap=round] (299.47,111.16) -- (303.51,111.16);

\path[draw=drawColor,draw opacity=0.50,line width= 0.4pt,line join=round,line cap=round] (301.49,109.14) -- (301.49,113.17);

\path[draw=drawColor,draw opacity=0.50,line width= 0.4pt,line join=round,line cap=round] (303.06,114.09) -- (305.92,116.94);

\path[draw=drawColor,draw opacity=0.50,line width= 0.4pt,line join=round,line cap=round] (303.06,116.94) -- (305.92,114.09);

\path[draw=drawColor,draw opacity=0.50,line width= 0.4pt,line join=round,line cap=round] (302.47,115.51) -- (306.51,115.51);

\path[draw=drawColor,draw opacity=0.50,line width= 0.4pt,line join=round,line cap=round] (304.49,113.50) -- (304.49,117.53);

\path[draw=drawColor,draw opacity=0.50,line width= 0.4pt,line join=round,line cap=round] (304.20,123.08) -- (307.05,125.93);

\path[draw=drawColor,draw opacity=0.50,line width= 0.4pt,line join=round,line cap=round] (304.20,125.93) -- (307.05,123.08);

\path[draw=drawColor,draw opacity=0.50,line width= 0.4pt,line join=round,line cap=round] (303.60,124.51) -- (307.64,124.51);

\path[draw=drawColor,draw opacity=0.50,line width= 0.4pt,line join=round,line cap=round] (305.62,122.49) -- (305.62,126.52);

\path[draw=drawColor,draw opacity=0.50,line width= 0.4pt,line join=round,line cap=round] (303.06,113.84) -- (305.92,116.70);

\path[draw=drawColor,draw opacity=0.50,line width= 0.4pt,line join=round,line cap=round] (303.06,116.70) -- (305.92,113.84);

\path[draw=drawColor,draw opacity=0.50,line width= 0.4pt,line join=round,line cap=round] (302.47,115.27) -- (306.51,115.27);

\path[draw=drawColor,draw opacity=0.50,line width= 0.4pt,line join=round,line cap=round] (304.49,113.25) -- (304.49,117.29);

\path[draw=drawColor,draw opacity=0.50,line width= 0.4pt,line join=round,line cap=round] (301.51,112.81) -- (304.36,115.66);

\path[draw=drawColor,draw opacity=0.50,line width= 0.4pt,line join=round,line cap=round] (301.51,115.66) -- (304.36,112.81);

\path[draw=drawColor,draw opacity=0.50,line width= 0.4pt,line join=round,line cap=round] (300.92,114.23) -- (304.95,114.23);

\path[draw=drawColor,draw opacity=0.50,line width= 0.4pt,line join=round,line cap=round] (302.94,112.22) -- (302.94,116.25);

\path[draw=drawColor,draw opacity=0.50,line width= 0.4pt,line join=round,line cap=round] (302.93,113.59) -- (305.78,116.45);

\path[draw=drawColor,draw opacity=0.50,line width= 0.4pt,line join=round,line cap=round] (302.93,116.45) -- (305.78,113.59);

\path[draw=drawColor,draw opacity=0.50,line width= 0.4pt,line join=round,line cap=round] (302.34,115.02) -- (306.37,115.02);

\path[draw=drawColor,draw opacity=0.50,line width= 0.4pt,line join=round,line cap=round] (304.36,113.00) -- (304.36,117.04);

\path[draw=drawColor,draw opacity=0.50,line width= 0.4pt,line join=round,line cap=round] (312.14,115.24) -- (314.99,118.09);

\path[draw=drawColor,draw opacity=0.50,line width= 0.4pt,line join=round,line cap=round] (312.14,118.09) -- (314.99,115.24);

\path[draw=drawColor,draw opacity=0.50,line width= 0.4pt,line join=round,line cap=round] (311.55,116.67) -- (315.58,116.67);

\path[draw=drawColor,draw opacity=0.50,line width= 0.4pt,line join=round,line cap=round] (313.56,114.65) -- (313.56,118.68);

\path[draw=drawColor,draw opacity=0.50,line width= 0.4pt,line join=round,line cap=round] (301.96,111.97) -- (304.81,114.82);

\path[draw=drawColor,draw opacity=0.50,line width= 0.4pt,line join=round,line cap=round] (301.96,114.82) -- (304.81,111.97);

\path[draw=drawColor,draw opacity=0.50,line width= 0.4pt,line join=round,line cap=round] (301.36,113.39) -- (305.40,113.39);

\path[draw=drawColor,draw opacity=0.50,line width= 0.4pt,line join=round,line cap=round] (303.38,111.38) -- (303.38,115.41);

\path[draw=drawColor,draw opacity=0.50,line width= 0.4pt,line join=round,line cap=round] (305.01,119.73) -- (307.86,122.59);

\path[draw=drawColor,draw opacity=0.50,line width= 0.4pt,line join=round,line cap=round] (305.01,122.59) -- (307.86,119.73);

\path[draw=drawColor,draw opacity=0.50,line width= 0.4pt,line join=round,line cap=round] (304.42,121.16) -- (308.45,121.16);

\path[draw=drawColor,draw opacity=0.50,line width= 0.4pt,line join=round,line cap=round] (306.43,119.14) -- (306.43,123.18);

\path[draw=drawColor,draw opacity=0.50,line width= 0.4pt,line join=round,line cap=round] (305.01,116.09) -- (307.86,118.94);

\path[draw=drawColor,draw opacity=0.50,line width= 0.4pt,line join=round,line cap=round] (305.01,118.94) -- (307.86,116.09);

\path[draw=drawColor,draw opacity=0.50,line width= 0.4pt,line join=round,line cap=round] (304.42,117.51) -- (308.45,117.51);

\path[draw=drawColor,draw opacity=0.50,line width= 0.4pt,line join=round,line cap=round] (306.43,115.49) -- (306.43,119.53);

\path[draw=drawColor,draw opacity=0.50,line width= 0.4pt,line join=round,line cap=round] (303.45,114.33) -- (306.31,117.18);

\path[draw=drawColor,draw opacity=0.50,line width= 0.4pt,line join=round,line cap=round] (303.45,117.18) -- (306.31,114.33);

\path[draw=drawColor,draw opacity=0.50,line width= 0.4pt,line join=round,line cap=round] (302.86,115.75) -- (306.90,115.75);

\path[draw=drawColor,draw opacity=0.50,line width= 0.4pt,line join=round,line cap=round] (304.88,113.74) -- (304.88,117.77);

\path[draw=drawColor,draw opacity=0.50,line width= 0.4pt,line join=round,line cap=round] (304.32,123.94) -- (307.17,126.79);

\path[draw=drawColor,draw opacity=0.50,line width= 0.4pt,line join=round,line cap=round] (304.32,126.79) -- (307.17,123.94);

\path[draw=drawColor,draw opacity=0.50,line width= 0.4pt,line join=round,line cap=round] (303.72,125.37) -- (307.76,125.37);

\path[draw=drawColor,draw opacity=0.50,line width= 0.4pt,line join=round,line cap=round] (305.74,123.35) -- (305.74,127.38);

\path[draw=drawColor,draw opacity=0.50,line width= 0.4pt,line join=round,line cap=round] (310.37,113.84) -- (313.23,116.70);

\path[draw=drawColor,draw opacity=0.50,line width= 0.4pt,line join=round,line cap=round] (310.37,116.70) -- (313.23,113.84);

\path[draw=drawColor,draw opacity=0.50,line width= 0.4pt,line join=round,line cap=round] (309.78,115.27) -- (313.82,115.27);

\path[draw=drawColor,draw opacity=0.50,line width= 0.4pt,line join=round,line cap=round] (311.80,113.25) -- (311.80,117.29);

\path[draw=drawColor,draw opacity=0.50,line width= 0.4pt,line join=round,line cap=round] (301.20,112.81) -- (304.06,115.66);

\path[draw=drawColor,draw opacity=0.50,line width= 0.4pt,line join=round,line cap=round] (301.20,115.66) -- (304.06,112.81);

\path[draw=drawColor,draw opacity=0.50,line width= 0.4pt,line join=round,line cap=round] (300.61,114.23) -- (304.65,114.23);

\path[draw=drawColor,draw opacity=0.50,line width= 0.4pt,line join=round,line cap=round] (302.63,112.22) -- (302.63,116.25);

\path[draw=drawColor,draw opacity=0.50,line width= 0.4pt,line join=round,line cap=round] (302.38,112.25) -- (305.24,115.11);

\path[draw=drawColor,draw opacity=0.50,line width= 0.4pt,line join=round,line cap=round] (302.38,115.11) -- (305.24,112.25);

\path[draw=drawColor,draw opacity=0.50,line width= 0.4pt,line join=round,line cap=round] (301.79,113.68) -- (305.83,113.68);

\path[draw=drawColor,draw opacity=0.50,line width= 0.4pt,line join=round,line cap=round] (303.81,111.66) -- (303.81,115.70);

\path[draw=drawColor,draw opacity=0.50,line width= 0.4pt,line join=round,line cap=round] (302.66,113.59) -- (305.51,116.45);

\path[draw=drawColor,draw opacity=0.50,line width= 0.4pt,line join=round,line cap=round] (302.66,116.45) -- (305.51,113.59);

\path[draw=drawColor,draw opacity=0.50,line width= 0.4pt,line join=round,line cap=round] (302.07,115.02) -- (306.10,115.02);

\path[draw=drawColor,draw opacity=0.50,line width= 0.4pt,line join=round,line cap=round] (304.09,113.00) -- (304.09,117.04);

\path[draw=drawColor,draw opacity=0.50,line width= 0.4pt,line join=round,line cap=round] (302.24,113.59) -- (305.10,116.45);

\path[draw=drawColor,draw opacity=0.50,line width= 0.4pt,line join=round,line cap=round] (302.24,116.45) -- (305.10,113.59);

\path[draw=drawColor,draw opacity=0.50,line width= 0.4pt,line join=round,line cap=round] (301.65,115.02) -- (305.69,115.02);

\path[draw=drawColor,draw opacity=0.50,line width= 0.4pt,line join=round,line cap=round] (303.67,113.00) -- (303.67,117.04);

\path[draw=drawColor,draw opacity=0.50,line width= 0.4pt,line join=round,line cap=round] (306.67,116.87) -- (309.52,119.73);

\path[draw=drawColor,draw opacity=0.50,line width= 0.4pt,line join=round,line cap=round] (306.67,119.73) -- (309.52,116.87);

\path[draw=drawColor,draw opacity=0.50,line width= 0.4pt,line join=round,line cap=round] (306.08,118.30) -- (310.11,118.30);

\path[draw=drawColor,draw opacity=0.50,line width= 0.4pt,line join=round,line cap=round] (308.10,116.28) -- (308.10,120.32);

\path[draw=drawColor,draw opacity=0.50,line width= 0.4pt,line join=round,line cap=round] (340.94,147.69) -- (343.80,150.54);

\path[draw=drawColor,draw opacity=0.50,line width= 0.4pt,line join=round,line cap=round] (340.94,150.54) -- (343.80,147.69);

\path[draw=drawColor,draw opacity=0.50,line width= 0.4pt,line join=round,line cap=round] (340.35,149.11) -- (344.39,149.11);

\path[draw=drawColor,draw opacity=0.50,line width= 0.4pt,line join=round,line cap=round] (342.37,147.09) -- (342.37,151.13);

\path[draw=drawColor,draw opacity=0.50,line width= 0.4pt,line join=round,line cap=round] (341.20,149.34) -- (344.06,152.19);

\path[draw=drawColor,draw opacity=0.50,line width= 0.4pt,line join=round,line cap=round] (341.20,152.19) -- (344.06,149.34);

\path[draw=drawColor,draw opacity=0.50,line width= 0.4pt,line join=round,line cap=round] (340.61,150.77) -- (344.65,150.77);

\path[draw=drawColor,draw opacity=0.50,line width= 0.4pt,line join=round,line cap=round] (342.63,148.75) -- (342.63,152.79);

\path[draw=drawColor,draw opacity=0.50,line width= 0.4pt,line join=round,line cap=round] (338.38,147.73) -- (341.23,150.58);

\path[draw=drawColor,draw opacity=0.50,line width= 0.4pt,line join=round,line cap=round] (338.38,150.58) -- (341.23,147.73);

\path[draw=drawColor,draw opacity=0.50,line width= 0.4pt,line join=round,line cap=round] (337.79,149.15) -- (341.82,149.15);

\path[draw=drawColor,draw opacity=0.50,line width= 0.4pt,line join=round,line cap=round] (339.81,147.14) -- (339.81,151.17);

\path[draw=drawColor,draw opacity=0.50,line width= 0.4pt,line join=round,line cap=round] (338.52,147.19) -- (341.37,150.04);

\path[draw=drawColor,draw opacity=0.50,line width= 0.4pt,line join=round,line cap=round] (338.52,150.04) -- (341.37,147.19);

\path[draw=drawColor,draw opacity=0.50,line width= 0.4pt,line join=round,line cap=round] (337.93,148.62) -- (341.96,148.62);

\path[draw=drawColor,draw opacity=0.50,line width= 0.4pt,line join=round,line cap=round] (339.95,146.60) -- (339.95,150.63);

\path[draw=drawColor,draw opacity=0.50,line width= 0.4pt,line join=round,line cap=round] (340.84,147.79) -- (343.69,150.65);

\path[draw=drawColor,draw opacity=0.50,line width= 0.4pt,line join=round,line cap=round] (340.84,150.65) -- (343.69,147.79);

\path[draw=drawColor,draw opacity=0.50,line width= 0.4pt,line join=round,line cap=round] (340.25,149.22) -- (344.28,149.22);

\path[draw=drawColor,draw opacity=0.50,line width= 0.4pt,line join=round,line cap=round] (342.27,147.20) -- (342.27,151.24);

\path[draw=drawColor,draw opacity=0.50,line width= 0.4pt,line join=round,line cap=round] (340.88,147.75) -- (343.73,150.61);

\path[draw=drawColor,draw opacity=0.50,line width= 0.4pt,line join=round,line cap=round] (340.88,150.61) -- (343.73,147.75);

\path[draw=drawColor,draw opacity=0.50,line width= 0.4pt,line join=round,line cap=round] (340.28,149.18) -- (344.32,149.18);

\path[draw=drawColor,draw opacity=0.50,line width= 0.4pt,line join=round,line cap=round] (342.30,147.16) -- (342.30,151.20);

\path[draw=drawColor,draw opacity=0.50,line width= 0.4pt,line join=round,line cap=round] (340.17,148.70) -- (343.03,151.56);

\path[draw=drawColor,draw opacity=0.50,line width= 0.4pt,line join=round,line cap=round] (340.17,151.56) -- (343.03,148.70);

\path[draw=drawColor,draw opacity=0.50,line width= 0.4pt,line join=round,line cap=round] (339.58,150.13) -- (343.62,150.13);

\path[draw=drawColor,draw opacity=0.50,line width= 0.4pt,line join=round,line cap=round] (341.60,148.11) -- (341.60,152.15);

\path[draw=drawColor,draw opacity=0.50,line width= 0.4pt,line join=round,line cap=round] (340.30,147.78) -- (343.16,150.63);

\path[draw=drawColor,draw opacity=0.50,line width= 0.4pt,line join=round,line cap=round] (340.30,150.63) -- (343.16,147.78);

\path[draw=drawColor,draw opacity=0.50,line width= 0.4pt,line join=round,line cap=round] (339.71,149.21) -- (343.75,149.21);

\path[draw=drawColor,draw opacity=0.50,line width= 0.4pt,line join=round,line cap=round] (341.73,147.19) -- (341.73,151.22);

\path[draw=drawColor,draw opacity=0.50,line width= 0.4pt,line join=round,line cap=round] (339.80,147.78) -- (342.66,150.63);

\path[draw=drawColor,draw opacity=0.50,line width= 0.4pt,line join=round,line cap=round] (339.80,150.63) -- (342.66,147.78);

\path[draw=drawColor,draw opacity=0.50,line width= 0.4pt,line join=round,line cap=round] (339.21,149.21) -- (343.25,149.21);

\path[draw=drawColor,draw opacity=0.50,line width= 0.4pt,line join=round,line cap=round] (341.23,147.19) -- (341.23,151.22);

\path[draw=drawColor,draw opacity=0.50,line width= 0.4pt,line join=round,line cap=round] (338.83,148.17) -- (341.69,151.03);

\path[draw=drawColor,draw opacity=0.50,line width= 0.4pt,line join=round,line cap=round] (338.83,151.03) -- (341.69,148.17);

\path[draw=drawColor,draw opacity=0.50,line width= 0.4pt,line join=round,line cap=round] (338.24,149.60) -- (342.28,149.60);

\path[draw=drawColor,draw opacity=0.50,line width= 0.4pt,line join=round,line cap=round] (340.26,147.58) -- (340.26,151.62);

\path[draw=drawColor,draw opacity=0.50,line width= 0.4pt,line join=round,line cap=round] (338.14,147.09) -- (340.99,149.94);

\path[draw=drawColor,draw opacity=0.50,line width= 0.4pt,line join=round,line cap=round] (338.14,149.94) -- (340.99,147.09);

\path[draw=drawColor,draw opacity=0.50,line width= 0.4pt,line join=round,line cap=round] (337.55,148.52) -- (341.58,148.52);

\path[draw=drawColor,draw opacity=0.50,line width= 0.4pt,line join=round,line cap=round] (339.56,146.50) -- (339.56,150.54);

\path[draw=drawColor,draw opacity=0.50,line width= 0.4pt,line join=round,line cap=round] (341.57,147.60) -- (344.42,150.46);

\path[draw=drawColor,draw opacity=0.50,line width= 0.4pt,line join=round,line cap=round] (341.57,150.46) -- (344.42,147.60);

\path[draw=drawColor,draw opacity=0.50,line width= 0.4pt,line join=round,line cap=round] (340.97,149.03) -- (345.01,149.03);

\path[draw=drawColor,draw opacity=0.50,line width= 0.4pt,line join=round,line cap=round] (342.99,147.01) -- (342.99,151.05);

\path[draw=drawColor,draw opacity=0.50,line width= 0.4pt,line join=round,line cap=round] (342.41,149.18) -- (345.26,152.03);

\path[draw=drawColor,draw opacity=0.50,line width= 0.4pt,line join=round,line cap=round] (342.41,152.03) -- (345.26,149.18);

\path[draw=drawColor,draw opacity=0.50,line width= 0.4pt,line join=round,line cap=round] (341.82,150.60) -- (345.85,150.60);

\path[draw=drawColor,draw opacity=0.50,line width= 0.4pt,line join=round,line cap=round] (343.83,148.59) -- (343.83,152.62);

\path[draw=drawColor,draw opacity=0.50,line width= 0.4pt,line join=round,line cap=round] (341.27,148.64) -- (344.12,151.50);

\path[draw=drawColor,draw opacity=0.50,line width= 0.4pt,line join=round,line cap=round] (341.27,151.50) -- (344.12,148.64);

\path[draw=drawColor,draw opacity=0.50,line width= 0.4pt,line join=round,line cap=round] (340.68,150.07) -- (344.71,150.07);

\path[draw=drawColor,draw opacity=0.50,line width= 0.4pt,line join=round,line cap=round] (342.69,148.05) -- (342.69,152.09);

\path[draw=drawColor,draw opacity=0.50,line width= 0.4pt,line join=round,line cap=round] (338.16,147.50) -- (341.01,150.35);

\path[draw=drawColor,draw opacity=0.50,line width= 0.4pt,line join=round,line cap=round] (338.16,150.35) -- (341.01,147.50);

\path[draw=drawColor,draw opacity=0.50,line width= 0.4pt,line join=round,line cap=round] (337.56,148.92) -- (341.60,148.92);

\path[draw=drawColor,draw opacity=0.50,line width= 0.4pt,line join=round,line cap=round] (339.58,146.90) -- (339.58,150.94);

\path[draw=drawColor,draw opacity=0.50,line width= 0.4pt,line join=round,line cap=round] (341.64,151.33) -- (344.50,154.19);

\path[draw=drawColor,draw opacity=0.50,line width= 0.4pt,line join=round,line cap=round] (341.64,154.19) -- (344.50,151.33);

\path[draw=drawColor,draw opacity=0.50,line width= 0.4pt,line join=round,line cap=round] (341.05,152.76) -- (345.09,152.76);

\path[draw=drawColor,draw opacity=0.50,line width= 0.4pt,line join=round,line cap=round] (343.07,150.74) -- (343.07,154.78);

\path[draw=drawColor,draw opacity=0.50,line width= 0.4pt,line join=round,line cap=round] (342.00,151.35) -- (344.85,154.21);

\path[draw=drawColor,draw opacity=0.50,line width= 0.4pt,line join=round,line cap=round] (342.00,154.21) -- (344.85,151.35);

\path[draw=drawColor,draw opacity=0.50,line width= 0.4pt,line join=round,line cap=round] (341.41,152.78) -- (345.45,152.78);

\path[draw=drawColor,draw opacity=0.50,line width= 0.4pt,line join=round,line cap=round] (343.43,150.76) -- (343.43,154.80);

\path[draw=drawColor,draw opacity=0.50,line width= 0.4pt,line join=round,line cap=round] (340.85,155.45) -- (343.71,158.31);

\path[draw=drawColor,draw opacity=0.50,line width= 0.4pt,line join=round,line cap=round] (340.85,158.31) -- (343.71,155.45);

\path[draw=drawColor,draw opacity=0.50,line width= 0.4pt,line join=round,line cap=round] (340.26,156.88) -- (344.30,156.88);

\path[draw=drawColor,draw opacity=0.50,line width= 0.4pt,line join=round,line cap=round] (342.28,154.86) -- (342.28,158.90);

\path[draw=drawColor,draw opacity=0.50,line width= 0.4pt,line join=round,line cap=round] (344.87,151.87) -- (347.72,154.73);

\path[draw=drawColor,draw opacity=0.50,line width= 0.4pt,line join=round,line cap=round] (344.87,154.73) -- (347.72,151.87);

\path[draw=drawColor,draw opacity=0.50,line width= 0.4pt,line join=round,line cap=round] (344.28,153.30) -- (348.31,153.30);

\path[draw=drawColor,draw opacity=0.50,line width= 0.4pt,line join=round,line cap=round] (346.29,151.28) -- (346.29,155.32);

\path[draw=drawColor,draw opacity=0.50,line width= 0.4pt,line join=round,line cap=round] (340.89,148.59) -- (343.74,151.45);

\path[draw=drawColor,draw opacity=0.50,line width= 0.4pt,line join=round,line cap=round] (340.89,151.45) -- (343.74,148.59);

\path[draw=drawColor,draw opacity=0.50,line width= 0.4pt,line join=round,line cap=round] (340.30,150.02) -- (344.33,150.02);

\path[draw=drawColor,draw opacity=0.50,line width= 0.4pt,line join=round,line cap=round] (342.32,148.00) -- (342.32,152.04);

\path[draw=drawColor,draw opacity=0.50,line width= 0.4pt,line join=round,line cap=round] (343.25,149.13) -- (346.10,151.98);

\path[draw=drawColor,draw opacity=0.50,line width= 0.4pt,line join=round,line cap=round] (343.25,151.98) -- (346.10,149.13);

\path[draw=drawColor,draw opacity=0.50,line width= 0.4pt,line join=round,line cap=round] (342.66,150.56) -- (346.69,150.56);

\path[draw=drawColor,draw opacity=0.50,line width= 0.4pt,line join=round,line cap=round] (344.67,148.54) -- (344.67,152.57);

\path[draw=drawColor,draw opacity=0.50,line width= 0.4pt,line join=round,line cap=round] (344.18,148.88) -- (347.03,151.73);

\path[draw=drawColor,draw opacity=0.50,line width= 0.4pt,line join=round,line cap=round] (344.18,151.73) -- (347.03,148.88);

\path[draw=drawColor,draw opacity=0.50,line width= 0.4pt,line join=round,line cap=round] (343.59,150.30) -- (347.62,150.30);

\path[draw=drawColor,draw opacity=0.50,line width= 0.4pt,line join=round,line cap=round] (345.61,148.28) -- (345.61,152.32);

\path[draw=drawColor,draw opacity=0.50,line width= 0.4pt,line join=round,line cap=round] (340.50,148.56) -- (343.35,151.41);

\path[draw=drawColor,draw opacity=0.50,line width= 0.4pt,line join=round,line cap=round] (340.50,151.41) -- (343.35,148.56);

\path[draw=drawColor,draw opacity=0.50,line width= 0.4pt,line join=round,line cap=round] (339.91,149.98) -- (343.95,149.98);

\path[draw=drawColor,draw opacity=0.50,line width= 0.4pt,line join=round,line cap=round] (341.93,147.96) -- (341.93,152.00);

\path[draw=drawColor,draw opacity=0.50,line width= 0.4pt,line join=round,line cap=round] (342.78,150.49) -- (345.63,153.34);

\path[draw=drawColor,draw opacity=0.50,line width= 0.4pt,line join=round,line cap=round] (342.78,153.34) -- (345.63,150.49);

\path[draw=drawColor,draw opacity=0.50,line width= 0.4pt,line join=round,line cap=round] (342.19,151.91) -- (346.22,151.91);

\path[draw=drawColor,draw opacity=0.50,line width= 0.4pt,line join=round,line cap=round] (344.20,149.90) -- (344.20,153.93);

\path[draw=drawColor,draw opacity=0.50,line width= 0.4pt,line join=round,line cap=round] (340.68,148.54) -- (343.54,151.40);

\path[draw=drawColor,draw opacity=0.50,line width= 0.4pt,line join=round,line cap=round] (340.68,151.40) -- (343.54,148.54);

\path[draw=drawColor,draw opacity=0.50,line width= 0.4pt,line join=round,line cap=round] (340.09,149.97) -- (344.13,149.97);

\path[draw=drawColor,draw opacity=0.50,line width= 0.4pt,line join=round,line cap=round] (342.11,147.95) -- (342.11,151.99);

\path[draw=drawColor,draw opacity=0.50,line width= 0.4pt,line join=round,line cap=round] (341.13,148.48) -- (343.99,151.33);

\path[draw=drawColor,draw opacity=0.50,line width= 0.4pt,line join=round,line cap=round] (341.13,151.33) -- (343.99,148.48);

\path[draw=drawColor,draw opacity=0.50,line width= 0.4pt,line join=round,line cap=round] (340.54,149.91) -- (344.58,149.91);

\path[draw=drawColor,draw opacity=0.50,line width= 0.4pt,line join=round,line cap=round] (342.56,147.89) -- (342.56,151.92);

\path[draw=drawColor,draw opacity=0.50,line width= 0.4pt,line join=round,line cap=round] (341.76,150.57) -- (344.61,153.42);

\path[draw=drawColor,draw opacity=0.50,line width= 0.4pt,line join=round,line cap=round] (341.76,153.42) -- (344.61,150.57);

\path[draw=drawColor,draw opacity=0.50,line width= 0.4pt,line join=round,line cap=round] (341.17,152.00) -- (345.21,152.00);

\path[draw=drawColor,draw opacity=0.50,line width= 0.4pt,line join=round,line cap=round] (343.19,149.98) -- (343.19,154.01);

\path[draw=drawColor,draw opacity=0.50,line width= 0.4pt,line join=round,line cap=round] (343.11,148.35) -- (345.97,151.21);

\path[draw=drawColor,draw opacity=0.50,line width= 0.4pt,line join=round,line cap=round] (343.11,151.21) -- (345.97,148.35);

\path[draw=drawColor,draw opacity=0.50,line width= 0.4pt,line join=round,line cap=round] (342.52,149.78) -- (346.56,149.78);

\path[draw=drawColor,draw opacity=0.50,line width= 0.4pt,line join=round,line cap=round] (344.54,147.76) -- (344.54,151.80);

\path[draw=drawColor,draw opacity=0.50,line width= 0.4pt,line join=round,line cap=round] (341.88,151.83) -- (344.74,154.68);

\path[draw=drawColor,draw opacity=0.50,line width= 0.4pt,line join=round,line cap=round] (341.88,154.68) -- (344.74,151.83);

\path[draw=drawColor,draw opacity=0.50,line width= 0.4pt,line join=round,line cap=round] (341.29,153.25) -- (345.33,153.25);

\path[draw=drawColor,draw opacity=0.50,line width= 0.4pt,line join=round,line cap=round] (343.31,151.23) -- (343.31,155.27);

\path[draw=drawColor,draw opacity=0.50,line width= 0.4pt,line join=round,line cap=round] (341.78,149.51) -- (344.63,152.37);

\path[draw=drawColor,draw opacity=0.50,line width= 0.4pt,line join=round,line cap=round] (341.78,152.37) -- (344.63,149.51);

\path[draw=drawColor,draw opacity=0.50,line width= 0.4pt,line join=round,line cap=round] (341.18,150.94) -- (345.22,150.94);

\path[draw=drawColor,draw opacity=0.50,line width= 0.4pt,line join=round,line cap=round] (343.20,148.92) -- (343.20,152.96);

\path[draw=drawColor,draw opacity=0.50,line width= 0.4pt,line join=round,line cap=round] (342.93,149.63) -- (345.78,152.48);

\path[draw=drawColor,draw opacity=0.50,line width= 0.4pt,line join=round,line cap=round] (342.93,152.48) -- (345.78,149.63);

\path[draw=drawColor,draw opacity=0.50,line width= 0.4pt,line join=round,line cap=round] (342.34,151.06) -- (346.37,151.06);

\path[draw=drawColor,draw opacity=0.50,line width= 0.4pt,line join=round,line cap=round] (344.36,149.04) -- (344.36,153.07);

\path[draw=drawColor,draw opacity=0.50,line width= 0.4pt,line join=round,line cap=round] (343.82,151.02) -- (346.67,153.88);

\path[draw=drawColor,draw opacity=0.50,line width= 0.4pt,line join=round,line cap=round] (343.82,153.88) -- (346.67,151.02);

\path[draw=drawColor,draw opacity=0.50,line width= 0.4pt,line join=round,line cap=round] (343.23,152.45) -- (347.26,152.45);

\path[draw=drawColor,draw opacity=0.50,line width= 0.4pt,line join=round,line cap=round] (345.24,150.43) -- (345.24,154.47);

\path[draw=drawColor,draw opacity=0.50,line width= 0.4pt,line join=round,line cap=round] (344.78,152.07) -- (347.63,154.92);

\path[draw=drawColor,draw opacity=0.50,line width= 0.4pt,line join=round,line cap=round] (344.78,154.92) -- (347.63,152.07);

\path[draw=drawColor,draw opacity=0.50,line width= 0.4pt,line join=round,line cap=round] (344.19,153.49) -- (348.22,153.49);

\path[draw=drawColor,draw opacity=0.50,line width= 0.4pt,line join=round,line cap=round] (346.20,151.48) -- (346.20,155.51);

\path[draw=drawColor,draw opacity=0.50,line width= 0.4pt,line join=round,line cap=round] (345.04,156.27) -- (347.89,159.12);

\path[draw=drawColor,draw opacity=0.50,line width= 0.4pt,line join=round,line cap=round] (345.04,159.12) -- (347.89,156.27);

\path[draw=drawColor,draw opacity=0.50,line width= 0.4pt,line join=round,line cap=round] (344.45,157.69) -- (348.48,157.69);

\path[draw=drawColor,draw opacity=0.50,line width= 0.4pt,line join=round,line cap=round] (346.47,155.68) -- (346.47,159.71);

\path[draw=drawColor,draw opacity=0.50,line width= 0.4pt,line join=round,line cap=round] (344.80,153.65) -- (347.66,156.50);

\path[draw=drawColor,draw opacity=0.50,line width= 0.4pt,line join=round,line cap=round] (344.80,156.50) -- (347.66,153.65);

\path[draw=drawColor,draw opacity=0.50,line width= 0.4pt,line join=round,line cap=round] (344.21,155.07) -- (348.25,155.07);

\path[draw=drawColor,draw opacity=0.50,line width= 0.4pt,line join=round,line cap=round] (346.23,153.06) -- (346.23,157.09);

\path[draw=drawColor,draw opacity=0.50,line width= 0.4pt,line join=round,line cap=round] (346.22,152.48) -- (349.08,155.33);

\path[draw=drawColor,draw opacity=0.50,line width= 0.4pt,line join=round,line cap=round] (346.22,155.33) -- (349.08,152.48);

\path[draw=drawColor,draw opacity=0.50,line width= 0.4pt,line join=round,line cap=round] (345.63,153.90) -- (349.67,153.90);

\path[draw=drawColor,draw opacity=0.50,line width= 0.4pt,line join=round,line cap=round] (347.65,151.89) -- (347.65,155.92);

\path[draw=drawColor,draw opacity=0.50,line width= 0.4pt,line join=round,line cap=round] (344.16,153.77) -- (347.01,156.62);

\path[draw=drawColor,draw opacity=0.50,line width= 0.4pt,line join=round,line cap=round] (344.16,156.62) -- (347.01,153.77);

\path[draw=drawColor,draw opacity=0.50,line width= 0.4pt,line join=round,line cap=round] (343.57,155.19) -- (347.60,155.19);

\path[draw=drawColor,draw opacity=0.50,line width= 0.4pt,line join=round,line cap=round] (345.59,153.18) -- (345.59,157.21);

\path[draw=drawColor,draw opacity=0.50,line width= 0.4pt,line join=round,line cap=round] (343.02,153.94) -- (345.87,156.80);

\path[draw=drawColor,draw opacity=0.50,line width= 0.4pt,line join=round,line cap=round] (343.02,156.80) -- (345.87,153.94);

\path[draw=drawColor,draw opacity=0.50,line width= 0.4pt,line join=round,line cap=round] (342.43,155.37) -- (346.46,155.37);

\path[draw=drawColor,draw opacity=0.50,line width= 0.4pt,line join=round,line cap=round] (344.45,153.35) -- (344.45,157.39);

\path[draw=drawColor,draw opacity=0.50,line width= 0.4pt,line join=round,line cap=round] (344.74,155.06) -- (347.59,157.91);

\path[draw=drawColor,draw opacity=0.50,line width= 0.4pt,line join=round,line cap=round] (344.74,157.91) -- (347.59,155.06);

\path[draw=drawColor,draw opacity=0.50,line width= 0.4pt,line join=round,line cap=round] (344.15,156.49) -- (348.18,156.49);

\path[draw=drawColor,draw opacity=0.50,line width= 0.4pt,line join=round,line cap=round] (346.16,154.47) -- (346.16,158.50);

\path[draw=drawColor,draw opacity=0.50,line width= 0.4pt,line join=round,line cap=round] (344.58,153.63) -- (347.44,156.48);

\path[draw=drawColor,draw opacity=0.50,line width= 0.4pt,line join=round,line cap=round] (344.58,156.48) -- (347.44,153.63);

\path[draw=drawColor,draw opacity=0.50,line width= 0.4pt,line join=round,line cap=round] (343.99,155.06) -- (348.03,155.06);

\path[draw=drawColor,draw opacity=0.50,line width= 0.4pt,line join=round,line cap=round] (346.01,153.04) -- (346.01,157.08);

\path[draw=drawColor,draw opacity=0.50,line width= 0.4pt,line join=round,line cap=round] (344.39,154.88) -- (347.24,157.73);

\path[draw=drawColor,draw opacity=0.50,line width= 0.4pt,line join=round,line cap=round] (344.39,157.73) -- (347.24,154.88);

\path[draw=drawColor,draw opacity=0.50,line width= 0.4pt,line join=round,line cap=round] (343.80,156.31) -- (347.83,156.31);

\path[draw=drawColor,draw opacity=0.50,line width= 0.4pt,line join=round,line cap=round] (345.81,154.29) -- (345.81,158.32);

\path[draw=drawColor,draw opacity=0.50,line width= 0.4pt,line join=round,line cap=round] (344.71,152.69) -- (347.56,155.54);

\path[draw=drawColor,draw opacity=0.50,line width= 0.4pt,line join=round,line cap=round] (344.71,155.54) -- (347.56,152.69);

\path[draw=drawColor,draw opacity=0.50,line width= 0.4pt,line join=round,line cap=round] (344.12,154.12) -- (348.15,154.12);

\path[draw=drawColor,draw opacity=0.50,line width= 0.4pt,line join=round,line cap=round] (346.13,152.10) -- (346.13,156.13);

\path[draw=drawColor,draw opacity=0.50,line width= 0.4pt,line join=round,line cap=round] (343.20,150.26) -- (346.05,153.12);

\path[draw=drawColor,draw opacity=0.50,line width= 0.4pt,line join=round,line cap=round] (343.20,153.12) -- (346.05,150.26);

\path[draw=drawColor,draw opacity=0.50,line width= 0.4pt,line join=round,line cap=round] (342.61,151.69) -- (346.64,151.69);

\path[draw=drawColor,draw opacity=0.50,line width= 0.4pt,line join=round,line cap=round] (344.63,149.67) -- (344.63,153.71);

\path[draw=drawColor,draw opacity=0.50,line width= 0.4pt,line join=round,line cap=round] (348.17,153.91) -- (351.02,156.76);

\path[draw=drawColor,draw opacity=0.50,line width= 0.4pt,line join=round,line cap=round] (348.17,156.76) -- (351.02,153.91);

\path[draw=drawColor,draw opacity=0.50,line width= 0.4pt,line join=round,line cap=round] (347.58,155.34) -- (351.61,155.34);

\path[draw=drawColor,draw opacity=0.50,line width= 0.4pt,line join=round,line cap=round] (349.60,153.32) -- (349.60,157.36);

\path[draw=drawColor,draw opacity=0.50,line width= 0.4pt,line join=round,line cap=round] (351.94,153.81) -- (354.80,156.66);

\path[draw=drawColor,draw opacity=0.50,line width= 0.4pt,line join=round,line cap=round] (351.94,156.66) -- (354.80,153.81);

\path[draw=drawColor,draw opacity=0.50,line width= 0.4pt,line join=round,line cap=round] (351.35,155.23) -- (355.39,155.23);

\path[draw=drawColor,draw opacity=0.50,line width= 0.4pt,line join=round,line cap=round] (353.37,153.22) -- (353.37,157.25);

\path[draw=drawColor,draw opacity=0.50,line width= 0.4pt,line join=round,line cap=round] (347.17,152.01) -- (350.02,154.87);

\path[draw=drawColor,draw opacity=0.50,line width= 0.4pt,line join=round,line cap=round] (347.17,154.87) -- (350.02,152.01);

\path[draw=drawColor,draw opacity=0.50,line width= 0.4pt,line join=round,line cap=round] (346.58,153.44) -- (350.62,153.44);

\path[draw=drawColor,draw opacity=0.50,line width= 0.4pt,line join=round,line cap=round] (348.60,151.42) -- (348.60,155.46);

\path[draw=drawColor,draw opacity=0.50,line width= 0.4pt,line join=round,line cap=round] (345.80,155.57) -- (348.66,158.42);

\path[draw=drawColor,draw opacity=0.50,line width= 0.4pt,line join=round,line cap=round] (345.80,158.42) -- (348.66,155.57);

\path[draw=drawColor,draw opacity=0.50,line width= 0.4pt,line join=round,line cap=round] (345.21,157.00) -- (349.25,157.00);

\path[draw=drawColor,draw opacity=0.50,line width= 0.4pt,line join=round,line cap=round] (347.23,154.98) -- (347.23,159.01);

\path[draw=drawColor,draw opacity=0.50,line width= 0.4pt,line join=round,line cap=round] (345.99,157.99) -- (348.84,160.85);

\path[draw=drawColor,draw opacity=0.50,line width= 0.4pt,line join=round,line cap=round] (345.99,160.85) -- (348.84,157.99);

\path[draw=drawColor,draw opacity=0.50,line width= 0.4pt,line join=round,line cap=round] (345.40,159.42) -- (349.43,159.42);

\path[draw=drawColor,draw opacity=0.50,line width= 0.4pt,line join=round,line cap=round] (347.41,157.40) -- (347.41,161.44);

\path[draw=drawColor,draw opacity=0.50,line width= 0.4pt,line join=round,line cap=round] (363.35,169.28) -- (366.20,172.14);

\path[draw=drawColor,draw opacity=0.50,line width= 0.4pt,line join=round,line cap=round] (363.35,172.14) -- (366.20,169.28);

\path[draw=drawColor,draw opacity=0.50,line width= 0.4pt,line join=round,line cap=round] (362.76,170.71) -- (366.79,170.71);

\path[draw=drawColor,draw opacity=0.50,line width= 0.4pt,line join=round,line cap=round] (364.77,168.69) -- (364.77,172.73);

\path[draw=drawColor,draw opacity=0.50,line width= 0.4pt,line join=round,line cap=round] (363.65,170.22) -- (366.50,173.07);

\path[draw=drawColor,draw opacity=0.50,line width= 0.4pt,line join=round,line cap=round] (363.65,173.07) -- (366.50,170.22);

\path[draw=drawColor,draw opacity=0.50,line width= 0.4pt,line join=round,line cap=round] (363.06,171.64) -- (367.09,171.64);

\path[draw=drawColor,draw opacity=0.50,line width= 0.4pt,line join=round,line cap=round] (365.07,169.63) -- (365.07,173.66);

\path[draw=drawColor,draw opacity=0.50,line width= 0.4pt,line join=round,line cap=round] (367.64,170.47) -- (370.49,173.33);

\path[draw=drawColor,draw opacity=0.50,line width= 0.4pt,line join=round,line cap=round] (367.64,173.33) -- (370.49,170.47);

\path[draw=drawColor,draw opacity=0.50,line width= 0.4pt,line join=round,line cap=round] (367.05,171.90) -- (371.08,171.90);

\path[draw=drawColor,draw opacity=0.50,line width= 0.4pt,line join=round,line cap=round] (369.06,169.88) -- (369.06,173.92);

\path[draw=drawColor,draw opacity=0.50,line width= 0.4pt,line join=round,line cap=round] (365.87,169.73) -- (368.73,172.59);

\path[draw=drawColor,draw opacity=0.50,line width= 0.4pt,line join=round,line cap=round] (365.87,172.59) -- (368.73,169.73);

\path[draw=drawColor,draw opacity=0.50,line width= 0.4pt,line join=round,line cap=round] (365.28,171.16) -- (369.32,171.16);

\path[draw=drawColor,draw opacity=0.50,line width= 0.4pt,line join=round,line cap=round] (367.30,169.14) -- (367.30,173.18);

\path[draw=drawColor,draw opacity=0.50,line width= 0.4pt,line join=round,line cap=round] (363.67,172.96) -- (366.53,175.81);

\path[draw=drawColor,draw opacity=0.50,line width= 0.4pt,line join=round,line cap=round] (363.67,175.81) -- (366.53,172.96);

\path[draw=drawColor,draw opacity=0.50,line width= 0.4pt,line join=round,line cap=round] (363.08,174.38) -- (367.12,174.38);

\path[draw=drawColor,draw opacity=0.50,line width= 0.4pt,line join=round,line cap=round] (365.10,172.36) -- (365.10,176.40);

\path[draw=drawColor,draw opacity=0.50,line width= 0.4pt,line join=round,line cap=round] (366.27,171.53) -- (369.12,174.39);

\path[draw=drawColor,draw opacity=0.50,line width= 0.4pt,line join=round,line cap=round] (366.27,174.39) -- (369.12,171.53);

\path[draw=drawColor,draw opacity=0.50,line width= 0.4pt,line join=round,line cap=round] (365.68,172.96) -- (369.71,172.96);

\path[draw=drawColor,draw opacity=0.50,line width= 0.4pt,line join=round,line cap=round] (367.69,170.94) -- (367.69,174.98);

\path[draw=drawColor,draw opacity=0.50,line width= 0.4pt,line join=round,line cap=round] (365.03,171.07) -- (367.88,173.92);

\path[draw=drawColor,draw opacity=0.50,line width= 0.4pt,line join=round,line cap=round] (365.03,173.92) -- (367.88,171.07);

\path[draw=drawColor,draw opacity=0.50,line width= 0.4pt,line join=round,line cap=round] (364.44,172.49) -- (368.47,172.49);

\path[draw=drawColor,draw opacity=0.50,line width= 0.4pt,line join=round,line cap=round] (366.46,170.47) -- (366.46,174.51);

\path[draw=drawColor,draw opacity=0.50,line width= 0.4pt,line join=round,line cap=round] (364.46,172.10) -- (367.31,174.95);

\path[draw=drawColor,draw opacity=0.50,line width= 0.4pt,line join=round,line cap=round] (364.46,174.95) -- (367.31,172.10);

\path[draw=drawColor,draw opacity=0.50,line width= 0.4pt,line join=round,line cap=round] (363.87,173.53) -- (367.90,173.53);

\path[draw=drawColor,draw opacity=0.50,line width= 0.4pt,line join=round,line cap=round] (365.88,171.51) -- (365.88,175.54);

\path[draw=drawColor,draw opacity=0.50,line width= 0.4pt,line join=round,line cap=round] (366.07,171.53) -- (368.92,174.39);

\path[draw=drawColor,draw opacity=0.50,line width= 0.4pt,line join=round,line cap=round] (366.07,174.39) -- (368.92,171.53);

\path[draw=drawColor,draw opacity=0.50,line width= 0.4pt,line join=round,line cap=round] (365.48,172.96) -- (369.51,172.96);

\path[draw=drawColor,draw opacity=0.50,line width= 0.4pt,line join=round,line cap=round] (367.50,170.94) -- (367.50,174.98);

\path[draw=drawColor,draw opacity=0.50,line width= 0.4pt,line join=round,line cap=round] (366.90,172.18) -- (369.75,175.03);

\path[draw=drawColor,draw opacity=0.50,line width= 0.4pt,line join=round,line cap=round] (366.90,175.03) -- (369.75,172.18);

\path[draw=drawColor,draw opacity=0.50,line width= 0.4pt,line join=round,line cap=round] (366.31,173.60) -- (370.34,173.60);

\path[draw=drawColor,draw opacity=0.50,line width= 0.4pt,line join=round,line cap=round] (368.32,171.59) -- (368.32,175.62);

\path[draw=drawColor,draw opacity=0.50,line width= 0.4pt,line join=round,line cap=round] (364.07,173.15) -- (366.93,176.00);

\path[draw=drawColor,draw opacity=0.50,line width= 0.4pt,line join=round,line cap=round] (364.07,176.00) -- (366.93,173.15);

\path[draw=drawColor,draw opacity=0.50,line width= 0.4pt,line join=round,line cap=round] (363.48,174.58) -- (367.52,174.58);

\path[draw=drawColor,draw opacity=0.50,line width= 0.4pt,line join=round,line cap=round] (365.50,172.56) -- (365.50,176.59);

\path[draw=drawColor,draw opacity=0.50,line width= 0.4pt,line join=round,line cap=round] (366.52,172.68) -- (369.37,175.53);

\path[draw=drawColor,draw opacity=0.50,line width= 0.4pt,line join=round,line cap=round] (366.52,175.53) -- (369.37,172.68);

\path[draw=drawColor,draw opacity=0.50,line width= 0.4pt,line join=round,line cap=round] (365.93,174.10) -- (369.96,174.10);

\path[draw=drawColor,draw opacity=0.50,line width= 0.4pt,line join=round,line cap=round] (367.94,172.08) -- (367.94,176.12);

\path[draw=drawColor,draw opacity=0.50,line width= 0.4pt,line join=round,line cap=round] (365.97,169.35) -- (368.82,172.20);

\path[draw=drawColor,draw opacity=0.50,line width= 0.4pt,line join=round,line cap=round] (365.97,172.20) -- (368.82,169.35);

\path[draw=drawColor,draw opacity=0.50,line width= 0.4pt,line join=round,line cap=round] (365.38,170.77) -- (369.41,170.77);

\path[draw=drawColor,draw opacity=0.50,line width= 0.4pt,line join=round,line cap=round] (367.40,168.76) -- (367.40,172.79);

\path[draw=drawColor,draw opacity=0.50,line width= 0.4pt,line join=round,line cap=round] (362.95,172.82) -- (365.81,175.67);

\path[draw=drawColor,draw opacity=0.50,line width= 0.4pt,line join=round,line cap=round] (362.95,175.67) -- (365.81,172.82);

\path[draw=drawColor,draw opacity=0.50,line width= 0.4pt,line join=round,line cap=round] (362.36,174.25) -- (366.40,174.25);

\path[draw=drawColor,draw opacity=0.50,line width= 0.4pt,line join=round,line cap=round] (364.38,172.23) -- (364.38,176.26);

\path[draw=drawColor,draw opacity=0.50,line width= 0.4pt,line join=round,line cap=round] (363.15,169.60) -- (366.00,172.45);

\path[draw=drawColor,draw opacity=0.50,line width= 0.4pt,line join=round,line cap=round] (363.15,172.45) -- (366.00,169.60);

\path[draw=drawColor,draw opacity=0.50,line width= 0.4pt,line join=round,line cap=round] (362.56,171.02) -- (366.59,171.02);

\path[draw=drawColor,draw opacity=0.50,line width= 0.4pt,line join=round,line cap=round] (364.57,169.01) -- (364.57,173.04);

\path[draw=drawColor,draw opacity=0.50,line width= 0.4pt,line join=round,line cap=round] (366.64,171.47) -- (369.49,174.32);

\path[draw=drawColor,draw opacity=0.50,line width= 0.4pt,line join=round,line cap=round] (366.64,174.32) -- (369.49,171.47);

\path[draw=drawColor,draw opacity=0.50,line width= 0.4pt,line join=round,line cap=round] (366.05,172.89) -- (370.08,172.89);

\path[draw=drawColor,draw opacity=0.50,line width= 0.4pt,line join=round,line cap=round] (368.07,170.88) -- (368.07,174.91);

\path[draw=drawColor,draw opacity=0.50,line width= 0.4pt,line join=round,line cap=round] (365.21,171.07) -- (368.07,173.93);

\path[draw=drawColor,draw opacity=0.50,line width= 0.4pt,line join=round,line cap=round] (365.21,173.93) -- (368.07,171.07);

\path[draw=drawColor,draw opacity=0.50,line width= 0.4pt,line join=round,line cap=round] (364.62,172.50) -- (368.66,172.50);

\path[draw=drawColor,draw opacity=0.50,line width= 0.4pt,line join=round,line cap=round] (366.64,170.48) -- (366.64,174.52);

\path[draw=drawColor,draw opacity=0.50,line width= 0.4pt,line join=round,line cap=round] (367.62,171.69) -- (370.48,174.54);

\path[draw=drawColor,draw opacity=0.50,line width= 0.4pt,line join=round,line cap=round] (367.62,174.54) -- (370.48,171.69);

\path[draw=drawColor,draw opacity=0.50,line width= 0.4pt,line join=round,line cap=round] (367.03,173.12) -- (371.07,173.12);

\path[draw=drawColor,draw opacity=0.50,line width= 0.4pt,line join=round,line cap=round] (369.05,171.10) -- (369.05,175.13);

\path[draw=drawColor,draw opacity=0.50,line width= 0.4pt,line join=round,line cap=round] (365.98,175.00) -- (368.83,177.86);

\path[draw=drawColor,draw opacity=0.50,line width= 0.4pt,line join=round,line cap=round] (365.98,177.86) -- (368.83,175.00);

\path[draw=drawColor,draw opacity=0.50,line width= 0.4pt,line join=round,line cap=round] (365.39,176.43) -- (369.42,176.43);

\path[draw=drawColor,draw opacity=0.50,line width= 0.4pt,line join=round,line cap=round] (367.40,174.41) -- (367.40,178.45);

\path[draw=drawColor,draw opacity=0.50,line width= 0.4pt,line join=round,line cap=round] (369.15,171.92) -- (372.00,174.77);

\path[draw=drawColor,draw opacity=0.50,line width= 0.4pt,line join=round,line cap=round] (369.15,174.77) -- (372.00,171.92);

\path[draw=drawColor,draw opacity=0.50,line width= 0.4pt,line join=round,line cap=round] (368.56,173.34) -- (372.60,173.34);

\path[draw=drawColor,draw opacity=0.50,line width= 0.4pt,line join=round,line cap=round] (370.58,171.33) -- (370.58,175.36);

\path[draw=drawColor,draw opacity=0.50,line width= 0.4pt,line join=round,line cap=round] (365.64,171.02) -- (368.49,173.87);

\path[draw=drawColor,draw opacity=0.50,line width= 0.4pt,line join=round,line cap=round] (365.64,173.87) -- (368.49,171.02);

\path[draw=drawColor,draw opacity=0.50,line width= 0.4pt,line join=round,line cap=round] (365.05,172.45) -- (369.09,172.45);

\path[draw=drawColor,draw opacity=0.50,line width= 0.4pt,line join=round,line cap=round] (367.07,170.43) -- (367.07,174.46);

\path[draw=drawColor,draw opacity=0.50,line width= 0.4pt,line join=round,line cap=round] (364.91,172.66) -- (367.77,175.52);

\path[draw=drawColor,draw opacity=0.50,line width= 0.4pt,line join=round,line cap=round] (364.91,175.52) -- (367.77,172.66);

\path[draw=drawColor,draw opacity=0.50,line width= 0.4pt,line join=round,line cap=round] (364.32,174.09) -- (368.36,174.09);

\path[draw=drawColor,draw opacity=0.50,line width= 0.4pt,line join=round,line cap=round] (366.34,172.07) -- (366.34,176.11);

\path[draw=drawColor,draw opacity=0.50,line width= 0.4pt,line join=round,line cap=round] (365.21,171.39) -- (368.07,174.24);

\path[draw=drawColor,draw opacity=0.50,line width= 0.4pt,line join=round,line cap=round] (365.21,174.24) -- (368.07,171.39);

\path[draw=drawColor,draw opacity=0.50,line width= 0.4pt,line join=round,line cap=round] (364.62,172.81) -- (368.66,172.81);

\path[draw=drawColor,draw opacity=0.50,line width= 0.4pt,line join=round,line cap=round] (366.64,170.80) -- (366.64,174.83);

\path[draw=drawColor,draw opacity=0.50,line width= 0.4pt,line join=round,line cap=round] (366.81,172.06) -- (369.67,174.91);

\path[draw=drawColor,draw opacity=0.50,line width= 0.4pt,line join=round,line cap=round] (366.81,174.91) -- (369.67,172.06);

\path[draw=drawColor,draw opacity=0.50,line width= 0.4pt,line join=round,line cap=round] (366.22,173.49) -- (370.26,173.49);

\path[draw=drawColor,draw opacity=0.50,line width= 0.4pt,line join=round,line cap=round] (368.24,171.47) -- (368.24,175.50);

\path[draw=drawColor,draw opacity=0.50,line width= 0.4pt,line join=round,line cap=round] (367.10,171.03) -- (369.96,173.88);

\path[draw=drawColor,draw opacity=0.50,line width= 0.4pt,line join=round,line cap=round] (367.10,173.88) -- (369.96,171.03);

\path[draw=drawColor,draw opacity=0.50,line width= 0.4pt,line join=round,line cap=round] (366.51,172.46) -- (370.55,172.46);

\path[draw=drawColor,draw opacity=0.50,line width= 0.4pt,line join=round,line cap=round] (368.53,170.44) -- (368.53,174.47);

\path[draw=drawColor,draw opacity=0.50,line width= 0.4pt,line join=round,line cap=round] (367.38,173.33) -- (370.23,176.18);

\path[draw=drawColor,draw opacity=0.50,line width= 0.4pt,line join=round,line cap=round] (367.38,176.18) -- (370.23,173.33);

\path[draw=drawColor,draw opacity=0.50,line width= 0.4pt,line join=round,line cap=round] (366.79,174.75) -- (370.82,174.75);

\path[draw=drawColor,draw opacity=0.50,line width= 0.4pt,line join=round,line cap=round] (368.81,172.74) -- (368.81,176.77);

\path[draw=drawColor,draw opacity=0.50,line width= 0.4pt,line join=round,line cap=round] (364.26,171.06) -- (367.11,173.92);

\path[draw=drawColor,draw opacity=0.50,line width= 0.4pt,line join=round,line cap=round] (364.26,173.92) -- (367.11,171.06);

\path[draw=drawColor,draw opacity=0.50,line width= 0.4pt,line join=round,line cap=round] (363.67,172.49) -- (367.70,172.49);

\path[draw=drawColor,draw opacity=0.50,line width= 0.4pt,line join=round,line cap=round] (365.69,170.47) -- (365.69,174.51);

\path[draw=drawColor,draw opacity=0.50,line width= 0.4pt,line join=round,line cap=round] (365.15,173.06) -- (368.01,175.91);

\path[draw=drawColor,draw opacity=0.50,line width= 0.4pt,line join=round,line cap=round] (365.15,175.91) -- (368.01,173.06);

\path[draw=drawColor,draw opacity=0.50,line width= 0.4pt,line join=round,line cap=round] (364.56,174.48) -- (368.60,174.48);

\path[draw=drawColor,draw opacity=0.50,line width= 0.4pt,line join=round,line cap=round] (366.58,172.47) -- (366.58,176.50);

\path[draw=drawColor,draw opacity=0.50,line width= 0.4pt,line join=round,line cap=round] (370.17,175.89) -- (373.02,178.74);

\path[draw=drawColor,draw opacity=0.50,line width= 0.4pt,line join=round,line cap=round] (370.17,178.74) -- (373.02,175.89);

\path[draw=drawColor,draw opacity=0.50,line width= 0.4pt,line join=round,line cap=round] (369.57,177.32) -- (373.61,177.32);

\path[draw=drawColor,draw opacity=0.50,line width= 0.4pt,line join=round,line cap=round] (371.59,175.30) -- (371.59,179.34);

\path[draw=drawColor,draw opacity=0.50,line width= 0.4pt,line join=round,line cap=round] (370.35,174.89) -- (373.20,177.74);

\path[draw=drawColor,draw opacity=0.50,line width= 0.4pt,line join=round,line cap=round] (370.35,177.74) -- (373.20,174.89);

\path[draw=drawColor,draw opacity=0.50,line width= 0.4pt,line join=round,line cap=round] (369.76,176.31) -- (373.79,176.31);

\path[draw=drawColor,draw opacity=0.50,line width= 0.4pt,line join=round,line cap=round] (371.77,174.30) -- (371.77,178.33);

\path[draw=drawColor,draw opacity=0.50,line width= 0.4pt,line join=round,line cap=round] (365.81,173.28) -- (368.67,176.14);

\path[draw=drawColor,draw opacity=0.50,line width= 0.4pt,line join=round,line cap=round] (365.81,176.14) -- (368.67,173.28);

\path[draw=drawColor,draw opacity=0.50,line width= 0.4pt,line join=round,line cap=round] (365.22,174.71) -- (369.26,174.71);

\path[draw=drawColor,draw opacity=0.50,line width= 0.4pt,line join=round,line cap=round] (367.24,172.69) -- (367.24,176.73);

\path[draw=drawColor,draw opacity=0.50,line width= 0.4pt,line join=round,line cap=round] (366.07,174.29) -- (368.93,177.14);

\path[draw=drawColor,draw opacity=0.50,line width= 0.4pt,line join=round,line cap=round] (366.07,177.14) -- (368.93,174.29);

\path[draw=drawColor,draw opacity=0.50,line width= 0.4pt,line join=round,line cap=round] (365.48,175.72) -- (369.52,175.72);

\path[draw=drawColor,draw opacity=0.50,line width= 0.4pt,line join=round,line cap=round] (367.50,173.70) -- (367.50,177.73);

\path[draw=drawColor,draw opacity=0.50,line width= 0.4pt,line join=round,line cap=round] (366.75,173.80) -- (369.60,176.65);

\path[draw=drawColor,draw opacity=0.50,line width= 0.4pt,line join=round,line cap=round] (366.75,176.65) -- (369.60,173.80);

\path[draw=drawColor,draw opacity=0.50,line width= 0.4pt,line join=round,line cap=round] (366.16,175.23) -- (370.19,175.23);

\path[draw=drawColor,draw opacity=0.50,line width= 0.4pt,line join=round,line cap=round] (368.18,173.21) -- (368.18,177.24);

\path[draw=drawColor,draw opacity=0.50,line width= 0.4pt,line join=round,line cap=round] (366.31,175.79) -- (369.17,178.64);

\path[draw=drawColor,draw opacity=0.50,line width= 0.4pt,line join=round,line cap=round] (366.31,178.64) -- (369.17,175.79);

\path[draw=drawColor,draw opacity=0.50,line width= 0.4pt,line join=round,line cap=round] (365.72,177.21) -- (369.76,177.21);

\path[draw=drawColor,draw opacity=0.50,line width= 0.4pt,line join=round,line cap=round] (367.74,175.20) -- (367.74,179.23);

\path[draw=drawColor,draw opacity=0.50,line width= 0.4pt,line join=round,line cap=round] (369.52,172.88) -- (372.37,175.74);

\path[draw=drawColor,draw opacity=0.50,line width= 0.4pt,line join=round,line cap=round] (369.52,175.74) -- (372.37,172.88);

\path[draw=drawColor,draw opacity=0.50,line width= 0.4pt,line join=round,line cap=round] (368.93,174.31) -- (372.96,174.31);

\path[draw=drawColor,draw opacity=0.50,line width= 0.4pt,line join=round,line cap=round] (370.94,172.29) -- (370.94,176.33);

\path[draw=drawColor,draw opacity=0.50,line width= 0.4pt,line join=round,line cap=round] (371.64,174.45) -- (374.49,177.30);

\path[draw=drawColor,draw opacity=0.50,line width= 0.4pt,line join=round,line cap=round] (371.64,177.30) -- (374.49,174.45);

\path[draw=drawColor,draw opacity=0.50,line width= 0.4pt,line join=round,line cap=round] (371.05,175.87) -- (375.08,175.87);

\path[draw=drawColor,draw opacity=0.50,line width= 0.4pt,line join=round,line cap=round] (373.06,173.86) -- (373.06,177.89);

\path[draw=drawColor,draw opacity=0.50,line width= 0.4pt,line join=round,line cap=round] (371.25,173.85) -- (374.11,176.70);

\path[draw=drawColor,draw opacity=0.50,line width= 0.4pt,line join=round,line cap=round] (371.25,176.70) -- (374.11,173.85);

\path[draw=drawColor,draw opacity=0.50,line width= 0.4pt,line join=round,line cap=round] (370.66,175.28) -- (374.70,175.28);

\path[draw=drawColor,draw opacity=0.50,line width= 0.4pt,line join=round,line cap=round] (372.68,173.26) -- (372.68,177.30);

\path[draw=drawColor,draw opacity=0.50,line width= 0.4pt,line join=round,line cap=round] (368.27,179.57) -- (371.12,182.43);

\path[draw=drawColor,draw opacity=0.50,line width= 0.4pt,line join=round,line cap=round] (368.27,182.43) -- (371.12,179.57);

\path[draw=drawColor,draw opacity=0.50,line width= 0.4pt,line join=round,line cap=round] (367.68,181.00) -- (371.71,181.00);

\path[draw=drawColor,draw opacity=0.50,line width= 0.4pt,line join=round,line cap=round] (369.70,178.98) -- (369.70,183.02);

\path[draw=drawColor,draw opacity=0.50,line width= 0.4pt,line join=round,line cap=round] (371.08,174.69) -- (373.94,177.55);

\path[draw=drawColor,draw opacity=0.50,line width= 0.4pt,line join=round,line cap=round] (371.08,177.55) -- (373.94,174.69);

\path[draw=drawColor,draw opacity=0.50,line width= 0.4pt,line join=round,line cap=round] (370.49,176.12) -- (374.53,176.12);

\path[draw=drawColor,draw opacity=0.50,line width= 0.4pt,line join=round,line cap=round] (372.51,174.10) -- (372.51,178.14);

\path[draw=drawColor,draw opacity=0.50,line width= 0.4pt,line join=round,line cap=round] (368.92,174.00) -- (371.78,176.85);

\path[draw=drawColor,draw opacity=0.50,line width= 0.4pt,line join=round,line cap=round] (368.92,176.85) -- (371.78,174.00);

\path[draw=drawColor,draw opacity=0.50,line width= 0.4pt,line join=round,line cap=round] (368.33,175.42) -- (372.37,175.42);

\path[draw=drawColor,draw opacity=0.50,line width= 0.4pt,line join=round,line cap=round] (370.35,173.41) -- (370.35,177.44);

\path[draw=drawColor,draw opacity=0.50,line width= 0.4pt,line join=round,line cap=round] (367.27,172.39) -- (370.12,175.24);

\path[draw=drawColor,draw opacity=0.50,line width= 0.4pt,line join=round,line cap=round] (367.27,175.24) -- (370.12,172.39);

\path[draw=drawColor,draw opacity=0.50,line width= 0.4pt,line join=round,line cap=round] (366.67,173.81) -- (370.71,173.81);

\path[draw=drawColor,draw opacity=0.50,line width= 0.4pt,line join=round,line cap=round] (368.69,171.80) -- (368.69,175.83);

\path[draw=drawColor,draw opacity=0.50,line width= 0.4pt,line join=round,line cap=round] (369.36,172.57) -- (372.21,175.42);

\path[draw=drawColor,draw opacity=0.50,line width= 0.4pt,line join=round,line cap=round] (369.36,175.42) -- (372.21,172.57);

\path[draw=drawColor,draw opacity=0.50,line width= 0.4pt,line join=round,line cap=round] (368.77,173.99) -- (372.80,173.99);

\path[draw=drawColor,draw opacity=0.50,line width= 0.4pt,line join=round,line cap=round] (370.78,171.97) -- (370.78,176.01);

\path[draw=drawColor,draw opacity=0.50,line width= 0.4pt,line join=round,line cap=round] (367.87,173.82) -- (370.72,176.68);

\path[draw=drawColor,draw opacity=0.50,line width= 0.4pt,line join=round,line cap=round] (367.87,176.68) -- (370.72,173.82);

\path[draw=drawColor,draw opacity=0.50,line width= 0.4pt,line join=round,line cap=round] (367.28,175.25) -- (371.31,175.25);

\path[draw=drawColor,draw opacity=0.50,line width= 0.4pt,line join=round,line cap=round] (369.30,173.23) -- (369.30,177.27);

\path[draw=drawColor,draw opacity=0.50,line width= 0.4pt,line join=round,line cap=round] (369.42,172.60) -- (372.28,175.45);

\path[draw=drawColor,draw opacity=0.50,line width= 0.4pt,line join=round,line cap=round] (369.42,175.45) -- (372.28,172.60);

\path[draw=drawColor,draw opacity=0.50,line width= 0.4pt,line join=round,line cap=round] (368.83,174.02) -- (372.87,174.02);

\path[draw=drawColor,draw opacity=0.50,line width= 0.4pt,line join=round,line cap=round] (370.85,172.01) -- (370.85,176.04);

\path[draw=drawColor,draw opacity=0.50,line width= 0.4pt,line join=round,line cap=round] (371.05,175.53) -- (373.90,178.38);

\path[draw=drawColor,draw opacity=0.50,line width= 0.4pt,line join=round,line cap=round] (371.05,178.38) -- (373.90,175.53);

\path[draw=drawColor,draw opacity=0.50,line width= 0.4pt,line join=round,line cap=round] (370.46,176.95) -- (374.49,176.95);

\path[draw=drawColor,draw opacity=0.50,line width= 0.4pt,line join=round,line cap=round] (372.48,174.94) -- (372.48,178.97);

\path[draw=drawColor,draw opacity=0.50,line width= 0.4pt,line join=round,line cap=round] (367.00,175.69) -- (369.85,178.54);

\path[draw=drawColor,draw opacity=0.50,line width= 0.4pt,line join=round,line cap=round] (367.00,178.54) -- (369.85,175.69);

\path[draw=drawColor,draw opacity=0.50,line width= 0.4pt,line join=round,line cap=round] (366.41,177.11) -- (370.44,177.11);

\path[draw=drawColor,draw opacity=0.50,line width= 0.4pt,line join=round,line cap=round] (368.42,175.10) -- (368.42,179.13);

\path[draw=drawColor,draw opacity=0.50,line width= 0.4pt,line join=round,line cap=round] (368.42,177.43) -- (371.27,180.28);

\path[draw=drawColor,draw opacity=0.50,line width= 0.4pt,line join=round,line cap=round] (368.42,180.28) -- (371.27,177.43);

\path[draw=drawColor,draw opacity=0.50,line width= 0.4pt,line join=round,line cap=round] (367.83,178.85) -- (371.86,178.85);

\path[draw=drawColor,draw opacity=0.50,line width= 0.4pt,line join=round,line cap=round] (369.84,176.83) -- (369.84,180.87);

\path[draw=drawColor,draw opacity=0.50,line width= 0.4pt,line join=round,line cap=round] (368.01,173.56) -- (370.86,176.41);

\path[draw=drawColor,draw opacity=0.50,line width= 0.4pt,line join=round,line cap=round] (368.01,176.41) -- (370.86,173.56);

\path[draw=drawColor,draw opacity=0.50,line width= 0.4pt,line join=round,line cap=round] (367.41,174.99) -- (371.45,174.99);

\path[draw=drawColor,draw opacity=0.50,line width= 0.4pt,line join=round,line cap=round] (369.43,172.97) -- (369.43,177.00);

\path[draw=drawColor,draw opacity=0.50,line width= 0.4pt,line join=round,line cap=round] (392.94,192.68) -- (395.79,195.53);

\path[draw=drawColor,draw opacity=0.50,line width= 0.4pt,line join=round,line cap=round] (392.94,195.53) -- (395.79,192.68);

\path[draw=drawColor,draw opacity=0.50,line width= 0.4pt,line join=round,line cap=round] (392.35,194.10) -- (396.39,194.10);

\path[draw=drawColor,draw opacity=0.50,line width= 0.4pt,line join=round,line cap=round] (394.37,192.09) -- (394.37,196.12);

\path[draw=drawColor,draw opacity=0.50,line width= 0.4pt,line join=round,line cap=round] (392.94,192.05) -- (395.79,194.90);

\path[draw=drawColor,draw opacity=0.50,line width= 0.4pt,line join=round,line cap=round] (392.94,194.90) -- (395.79,192.05);

\path[draw=drawColor,draw opacity=0.50,line width= 0.4pt,line join=round,line cap=round] (392.35,193.48) -- (396.39,193.48);

\path[draw=drawColor,draw opacity=0.50,line width= 0.4pt,line join=round,line cap=round] (394.37,191.46) -- (394.37,195.49);

\path[draw=drawColor,draw opacity=0.50,line width= 0.4pt,line join=round,line cap=round] (392.94,194.64) -- (395.79,197.50);

\path[draw=drawColor,draw opacity=0.50,line width= 0.4pt,line join=round,line cap=round] (392.94,197.50) -- (395.79,194.64);

\path[draw=drawColor,draw opacity=0.50,line width= 0.4pt,line join=round,line cap=round] (392.35,196.07) -- (396.39,196.07);

\path[draw=drawColor,draw opacity=0.50,line width= 0.4pt,line join=round,line cap=round] (394.37,194.05) -- (394.37,198.09);

\path[draw=drawColor,draw opacity=0.50,line width= 0.4pt,line join=round,line cap=round] (384.20,190.24) -- (387.05,193.09);

\path[draw=drawColor,draw opacity=0.50,line width= 0.4pt,line join=round,line cap=round] (384.20,193.09) -- (387.05,190.24);

\path[draw=drawColor,draw opacity=0.50,line width= 0.4pt,line join=round,line cap=round] (383.61,191.67) -- (387.64,191.67);

\path[draw=drawColor,draw opacity=0.50,line width= 0.4pt,line join=round,line cap=round] (385.63,189.65) -- (385.63,193.68);

\path[draw=drawColor,draw opacity=0.50,line width= 0.4pt,line join=round,line cap=round] (392.94,191.71) -- (395.79,194.57);

\path[draw=drawColor,draw opacity=0.50,line width= 0.4pt,line join=round,line cap=round] (392.94,194.57) -- (395.79,191.71);

\path[draw=drawColor,draw opacity=0.50,line width= 0.4pt,line join=round,line cap=round] (392.35,193.14) -- (396.39,193.14);

\path[draw=drawColor,draw opacity=0.50,line width= 0.4pt,line join=round,line cap=round] (394.37,191.12) -- (394.37,195.16);

\path[draw=drawColor,draw opacity=0.50,line width= 0.4pt,line join=round,line cap=round] (392.94,191.56) -- (395.79,194.42);

\path[draw=drawColor,draw opacity=0.50,line width= 0.4pt,line join=round,line cap=round] (392.94,194.42) -- (395.79,191.56);

\path[draw=drawColor,draw opacity=0.50,line width= 0.4pt,line join=round,line cap=round] (392.35,192.99) -- (396.39,192.99);

\path[draw=drawColor,draw opacity=0.50,line width= 0.4pt,line join=round,line cap=round] (394.37,190.97) -- (394.37,195.01);

\path[draw=drawColor,draw opacity=0.50,line width= 0.4pt,line join=round,line cap=round] (392.94,191.25) -- (395.79,194.10);

\path[draw=drawColor,draw opacity=0.50,line width= 0.4pt,line join=round,line cap=round] (392.94,194.10) -- (395.79,191.25);

\path[draw=drawColor,draw opacity=0.50,line width= 0.4pt,line join=round,line cap=round] (392.35,192.67) -- (396.39,192.67);

\path[draw=drawColor,draw opacity=0.50,line width= 0.4pt,line join=round,line cap=round] (394.37,190.66) -- (394.37,194.69);

\path[draw=drawColor,draw opacity=0.50,line width= 0.4pt,line join=round,line cap=round] (392.94,194.93) -- (395.79,197.78);

\path[draw=drawColor,draw opacity=0.50,line width= 0.4pt,line join=round,line cap=round] (392.94,197.78) -- (395.79,194.93);

\path[draw=drawColor,draw opacity=0.50,line width= 0.4pt,line join=round,line cap=round] (392.35,196.36) -- (396.39,196.36);

\path[draw=drawColor,draw opacity=0.50,line width= 0.4pt,line join=round,line cap=round] (394.37,194.34) -- (394.37,198.37);

\path[draw=drawColor,draw opacity=0.50,line width= 0.4pt,line join=round,line cap=round] (385.10,192.37) -- (387.96,195.22);

\path[draw=drawColor,draw opacity=0.50,line width= 0.4pt,line join=round,line cap=round] (385.10,195.22) -- (387.96,192.37);

\path[draw=drawColor,draw opacity=0.50,line width= 0.4pt,line join=round,line cap=round] (384.51,193.79) -- (388.55,193.79);

\path[draw=drawColor,draw opacity=0.50,line width= 0.4pt,line join=round,line cap=round] (386.53,191.78) -- (386.53,195.81);

\path[draw=drawColor,draw opacity=0.50,line width= 0.4pt,line join=round,line cap=round] (383.46,188.99) -- (386.32,191.84);

\path[draw=drawColor,draw opacity=0.50,line width= 0.4pt,line join=round,line cap=round] (383.46,191.84) -- (386.32,188.99);

\path[draw=drawColor,draw opacity=0.50,line width= 0.4pt,line join=round,line cap=round] (382.87,190.42) -- (386.91,190.42);

\path[draw=drawColor,draw opacity=0.50,line width= 0.4pt,line join=round,line cap=round] (384.89,188.40) -- (384.89,192.43);

\path[draw=drawColor,draw opacity=0.50,line width= 0.4pt,line join=round,line cap=round] (385.50,193.99) -- (388.36,196.84);

\path[draw=drawColor,draw opacity=0.50,line width= 0.4pt,line join=round,line cap=round] (385.50,196.84) -- (388.36,193.99);

\path[draw=drawColor,draw opacity=0.50,line width= 0.4pt,line join=round,line cap=round] (384.91,195.41) -- (388.95,195.41);

\path[draw=drawColor,draw opacity=0.50,line width= 0.4pt,line join=round,line cap=round] (386.93,193.39) -- (386.93,197.43);

\path[draw=drawColor,draw opacity=0.50,line width= 0.4pt,line join=round,line cap=round] (392.94,191.67) -- (395.79,194.53);

\path[draw=drawColor,draw opacity=0.50,line width= 0.4pt,line join=round,line cap=round] (392.94,194.53) -- (395.79,191.67);

\path[draw=drawColor,draw opacity=0.50,line width= 0.4pt,line join=round,line cap=round] (392.35,193.10) -- (396.39,193.10);

\path[draw=drawColor,draw opacity=0.50,line width= 0.4pt,line join=round,line cap=round] (394.37,191.08) -- (394.37,195.12);

\path[draw=drawColor,draw opacity=0.50,line width= 0.4pt,line join=round,line cap=round] (392.94,191.66) -- (395.79,194.51);

\path[draw=drawColor,draw opacity=0.50,line width= 0.4pt,line join=round,line cap=round] (392.94,194.51) -- (395.79,191.66);

\path[draw=drawColor,draw opacity=0.50,line width= 0.4pt,line join=round,line cap=round] (392.35,193.08) -- (396.39,193.08);

\path[draw=drawColor,draw opacity=0.50,line width= 0.4pt,line join=round,line cap=round] (394.37,191.07) -- (394.37,195.10);

\path[draw=drawColor,draw opacity=0.50,line width= 0.4pt,line join=round,line cap=round] (392.94,191.21) -- (395.79,194.06);

\path[draw=drawColor,draw opacity=0.50,line width= 0.4pt,line join=round,line cap=round] (392.94,194.06) -- (395.79,191.21);

\path[draw=drawColor,draw opacity=0.50,line width= 0.4pt,line join=round,line cap=round] (392.35,192.63) -- (396.39,192.63);

\path[draw=drawColor,draw opacity=0.50,line width= 0.4pt,line join=round,line cap=round] (394.37,190.61) -- (394.37,194.65);

\path[draw=drawColor,draw opacity=0.50,line width= 0.4pt,line join=round,line cap=round] (392.94,192.74) -- (395.79,195.59);

\path[draw=drawColor,draw opacity=0.50,line width= 0.4pt,line join=round,line cap=round] (392.94,195.59) -- (395.79,192.74);

\path[draw=drawColor,draw opacity=0.50,line width= 0.4pt,line join=round,line cap=round] (392.35,194.16) -- (396.39,194.16);

\path[draw=drawColor,draw opacity=0.50,line width= 0.4pt,line join=round,line cap=round] (394.37,192.14) -- (394.37,196.18);

\path[draw=drawColor,draw opacity=0.50,line width= 0.4pt,line join=round,line cap=round] (383.84,193.94) -- (386.70,196.80);

\path[draw=drawColor,draw opacity=0.50,line width= 0.4pt,line join=round,line cap=round] (383.84,196.80) -- (386.70,193.94);

\path[draw=drawColor,draw opacity=0.50,line width= 0.4pt,line join=round,line cap=round] (383.25,195.37) -- (387.29,195.37);

\path[draw=drawColor,draw opacity=0.50,line width= 0.4pt,line join=round,line cap=round] (385.27,193.35) -- (385.27,197.39);

\path[draw=drawColor,draw opacity=0.50,line width= 0.4pt,line join=round,line cap=round] (392.94,193.81) -- (395.79,196.66);

\path[draw=drawColor,draw opacity=0.50,line width= 0.4pt,line join=round,line cap=round] (392.94,196.66) -- (395.79,193.81);

\path[draw=drawColor,draw opacity=0.50,line width= 0.4pt,line join=round,line cap=round] (392.35,195.24) -- (396.39,195.24);

\path[draw=drawColor,draw opacity=0.50,line width= 0.4pt,line join=round,line cap=round] (394.37,193.22) -- (394.37,197.25);

\path[draw=drawColor,draw opacity=0.50,line width= 0.4pt,line join=round,line cap=round] (392.94,190.82) -- (395.79,193.67);

\path[draw=drawColor,draw opacity=0.50,line width= 0.4pt,line join=round,line cap=round] (392.94,193.67) -- (395.79,190.82);

\path[draw=drawColor,draw opacity=0.50,line width= 0.4pt,line join=round,line cap=round] (392.35,192.25) -- (396.39,192.25);

\path[draw=drawColor,draw opacity=0.50,line width= 0.4pt,line join=round,line cap=round] (394.37,190.23) -- (394.37,194.26);

\path[draw=drawColor,draw opacity=0.50,line width= 0.4pt,line join=round,line cap=round] (392.94,193.71) -- (395.79,196.56);

\path[draw=drawColor,draw opacity=0.50,line width= 0.4pt,line join=round,line cap=round] (392.94,196.56) -- (395.79,193.71);

\path[draw=drawColor,draw opacity=0.50,line width= 0.4pt,line join=round,line cap=round] (392.35,195.14) -- (396.39,195.14);

\path[draw=drawColor,draw opacity=0.50,line width= 0.4pt,line join=round,line cap=round] (394.37,193.12) -- (394.37,197.16);

\path[draw=drawColor,draw opacity=0.50,line width= 0.4pt,line join=round,line cap=round] (392.94,192.70) -- (395.79,195.55);

\path[draw=drawColor,draw opacity=0.50,line width= 0.4pt,line join=round,line cap=round] (392.94,195.55) -- (395.79,192.70);

\path[draw=drawColor,draw opacity=0.50,line width= 0.4pt,line join=round,line cap=round] (392.35,194.12) -- (396.39,194.12);

\path[draw=drawColor,draw opacity=0.50,line width= 0.4pt,line join=round,line cap=round] (394.37,192.11) -- (394.37,196.14);

\path[draw=drawColor,draw opacity=0.50,line width= 0.4pt,line join=round,line cap=round] (392.94,197.59) -- (395.79,200.44);

\path[draw=drawColor,draw opacity=0.50,line width= 0.4pt,line join=round,line cap=round] (392.94,200.44) -- (395.79,197.59);

\path[draw=drawColor,draw opacity=0.50,line width= 0.4pt,line join=round,line cap=round] (392.35,199.01) -- (396.39,199.01);

\path[draw=drawColor,draw opacity=0.50,line width= 0.4pt,line join=round,line cap=round] (394.37,197.00) -- (394.37,201.03);

\path[draw=drawColor,draw opacity=0.50,line width= 0.4pt,line join=round,line cap=round] (392.94,193.73) -- (395.79,196.58);

\path[draw=drawColor,draw opacity=0.50,line width= 0.4pt,line join=round,line cap=round] (392.94,196.58) -- (395.79,193.73);

\path[draw=drawColor,draw opacity=0.50,line width= 0.4pt,line join=round,line cap=round] (392.35,195.16) -- (396.39,195.16);

\path[draw=drawColor,draw opacity=0.50,line width= 0.4pt,line join=round,line cap=round] (394.37,193.14) -- (394.37,197.18);

\path[draw=drawColor,draw opacity=0.50,line width= 0.4pt,line join=round,line cap=round] (392.94,192.89) -- (395.79,195.74);

\path[draw=drawColor,draw opacity=0.50,line width= 0.4pt,line join=round,line cap=round] (392.94,195.74) -- (395.79,192.89);

\path[draw=drawColor,draw opacity=0.50,line width= 0.4pt,line join=round,line cap=round] (392.35,194.32) -- (396.39,194.32);

\path[draw=drawColor,draw opacity=0.50,line width= 0.4pt,line join=round,line cap=round] (394.37,192.30) -- (394.37,196.34);

\path[draw=drawColor,draw opacity=0.50,line width= 0.4pt,line join=round,line cap=round] (392.94,193.14) -- (395.79,196.00);

\path[draw=drawColor,draw opacity=0.50,line width= 0.4pt,line join=round,line cap=round] (392.94,196.00) -- (395.79,193.14);

\path[draw=drawColor,draw opacity=0.50,line width= 0.4pt,line join=round,line cap=round] (392.35,194.57) -- (396.39,194.57);

\path[draw=drawColor,draw opacity=0.50,line width= 0.4pt,line join=round,line cap=round] (394.37,192.55) -- (394.37,196.59);

\path[draw=drawColor,draw opacity=0.50,line width= 0.4pt,line join=round,line cap=round] (392.94,193.42) -- (395.79,196.27);

\path[draw=drawColor,draw opacity=0.50,line width= 0.4pt,line join=round,line cap=round] (392.94,196.27) -- (395.79,193.42);

\path[draw=drawColor,draw opacity=0.50,line width= 0.4pt,line join=round,line cap=round] (392.35,194.84) -- (396.39,194.84);

\path[draw=drawColor,draw opacity=0.50,line width= 0.4pt,line join=round,line cap=round] (394.37,192.82) -- (394.37,196.86);

\path[draw=drawColor,draw opacity=0.50,line width= 0.4pt,line join=round,line cap=round] (392.94,190.72) -- (395.79,193.58);

\path[draw=drawColor,draw opacity=0.50,line width= 0.4pt,line join=round,line cap=round] (392.94,193.58) -- (395.79,190.72);

\path[draw=drawColor,draw opacity=0.50,line width= 0.4pt,line join=round,line cap=round] (392.35,192.15) -- (396.39,192.15);

\path[draw=drawColor,draw opacity=0.50,line width= 0.4pt,line join=round,line cap=round] (394.37,190.13) -- (394.37,194.17);

\path[draw=drawColor,draw opacity=0.50,line width= 0.4pt,line join=round,line cap=round] (385.26,193.78) -- (388.11,196.63);

\path[draw=drawColor,draw opacity=0.50,line width= 0.4pt,line join=round,line cap=round] (385.26,196.63) -- (388.11,193.78);

\path[draw=drawColor,draw opacity=0.50,line width= 0.4pt,line join=round,line cap=round] (384.67,195.21) -- (388.70,195.21);

\path[draw=drawColor,draw opacity=0.50,line width= 0.4pt,line join=round,line cap=round] (386.69,193.19) -- (386.69,197.22);

\path[draw=drawColor,draw opacity=0.50,line width= 0.4pt,line join=round,line cap=round] (392.94,192.10) -- (395.79,194.95);

\path[draw=drawColor,draw opacity=0.50,line width= 0.4pt,line join=round,line cap=round] (392.94,194.95) -- (395.79,192.10);

\path[draw=drawColor,draw opacity=0.50,line width= 0.4pt,line join=round,line cap=round] (392.35,193.52) -- (396.39,193.52);

\path[draw=drawColor,draw opacity=0.50,line width= 0.4pt,line join=round,line cap=round] (394.37,191.51) -- (394.37,195.54);

\path[draw=drawColor,draw opacity=0.50,line width= 0.4pt,line join=round,line cap=round] (392.94,192.64) -- (395.79,195.49);

\path[draw=drawColor,draw opacity=0.50,line width= 0.4pt,line join=round,line cap=round] (392.94,195.49) -- (395.79,192.64);

\path[draw=drawColor,draw opacity=0.50,line width= 0.4pt,line join=round,line cap=round] (392.35,194.07) -- (396.39,194.07);

\path[draw=drawColor,draw opacity=0.50,line width= 0.4pt,line join=round,line cap=round] (394.37,192.05) -- (394.37,196.08);

\path[draw=drawColor,draw opacity=0.50,line width= 0.4pt,line join=round,line cap=round] (392.94,191.70) -- (395.79,194.55);

\path[draw=drawColor,draw opacity=0.50,line width= 0.4pt,line join=round,line cap=round] (392.94,194.55) -- (395.79,191.70);

\path[draw=drawColor,draw opacity=0.50,line width= 0.4pt,line join=round,line cap=round] (392.35,193.13) -- (396.39,193.13);

\path[draw=drawColor,draw opacity=0.50,line width= 0.4pt,line join=round,line cap=round] (394.37,191.11) -- (394.37,195.15);

\path[draw=drawColor,draw opacity=0.50,line width= 0.4pt,line join=round,line cap=round] (385.66,192.97) -- (388.51,195.82);

\path[draw=drawColor,draw opacity=0.50,line width= 0.4pt,line join=round,line cap=round] (385.66,195.82) -- (388.51,192.97);

\path[draw=drawColor,draw opacity=0.50,line width= 0.4pt,line join=round,line cap=round] (385.07,194.39) -- (389.11,194.39);

\path[draw=drawColor,draw opacity=0.50,line width= 0.4pt,line join=round,line cap=round] (387.09,192.37) -- (387.09,196.41);

\path[draw=drawColor,draw opacity=0.50,line width= 0.4pt,line join=round,line cap=round] (385.84,194.55) -- (388.70,197.41);

\path[draw=drawColor,draw opacity=0.50,line width= 0.4pt,line join=round,line cap=round] (385.84,197.41) -- (388.70,194.55);

\path[draw=drawColor,draw opacity=0.50,line width= 0.4pt,line join=round,line cap=round] (385.25,195.98) -- (389.29,195.98);

\path[draw=drawColor,draw opacity=0.50,line width= 0.4pt,line join=round,line cap=round] (387.27,193.96) -- (387.27,198.00);

\path[draw=drawColor,draw opacity=0.50,line width= 0.4pt,line join=round,line cap=round] (392.94,196.86) -- (395.79,199.71);

\path[draw=drawColor,draw opacity=0.50,line width= 0.4pt,line join=round,line cap=round] (392.94,199.71) -- (395.79,196.86);

\path[draw=drawColor,draw opacity=0.50,line width= 0.4pt,line join=round,line cap=round] (392.35,198.29) -- (396.39,198.29);

\path[draw=drawColor,draw opacity=0.50,line width= 0.4pt,line join=round,line cap=round] (394.37,196.27) -- (394.37,200.31);

\path[draw=drawColor,draw opacity=0.50,line width= 0.4pt,line join=round,line cap=round] (392.94,196.17) -- (395.79,199.02);

\path[draw=drawColor,draw opacity=0.50,line width= 0.4pt,line join=round,line cap=round] (392.94,199.02) -- (395.79,196.17);

\path[draw=drawColor,draw opacity=0.50,line width= 0.4pt,line join=round,line cap=round] (392.35,197.60) -- (396.39,197.60);

\path[draw=drawColor,draw opacity=0.50,line width= 0.4pt,line join=round,line cap=round] (394.37,195.58) -- (394.37,199.61);

\path[draw=drawColor,draw opacity=0.50,line width= 0.4pt,line join=round,line cap=round] (392.94,196.85) -- (395.79,199.70);

\path[draw=drawColor,draw opacity=0.50,line width= 0.4pt,line join=round,line cap=round] (392.94,199.70) -- (395.79,196.85);

\path[draw=drawColor,draw opacity=0.50,line width= 0.4pt,line join=round,line cap=round] (392.35,198.28) -- (396.39,198.28);

\path[draw=drawColor,draw opacity=0.50,line width= 0.4pt,line join=round,line cap=round] (394.37,196.26) -- (394.37,200.29);

\path[draw=drawColor,draw opacity=0.50,line width= 0.4pt,line join=round,line cap=round] (392.94,193.45) -- (395.79,196.30);

\path[draw=drawColor,draw opacity=0.50,line width= 0.4pt,line join=round,line cap=round] (392.94,196.30) -- (395.79,193.45);

\path[draw=drawColor,draw opacity=0.50,line width= 0.4pt,line join=round,line cap=round] (392.35,194.87) -- (396.39,194.87);

\path[draw=drawColor,draw opacity=0.50,line width= 0.4pt,line join=round,line cap=round] (394.37,192.86) -- (394.37,196.89);

\path[draw=drawColor,draw opacity=0.50,line width= 0.4pt,line join=round,line cap=round] (392.94,198.27) -- (395.79,201.13);

\path[draw=drawColor,draw opacity=0.50,line width= 0.4pt,line join=round,line cap=round] (392.94,201.13) -- (395.79,198.27);

\path[draw=drawColor,draw opacity=0.50,line width= 0.4pt,line join=round,line cap=round] (392.35,199.70) -- (396.39,199.70);

\path[draw=drawColor,draw opacity=0.50,line width= 0.4pt,line join=round,line cap=round] (394.37,197.68) -- (394.37,201.72);

\path[draw=drawColor,draw opacity=0.50,line width= 0.4pt,line join=round,line cap=round] (392.94,194.34) -- (395.79,197.20);

\path[draw=drawColor,draw opacity=0.50,line width= 0.4pt,line join=round,line cap=round] (392.94,197.20) -- (395.79,194.34);

\path[draw=drawColor,draw opacity=0.50,line width= 0.4pt,line join=round,line cap=round] (392.35,195.77) -- (396.39,195.77);

\path[draw=drawColor,draw opacity=0.50,line width= 0.4pt,line join=round,line cap=round] (394.37,193.75) -- (394.37,197.79);

\path[draw=drawColor,draw opacity=0.50,line width= 0.4pt,line join=round,line cap=round] (385.90,191.31) -- (388.76,194.16);

\path[draw=drawColor,draw opacity=0.50,line width= 0.4pt,line join=round,line cap=round] (385.90,194.16) -- (388.76,191.31);

\path[draw=drawColor,draw opacity=0.50,line width= 0.4pt,line join=round,line cap=round] (385.31,192.73) -- (389.35,192.73);

\path[draw=drawColor,draw opacity=0.50,line width= 0.4pt,line join=round,line cap=round] (387.33,190.71) -- (387.33,194.75);

\path[draw=drawColor,draw opacity=0.50,line width= 0.4pt,line join=round,line cap=round] (385.75,195.59) -- (388.60,198.44);

\path[draw=drawColor,draw opacity=0.50,line width= 0.4pt,line join=round,line cap=round] (385.75,198.44) -- (388.60,195.59);

\path[draw=drawColor,draw opacity=0.50,line width= 0.4pt,line join=round,line cap=round] (385.16,197.02) -- (389.19,197.02);

\path[draw=drawColor,draw opacity=0.50,line width= 0.4pt,line join=round,line cap=round] (387.18,195.00) -- (387.18,199.04);

\path[draw=drawColor,draw opacity=0.50,line width= 0.4pt,line join=round,line cap=round] (392.94,196.82) -- (395.79,199.68);

\path[draw=drawColor,draw opacity=0.50,line width= 0.4pt,line join=round,line cap=round] (392.94,199.68) -- (395.79,196.82);

\path[draw=drawColor,draw opacity=0.50,line width= 0.4pt,line join=round,line cap=round] (392.35,198.25) -- (396.39,198.25);

\path[draw=drawColor,draw opacity=0.50,line width= 0.4pt,line join=round,line cap=round] (394.37,196.23) -- (394.37,200.27);

\path[draw=drawColor,draw opacity=0.50,line width= 0.4pt,line join=round,line cap=round] (392.94,192.61) -- (395.79,195.46);

\path[draw=drawColor,draw opacity=0.50,line width= 0.4pt,line join=round,line cap=round] (392.94,195.46) -- (395.79,192.61);

\path[draw=drawColor,draw opacity=0.50,line width= 0.4pt,line join=round,line cap=round] (392.35,194.03) -- (396.39,194.03);

\path[draw=drawColor,draw opacity=0.50,line width= 0.4pt,line join=round,line cap=round] (394.37,192.02) -- (394.37,196.05);

\path[draw=drawColor,draw opacity=0.50,line width= 0.4pt,line join=round,line cap=round] (392.94,193.19) -- (395.79,196.05);

\path[draw=drawColor,draw opacity=0.50,line width= 0.4pt,line join=round,line cap=round] (392.94,196.05) -- (395.79,193.19);

\path[draw=drawColor,draw opacity=0.50,line width= 0.4pt,line join=round,line cap=round] (392.35,194.62) -- (396.39,194.62);

\path[draw=drawColor,draw opacity=0.50,line width= 0.4pt,line join=round,line cap=round] (394.37,192.60) -- (394.37,196.64);

\path[draw=drawColor,draw opacity=0.50,line width= 0.4pt,line join=round,line cap=round] (392.94,193.33) -- (395.79,196.19);

\path[draw=drawColor,draw opacity=0.50,line width= 0.4pt,line join=round,line cap=round] (392.94,196.19) -- (395.79,193.33);

\path[draw=drawColor,draw opacity=0.50,line width= 0.4pt,line join=round,line cap=round] (392.35,194.76) -- (396.39,194.76);

\path[draw=drawColor,draw opacity=0.50,line width= 0.4pt,line join=round,line cap=round] (394.37,192.74) -- (394.37,196.78);

\path[draw=drawColor,draw opacity=0.50,line width= 0.4pt,line join=round,line cap=round] (392.94,193.38) -- (395.79,196.23);

\path[draw=drawColor,draw opacity=0.50,line width= 0.4pt,line join=round,line cap=round] (392.94,196.23) -- (395.79,193.38);

\path[draw=drawColor,draw opacity=0.50,line width= 0.4pt,line join=round,line cap=round] (392.35,194.81) -- (396.39,194.81);

\path[draw=drawColor,draw opacity=0.50,line width= 0.4pt,line join=round,line cap=round] (394.37,192.79) -- (394.37,196.82);

\path[draw=drawColor,draw opacity=0.50,line width= 0.4pt,line join=round,line cap=round] (392.94,195.52) -- (395.79,198.37);

\path[draw=drawColor,draw opacity=0.50,line width= 0.4pt,line join=round,line cap=round] (392.94,198.37) -- (395.79,195.52);

\path[draw=drawColor,draw opacity=0.50,line width= 0.4pt,line join=round,line cap=round] (392.35,196.94) -- (396.39,196.94);

\path[draw=drawColor,draw opacity=0.50,line width= 0.4pt,line join=round,line cap=round] (394.37,194.93) -- (394.37,198.96);

\path[draw=drawColor,draw opacity=0.50,line width= 0.4pt,line join=round,line cap=round] (392.94,195.64) -- (395.79,198.49);

\path[draw=drawColor,draw opacity=0.50,line width= 0.4pt,line join=round,line cap=round] (392.94,198.49) -- (395.79,195.64);

\path[draw=drawColor,draw opacity=0.50,line width= 0.4pt,line join=round,line cap=round] (392.35,197.07) -- (396.39,197.07);

\path[draw=drawColor,draw opacity=0.50,line width= 0.4pt,line join=round,line cap=round] (394.37,195.05) -- (394.37,199.08);

\path[draw=drawColor,draw opacity=0.50,line width= 0.4pt,line join=round,line cap=round] (392.94,199.49) -- (395.79,202.34);

\path[draw=drawColor,draw opacity=0.50,line width= 0.4pt,line join=round,line cap=round] (392.94,202.34) -- (395.79,199.49);

\path[draw=drawColor,draw opacity=0.50,line width= 0.4pt,line join=round,line cap=round] (392.35,200.91) -- (396.39,200.91);

\path[draw=drawColor,draw opacity=0.50,line width= 0.4pt,line join=round,line cap=round] (394.37,198.89) -- (394.37,202.93);

\path[draw=drawColor,draw opacity=0.50,line width= 0.4pt,line join=round,line cap=round] (392.94,204.00) -- (395.79,206.86);

\path[draw=drawColor,draw opacity=0.50,line width= 0.4pt,line join=round,line cap=round] (392.94,206.86) -- (395.79,204.00);

\path[draw=drawColor,draw opacity=0.50,line width= 0.4pt,line join=round,line cap=round] (392.35,205.43) -- (396.39,205.43);

\path[draw=drawColor,draw opacity=0.50,line width= 0.4pt,line join=round,line cap=round] (394.37,203.41) -- (394.37,207.45);

\path[draw=drawColor,draw opacity=0.50,line width= 0.4pt,line join=round,line cap=round] (392.94,211.01) -- (395.79,213.87);

\path[draw=drawColor,draw opacity=0.50,line width= 0.4pt,line join=round,line cap=round] (392.94,213.87) -- (395.79,211.01);

\path[draw=drawColor,draw opacity=0.50,line width= 0.4pt,line join=round,line cap=round] (392.35,212.44) -- (396.39,212.44);

\path[draw=drawColor,draw opacity=0.50,line width= 0.4pt,line join=round,line cap=round] (394.37,210.42) -- (394.37,214.46);

\path[draw=drawColor,draw opacity=0.50,line width= 0.4pt,line join=round,line cap=round] (392.94,204.24) -- (395.79,207.09);

\path[draw=drawColor,draw opacity=0.50,line width= 0.4pt,line join=round,line cap=round] (392.94,207.09) -- (395.79,204.24);

\path[draw=drawColor,draw opacity=0.50,line width= 0.4pt,line join=round,line cap=round] (392.35,205.66) -- (396.39,205.66);

\path[draw=drawColor,draw opacity=0.50,line width= 0.4pt,line join=round,line cap=round] (394.37,203.64) -- (394.37,207.68);

\path[draw=drawColor,draw opacity=0.50,line width= 0.4pt,line join=round,line cap=round] (392.94,199.83) -- (395.79,202.68);

\path[draw=drawColor,draw opacity=0.50,line width= 0.4pt,line join=round,line cap=round] (392.94,202.68) -- (395.79,199.83);

\path[draw=drawColor,draw opacity=0.50,line width= 0.4pt,line join=round,line cap=round] (392.35,201.26) -- (396.39,201.26);

\path[draw=drawColor,draw opacity=0.50,line width= 0.4pt,line join=round,line cap=round] (394.37,199.24) -- (394.37,203.28);

\path[draw=drawColor,draw opacity=0.50,line width= 0.4pt,line join=round,line cap=round] (392.94,201.48) -- (395.79,204.33);

\path[draw=drawColor,draw opacity=0.50,line width= 0.4pt,line join=round,line cap=round] (392.94,204.33) -- (395.79,201.48);

\path[draw=drawColor,draw opacity=0.50,line width= 0.4pt,line join=round,line cap=round] (392.35,202.91) -- (396.39,202.91);

\path[draw=drawColor,draw opacity=0.50,line width= 0.4pt,line join=round,line cap=round] (394.37,200.89) -- (394.37,204.92);

\path[draw=drawColor,draw opacity=0.50,line width= 0.4pt,line join=round,line cap=round] (392.94,201.69) -- (395.79,204.54);

\path[draw=drawColor,draw opacity=0.50,line width= 0.4pt,line join=round,line cap=round] (392.94,204.54) -- (395.79,201.69);

\path[draw=drawColor,draw opacity=0.50,line width= 0.4pt,line join=round,line cap=round] (392.35,203.11) -- (396.39,203.11);

\path[draw=drawColor,draw opacity=0.50,line width= 0.4pt,line join=round,line cap=round] (394.37,201.09) -- (394.37,205.13);

\path[draw=drawColor,draw opacity=0.50,line width= 0.4pt,line join=round,line cap=round] (392.94,199.25) -- (395.79,202.10);

\path[draw=drawColor,draw opacity=0.50,line width= 0.4pt,line join=round,line cap=round] (392.94,202.10) -- (395.79,199.25);

\path[draw=drawColor,draw opacity=0.50,line width= 0.4pt,line join=round,line cap=round] (392.35,200.68) -- (396.39,200.68);

\path[draw=drawColor,draw opacity=0.50,line width= 0.4pt,line join=round,line cap=round] (394.37,198.66) -- (394.37,202.70);

\path[draw=drawColor,draw opacity=0.50,line width= 0.4pt,line join=round,line cap=round] (392.94,197.37) -- (395.79,200.22);

\path[draw=drawColor,draw opacity=0.50,line width= 0.4pt,line join=round,line cap=round] (392.94,200.22) -- (395.79,197.37);

\path[draw=drawColor,draw opacity=0.50,line width= 0.4pt,line join=round,line cap=round] (392.35,198.80) -- (396.39,198.80);

\path[draw=drawColor,draw opacity=0.50,line width= 0.4pt,line join=round,line cap=round] (394.37,196.78) -- (394.37,200.81);

\path[draw=drawColor,draw opacity=0.50,line width= 0.4pt,line join=round,line cap=round] (392.94,197.28) -- (395.79,200.14);

\path[draw=drawColor,draw opacity=0.50,line width= 0.4pt,line join=round,line cap=round] (392.94,200.14) -- (395.79,197.28);

\path[draw=drawColor,draw opacity=0.50,line width= 0.4pt,line join=round,line cap=round] (392.35,198.71) -- (396.39,198.71);

\path[draw=drawColor,draw opacity=0.50,line width= 0.4pt,line join=round,line cap=round] (394.37,196.69) -- (394.37,200.73);

\path[draw=drawColor,draw opacity=0.50,line width= 0.4pt,line join=round,line cap=round] (392.94,202.36) -- (395.79,205.21);

\path[draw=drawColor,draw opacity=0.50,line width= 0.4pt,line join=round,line cap=round] (392.94,205.21) -- (395.79,202.36);

\path[draw=drawColor,draw opacity=0.50,line width= 0.4pt,line join=round,line cap=round] (392.35,203.78) -- (396.39,203.78);

\path[draw=drawColor,draw opacity=0.50,line width= 0.4pt,line join=round,line cap=round] (394.37,201.77) -- (394.37,205.80);

\path[draw=drawColor,draw opacity=0.50,line width= 0.4pt,line join=round,line cap=round] (392.94,200.25) -- (395.79,203.11);

\path[draw=drawColor,draw opacity=0.50,line width= 0.4pt,line join=round,line cap=round] (392.94,203.11) -- (395.79,200.25);

\path[draw=drawColor,draw opacity=0.50,line width= 0.4pt,line join=round,line cap=round] (392.35,201.68) -- (396.39,201.68);

\path[draw=drawColor,draw opacity=0.50,line width= 0.4pt,line join=round,line cap=round] (394.37,199.66) -- (394.37,203.70);

\path[draw=drawColor,draw opacity=0.50,line width= 0.4pt,line join=round,line cap=round] (392.94,198.02) -- (395.79,200.88);

\path[draw=drawColor,draw opacity=0.50,line width= 0.4pt,line join=round,line cap=round] (392.94,200.88) -- (395.79,198.02);

\path[draw=drawColor,draw opacity=0.50,line width= 0.4pt,line join=round,line cap=round] (392.35,199.45) -- (396.39,199.45);

\path[draw=drawColor,draw opacity=0.50,line width= 0.4pt,line join=round,line cap=round] (394.37,197.43) -- (394.37,201.47);

\path[draw=drawColor,draw opacity=0.50,line width= 0.4pt,line join=round,line cap=round] (392.94,204.14) -- (395.79,206.99);

\path[draw=drawColor,draw opacity=0.50,line width= 0.4pt,line join=round,line cap=round] (392.94,206.99) -- (395.79,204.14);

\path[draw=drawColor,draw opacity=0.50,line width= 0.4pt,line join=round,line cap=round] (392.35,205.56) -- (396.39,205.56);

\path[draw=drawColor,draw opacity=0.50,line width= 0.4pt,line join=round,line cap=round] (394.37,203.54) -- (394.37,207.58);

\path[draw=drawColor,draw opacity=0.50,line width= 0.4pt,line join=round,line cap=round] (392.94,199.46) -- (395.79,202.31);

\path[draw=drawColor,draw opacity=0.50,line width= 0.4pt,line join=round,line cap=round] (392.94,202.31) -- (395.79,199.46);

\path[draw=drawColor,draw opacity=0.50,line width= 0.4pt,line join=round,line cap=round] (392.35,200.89) -- (396.39,200.89);

\path[draw=drawColor,draw opacity=0.50,line width= 0.4pt,line join=round,line cap=round] (394.37,198.87) -- (394.37,202.90);

\path[draw=drawColor,draw opacity=0.50,line width= 0.4pt,line join=round,line cap=round] (392.94,203.22) -- (395.79,206.07);

\path[draw=drawColor,draw opacity=0.50,line width= 0.4pt,line join=round,line cap=round] (392.94,206.07) -- (395.79,203.22);

\path[draw=drawColor,draw opacity=0.50,line width= 0.4pt,line join=round,line cap=round] (392.35,204.64) -- (396.39,204.64);

\path[draw=drawColor,draw opacity=0.50,line width= 0.4pt,line join=round,line cap=round] (394.37,202.62) -- (394.37,206.66);

\path[draw=drawColor,draw opacity=0.50,line width= 0.4pt,line join=round,line cap=round] (392.94,203.79) -- (395.79,206.65);

\path[draw=drawColor,draw opacity=0.50,line width= 0.4pt,line join=round,line cap=round] (392.94,206.65) -- (395.79,203.79);

\path[draw=drawColor,draw opacity=0.50,line width= 0.4pt,line join=round,line cap=round] (392.35,205.22) -- (396.39,205.22);

\path[draw=drawColor,draw opacity=0.50,line width= 0.4pt,line join=round,line cap=round] (394.37,203.20) -- (394.37,207.24);

\path[draw=drawColor,draw opacity=0.50,line width= 0.4pt,line join=round,line cap=round] (392.94,203.73) -- (395.79,206.58);

\path[draw=drawColor,draw opacity=0.50,line width= 0.4pt,line join=round,line cap=round] (392.94,206.58) -- (395.79,203.73);

\path[draw=drawColor,draw opacity=0.50,line width= 0.4pt,line join=round,line cap=round] (392.35,205.15) -- (396.39,205.15);

\path[draw=drawColor,draw opacity=0.50,line width= 0.4pt,line join=round,line cap=round] (394.37,203.14) -- (394.37,207.17);

\path[draw=drawColor,draw opacity=0.50,line width= 0.4pt,line join=round,line cap=round] (392.94,211.01) -- (395.79,213.87);

\path[draw=drawColor,draw opacity=0.50,line width= 0.4pt,line join=round,line cap=round] (392.94,213.87) -- (395.79,211.01);

\path[draw=drawColor,draw opacity=0.50,line width= 0.4pt,line join=round,line cap=round] (392.35,212.44) -- (396.39,212.44);

\path[draw=drawColor,draw opacity=0.50,line width= 0.4pt,line join=round,line cap=round] (394.37,210.42) -- (394.37,214.46);

\path[draw=drawColor,draw opacity=0.50,line width= 0.4pt,line join=round,line cap=round] (392.94,211.01) -- (395.79,213.87);

\path[draw=drawColor,draw opacity=0.50,line width= 0.4pt,line join=round,line cap=round] (392.94,213.87) -- (395.79,211.01);

\path[draw=drawColor,draw opacity=0.50,line width= 0.4pt,line join=round,line cap=round] (392.35,212.44) -- (396.39,212.44);

\path[draw=drawColor,draw opacity=0.50,line width= 0.4pt,line join=round,line cap=round] (394.37,210.42) -- (394.37,214.46);

\path[draw=drawColor,draw opacity=0.50,line width= 0.4pt,line join=round,line cap=round] (392.94,211.01) -- (395.79,213.87);

\path[draw=drawColor,draw opacity=0.50,line width= 0.4pt,line join=round,line cap=round] (392.94,213.87) -- (395.79,211.01);

\path[draw=drawColor,draw opacity=0.50,line width= 0.4pt,line join=round,line cap=round] (392.35,212.44) -- (396.39,212.44);

\path[draw=drawColor,draw opacity=0.50,line width= 0.4pt,line join=round,line cap=round] (394.37,210.42) -- (394.37,214.46);

\path[draw=drawColor,draw opacity=0.50,line width= 0.4pt,line join=round,line cap=round] (392.94,211.01) -- (395.79,213.87);

\path[draw=drawColor,draw opacity=0.50,line width= 0.4pt,line join=round,line cap=round] (392.94,213.87) -- (395.79,211.01);

\path[draw=drawColor,draw opacity=0.50,line width= 0.4pt,line join=round,line cap=round] (392.35,212.44) -- (396.39,212.44);

\path[draw=drawColor,draw opacity=0.50,line width= 0.4pt,line join=round,line cap=round] (394.37,210.42) -- (394.37,214.46);

\path[draw=drawColor,draw opacity=0.50,line width= 0.4pt,line join=round,line cap=round] (392.94,211.01) -- (395.79,213.87);

\path[draw=drawColor,draw opacity=0.50,line width= 0.4pt,line join=round,line cap=round] (392.94,213.87) -- (395.79,211.01);

\path[draw=drawColor,draw opacity=0.50,line width= 0.4pt,line join=round,line cap=round] (392.35,212.44) -- (396.39,212.44);

\path[draw=drawColor,draw opacity=0.50,line width= 0.4pt,line join=round,line cap=round] (394.37,210.42) -- (394.37,214.46);

\path[draw=drawColor,draw opacity=0.50,line width= 0.4pt,line join=round,line cap=round] (392.94,202.52) -- (395.79,205.37);

\path[draw=drawColor,draw opacity=0.50,line width= 0.4pt,line join=round,line cap=round] (392.94,205.37) -- (395.79,202.52);

\path[draw=drawColor,draw opacity=0.50,line width= 0.4pt,line join=round,line cap=round] (392.35,203.94) -- (396.39,203.94);

\path[draw=drawColor,draw opacity=0.50,line width= 0.4pt,line join=round,line cap=round] (394.37,201.93) -- (394.37,205.96);

\path[draw=drawColor,draw opacity=0.50,line width= 0.4pt,line join=round,line cap=round] (392.94,198.34) -- (395.79,201.19);

\path[draw=drawColor,draw opacity=0.50,line width= 0.4pt,line join=round,line cap=round] (392.94,201.19) -- (395.79,198.34);

\path[draw=drawColor,draw opacity=0.50,line width= 0.4pt,line join=round,line cap=round] (392.35,199.77) -- (396.39,199.77);

\path[draw=drawColor,draw opacity=0.50,line width= 0.4pt,line join=round,line cap=round] (394.37,197.75) -- (394.37,201.79);

\path[draw=drawColor,draw opacity=0.50,line width= 0.4pt,line join=round,line cap=round] (392.94,201.36) -- (395.79,204.21);

\path[draw=drawColor,draw opacity=0.50,line width= 0.4pt,line join=round,line cap=round] (392.94,204.21) -- (395.79,201.36);

\path[draw=drawColor,draw opacity=0.50,line width= 0.4pt,line join=round,line cap=round] (392.35,202.78) -- (396.39,202.78);

\path[draw=drawColor,draw opacity=0.50,line width= 0.4pt,line join=round,line cap=round] (394.37,200.77) -- (394.37,204.80);

\path[draw=drawColor,draw opacity=0.50,line width= 0.4pt,line join=round,line cap=round] (392.94,198.80) -- (395.79,201.66);

\path[draw=drawColor,draw opacity=0.50,line width= 0.4pt,line join=round,line cap=round] (392.94,201.66) -- (395.79,198.80);

\path[draw=drawColor,draw opacity=0.50,line width= 0.4pt,line join=round,line cap=round] (392.35,200.23) -- (396.39,200.23);

\path[draw=drawColor,draw opacity=0.50,line width= 0.4pt,line join=round,line cap=round] (394.37,198.21) -- (394.37,202.25);

\path[draw=drawColor,draw opacity=0.50,line width= 0.4pt,line join=round,line cap=round] (392.94,201.58) -- (395.79,204.43);

\path[draw=drawColor,draw opacity=0.50,line width= 0.4pt,line join=round,line cap=round] (392.94,204.43) -- (395.79,201.58);

\path[draw=drawColor,draw opacity=0.50,line width= 0.4pt,line join=round,line cap=round] (392.35,203.01) -- (396.39,203.01);

\path[draw=drawColor,draw opacity=0.50,line width= 0.4pt,line join=round,line cap=round] (394.37,200.99) -- (394.37,205.03);

\path[draw=drawColor,draw opacity=0.50,line width= 0.4pt,line join=round,line cap=round] (392.94,203.58) -- (395.79,206.44);

\path[draw=drawColor,draw opacity=0.50,line width= 0.4pt,line join=round,line cap=round] (392.94,206.44) -- (395.79,203.58);

\path[draw=drawColor,draw opacity=0.50,line width= 0.4pt,line join=round,line cap=round] (392.35,205.01) -- (396.39,205.01);

\path[draw=drawColor,draw opacity=0.50,line width= 0.4pt,line join=round,line cap=round] (394.37,202.99) -- (394.37,207.03);

\path[draw=drawColor,draw opacity=0.50,line width= 0.4pt,line join=round,line cap=round] (392.94,211.01) -- (395.79,213.87);

\path[draw=drawColor,draw opacity=0.50,line width= 0.4pt,line join=round,line cap=round] (392.94,213.87) -- (395.79,211.01);

\path[draw=drawColor,draw opacity=0.50,line width= 0.4pt,line join=round,line cap=round] (392.35,212.44) -- (396.39,212.44);

\path[draw=drawColor,draw opacity=0.50,line width= 0.4pt,line join=round,line cap=round] (394.37,210.42) -- (394.37,214.46);

\path[draw=drawColor,draw opacity=0.50,line width= 0.4pt,line join=round,line cap=round] (392.94,204.45) -- (395.79,207.30);

\path[draw=drawColor,draw opacity=0.50,line width= 0.4pt,line join=round,line cap=round] (392.94,207.30) -- (395.79,204.45);

\path[draw=drawColor,draw opacity=0.50,line width= 0.4pt,line join=round,line cap=round] (392.35,205.88) -- (396.39,205.88);

\path[draw=drawColor,draw opacity=0.50,line width= 0.4pt,line join=round,line cap=round] (394.37,203.86) -- (394.37,207.89);

\path[draw=drawColor,draw opacity=0.50,line width= 0.4pt,line join=round,line cap=round] (392.94,211.01) -- (395.79,213.87);

\path[draw=drawColor,draw opacity=0.50,line width= 0.4pt,line join=round,line cap=round] (392.94,213.87) -- (395.79,211.01);

\path[draw=drawColor,draw opacity=0.50,line width= 0.4pt,line join=round,line cap=round] (392.35,212.44) -- (396.39,212.44);

\path[draw=drawColor,draw opacity=0.50,line width= 0.4pt,line join=round,line cap=round] (394.37,210.42) -- (394.37,214.46);

\path[draw=drawColor,draw opacity=0.50,line width= 0.4pt,line join=round,line cap=round] (392.94,203.92) -- (395.79,206.77);

\path[draw=drawColor,draw opacity=0.50,line width= 0.4pt,line join=round,line cap=round] (392.94,206.77) -- (395.79,203.92);

\path[draw=drawColor,draw opacity=0.50,line width= 0.4pt,line join=round,line cap=round] (392.35,205.34) -- (396.39,205.34);

\path[draw=drawColor,draw opacity=0.50,line width= 0.4pt,line join=round,line cap=round] (394.37,203.33) -- (394.37,207.36);

\path[draw=drawColor,draw opacity=0.50,line width= 0.4pt,line join=round,line cap=round] (392.94,211.01) -- (395.79,213.87);

\path[draw=drawColor,draw opacity=0.50,line width= 0.4pt,line join=round,line cap=round] (392.94,213.87) -- (395.79,211.01);

\path[draw=drawColor,draw opacity=0.50,line width= 0.4pt,line join=round,line cap=round] (392.35,212.44) -- (396.39,212.44);

\path[draw=drawColor,draw opacity=0.50,line width= 0.4pt,line join=round,line cap=round] (394.37,210.42) -- (394.37,214.46);

\path[draw=drawColor,draw opacity=0.50,line width= 0.4pt,line join=round,line cap=round] (392.94,203.40) -- (395.79,206.25);

\path[draw=drawColor,draw opacity=0.50,line width= 0.4pt,line join=round,line cap=round] (392.94,206.25) -- (395.79,203.40);

\path[draw=drawColor,draw opacity=0.50,line width= 0.4pt,line join=round,line cap=round] (392.35,204.82) -- (396.39,204.82);

\path[draw=drawColor,draw opacity=0.50,line width= 0.4pt,line join=round,line cap=round] (394.37,202.81) -- (394.37,206.84);

\path[draw=drawColor,draw opacity=0.50,line width= 0.4pt,line join=round,line cap=round] (392.94,202.64) -- (395.79,205.50);

\path[draw=drawColor,draw opacity=0.50,line width= 0.4pt,line join=round,line cap=round] (392.94,205.50) -- (395.79,202.64);

\path[draw=drawColor,draw opacity=0.50,line width= 0.4pt,line join=round,line cap=round] (392.35,204.07) -- (396.39,204.07);

\path[draw=drawColor,draw opacity=0.50,line width= 0.4pt,line join=round,line cap=round] (394.37,202.05) -- (394.37,206.09);

\path[draw=drawColor,draw opacity=0.50,line width= 0.4pt,line join=round,line cap=round] (392.94,202.53) -- (395.79,205.38);

\path[draw=drawColor,draw opacity=0.50,line width= 0.4pt,line join=round,line cap=round] (392.94,205.38) -- (395.79,202.53);

\path[draw=drawColor,draw opacity=0.50,line width= 0.4pt,line join=round,line cap=round] (392.35,203.96) -- (396.39,203.96);

\path[draw=drawColor,draw opacity=0.50,line width= 0.4pt,line join=round,line cap=round] (394.37,201.94) -- (394.37,205.97);

\path[draw=drawColor,draw opacity=0.50,line width= 0.4pt,line join=round,line cap=round] (392.94,204.60) -- (395.79,207.46);

\path[draw=drawColor,draw opacity=0.50,line width= 0.4pt,line join=round,line cap=round] (392.94,207.46) -- (395.79,204.60);

\path[draw=drawColor,draw opacity=0.50,line width= 0.4pt,line join=round,line cap=round] (392.35,206.03) -- (396.39,206.03);

\path[draw=drawColor,draw opacity=0.50,line width= 0.4pt,line join=round,line cap=round] (394.37,204.01) -- (394.37,208.05);

\path[draw=drawColor,draw opacity=0.50,line width= 0.4pt,line join=round,line cap=round] (392.94,204.42) -- (395.79,207.27);

\path[draw=drawColor,draw opacity=0.50,line width= 0.4pt,line join=round,line cap=round] (392.94,207.27) -- (395.79,204.42);

\path[draw=drawColor,draw opacity=0.50,line width= 0.4pt,line join=round,line cap=round] (392.35,205.85) -- (396.39,205.85);

\path[draw=drawColor,draw opacity=0.50,line width= 0.4pt,line join=round,line cap=round] (394.37,203.83) -- (394.37,207.87);

\path[draw=drawColor,draw opacity=0.50,line width= 0.4pt,line join=round,line cap=round] (392.94,202.81) -- (395.79,205.67);

\path[draw=drawColor,draw opacity=0.50,line width= 0.4pt,line join=round,line cap=round] (392.94,205.67) -- (395.79,202.81);

\path[draw=drawColor,draw opacity=0.50,line width= 0.4pt,line join=round,line cap=round] (392.35,204.24) -- (396.39,204.24);

\path[draw=drawColor,draw opacity=0.50,line width= 0.4pt,line join=round,line cap=round] (394.37,202.22) -- (394.37,206.26);

\path[draw=drawColor,draw opacity=0.50,line width= 0.4pt,line join=round,line cap=round] (392.94,199.25) -- (395.79,202.10);

\path[draw=drawColor,draw opacity=0.50,line width= 0.4pt,line join=round,line cap=round] (392.94,202.10) -- (395.79,199.25);

\path[draw=drawColor,draw opacity=0.50,line width= 0.4pt,line join=round,line cap=round] (392.35,200.67) -- (396.39,200.67);

\path[draw=drawColor,draw opacity=0.50,line width= 0.4pt,line join=round,line cap=round] (394.37,198.66) -- (394.37,202.69);

\path[draw=drawColor,draw opacity=0.50,line width= 0.4pt,line join=round,line cap=round] (392.94,203.50) -- (395.79,206.35);

\path[draw=drawColor,draw opacity=0.50,line width= 0.4pt,line join=round,line cap=round] (392.94,206.35) -- (395.79,203.50);

\path[draw=drawColor,draw opacity=0.50,line width= 0.4pt,line join=round,line cap=round] (392.35,204.92) -- (396.39,204.92);

\path[draw=drawColor,draw opacity=0.50,line width= 0.4pt,line join=round,line cap=round] (394.37,202.91) -- (394.37,206.94);

\path[draw=drawColor,draw opacity=0.50,line width= 0.4pt,line join=round,line cap=round] (392.94,198.48) -- (395.79,201.33);

\path[draw=drawColor,draw opacity=0.50,line width= 0.4pt,line join=round,line cap=round] (392.94,201.33) -- (395.79,198.48);

\path[draw=drawColor,draw opacity=0.50,line width= 0.4pt,line join=round,line cap=round] (392.35,199.91) -- (396.39,199.91);

\path[draw=drawColor,draw opacity=0.50,line width= 0.4pt,line join=round,line cap=round] (394.37,197.89) -- (394.37,201.92);

\path[draw=drawColor,draw opacity=0.50,line width= 0.4pt,line join=round,line cap=round] (392.94,202.35) -- (395.79,205.20);

\path[draw=drawColor,draw opacity=0.50,line width= 0.4pt,line join=round,line cap=round] (392.94,205.20) -- (395.79,202.35);

\path[draw=drawColor,draw opacity=0.50,line width= 0.4pt,line join=round,line cap=round] (392.35,203.77) -- (396.39,203.77);

\path[draw=drawColor,draw opacity=0.50,line width= 0.4pt,line join=round,line cap=round] (394.37,201.76) -- (394.37,205.79);

\path[draw=drawColor,draw opacity=0.50,line width= 0.4pt,line join=round,line cap=round] (392.94,200.05) -- (395.79,202.91);

\path[draw=drawColor,draw opacity=0.50,line width= 0.4pt,line join=round,line cap=round] (392.94,202.91) -- (395.79,200.05);

\path[draw=drawColor,draw opacity=0.50,line width= 0.4pt,line join=round,line cap=round] (392.35,201.48) -- (396.39,201.48);

\path[draw=drawColor,draw opacity=0.50,line width= 0.4pt,line join=round,line cap=round] (394.37,199.46) -- (394.37,203.50);

\path[draw=drawColor,draw opacity=0.50,line width= 0.4pt,line join=round,line cap=round] (392.94,211.01) -- (395.79,213.87);

\path[draw=drawColor,draw opacity=0.50,line width= 0.4pt,line join=round,line cap=round] (392.94,213.87) -- (395.79,211.01);

\path[draw=drawColor,draw opacity=0.50,line width= 0.4pt,line join=round,line cap=round] (392.35,212.44) -- (396.39,212.44);

\path[draw=drawColor,draw opacity=0.50,line width= 0.4pt,line join=round,line cap=round] (394.37,210.42) -- (394.37,214.46);

\path[draw=drawColor,draw opacity=0.50,line width= 0.4pt,line join=round,line cap=round] (392.94,202.96) -- (395.79,205.82);

\path[draw=drawColor,draw opacity=0.50,line width= 0.4pt,line join=round,line cap=round] (392.94,205.82) -- (395.79,202.96);

\path[draw=drawColor,draw opacity=0.50,line width= 0.4pt,line join=round,line cap=round] (392.35,204.39) -- (396.39,204.39);

\path[draw=drawColor,draw opacity=0.50,line width= 0.4pt,line join=round,line cap=round] (394.37,202.37) -- (394.37,206.41);

\path[draw=drawColor,draw opacity=0.50,line width= 0.4pt,line join=round,line cap=round] (392.94,206.06) -- (395.79,208.91);

\path[draw=drawColor,draw opacity=0.50,line width= 0.4pt,line join=round,line cap=round] (392.94,208.91) -- (395.79,206.06);

\path[draw=drawColor,draw opacity=0.50,line width= 0.4pt,line join=round,line cap=round] (392.35,207.49) -- (396.39,207.49);

\path[draw=drawColor,draw opacity=0.50,line width= 0.4pt,line join=round,line cap=round] (394.37,205.47) -- (394.37,209.50);

\path[draw=drawColor,draw opacity=0.50,line width= 0.4pt,line join=round,line cap=round] (392.94,205.30) -- (395.79,208.16);

\path[draw=drawColor,draw opacity=0.50,line width= 0.4pt,line join=round,line cap=round] (392.94,208.16) -- (395.79,205.30);

\path[draw=drawColor,draw opacity=0.50,line width= 0.4pt,line join=round,line cap=round] (392.35,206.73) -- (396.39,206.73);

\path[draw=drawColor,draw opacity=0.50,line width= 0.4pt,line join=round,line cap=round] (394.37,204.71) -- (394.37,208.75);
\definecolor{fillColor}{RGB}{117,112,179}

\path[fill=fillColor,fill opacity=0.50] (304.78,115.88) --
	(307.63,115.88) --
	(307.63,118.73) --
	(304.78,118.73) --
	cycle;

\path[fill=fillColor,fill opacity=0.50] (306.86,117.43) --
	(309.72,117.43) --
	(309.72,120.29) --
	(306.86,120.29) --
	cycle;

\path[fill=fillColor,fill opacity=0.50] (306.86,125.31) --
	(309.72,125.31) --
	(309.72,128.16) --
	(306.86,128.16) --
	cycle;

\path[fill=fillColor,fill opacity=0.50] (303.83,113.84) --
	(306.68,113.84) --
	(306.68,116.70) --
	(303.83,116.70) --
	cycle;

\path[fill=fillColor,fill opacity=0.50] (306.18,115.67) --
	(309.03,115.67) --
	(309.03,118.52) --
	(306.18,118.52) --
	cycle;

\path[fill=fillColor,fill opacity=0.50] (305.76,122.27) --
	(308.62,122.27) --
	(308.62,125.12) --
	(305.76,125.12) --
	cycle;

\path[fill=fillColor,fill opacity=0.50] (307.51,116.49) --
	(310.36,116.49) --
	(310.36,119.34) --
	(307.51,119.34) --
	cycle;

\path[fill=fillColor,fill opacity=0.50] (305.12,113.59) --
	(307.97,113.59) --
	(307.97,116.45) --
	(305.12,116.45) --
	cycle;

\path[fill=fillColor,fill opacity=0.50] (304.78,113.59) --
	(307.63,113.59) --
	(307.63,116.45) --
	(304.78,116.45) --
	cycle;

\path[fill=fillColor,fill opacity=0.50] (307.33,115.46) --
	(310.18,115.46) --
	(310.18,118.31) --
	(307.33,118.31) --
	cycle;

\path[fill=fillColor,fill opacity=0.50] (307.60,117.97) --
	(310.45,117.97) --
	(310.45,120.82) --
	(307.60,120.82) --
	cycle;

\path[fill=fillColor,fill opacity=0.50] (306.07,115.88) --
	(308.93,115.88) --
	(308.93,118.73) --
	(306.07,118.73) --
	cycle;

\path[fill=fillColor,fill opacity=0.50] (305.66,125.03) --
	(308.51,125.03) --
	(308.51,127.88) --
	(305.66,127.88) --
	cycle;

\path[fill=fillColor,fill opacity=0.50] (305.97,116.49) --
	(308.82,116.49) --
	(308.82,119.34) --
	(305.97,119.34) --
	cycle;

\path[fill=fillColor,fill opacity=0.50] (307.15,116.09) --
	(310.00,116.09) --
	(310.00,118.94) --
	(307.15,118.94) --
	cycle;

\path[fill=fillColor,fill opacity=0.50] (306.07,115.46) --
	(308.93,115.46) --
	(308.93,118.31) --
	(306.07,118.31) --
	cycle;

\path[fill=fillColor,fill opacity=0.50] (322.74,133.82) --
	(325.60,133.82) --
	(325.60,136.68) --
	(322.74,136.68) --
	cycle;

\path[fill=fillColor,fill opacity=0.50] (320.07,128.09) --
	(322.93,128.09) --
	(322.93,130.94) --
	(320.07,130.94) --
	cycle;

\path[fill=fillColor,fill opacity=0.50] (318.26,125.67) --
	(321.11,125.67) --
	(321.11,128.53) --
	(318.26,128.53) --
	cycle;

\path[fill=fillColor,fill opacity=0.50] (318.04,126.86) --
	(320.89,126.86) --
	(320.89,129.72) --
	(318.04,129.72) --
	cycle;

\path[fill=fillColor,fill opacity=0.50] (317.93,126.62) --
	(320.78,126.62) --
	(320.78,129.47) --
	(317.93,129.47) --
	cycle;

\path[fill=fillColor,fill opacity=0.50] (319.54,127.26) --
	(322.40,127.26) --
	(322.40,130.11) --
	(319.54,130.11) --
	cycle;

\path[fill=fillColor,fill opacity=0.50] (319.28,127.34) --
	(322.14,127.34) --
	(322.14,130.19) --
	(319.28,130.19) --
	cycle;

\path[fill=fillColor,fill opacity=0.50] (318.61,126.28) --
	(321.46,126.28) --
	(321.46,129.14) --
	(318.61,129.14) --
	cycle;

\path[fill=fillColor,fill opacity=0.50] (320.58,128.45) --
	(323.44,128.45) --
	(323.44,131.30) --
	(320.58,131.30) --
	cycle;

\path[fill=fillColor,fill opacity=0.50] (320.11,135.82) --
	(322.96,135.82) --
	(322.96,138.67) --
	(320.11,138.67) --
	cycle;

\path[fill=fillColor,fill opacity=0.50] (320.76,131.03) --
	(323.61,131.03) --
	(323.61,133.88) --
	(320.76,133.88) --
	cycle;

\path[fill=fillColor,fill opacity=0.50] (321.04,130.74) --
	(323.90,130.74) --
	(323.90,133.59) --
	(321.04,133.59) --
	cycle;

\path[fill=fillColor,fill opacity=0.50] (319.80,130.51) --
	(322.65,130.51) --
	(322.65,133.36) --
	(319.80,133.36) --
	cycle;

\path[fill=fillColor,fill opacity=0.50] (324.44,135.12) --
	(327.29,135.12) --
	(327.29,137.98) --
	(324.44,137.98) --
	cycle;

\path[fill=fillColor,fill opacity=0.50] (326.17,133.96) --
	(329.03,133.96) --
	(329.03,136.81) --
	(326.17,136.81) --
	cycle;

\path[fill=fillColor,fill opacity=0.50] (320.79,130.39) --
	(323.64,130.39) --
	(323.64,133.24) --
	(320.79,133.24) --
	cycle;

\path[fill=fillColor,fill opacity=0.50] (345.98,158.61) --
	(348.83,158.61) --
	(348.83,161.46) --
	(345.98,161.46) --
	cycle;

\path[fill=fillColor,fill opacity=0.50] (334.46,145.30) --
	(337.31,145.30) --
	(337.31,148.15) --
	(334.46,148.15) --
	cycle;

\path[fill=fillColor,fill opacity=0.50] (337.93,140.54) --
	(340.79,140.54) --
	(340.79,143.40) --
	(337.93,143.40) --
	cycle;

\path[fill=fillColor,fill opacity=0.50] (334.73,140.67) --
	(337.58,140.67) --
	(337.58,143.52) --
	(334.73,143.52) --
	cycle;

\path[fill=fillColor,fill opacity=0.50] (331.34,143.28) --
	(334.19,143.28) --
	(334.19,146.14) --
	(331.34,146.14) --
	cycle;

\path[fill=fillColor,fill opacity=0.50] (332.10,142.26) --
	(334.96,142.26) --
	(334.96,145.11) --
	(332.10,145.11) --
	cycle;

\path[fill=fillColor,fill opacity=0.50] (332.75,138.71) --
	(335.60,138.71) --
	(335.60,141.57) --
	(332.75,141.57) --
	cycle;

\path[fill=fillColor,fill opacity=0.50] (328.88,136.29) --
	(331.73,136.29) --
	(331.73,139.15) --
	(328.88,139.15) --
	cycle;

\path[fill=fillColor,fill opacity=0.50] (331.44,140.82) --
	(334.30,140.82) --
	(334.30,143.67) --
	(331.44,143.67) --
	cycle;

\path[fill=fillColor,fill opacity=0.50] (328.04,135.20) --
	(330.90,135.20) --
	(330.90,138.06) --
	(328.04,138.06) --
	cycle;

\path[fill=fillColor,fill opacity=0.50] (330.24,137.61) --
	(333.10,137.61) --
	(333.10,140.47) --
	(330.24,140.47) --
	cycle;

\path[fill=fillColor,fill opacity=0.50] (343.44,157.22) --
	(346.29,157.22) --
	(346.29,160.07) --
	(343.44,160.07) --
	cycle;

\path[fill=fillColor,fill opacity=0.50] (341.07,159.27) --
	(343.92,159.27) --
	(343.92,162.12) --
	(341.07,162.12) --
	cycle;

\path[fill=fillColor,fill opacity=0.50] (343.32,156.65) --
	(346.17,156.65) --
	(346.17,159.50) --
	(343.32,159.50) --
	cycle;

\path[fill=fillColor,fill opacity=0.50] (347.80,160.85) --
	(350.65,160.85) --
	(350.65,163.71) --
	(347.80,163.71) --
	cycle;

\path[fill=fillColor,fill opacity=0.50] (341.71,159.78) --
	(344.57,159.78) --
	(344.57,162.63) --
	(341.71,162.63) --
	cycle;

\path[fill=fillColor,fill opacity=0.50] (314.29,118.96) --
	(317.14,118.96) --
	(317.14,121.82) --
	(314.29,121.82) --
	cycle;

\path[fill=fillColor,fill opacity=0.50] (307.15,116.68) --
	(310.00,116.68) --
	(310.00,119.54) --
	(307.15,119.54) --
	cycle;

\path[fill=fillColor,fill opacity=0.50] (307.51,116.49) --
	(310.36,116.49) --
	(310.36,119.34) --
	(307.51,119.34) --
	cycle;

\path[fill=fillColor,fill opacity=0.50] (307.42,124.65) --
	(310.27,124.65) --
	(310.27,127.50) --
	(307.42,127.50) --
	cycle;

\path[fill=fillColor,fill opacity=0.50] (308.55,117.79) --
	(311.40,117.79) --
	(311.40,120.64) --
	(308.55,120.64) --
	cycle;

\path[fill=fillColor,fill opacity=0.50] (314.24,116.68) --
	(317.09,116.68) --
	(317.09,119.54) --
	(314.24,119.54) --
	cycle;

\path[fill=fillColor,fill opacity=0.50] (310.09,118.31) --
	(312.94,118.31) --
	(312.94,121.16) --
	(310.09,121.16) --
	cycle;

\path[fill=fillColor,fill opacity=0.50] (308.38,118.31) --
	(311.23,118.31) --
	(311.23,121.16) --
	(308.38,121.16) --
	cycle;

\path[fill=fillColor,fill opacity=0.50] (308.13,116.29) --
	(310.98,116.29) --
	(310.98,119.14) --
	(308.13,119.14) --
	cycle;

\path[fill=fillColor,fill opacity=0.50] (310.65,120.46) --
	(313.51,120.46) --
	(313.51,123.31) --
	(310.65,123.31) --
	cycle;

\path[fill=fillColor,fill opacity=0.50] (308.79,118.47) --
	(311.64,118.47) --
	(311.64,121.33) --
	(308.79,121.33) --
	cycle;

\path[fill=fillColor,fill opacity=0.50] (309.11,118.14) --
	(311.96,118.14) --
	(311.96,120.99) --
	(309.11,120.99) --
	cycle;

\path[fill=fillColor,fill opacity=0.50] (313.61,120.31) --
	(316.47,120.31) --
	(316.47,123.17) --
	(313.61,123.17) --
	cycle;

\path[fill=fillColor,fill opacity=0.50] (310.58,119.73) --
	(313.44,119.73) --
	(313.44,122.59) --
	(310.58,122.59) --
	cycle;

\path[fill=fillColor,fill opacity=0.50] (313.72,120.31) --
	(316.57,120.31) --
	(316.57,123.17) --
	(313.72,123.17) --
	cycle;

\path[fill=fillColor,fill opacity=0.50] (311.06,119.73) --
	(313.91,119.73) --
	(313.91,122.59) --
	(311.06,122.59) --
	cycle;

\path[fill=fillColor,fill opacity=0.50] (331.68,155.52) --
	(334.54,155.52) --
	(334.54,158.38) --
	(331.68,158.38) --
	cycle;

\path[fill=fillColor,fill opacity=0.50] (323.06,129.59) --
	(325.91,129.59) --
	(325.91,132.44) --
	(323.06,132.44) --
	cycle;

\path[fill=fillColor,fill opacity=0.50] (322.57,129.78) --
	(325.42,129.78) --
	(325.42,132.63) --
	(322.57,132.63) --
	cycle;

\path[fill=fillColor,fill opacity=0.50] (324.37,133.18) --
	(327.23,133.18) --
	(327.23,136.04) --
	(324.37,136.04) --
	cycle;

\path[fill=fillColor,fill opacity=0.50] (322.35,130.39) --
	(325.20,130.39) --
	(325.20,133.24) --
	(322.35,133.24) --
	cycle;

\path[fill=fillColor,fill opacity=0.50] (322.45,130.74) --
	(325.30,130.74) --
	(325.30,133.59) --
	(322.45,133.59) --
	cycle;

\path[fill=fillColor,fill opacity=0.50] (323.59,132.05) --
	(326.44,132.05) --
	(326.44,134.91) --
	(323.59,134.91) --
	cycle;

\path[fill=fillColor,fill opacity=0.50] (323.20,131.31) --
	(326.05,131.31) --
	(326.05,134.16) --
	(323.20,134.16) --
	cycle;

\path[fill=fillColor,fill opacity=0.50] (321.83,135.24) --
	(324.69,135.24) --
	(324.69,138.10) --
	(321.83,138.10) --
	cycle;

\path[fill=fillColor,fill opacity=0.50] (321.65,129.52) --
	(324.50,129.52) --
	(324.50,132.38) --
	(321.65,132.38) --
	cycle;

\path[fill=fillColor,fill opacity=0.50] (320.79,128.59) --
	(323.64,128.59) --
	(323.64,131.44) --
	(320.79,131.44) --
	cycle;

\path[fill=fillColor,fill opacity=0.50] (329.06,143.97) --
	(331.91,143.97) --
	(331.91,146.82) --
	(329.06,146.82) --
	cycle;

\path[fill=fillColor,fill opacity=0.50] (330.84,139.83) --
	(333.69,139.83) --
	(333.69,142.68) --
	(330.84,142.68) --
	cycle;

\path[fill=fillColor,fill opacity=0.50] (327.46,139.69) --
	(330.31,139.69) --
	(330.31,142.55) --
	(327.46,142.55) --
	cycle;

\path[fill=fillColor,fill opacity=0.50] (327.71,140.37) --
	(330.57,140.37) --
	(330.57,143.22) --
	(327.71,143.22) --
	cycle;

\path[fill=fillColor,fill opacity=0.50] (330.27,142.94) --
	(333.12,142.94) --
	(333.12,145.80) --
	(330.27,145.80) --
	cycle;

\path[fill=fillColor,fill opacity=0.50] (318.61,133.28) --
	(321.46,133.28) --
	(321.46,136.13) --
	(318.61,136.13) --
	cycle;

\path[fill=fillColor,fill opacity=0.50] (313.82,121.27) --
	(316.68,121.27) --
	(316.68,124.12) --
	(313.82,124.12) --
	cycle;

\path[fill=fillColor,fill opacity=0.50] (314.29,122.85) --
	(317.14,122.85) --
	(317.14,125.71) --
	(314.29,125.71) --
	cycle;

\path[fill=fillColor,fill opacity=0.50] (312.49,121.27) --
	(315.35,121.27) --
	(315.35,124.12) --
	(312.49,124.12) --
	cycle;

\path[fill=fillColor,fill opacity=0.50] (313.18,123.30) --
	(316.03,123.30) --
	(316.03,126.15) --
	(313.18,126.15) --
	cycle;

\path[fill=fillColor,fill opacity=0.50] (319.76,120.87) --
	(322.62,120.87) --
	(322.62,123.72) --
	(319.76,123.72) --
	cycle;

\path[fill=fillColor,fill opacity=0.50] (312.38,122.62) --
	(315.23,122.62) --
	(315.23,125.48) --
	(312.38,125.48) --
	cycle;

\path[fill=fillColor,fill opacity=0.50] (314.08,121.78) --
	(316.94,121.78) --
	(316.94,124.63) --
	(314.08,124.63) --
	cycle;

\path[fill=fillColor,fill opacity=0.50] (320.01,121.78) --
	(322.87,121.78) --
	(322.87,124.63) --
	(320.01,124.63) --
	cycle;

\path[fill=fillColor,fill opacity=0.50] (313.45,124.35) --
	(316.31,124.35) --
	(316.31,127.20) --
	(313.45,127.20) --
	cycle;

\path[fill=fillColor,fill opacity=0.50] (315.34,121.78) --
	(318.20,121.78) --
	(318.20,124.63) --
	(315.34,124.63) --
	cycle;

\path[fill=fillColor,fill opacity=0.50] (315.66,125.03) --
	(318.52,125.03) --
	(318.52,127.88) --
	(315.66,127.88) --
	cycle;

\path[fill=fillColor,fill opacity=0.50] (316.31,129.00) --
	(319.17,129.00) --
	(319.17,131.85) --
	(316.31,131.85) --
	cycle;

\path[fill=fillColor,fill opacity=0.50] (312.84,124.74) --
	(315.69,124.74) --
	(315.69,127.60) --
	(312.84,127.60) --
	cycle;

\path[fill=fillColor,fill opacity=0.50] (314.49,122.85) --
	(317.34,122.85) --
	(317.34,125.71) --
	(314.49,125.71) --
	cycle;

\path[fill=fillColor,fill opacity=0.50] (317.56,128.93) --
	(320.41,128.93) --
	(320.41,131.78) --
	(317.56,131.78) --
	cycle;

\path[fill=fillColor,fill opacity=0.50] (329.83,149.19) --
	(332.68,149.19) --
	(332.68,152.04) --
	(329.83,152.04) --
	cycle;

\path[fill=fillColor,fill opacity=0.50] (315.53,123.41) --
	(318.38,123.41) --
	(318.38,126.26) --
	(315.53,126.26) --
	cycle;

\path[fill=fillColor,fill opacity=0.50] (314.03,124.93) --
	(316.89,124.93) --
	(316.89,127.79) --
	(314.03,127.79) --
	cycle;

\path[fill=fillColor,fill opacity=0.50] (315.02,124.45) --
	(317.87,124.45) --
	(317.87,127.30) --
	(315.02,127.30) --
	cycle;

\path[fill=fillColor,fill opacity=0.50] (317.82,126.03) --
	(320.67,126.03) --
	(320.67,128.88) --
	(317.82,128.88) --
	cycle;

\path[fill=fillColor,fill opacity=0.50] (315.11,125.49) --
	(317.97,125.49) --
	(317.97,128.35) --
	(315.11,128.35) --
	cycle;

\path[fill=fillColor,fill opacity=0.50] (313.88,122.85) --
	(316.73,122.85) --
	(316.73,125.71) --
	(313.88,125.71) --
	cycle;

\path[fill=fillColor,fill opacity=0.50] (317.44,126.28) --
	(320.29,126.28) --
	(320.29,129.14) --
	(317.44,129.14) --
	cycle;

\path[fill=fillColor,fill opacity=0.50] (316.61,124.55) --
	(319.46,124.55) --
	(319.46,127.40) --
	(316.61,127.40) --
	cycle;

\path[fill=fillColor,fill opacity=0.50] (321.75,131.52) --
	(324.61,131.52) --
	(324.61,134.38) --
	(321.75,134.38) --
	cycle;

\path[fill=fillColor,fill opacity=0.50] (315.02,126.28) --
	(317.87,126.28) --
	(317.87,129.14) --
	(315.02,129.14) --
	cycle;

\path[fill=fillColor,fill opacity=0.50] (324.83,138.69) --
	(327.68,138.69) --
	(327.68,141.54) --
	(324.83,141.54) --
	cycle;

\path[fill=fillColor,fill opacity=0.50] (322.84,139.80) --
	(325.69,139.80) --
	(325.69,142.65) --
	(322.84,142.65) --
	cycle;

\path[fill=fillColor,fill opacity=0.50] (325.05,143.08) --
	(327.91,143.08) --
	(327.91,145.94) --
	(325.05,145.94) --
	cycle;

\path[fill=fillColor,fill opacity=0.50] (323.56,143.14) --
	(326.42,143.14) --
	(326.42,146.00) --
	(323.56,146.00) --
	cycle;

\path[fill=fillColor,fill opacity=0.50] (326.51,145.17) --
	(329.37,145.17) --
	(329.37,148.02) --
	(326.51,148.02) --
	cycle;

\path[fill=fillColor,fill opacity=0.50] (392.94,211.01) --
	(395.79,211.01) --
	(395.79,213.87) --
	(392.94,213.87) --
	cycle;

\path[fill=fillColor,fill opacity=0.50] (392.94,200.30) --
	(395.79,200.30) --
	(395.79,203.16) --
	(392.94,203.16) --
	cycle;

\path[fill=fillColor,fill opacity=0.50] (360.82,188.91) --
	(363.67,188.91) --
	(363.67,191.76) --
	(360.82,191.76) --
	cycle;

\path[fill=fillColor,fill opacity=0.50] (359.77,171.17) --
	(362.62,171.17) --
	(362.62,174.02) --
	(359.77,174.02) --
	cycle;

\path[fill=fillColor,fill opacity=0.50] (363.27,177.74) --
	(366.12,177.74) --
	(366.12,180.59) --
	(363.27,180.59) --
	cycle;

\path[fill=fillColor,fill opacity=0.50] (356.27,190.68) --
	(359.13,190.68) --
	(359.13,193.53) --
	(356.27,193.53) --
	cycle;

\path[fill=fillColor,fill opacity=0.50] (324.10,130.33) --
	(326.95,130.33) --
	(326.95,133.18) --
	(324.10,133.18) --
	cycle;

\path[fill=fillColor,fill opacity=0.50] (324.31,132.85) --
	(327.16,132.85) --
	(327.16,135.70) --
	(324.31,135.70) --
	cycle;

\path[fill=fillColor,fill opacity=0.50] (323.27,131.79) --
	(326.12,131.79) --
	(326.12,134.64) --
	(323.27,134.64) --
	cycle;

\path[fill=fillColor,fill opacity=0.50] (327.21,133.37) --
	(330.07,133.37) --
	(330.07,136.22) --
	(327.21,136.22) --
	cycle;

\path[fill=fillColor,fill opacity=0.50] (392.94,211.01) --
	(395.79,211.01) --
	(395.79,213.87) --
	(392.94,213.87) --
	cycle;

\path[fill=fillColor,fill opacity=0.50] (392.94,211.01) --
	(395.79,211.01) --
	(395.79,213.87) --
	(392.94,213.87) --
	cycle;

\path[fill=fillColor,fill opacity=0.50] (392.94,211.01) --
	(395.79,211.01) --
	(395.79,213.87) --
	(392.94,213.87) --
	cycle;

\path[fill=fillColor,fill opacity=0.50] (392.94,211.01) --
	(395.79,211.01) --
	(395.79,213.87) --
	(392.94,213.87) --
	cycle;

\path[fill=fillColor,fill opacity=0.50] (392.94,211.01) --
	(395.79,211.01) --
	(395.79,213.87) --
	(392.94,213.87) --
	cycle;

\path[fill=fillColor,fill opacity=0.50] (392.94,211.01) --
	(395.79,211.01) --
	(395.79,213.87) --
	(392.94,213.87) --
	cycle;
\definecolor{drawColor}{RGB}{231,41,138}

\path[draw=drawColor,draw opacity=0.50,line width= 0.4pt,line join=round,line cap=round] (281.90,103.44) -- (285.93,103.44);

\path[draw=drawColor,draw opacity=0.50,line width= 0.4pt,line join=round,line cap=round] (283.92,101.43) -- (283.92,105.46);

\path[draw=drawColor,draw opacity=0.50,line width= 0.4pt,line join=round,line cap=round] (280.23, 99.39) -- (284.27, 99.39);

\path[draw=drawColor,draw opacity=0.50,line width= 0.4pt,line join=round,line cap=round] (282.25, 97.37) -- (282.25,101.41);

\path[draw=drawColor,draw opacity=0.50,line width= 0.4pt,line join=round,line cap=round] (278.29, 98.38) -- (282.33, 98.38);

\path[draw=drawColor,draw opacity=0.50,line width= 0.4pt,line join=round,line cap=round] (280.31, 96.36) -- (280.31,100.40);

\path[draw=drawColor,draw opacity=0.50,line width= 0.4pt,line join=round,line cap=round] (280.23,100.32) -- (284.27,100.32);

\path[draw=drawColor,draw opacity=0.50,line width= 0.4pt,line join=round,line cap=round] (282.25, 98.31) -- (282.25,102.34);

\path[draw=drawColor,draw opacity=0.50,line width= 0.4pt,line join=round,line cap=round] (281.10,100.32) -- (285.13,100.32);

\path[draw=drawColor,draw opacity=0.50,line width= 0.4pt,line join=round,line cap=round] (283.11, 98.31) -- (283.11,102.34);

\path[draw=drawColor,draw opacity=0.50,line width= 0.4pt,line join=round,line cap=round] (281.10,101.19) -- (285.13,101.19);

\path[draw=drawColor,draw opacity=0.50,line width= 0.4pt,line join=round,line cap=round] (283.11, 99.17) -- (283.11,103.20);

\path[draw=drawColor,draw opacity=0.50,line width= 0.4pt,line join=round,line cap=round] (279.30, 99.39) -- (283.34, 99.39);

\path[draw=drawColor,draw opacity=0.50,line width= 0.4pt,line join=round,line cap=round] (281.32, 97.37) -- (281.32,101.41);

\path[draw=drawColor,draw opacity=0.50,line width= 0.4pt,line join=round,line cap=round] (280.23,100.32) -- (284.27,100.32);

\path[draw=drawColor,draw opacity=0.50,line width= 0.4pt,line join=round,line cap=round] (282.25, 98.31) -- (282.25,102.34);

\path[draw=drawColor,draw opacity=0.50,line width= 0.4pt,line join=round,line cap=round] (281.90,100.32) -- (285.93,100.32);

\path[draw=drawColor,draw opacity=0.50,line width= 0.4pt,line join=round,line cap=round] (283.92, 98.31) -- (283.92,102.34);

\path[draw=drawColor,draw opacity=0.50,line width= 0.4pt,line join=round,line cap=round] (281.90,100.32) -- (285.93,100.32);

\path[draw=drawColor,draw opacity=0.50,line width= 0.4pt,line join=round,line cap=round] (283.92, 98.31) -- (283.92,102.34);

\path[draw=drawColor,draw opacity=0.50,line width= 0.4pt,line join=round,line cap=round] (280.23, 99.39) -- (284.27, 99.39);

\path[draw=drawColor,draw opacity=0.50,line width= 0.4pt,line join=round,line cap=round] (282.25, 97.37) -- (282.25,101.41);

\path[draw=drawColor,draw opacity=0.50,line width= 0.4pt,line join=round,line cap=round] (280.23, 99.39) -- (284.27, 99.39);

\path[draw=drawColor,draw opacity=0.50,line width= 0.4pt,line join=round,line cap=round] (282.25, 97.37) -- (282.25,101.41);

\path[draw=drawColor,draw opacity=0.50,line width= 0.4pt,line join=round,line cap=round] (281.10,100.32) -- (285.13,100.32);

\path[draw=drawColor,draw opacity=0.50,line width= 0.4pt,line join=round,line cap=round] (283.11, 98.31) -- (283.11,102.34);

\path[draw=drawColor,draw opacity=0.50,line width= 0.4pt,line join=round,line cap=round] (280.23, 98.38) -- (284.27, 98.38);

\path[draw=drawColor,draw opacity=0.50,line width= 0.4pt,line join=round,line cap=round] (282.25, 96.36) -- (282.25,100.40);

\path[draw=drawColor,draw opacity=0.50,line width= 0.4pt,line join=round,line cap=round] (281.10,100.32) -- (285.13,100.32);

\path[draw=drawColor,draw opacity=0.50,line width= 0.4pt,line join=round,line cap=round] (283.11, 98.31) -- (283.11,102.34);

\path[draw=drawColor,draw opacity=0.50,line width= 0.4pt,line join=round,line cap=round] (281.90,101.19) -- (285.93,101.19);

\path[draw=drawColor,draw opacity=0.50,line width= 0.4pt,line join=round,line cap=round] (283.92, 99.17) -- (283.92,103.20);
\definecolor{drawColor}{RGB}{230,171,2}

\path[draw=drawColor,draw opacity=0.50,line width= 0.4pt,line join=round,line cap=round] (305.66,117.97) -- (308.51,120.82);

\path[draw=drawColor,draw opacity=0.50,line width= 0.4pt,line join=round,line cap=round] (305.66,120.82) -- (308.51,117.97);

\path[draw=drawColor,draw opacity=0.50,line width= 0.4pt,line join=round,line cap=round] (305.07,119.39) -- (309.10,119.39);

\path[draw=drawColor,draw opacity=0.50,line width= 0.4pt,line join=round,line cap=round] (307.09,117.37) -- (307.09,121.41);

\path[draw=drawColor,draw opacity=0.50,line width= 0.4pt,line join=round,line cap=round] (305.55,117.06) -- (308.41,119.92);

\path[draw=drawColor,draw opacity=0.50,line width= 0.4pt,line join=round,line cap=round] (305.55,119.92) -- (308.41,117.06);

\path[draw=drawColor,draw opacity=0.50,line width= 0.4pt,line join=round,line cap=round] (304.96,118.49) -- (309.00,118.49);

\path[draw=drawColor,draw opacity=0.50,line width= 0.4pt,line join=round,line cap=round] (306.98,116.47) -- (306.98,120.51);

\path[draw=drawColor,draw opacity=0.50,line width= 0.4pt,line join=round,line cap=round] (306.67,116.87) -- (309.52,119.73);

\path[draw=drawColor,draw opacity=0.50,line width= 0.4pt,line join=round,line cap=round] (306.67,119.73) -- (309.52,116.87);

\path[draw=drawColor,draw opacity=0.50,line width= 0.4pt,line join=round,line cap=round] (306.08,118.30) -- (310.11,118.30);

\path[draw=drawColor,draw opacity=0.50,line width= 0.4pt,line join=round,line cap=round] (308.10,116.28) -- (308.10,120.32);

\path[draw=drawColor,draw opacity=0.50,line width= 0.4pt,line join=round,line cap=round] (306.67,117.61) -- (309.52,120.47);

\path[draw=drawColor,draw opacity=0.50,line width= 0.4pt,line join=round,line cap=round] (306.67,120.47) -- (309.52,117.61);

\path[draw=drawColor,draw opacity=0.50,line width= 0.4pt,line join=round,line cap=round] (306.08,119.04) -- (310.11,119.04);

\path[draw=drawColor,draw opacity=0.50,line width= 0.4pt,line join=round,line cap=round] (308.10,117.02) -- (308.10,121.06);

\path[draw=drawColor,draw opacity=0.50,line width= 0.4pt,line join=round,line cap=round] (306.67,117.25) -- (309.52,120.10);

\path[draw=drawColor,draw opacity=0.50,line width= 0.4pt,line join=round,line cap=round] (306.67,120.10) -- (309.52,117.25);

\path[draw=drawColor,draw opacity=0.50,line width= 0.4pt,line join=round,line cap=round] (306.08,118.68) -- (310.11,118.68);

\path[draw=drawColor,draw opacity=0.50,line width= 0.4pt,line join=round,line cap=round] (308.10,116.66) -- (308.10,120.69);

\path[draw=drawColor,draw opacity=0.50,line width= 0.4pt,line join=round,line cap=round] (306.77,117.79) -- (309.62,120.64);

\path[draw=drawColor,draw opacity=0.50,line width= 0.4pt,line join=round,line cap=round] (306.77,120.64) -- (309.62,117.79);

\path[draw=drawColor,draw opacity=0.50,line width= 0.4pt,line join=round,line cap=round] (306.18,119.22) -- (310.21,119.22);

\path[draw=drawColor,draw opacity=0.50,line width= 0.4pt,line join=round,line cap=round] (308.19,117.20) -- (308.19,121.23);

\path[draw=drawColor,draw opacity=0.50,line width= 0.4pt,line join=round,line cap=round] (303.83,115.46) -- (306.68,118.31);

\path[draw=drawColor,draw opacity=0.50,line width= 0.4pt,line join=round,line cap=round] (303.83,118.31) -- (306.68,115.46);

\path[draw=drawColor,draw opacity=0.50,line width= 0.4pt,line join=round,line cap=round] (303.24,116.88) -- (307.27,116.88);

\path[draw=drawColor,draw opacity=0.50,line width= 0.4pt,line join=round,line cap=round] (305.26,114.87) -- (305.26,118.90);

\path[draw=drawColor,draw opacity=0.50,line width= 0.4pt,line join=round,line cap=round] (305.76,116.09) -- (308.62,118.94);

\path[draw=drawColor,draw opacity=0.50,line width= 0.4pt,line join=round,line cap=round] (305.76,118.94) -- (308.62,116.09);

\path[draw=drawColor,draw opacity=0.50,line width= 0.4pt,line join=round,line cap=round] (305.17,117.51) -- (309.21,117.51);

\path[draw=drawColor,draw opacity=0.50,line width= 0.4pt,line join=round,line cap=round] (307.19,115.49) -- (307.19,119.53);

\path[draw=drawColor,draw opacity=0.50,line width= 0.4pt,line join=round,line cap=round] (305.66,117.43) -- (308.51,120.29);

\path[draw=drawColor,draw opacity=0.50,line width= 0.4pt,line join=round,line cap=round] (305.66,120.29) -- (308.51,117.43);

\path[draw=drawColor,draw opacity=0.50,line width= 0.4pt,line join=round,line cap=round] (305.07,118.86) -- (309.10,118.86);

\path[draw=drawColor,draw opacity=0.50,line width= 0.4pt,line join=round,line cap=round] (307.09,116.84) -- (307.09,120.88);

\path[draw=drawColor,draw opacity=0.50,line width= 0.4pt,line join=round,line cap=round] (305.76,116.49) -- (308.62,119.34);

\path[draw=drawColor,draw opacity=0.50,line width= 0.4pt,line join=round,line cap=round] (305.76,119.34) -- (308.62,116.49);

\path[draw=drawColor,draw opacity=0.50,line width= 0.4pt,line join=round,line cap=round] (305.17,117.91) -- (309.21,117.91);

\path[draw=drawColor,draw opacity=0.50,line width= 0.4pt,line join=round,line cap=round] (307.19,115.90) -- (307.19,119.93);

\path[draw=drawColor,draw opacity=0.50,line width= 0.4pt,line join=round,line cap=round] (305.66,116.68) -- (308.51,119.54);

\path[draw=drawColor,draw opacity=0.50,line width= 0.4pt,line join=round,line cap=round] (305.66,119.54) -- (308.51,116.68);

\path[draw=drawColor,draw opacity=0.50,line width= 0.4pt,line join=round,line cap=round] (305.07,118.11) -- (309.10,118.11);

\path[draw=drawColor,draw opacity=0.50,line width= 0.4pt,line join=round,line cap=round] (307.09,116.09) -- (307.09,120.13);

\path[draw=drawColor,draw opacity=0.50,line width= 0.4pt,line join=round,line cap=round] (307.05,116.87) -- (309.90,119.73);

\path[draw=drawColor,draw opacity=0.50,line width= 0.4pt,line join=round,line cap=round] (307.05,119.73) -- (309.90,116.87);

\path[draw=drawColor,draw opacity=0.50,line width= 0.4pt,line join=round,line cap=round] (306.46,118.30) -- (310.50,118.30);

\path[draw=drawColor,draw opacity=0.50,line width= 0.4pt,line join=round,line cap=round] (308.48,116.28) -- (308.48,120.32);

\path[draw=drawColor,draw opacity=0.50,line width= 0.4pt,line join=round,line cap=round] (305.55,116.68) -- (308.41,119.54);

\path[draw=drawColor,draw opacity=0.50,line width= 0.4pt,line join=round,line cap=round] (305.55,119.54) -- (308.41,116.68);

\path[draw=drawColor,draw opacity=0.50,line width= 0.4pt,line join=round,line cap=round] (304.96,118.11) -- (309.00,118.11);

\path[draw=drawColor,draw opacity=0.50,line width= 0.4pt,line join=round,line cap=round] (306.98,116.09) -- (306.98,120.13);

\path[draw=drawColor,draw opacity=0.50,line width= 0.4pt,line join=round,line cap=round] (307.05,117.06) -- (309.90,119.92);

\path[draw=drawColor,draw opacity=0.50,line width= 0.4pt,line join=round,line cap=round] (307.05,119.92) -- (309.90,117.06);

\path[draw=drawColor,draw opacity=0.50,line width= 0.4pt,line join=round,line cap=round] (306.46,118.49) -- (310.50,118.49);

\path[draw=drawColor,draw opacity=0.50,line width= 0.4pt,line join=round,line cap=round] (308.48,116.47) -- (308.48,120.51);

\path[draw=drawColor,draw opacity=0.50,line width= 0.4pt,line join=round,line cap=round] (305.66,117.06) -- (308.51,119.92);

\path[draw=drawColor,draw opacity=0.50,line width= 0.4pt,line join=round,line cap=round] (305.66,119.92) -- (308.51,117.06);

\path[draw=drawColor,draw opacity=0.50,line width= 0.4pt,line join=round,line cap=round] (305.07,118.49) -- (309.10,118.49);

\path[draw=drawColor,draw opacity=0.50,line width= 0.4pt,line join=round,line cap=round] (307.09,116.47) -- (307.09,120.51);

\path[draw=drawColor,draw opacity=0.50,line width= 0.4pt,line join=round,line cap=round] (306.18,117.25) -- (309.03,120.10);

\path[draw=drawColor,draw opacity=0.50,line width= 0.4pt,line join=round,line cap=round] (306.18,120.10) -- (309.03,117.25);

\path[draw=drawColor,draw opacity=0.50,line width= 0.4pt,line join=round,line cap=round] (305.58,118.68) -- (309.62,118.68);

\path[draw=drawColor,draw opacity=0.50,line width= 0.4pt,line join=round,line cap=round] (307.60,116.66) -- (307.60,120.69);
\definecolor{drawColor}{RGB}{231,41,138}

\path[draw=drawColor,draw opacity=0.50,line width= 0.4pt,line join=round,line cap=round] (361.61,174.26) -- (365.64,174.26);

\path[draw=drawColor,draw opacity=0.50,line width= 0.4pt,line join=round,line cap=round] (363.62,172.24) -- (363.62,176.27);

\path[draw=drawColor,draw opacity=0.50,line width= 0.4pt,line join=round,line cap=round] (361.04,173.60) -- (365.07,173.60);

\path[draw=drawColor,draw opacity=0.50,line width= 0.4pt,line join=round,line cap=round] (363.06,171.58) -- (363.06,175.61);

\path[draw=drawColor,draw opacity=0.50,line width= 0.4pt,line join=round,line cap=round] (362.62,174.39) -- (366.65,174.39);

\path[draw=drawColor,draw opacity=0.50,line width= 0.4pt,line join=round,line cap=round] (364.63,172.37) -- (364.63,176.40);

\path[draw=drawColor,draw opacity=0.50,line width= 0.4pt,line join=round,line cap=round] (363.21,175.19) -- (367.24,175.19);

\path[draw=drawColor,draw opacity=0.50,line width= 0.4pt,line join=round,line cap=round] (365.23,173.18) -- (365.23,177.21);

\path[draw=drawColor,draw opacity=0.50,line width= 0.4pt,line join=round,line cap=round] (362.92,175.17) -- (366.95,175.17);

\path[draw=drawColor,draw opacity=0.50,line width= 0.4pt,line join=round,line cap=round] (364.93,173.15) -- (364.93,177.19);

\path[draw=drawColor,draw opacity=0.50,line width= 0.4pt,line join=round,line cap=round] (363.62,175.61) -- (367.65,175.61);

\path[draw=drawColor,draw opacity=0.50,line width= 0.4pt,line join=round,line cap=round] (365.63,173.60) -- (365.63,177.63);

\path[draw=drawColor,draw opacity=0.50,line width= 0.4pt,line join=round,line cap=round] (362.87,174.99) -- (366.91,174.99);

\path[draw=drawColor,draw opacity=0.50,line width= 0.4pt,line join=round,line cap=round] (364.89,172.97) -- (364.89,177.01);

\path[draw=drawColor,draw opacity=0.50,line width= 0.4pt,line join=round,line cap=round] (362.87,175.14) -- (366.90,175.14);

\path[draw=drawColor,draw opacity=0.50,line width= 0.4pt,line join=round,line cap=round] (364.88,173.12) -- (364.88,177.16);

\path[draw=drawColor,draw opacity=0.50,line width= 0.4pt,line join=round,line cap=round] (362.68,175.35) -- (366.71,175.35);

\path[draw=drawColor,draw opacity=0.50,line width= 0.4pt,line join=round,line cap=round] (364.69,173.33) -- (364.69,177.37);

\path[draw=drawColor,draw opacity=0.50,line width= 0.4pt,line join=round,line cap=round] (361.58,173.49) -- (365.62,173.49);

\path[draw=drawColor,draw opacity=0.50,line width= 0.4pt,line join=round,line cap=round] (363.60,171.48) -- (363.60,175.51);

\path[draw=drawColor,draw opacity=0.50,line width= 0.4pt,line join=round,line cap=round] (361.29,173.11) -- (365.33,173.11);

\path[draw=drawColor,draw opacity=0.50,line width= 0.4pt,line join=round,line cap=round] (363.31,171.10) -- (363.31,175.13);

\path[draw=drawColor,draw opacity=0.50,line width= 0.4pt,line join=round,line cap=round] (362.01,174.23) -- (366.04,174.23);

\path[draw=drawColor,draw opacity=0.50,line width= 0.4pt,line join=round,line cap=round] (364.02,172.22) -- (364.02,176.25);

\path[draw=drawColor,draw opacity=0.50,line width= 0.4pt,line join=round,line cap=round] (361.60,174.61) -- (365.63,174.61);

\path[draw=drawColor,draw opacity=0.50,line width= 0.4pt,line join=round,line cap=round] (363.61,172.60) -- (363.61,176.63);

\path[draw=drawColor,draw opacity=0.50,line width= 0.4pt,line join=round,line cap=round] (363.34,176.23) -- (367.37,176.23);

\path[draw=drawColor,draw opacity=0.50,line width= 0.4pt,line join=round,line cap=round] (365.35,174.22) -- (365.35,178.25);

\path[draw=drawColor,draw opacity=0.50,line width= 0.4pt,line join=round,line cap=round] (363.56,176.42) -- (367.60,176.42);

\path[draw=drawColor,draw opacity=0.50,line width= 0.4pt,line join=round,line cap=round] (365.58,174.40) -- (365.58,178.44);

\path[draw=drawColor,draw opacity=0.50,line width= 0.4pt,line join=round,line cap=round] (363.41,176.38) -- (367.44,176.38);

\path[draw=drawColor,draw opacity=0.50,line width= 0.4pt,line join=round,line cap=round] (365.42,174.36) -- (365.42,178.40);

\path[draw=drawColor,draw opacity=0.50,line width= 0.4pt,line join=round,line cap=round] (297.01,112.80) -- (301.04,112.80);

\path[draw=drawColor,draw opacity=0.50,line width= 0.4pt,line join=round,line cap=round] (299.02,110.78) -- (299.02,114.81);

\path[draw=drawColor,draw opacity=0.50,line width= 0.4pt,line join=round,line cap=round] (298.21,113.39) -- (302.25,113.39);

\path[draw=drawColor,draw opacity=0.50,line width= 0.4pt,line join=round,line cap=round] (300.23,111.38) -- (300.23,115.41);

\path[draw=drawColor,draw opacity=0.50,line width= 0.4pt,line join=round,line cap=round] (299.13,113.96) -- (303.16,113.96);

\path[draw=drawColor,draw opacity=0.50,line width= 0.4pt,line join=round,line cap=round] (301.14,111.94) -- (301.14,115.98);

\path[draw=drawColor,draw opacity=0.50,line width= 0.4pt,line join=round,line cap=round] (298.77,113.96) -- (302.80,113.96);

\path[draw=drawColor,draw opacity=0.50,line width= 0.4pt,line join=round,line cap=round] (300.79,111.94) -- (300.79,115.98);

\path[draw=drawColor,draw opacity=0.50,line width= 0.4pt,line join=round,line cap=round] (298.95,114.50) -- (302.98,114.50);

\path[draw=drawColor,draw opacity=0.50,line width= 0.4pt,line join=round,line cap=round] (300.97,112.48) -- (300.97,116.52);

\path[draw=drawColor,draw opacity=0.50,line width= 0.4pt,line join=round,line cap=round] (298.77,113.96) -- (302.80,113.96);

\path[draw=drawColor,draw opacity=0.50,line width= 0.4pt,line join=round,line cap=round] (300.79,111.94) -- (300.79,115.98);

\path[draw=drawColor,draw opacity=0.50,line width= 0.4pt,line join=round,line cap=round] (298.59,113.68) -- (302.62,113.68);

\path[draw=drawColor,draw opacity=0.50,line width= 0.4pt,line join=round,line cap=round] (300.60,111.66) -- (300.60,115.70);

\path[draw=drawColor,draw opacity=0.50,line width= 0.4pt,line join=round,line cap=round] (298.77,113.96) -- (302.80,113.96);

\path[draw=drawColor,draw opacity=0.50,line width= 0.4pt,line join=round,line cap=round] (300.79,111.94) -- (300.79,115.98);

\path[draw=drawColor,draw opacity=0.50,line width= 0.4pt,line join=round,line cap=round] (298.95,115.51) -- (302.98,115.51);

\path[draw=drawColor,draw opacity=0.50,line width= 0.4pt,line join=round,line cap=round] (300.97,113.50) -- (300.97,117.53);

\path[draw=drawColor,draw opacity=0.50,line width= 0.4pt,line join=round,line cap=round] (297.01,112.80) -- (301.04,112.80);

\path[draw=drawColor,draw opacity=0.50,line width= 0.4pt,line join=round,line cap=round] (299.02,110.78) -- (299.02,114.81);

\path[draw=drawColor,draw opacity=0.50,line width= 0.4pt,line join=round,line cap=round] (298.02,113.96) -- (302.05,113.96);

\path[draw=drawColor,draw opacity=0.50,line width= 0.4pt,line join=round,line cap=round] (300.04,111.94) -- (300.04,115.98);

\path[draw=drawColor,draw opacity=0.50,line width= 0.4pt,line join=round,line cap=round] (298.40,113.39) -- (302.44,113.39);

\path[draw=drawColor,draw opacity=0.50,line width= 0.4pt,line join=round,line cap=round] (300.42,111.38) -- (300.42,115.41);

\path[draw=drawColor,draw opacity=0.50,line width= 0.4pt,line join=round,line cap=round] (298.21,113.96) -- (302.25,113.96);

\path[draw=drawColor,draw opacity=0.50,line width= 0.4pt,line join=round,line cap=round] (300.23,111.94) -- (300.23,115.98);

\path[draw=drawColor,draw opacity=0.50,line width= 0.4pt,line join=round,line cap=round] (299.13,113.96) -- (303.16,113.96);

\path[draw=drawColor,draw opacity=0.50,line width= 0.4pt,line join=round,line cap=round] (301.14,111.94) -- (301.14,115.98);

\path[draw=drawColor,draw opacity=0.50,line width= 0.4pt,line join=round,line cap=round] (298.77,114.50) -- (302.80,114.50);

\path[draw=drawColor,draw opacity=0.50,line width= 0.4pt,line join=round,line cap=round] (300.79,112.48) -- (300.79,116.52);

\path[draw=drawColor,draw opacity=0.50,line width= 0.4pt,line join=round,line cap=round] (297.01,113.10) -- (301.04,113.10);

\path[draw=drawColor,draw opacity=0.50,line width= 0.4pt,line join=round,line cap=round] (299.02,111.08) -- (299.02,115.12);

\path[fill=fillColor,fill opacity=0.50] (307.51,116.29) --
	(310.36,116.29) --
	(310.36,119.14) --
	(307.51,119.14) --
	cycle;

\path[fill=fillColor,fill opacity=0.50] (309.11,117.97) --
	(311.96,117.97) --
	(311.96,120.82) --
	(309.11,120.82) --
	cycle;

\path[fill=fillColor,fill opacity=0.50] (307.05,116.87) --
	(309.90,116.87) --
	(309.90,119.73) --
	(307.05,119.73) --
	cycle;

\path[fill=fillColor,fill opacity=0.50] (308.21,118.14) --
	(311.06,118.14) --
	(311.06,120.99) --
	(308.21,120.99) --
	cycle;

\path[fill=fillColor,fill opacity=0.50] (307.15,117.79) --
	(310.00,117.79) --
	(310.00,120.64) --
	(307.15,120.64) --
	cycle;

\path[fill=fillColor,fill opacity=0.50] (308.71,117.79) --
	(311.56,117.79) --
	(311.56,120.64) --
	(308.71,120.64) --
	cycle;

\path[fill=fillColor,fill opacity=0.50] (308.71,118.31) --
	(311.56,118.31) --
	(311.56,121.16) --
	(308.71,121.16) --
	cycle;

\path[fill=fillColor,fill opacity=0.50] (308.13,118.96) --
	(310.98,118.96) --
	(310.98,121.82) --
	(308.13,121.82) --
	cycle;

\path[fill=fillColor,fill opacity=0.50] (308.13,118.96) --
	(310.98,118.96) --
	(310.98,121.82) --
	(308.13,121.82) --
	cycle;

\path[fill=fillColor,fill opacity=0.50] (307.60,116.29) --
	(310.45,116.29) --
	(310.45,119.14) --
	(307.60,119.14) --
	cycle;

\path[fill=fillColor,fill opacity=0.50] (309.87,117.97) --
	(312.72,117.97) --
	(312.72,120.82) --
	(309.87,120.82) --
	cycle;

\path[fill=fillColor,fill opacity=0.50] (308.55,118.80) --
	(311.40,118.80) --
	(311.40,121.66) --
	(308.55,121.66) --
	cycle;

\path[fill=fillColor,fill opacity=0.50] (307.42,116.68) --
	(310.27,116.68) --
	(310.27,119.54) --
	(307.42,119.54) --
	cycle;

\path[fill=fillColor,fill opacity=0.50] (309.50,119.28) --
	(312.35,119.28) --
	(312.35,122.13) --
	(309.50,122.13) --
	cycle;

\path[fill=fillColor,fill opacity=0.50] (308.30,118.96) --
	(311.15,118.96) --
	(311.15,121.82) --
	(308.30,121.82) --
	cycle;

\path[fill=fillColor,fill opacity=0.50] (308.95,119.12) --
	(311.80,119.12) --
	(311.80,121.97) --
	(308.95,121.97) --
	cycle;

\path[draw=drawColor,draw opacity=0.50,line width= 0.4pt,line join=round,line cap=round] (376.28,192.66) -- (380.32,192.66);

\path[draw=drawColor,draw opacity=0.50,line width= 0.4pt,line join=round,line cap=round] (378.30,190.65) -- (378.30,194.68);

\path[draw=drawColor,draw opacity=0.50,line width= 0.4pt,line join=round,line cap=round] (375.88,192.11) -- (379.92,192.11);

\path[draw=drawColor,draw opacity=0.50,line width= 0.4pt,line join=round,line cap=round] (377.90,190.09) -- (377.90,194.13);

\path[draw=drawColor,draw opacity=0.50,line width= 0.4pt,line join=round,line cap=round] (375.42,192.03) -- (379.45,192.03);

\path[draw=drawColor,draw opacity=0.50,line width= 0.4pt,line join=round,line cap=round] (377.43,190.01) -- (377.43,194.05);

\path[draw=drawColor,draw opacity=0.50,line width= 0.4pt,line join=round,line cap=round] (375.85,191.33) -- (379.89,191.33);

\path[draw=drawColor,draw opacity=0.50,line width= 0.4pt,line join=round,line cap=round] (377.87,189.32) -- (377.87,193.35);

\path[draw=drawColor,draw opacity=0.50,line width= 0.4pt,line join=round,line cap=round] (376.10,191.21) -- (380.14,191.21);

\path[draw=drawColor,draw opacity=0.50,line width= 0.4pt,line join=round,line cap=round] (378.12,189.19) -- (378.12,193.23);

\path[draw=drawColor,draw opacity=0.50,line width= 0.4pt,line join=round,line cap=round] (375.02,190.81) -- (379.06,190.81);

\path[draw=drawColor,draw opacity=0.50,line width= 0.4pt,line join=round,line cap=round] (377.04,188.79) -- (377.04,192.83);

\path[draw=drawColor,draw opacity=0.50,line width= 0.4pt,line join=round,line cap=round] (375.34,191.99) -- (379.37,191.99);

\path[draw=drawColor,draw opacity=0.50,line width= 0.4pt,line join=round,line cap=round] (377.35,189.97) -- (377.35,194.01);

\path[draw=drawColor,draw opacity=0.50,line width= 0.4pt,line join=round,line cap=round] (375.36,192.57) -- (379.39,192.57);

\path[draw=drawColor,draw opacity=0.50,line width= 0.4pt,line join=round,line cap=round] (377.37,190.55) -- (377.37,194.58);

\path[draw=drawColor,draw opacity=0.50,line width= 0.4pt,line join=round,line cap=round] (373.94,191.56) -- (377.97,191.56);

\path[draw=drawColor,draw opacity=0.50,line width= 0.4pt,line join=round,line cap=round] (375.95,189.54) -- (375.95,193.57);

\path[draw=drawColor,draw opacity=0.50,line width= 0.4pt,line join=round,line cap=round] (375.08,191.33) -- (379.12,191.33);

\path[draw=drawColor,draw opacity=0.50,line width= 0.4pt,line join=round,line cap=round] (377.10,189.31) -- (377.10,193.35);

\path[draw=drawColor,draw opacity=0.50,line width= 0.4pt,line join=round,line cap=round] (374.34,191.89) -- (378.37,191.89);

\path[draw=drawColor,draw opacity=0.50,line width= 0.4pt,line join=round,line cap=round] (376.36,189.87) -- (376.36,193.91);

\path[draw=drawColor,draw opacity=0.50,line width= 0.4pt,line join=round,line cap=round] (374.29,190.99) -- (378.33,190.99);

\path[draw=drawColor,draw opacity=0.50,line width= 0.4pt,line join=round,line cap=round] (376.31,188.98) -- (376.31,193.01);

\path[draw=drawColor,draw opacity=0.50,line width= 0.4pt,line join=round,line cap=round] (374.72,190.94) -- (378.76,190.94);

\path[draw=drawColor,draw opacity=0.50,line width= 0.4pt,line join=round,line cap=round] (376.74,188.92) -- (376.74,192.96);

\path[draw=drawColor,draw opacity=0.50,line width= 0.4pt,line join=round,line cap=round] (374.67,192.02) -- (378.71,192.02);

\path[draw=drawColor,draw opacity=0.50,line width= 0.4pt,line join=round,line cap=round] (376.69,190.00) -- (376.69,194.04);

\path[draw=drawColor,draw opacity=0.50,line width= 0.4pt,line join=round,line cap=round] (374.34,190.05) -- (378.37,190.05);

\path[draw=drawColor,draw opacity=0.50,line width= 0.4pt,line join=round,line cap=round] (376.35,188.03) -- (376.35,192.06);

\path[draw=drawColor,draw opacity=0.50,line width= 0.4pt,line join=round,line cap=round] (375.40,192.27) -- (379.43,192.27);

\path[draw=drawColor,draw opacity=0.50,line width= 0.4pt,line join=round,line cap=round] (377.42,190.26) -- (377.42,194.29);

\path[draw=drawColor,draw opacity=0.50,line width= 0.4pt,line join=round,line cap=round] (343.80,164.83) -- (347.84,164.83);

\path[draw=drawColor,draw opacity=0.50,line width= 0.4pt,line join=round,line cap=round] (345.82,162.81) -- (345.82,166.84);

\path[draw=drawColor,draw opacity=0.50,line width= 0.4pt,line join=round,line cap=round] (328.23,140.54) -- (332.26,140.54);

\path[draw=drawColor,draw opacity=0.50,line width= 0.4pt,line join=round,line cap=round] (330.25,138.52) -- (330.25,142.56);

\path[draw=drawColor,draw opacity=0.50,line width= 0.4pt,line join=round,line cap=round] (323.76,129.80) -- (327.80,129.80);

\path[draw=drawColor,draw opacity=0.50,line width= 0.4pt,line join=round,line cap=round] (325.78,127.78) -- (325.78,131.82);

\path[draw=drawColor,draw opacity=0.50,line width= 0.4pt,line join=round,line cap=round] (321.73,129.07) -- (325.77,129.07);

\path[draw=drawColor,draw opacity=0.50,line width= 0.4pt,line join=round,line cap=round] (323.75,127.05) -- (323.75,131.09);

\path[draw=drawColor,draw opacity=0.50,line width= 0.4pt,line join=round,line cap=round] (327.71,137.02) -- (331.75,137.02);

\path[draw=drawColor,draw opacity=0.50,line width= 0.4pt,line join=round,line cap=round] (329.73,135.00) -- (329.73,139.03);

\path[draw=drawColor,draw opacity=0.50,line width= 0.4pt,line join=round,line cap=round] (324.81,131.70) -- (328.85,131.70);

\path[draw=drawColor,draw opacity=0.50,line width= 0.4pt,line join=round,line cap=round] (326.83,129.68) -- (326.83,133.71);

\path[draw=drawColor,draw opacity=0.50,line width= 0.4pt,line join=round,line cap=round] (329.98,141.07) -- (334.01,141.07);

\path[draw=drawColor,draw opacity=0.50,line width= 0.4pt,line join=round,line cap=round] (332.00,139.05) -- (332.00,143.08);

\path[draw=drawColor,draw opacity=0.50,line width= 0.4pt,line join=round,line cap=round] (324.40,134.33) -- (328.44,134.33);

\path[draw=drawColor,draw opacity=0.50,line width= 0.4pt,line join=round,line cap=round] (326.42,132.31) -- (326.42,136.34);

\path[draw=drawColor,draw opacity=0.50,line width= 0.4pt,line join=round,line cap=round] (320.98,127.19) -- (325.01,127.19);

\path[draw=drawColor,draw opacity=0.50,line width= 0.4pt,line join=round,line cap=round] (322.99,125.17) -- (322.99,129.21);

\path[draw=drawColor,draw opacity=0.50,line width= 0.4pt,line join=round,line cap=round] (325.14,129.80) -- (329.17,129.80);

\path[draw=drawColor,draw opacity=0.50,line width= 0.4pt,line join=round,line cap=round] (327.16,127.78) -- (327.16,131.82);

\path[draw=drawColor,draw opacity=0.50,line width= 0.4pt,line join=round,line cap=round] (333.73,148.77) -- (337.77,148.77);

\path[draw=drawColor,draw opacity=0.50,line width= 0.4pt,line join=round,line cap=round] (335.75,146.75) -- (335.75,150.79);

\path[draw=drawColor,draw opacity=0.50,line width= 0.4pt,line join=round,line cap=round] (345.65,164.26) -- (349.69,164.26);

\path[draw=drawColor,draw opacity=0.50,line width= 0.4pt,line join=round,line cap=round] (347.67,162.24) -- (347.67,166.28);

\path[draw=drawColor,draw opacity=0.50,line width= 0.4pt,line join=round,line cap=round] (357.40,176.33) -- (361.44,176.33);

\path[draw=drawColor,draw opacity=0.50,line width= 0.4pt,line join=round,line cap=round] (359.42,174.31) -- (359.42,178.34);

\path[draw=drawColor,draw opacity=0.50,line width= 0.4pt,line join=round,line cap=round] (358.41,175.54) -- (362.44,175.54);

\path[draw=drawColor,draw opacity=0.50,line width= 0.4pt,line join=round,line cap=round] (360.43,173.53) -- (360.43,177.56);

\path[draw=drawColor,draw opacity=0.50,line width= 0.4pt,line join=round,line cap=round] (360.64,177.73) -- (364.67,177.73);

\path[draw=drawColor,draw opacity=0.50,line width= 0.4pt,line join=round,line cap=round] (362.66,175.71) -- (362.66,179.75);

\path[draw=drawColor,draw opacity=0.50,line width= 0.4pt,line join=round,line cap=round] (353.48,174.46) -- (357.52,174.46);

\path[draw=drawColor,draw opacity=0.50,line width= 0.4pt,line join=round,line cap=round] (355.50,172.44) -- (355.50,176.48);

\path[draw=drawColor,draw opacity=0.50,line width= 0.4pt,line join=round,line cap=round] (355.69,150.60) -- (359.72,150.60);

\path[draw=drawColor,draw opacity=0.50,line width= 0.4pt,line join=round,line cap=round] (357.70,148.59) -- (357.70,152.62);

\path[draw=drawColor,draw opacity=0.50,line width= 0.4pt,line join=round,line cap=round] (357.36,149.34) -- (361.40,149.34);

\path[draw=drawColor,draw opacity=0.50,line width= 0.4pt,line join=round,line cap=round] (359.38,147.32) -- (359.38,151.36);

\path[draw=drawColor,draw opacity=0.50,line width= 0.4pt,line join=round,line cap=round] (356.33,149.07) -- (360.36,149.07);

\path[draw=drawColor,draw opacity=0.50,line width= 0.4pt,line join=round,line cap=round] (358.34,147.05) -- (358.34,151.09);

\path[draw=drawColor,draw opacity=0.50,line width= 0.4pt,line join=round,line cap=round] (392.35,148.07) -- (396.39,148.07);

\path[draw=drawColor,draw opacity=0.50,line width= 0.4pt,line join=round,line cap=round] (394.37,146.05) -- (394.37,150.09);

\path[draw=drawColor,draw opacity=0.50,line width= 0.4pt,line join=round,line cap=round] (353.32,147.42) -- (357.36,147.42);

\path[draw=drawColor,draw opacity=0.50,line width= 0.4pt,line join=round,line cap=round] (355.34,145.40) -- (355.34,149.44);

\path[draw=drawColor,draw opacity=0.50,line width= 0.4pt,line join=round,line cap=round] (354.74,148.29) -- (358.78,148.29);

\path[draw=drawColor,draw opacity=0.50,line width= 0.4pt,line join=round,line cap=round] (356.76,146.27) -- (356.76,150.31);

\path[draw=drawColor,draw opacity=0.50,line width= 0.4pt,line join=round,line cap=round] (354.99,147.68) -- (359.03,147.68);

\path[draw=drawColor,draw opacity=0.50,line width= 0.4pt,line join=round,line cap=round] (357.01,145.66) -- (357.01,149.70);

\path[draw=drawColor,draw opacity=0.50,line width= 0.4pt,line join=round,line cap=round] (354.17,147.39) -- (358.21,147.39);

\path[draw=drawColor,draw opacity=0.50,line width= 0.4pt,line join=round,line cap=round] (356.19,145.37) -- (356.19,149.40);

\path[draw=drawColor,draw opacity=0.50,line width= 0.4pt,line join=round,line cap=round] (356.00,148.64) -- (360.04,148.64);

\path[draw=drawColor,draw opacity=0.50,line width= 0.4pt,line join=round,line cap=round] (358.02,146.63) -- (358.02,150.66);

\path[draw=drawColor,draw opacity=0.50,line width= 0.4pt,line join=round,line cap=round] (355.32,147.98) -- (359.36,147.98);

\path[draw=drawColor,draw opacity=0.50,line width= 0.4pt,line join=round,line cap=round] (357.34,145.96) -- (357.34,150.00);

\path[draw=drawColor,draw opacity=0.50,line width= 0.4pt,line join=round,line cap=round] (355.50,148.56) -- (359.53,148.56);

\path[draw=drawColor,draw opacity=0.50,line width= 0.4pt,line join=round,line cap=round] (357.52,146.54) -- (357.52,150.58);

\path[draw=drawColor,draw opacity=0.50,line width= 0.4pt,line join=round,line cap=round] (356.07,151.50) -- (360.11,151.50);

\path[draw=drawColor,draw opacity=0.50,line width= 0.4pt,line join=round,line cap=round] (358.09,149.49) -- (358.09,153.52);

\path[draw=drawColor,draw opacity=0.50,line width= 0.4pt,line join=round,line cap=round] (392.35,153.59) -- (396.39,153.59);

\path[draw=drawColor,draw opacity=0.50,line width= 0.4pt,line join=round,line cap=round] (394.37,151.57) -- (394.37,155.60);

\path[draw=drawColor,draw opacity=0.50,line width= 0.4pt,line join=round,line cap=round] (355.85,151.69) -- (359.89,151.69);

\path[draw=drawColor,draw opacity=0.50,line width= 0.4pt,line join=round,line cap=round] (357.87,149.67) -- (357.87,153.71);

\path[draw=drawColor,draw opacity=0.50,line width= 0.4pt,line join=round,line cap=round] (355.79,151.66) -- (359.82,151.66);

\path[draw=drawColor,draw opacity=0.50,line width= 0.4pt,line join=round,line cap=round] (357.81,149.64) -- (357.81,153.67);

\path[draw=drawColor,draw opacity=0.50,line width= 0.4pt,line join=round,line cap=round] (355.96,151.33) -- (360.00,151.33);

\path[draw=drawColor,draw opacity=0.50,line width= 0.4pt,line join=round,line cap=round] (357.98,149.31) -- (357.98,153.34);

\path[draw=drawColor,draw opacity=0.50,line width= 0.4pt,line join=round,line cap=round] (355.58,149.02) -- (359.61,149.02);

\path[draw=drawColor,draw opacity=0.50,line width= 0.4pt,line join=round,line cap=round] (357.60,147.00) -- (357.60,151.04);

\path[draw=drawColor,draw opacity=0.50,line width= 0.4pt,line join=round,line cap=round] (356.89,149.51) -- (360.92,149.51);

\path[draw=drawColor,draw opacity=0.50,line width= 0.4pt,line join=round,line cap=round] (358.91,147.49) -- (358.91,151.53);

\path[draw=drawColor,draw opacity=0.50,line width= 0.4pt,line join=round,line cap=round] (358.78,149.43) -- (362.82,149.43);

\path[draw=drawColor,draw opacity=0.50,line width= 0.4pt,line join=round,line cap=round] (360.80,147.41) -- (360.80,151.45);

\path[draw=drawColor,draw opacity=0.50,line width= 0.4pt,line join=round,line cap=round] (357.14,149.65) -- (361.18,149.65);

\path[draw=drawColor,draw opacity=0.50,line width= 0.4pt,line join=round,line cap=round] (359.16,147.63) -- (359.16,151.67);

\path[draw=drawColor,draw opacity=0.50,line width= 0.4pt,line join=round,line cap=round] (357.31,149.21) -- (361.35,149.21);

\path[draw=drawColor,draw opacity=0.50,line width= 0.4pt,line join=round,line cap=round] (359.33,147.19) -- (359.33,151.22);

\path[draw=drawColor,draw opacity=0.50,line width= 0.4pt,line join=round,line cap=round] (354.49,147.13) -- (358.52,147.13);

\path[draw=drawColor,draw opacity=0.50,line width= 0.4pt,line join=round,line cap=round] (356.51,145.12) -- (356.51,149.15);

\path[draw=drawColor,draw opacity=0.50,line width= 0.4pt,line join=round,line cap=round] (354.87,147.25) -- (358.91,147.25);

\path[draw=drawColor,draw opacity=0.50,line width= 0.4pt,line join=round,line cap=round] (356.89,145.23) -- (356.89,149.26);

\path[draw=drawColor,draw opacity=0.50,line width= 0.4pt,line join=round,line cap=round] (356.64,148.53) -- (360.68,148.53);

\path[draw=drawColor,draw opacity=0.50,line width= 0.4pt,line join=round,line cap=round] (358.66,146.51) -- (358.66,150.55);

\path[draw=drawColor,draw opacity=0.50,line width= 0.4pt,line join=round,line cap=round] (356.74,148.16) -- (360.78,148.16);

\path[draw=drawColor,draw opacity=0.50,line width= 0.4pt,line join=round,line cap=round] (358.76,146.14) -- (358.76,150.18);

\path[draw=drawColor,draw opacity=0.50,line width= 0.4pt,line join=round,line cap=round] (354.31,146.48) -- (358.35,146.48);

\path[draw=drawColor,draw opacity=0.50,line width= 0.4pt,line join=round,line cap=round] (356.33,144.46) -- (356.33,148.49);

\path[draw=drawColor,draw opacity=0.50,line width= 0.4pt,line join=round,line cap=round] (356.68,148.67) -- (360.72,148.67);

\path[draw=drawColor,draw opacity=0.50,line width= 0.4pt,line join=round,line cap=round] (358.70,146.66) -- (358.70,150.69);

\path[draw=drawColor,draw opacity=0.50,line width= 0.4pt,line join=round,line cap=round] (356.75,150.65) -- (360.79,150.65);

\path[draw=drawColor,draw opacity=0.50,line width= 0.4pt,line join=round,line cap=round] (358.77,148.63) -- (358.77,152.67);

\path[draw=drawColor,draw opacity=0.50,line width= 0.4pt,line join=round,line cap=round] (392.35,150.50) -- (396.39,150.50);

\path[draw=drawColor,draw opacity=0.50,line width= 0.4pt,line join=round,line cap=round] (394.37,148.48) -- (394.37,152.51);

\path[draw=drawColor,draw opacity=0.50,line width= 0.4pt,line join=round,line cap=round] (356.18,149.94) -- (360.22,149.94);

\path[draw=drawColor,draw opacity=0.50,line width= 0.4pt,line join=round,line cap=round] (358.20,147.93) -- (358.20,151.96);

\path[draw=drawColor,draw opacity=0.50,line width= 0.4pt,line join=round,line cap=round] (355.56,149.04) -- (359.60,149.04);

\path[draw=drawColor,draw opacity=0.50,line width= 0.4pt,line join=round,line cap=round] (357.58,147.03) -- (357.58,151.06);

\path[draw=drawColor,draw opacity=0.50,line width= 0.4pt,line join=round,line cap=round] (357.04,149.47) -- (361.07,149.47);

\path[draw=drawColor,draw opacity=0.50,line width= 0.4pt,line join=round,line cap=round] (359.05,147.45) -- (359.05,151.49);

\path[draw=drawColor,draw opacity=0.50,line width= 0.4pt,line join=round,line cap=round] (357.15,154.76) -- (361.18,154.76);

\path[draw=drawColor,draw opacity=0.50,line width= 0.4pt,line join=round,line cap=round] (359.16,152.75) -- (359.16,156.78);

\path[draw=drawColor,draw opacity=0.50,line width= 0.4pt,line join=round,line cap=round] (358.06,154.07) -- (362.09,154.07);

\path[draw=drawColor,draw opacity=0.50,line width= 0.4pt,line join=round,line cap=round] (360.08,152.05) -- (360.08,156.09);

\path[draw=drawColor,draw opacity=0.50,line width= 0.4pt,line join=round,line cap=round] (356.07,152.62) -- (360.10,152.62);

\path[draw=drawColor,draw opacity=0.50,line width= 0.4pt,line join=round,line cap=round] (358.08,150.60) -- (358.08,154.64);

\path[draw=drawColor,draw opacity=0.50,line width= 0.4pt,line join=round,line cap=round] (357.84,152.82) -- (361.88,152.82);

\path[draw=drawColor,draw opacity=0.50,line width= 0.4pt,line join=round,line cap=round] (359.86,150.80) -- (359.86,154.84);

\path[draw=drawColor,draw opacity=0.50,line width= 0.4pt,line join=round,line cap=round] (392.35,151.93) -- (396.39,151.93);

\path[draw=drawColor,draw opacity=0.50,line width= 0.4pt,line join=round,line cap=round] (394.37,149.92) -- (394.37,153.95);

\path[draw=drawColor,draw opacity=0.50,line width= 0.4pt,line join=round,line cap=round] (357.68,154.19) -- (361.72,154.19);

\path[draw=drawColor,draw opacity=0.50,line width= 0.4pt,line join=round,line cap=round] (359.70,152.17) -- (359.70,156.20);

\path[draw=drawColor,draw opacity=0.50,line width= 0.4pt,line join=round,line cap=round] (356.67,154.78) -- (360.71,154.78);

\path[draw=drawColor,draw opacity=0.50,line width= 0.4pt,line join=round,line cap=round] (358.69,152.76) -- (358.69,156.80);

\path[draw=drawColor,draw opacity=0.50,line width= 0.4pt,line join=round,line cap=round] (358.29,151.49) -- (362.33,151.49);

\path[draw=drawColor,draw opacity=0.50,line width= 0.4pt,line join=round,line cap=round] (360.31,149.47) -- (360.31,153.51);

\path[draw=drawColor,draw opacity=0.50,line width= 0.4pt,line join=round,line cap=round] (357.63,149.75) -- (361.66,149.75);

\path[draw=drawColor,draw opacity=0.50,line width= 0.4pt,line join=round,line cap=round] (359.64,147.74) -- (359.64,151.77);

\path[draw=drawColor,draw opacity=0.50,line width= 0.4pt,line join=round,line cap=round] (357.17,149.33) -- (361.21,149.33);

\path[draw=drawColor,draw opacity=0.50,line width= 0.4pt,line join=round,line cap=round] (359.19,147.31) -- (359.19,151.34);

\path[draw=drawColor,draw opacity=0.50,line width= 0.4pt,line join=round,line cap=round] (357.89,150.73) -- (361.93,150.73);

\path[draw=drawColor,draw opacity=0.50,line width= 0.4pt,line join=round,line cap=round] (359.91,148.71) -- (359.91,152.75);

\path[draw=drawColor,draw opacity=0.50,line width= 0.4pt,line join=round,line cap=round] (358.70,158.41) -- (362.73,158.41);

\path[draw=drawColor,draw opacity=0.50,line width= 0.4pt,line join=round,line cap=round] (360.71,156.39) -- (360.71,160.42);

\path[draw=drawColor,draw opacity=0.50,line width= 0.4pt,line join=round,line cap=round] (358.58,155.91) -- (362.62,155.91);

\path[draw=drawColor,draw opacity=0.50,line width= 0.4pt,line join=round,line cap=round] (360.60,153.89) -- (360.60,157.92);

\path[draw=drawColor,draw opacity=0.50,line width= 0.4pt,line join=round,line cap=round] (355.82,154.48) -- (359.85,154.48);

\path[draw=drawColor,draw opacity=0.50,line width= 0.4pt,line join=round,line cap=round] (357.83,152.46) -- (357.83,156.50);

\path[draw=drawColor,draw opacity=0.50,line width= 0.4pt,line join=round,line cap=round] (357.88,157.00) -- (361.92,157.00);

\path[draw=drawColor,draw opacity=0.50,line width= 0.4pt,line join=round,line cap=round] (359.90,154.99) -- (359.90,159.02);

\path[draw=drawColor,draw opacity=0.50,line width= 0.4pt,line join=round,line cap=round] (358.13,157.68) -- (362.17,157.68);

\path[draw=drawColor,draw opacity=0.50,line width= 0.4pt,line join=round,line cap=round] (360.15,155.66) -- (360.15,159.70);

\path[draw=drawColor,draw opacity=0.50,line width= 0.4pt,line join=round,line cap=round] (339.50,145.83) -- (343.53,145.83);

\path[draw=drawColor,draw opacity=0.50,line width= 0.4pt,line join=round,line cap=round] (341.51,143.82) -- (341.51,147.85);

\path[draw=drawColor,draw opacity=0.50,line width= 0.4pt,line join=round,line cap=round] (338.54,145.49) -- (342.58,145.49);

\path[draw=drawColor,draw opacity=0.50,line width= 0.4pt,line join=round,line cap=round] (340.56,143.47) -- (340.56,147.51);

\path[draw=drawColor,draw opacity=0.50,line width= 0.4pt,line join=round,line cap=round] (340.35,145.76) -- (344.39,145.76);

\path[draw=drawColor,draw opacity=0.50,line width= 0.4pt,line join=round,line cap=round] (342.37,143.74) -- (342.37,147.78);

\path[draw=drawColor,draw opacity=0.50,line width= 0.4pt,line join=round,line cap=round] (339.32,145.85) -- (343.35,145.85);

\path[draw=drawColor,draw opacity=0.50,line width= 0.4pt,line join=round,line cap=round] (341.34,143.83) -- (341.34,147.87);

\path[draw=drawColor,draw opacity=0.50,line width= 0.4pt,line join=round,line cap=round] (337.00,143.29) -- (341.03,143.29);

\path[draw=drawColor,draw opacity=0.50,line width= 0.4pt,line join=round,line cap=round] (339.01,141.27) -- (339.01,145.31);

\path[draw=drawColor,draw opacity=0.50,line width= 0.4pt,line join=round,line cap=round] (337.30,143.44) -- (341.34,143.44);

\path[draw=drawColor,draw opacity=0.50,line width= 0.4pt,line join=round,line cap=round] (339.32,141.43) -- (339.32,145.46);

\path[draw=drawColor,draw opacity=0.50,line width= 0.4pt,line join=round,line cap=round] (338.81,145.29) -- (342.84,145.29);

\path[draw=drawColor,draw opacity=0.50,line width= 0.4pt,line join=round,line cap=round] (340.83,143.27) -- (340.83,147.30);

\path[draw=drawColor,draw opacity=0.50,line width= 0.4pt,line join=round,line cap=round] (339.08,145.80) -- (343.11,145.80);

\path[draw=drawColor,draw opacity=0.50,line width= 0.4pt,line join=round,line cap=round] (341.10,143.78) -- (341.10,147.82);

\path[draw=drawColor,draw opacity=0.50,line width= 0.4pt,line join=round,line cap=round] (339.45,148.62) -- (343.49,148.62);

\path[draw=drawColor,draw opacity=0.50,line width= 0.4pt,line join=round,line cap=round] (341.47,146.60) -- (341.47,150.63);

\path[draw=drawColor,draw opacity=0.50,line width= 0.4pt,line join=round,line cap=round] (339.31,145.89) -- (343.35,145.89);

\path[draw=drawColor,draw opacity=0.50,line width= 0.4pt,line join=round,line cap=round] (341.33,143.87) -- (341.33,147.91);

\path[draw=drawColor,draw opacity=0.50,line width= 0.4pt,line join=round,line cap=round] (338.92,145.10) -- (342.95,145.10);

\path[draw=drawColor,draw opacity=0.50,line width= 0.4pt,line join=round,line cap=round] (340.94,143.08) -- (340.94,147.11);

\path[draw=drawColor,draw opacity=0.50,line width= 0.4pt,line join=round,line cap=round] (339.44,145.76) -- (343.47,145.76);

\path[draw=drawColor,draw opacity=0.50,line width= 0.4pt,line join=round,line cap=round] (341.45,143.74) -- (341.45,147.78);

\path[draw=drawColor,draw opacity=0.50,line width= 0.4pt,line join=round,line cap=round] (338.27,145.32) -- (342.31,145.32);

\path[draw=drawColor,draw opacity=0.50,line width= 0.4pt,line join=round,line cap=round] (340.29,143.31) -- (340.29,147.34);

\path[draw=drawColor,draw opacity=0.50,line width= 0.4pt,line join=round,line cap=round] (339.52,145.58) -- (343.55,145.58);

\path[draw=drawColor,draw opacity=0.50,line width= 0.4pt,line join=round,line cap=round] (341.54,143.56) -- (341.54,147.60);

\path[draw=drawColor,draw opacity=0.50,line width= 0.4pt,line join=round,line cap=round] (338.61,144.47) -- (342.65,144.47);

\path[draw=drawColor,draw opacity=0.50,line width= 0.4pt,line join=round,line cap=round] (340.63,142.45) -- (340.63,146.49);

\path[draw=drawColor,draw opacity=0.50,line width= 0.4pt,line join=round,line cap=round] (339.05,145.45) -- (343.08,145.45);

\path[draw=drawColor,draw opacity=0.50,line width= 0.4pt,line join=round,line cap=round] (341.07,143.44) -- (341.07,147.47);

\path[draw=drawColor,draw opacity=0.50,line width= 0.4pt,line join=round,line cap=round] (305.98,137.90) -- (310.02,137.90);

\path[draw=drawColor,draw opacity=0.50,line width= 0.4pt,line join=round,line cap=round] (308.00,135.88) -- (308.00,139.92);

\path[draw=drawColor,draw opacity=0.50,line width= 0.4pt,line join=round,line cap=round] (305.28,121.88) -- (309.31,121.88);

\path[draw=drawColor,draw opacity=0.50,line width= 0.4pt,line join=round,line cap=round] (307.29,119.86) -- (307.29,123.90);

\path[draw=drawColor,draw opacity=0.50,line width= 0.4pt,line join=round,line cap=round] (304.96,120.70) -- (309.00,120.70);

\path[draw=drawColor,draw opacity=0.50,line width= 0.4pt,line join=round,line cap=round] (306.98,118.69) -- (306.98,122.72);

\path[draw=drawColor,draw opacity=0.50,line width= 0.4pt,line join=round,line cap=round] (304.07,116.44) -- (308.11,116.44);

\path[draw=drawColor,draw opacity=0.50,line width= 0.4pt,line join=round,line cap=round] (306.09,114.43) -- (306.09,118.46);

\path[draw=drawColor,draw opacity=0.50,line width= 0.4pt,line join=round,line cap=round] (303.72,116.67) -- (307.76,116.67);

\path[draw=drawColor,draw opacity=0.50,line width= 0.4pt,line join=round,line cap=round] (305.74,114.65) -- (305.74,118.68);

\path[draw=drawColor,draw opacity=0.50,line width= 0.4pt,line join=round,line cap=round] (305.07,122.69) -- (309.10,122.69);

\path[draw=drawColor,draw opacity=0.50,line width= 0.4pt,line join=round,line cap=round] (307.09,120.68) -- (307.09,124.71);

\path[draw=drawColor,draw opacity=0.50,line width= 0.4pt,line join=round,line cap=round] (305.17,116.88) -- (309.21,116.88);

\path[draw=drawColor,draw opacity=0.50,line width= 0.4pt,line join=round,line cap=round] (307.19,114.87) -- (307.19,118.90);

\path[draw=drawColor,draw opacity=0.50,line width= 0.4pt,line join=round,line cap=round] (306.27,117.31) -- (310.31,117.31);

\path[draw=drawColor,draw opacity=0.50,line width= 0.4pt,line join=round,line cap=round] (308.29,115.29) -- (308.29,119.32);

\path[draw=drawColor,draw opacity=0.50,line width= 0.4pt,line join=round,line cap=round] (312.08,117.10) -- (316.11,117.10);

\path[draw=drawColor,draw opacity=0.50,line width= 0.4pt,line join=round,line cap=round] (314.10,115.08) -- (314.10,119.11);

\path[draw=drawColor,draw opacity=0.50,line width= 0.4pt,line join=round,line cap=round] (305.88,119.90) -- (309.92,119.90);

\path[draw=drawColor,draw opacity=0.50,line width= 0.4pt,line join=round,line cap=round] (307.90,117.88) -- (307.90,121.92);

\path[draw=drawColor,draw opacity=0.50,line width= 0.4pt,line join=round,line cap=round] (305.38,116.22) -- (309.42,116.22);

\path[draw=drawColor,draw opacity=0.50,line width= 0.4pt,line join=round,line cap=round] (307.40,114.20) -- (307.40,118.24);

\path[draw=drawColor,draw opacity=0.50,line width= 0.4pt,line join=round,line cap=round] (307.01,124.73) -- (311.05,124.73);

\path[draw=drawColor,draw opacity=0.50,line width= 0.4pt,line join=round,line cap=round] (309.03,122.71) -- (309.03,126.74);

\path[draw=drawColor,draw opacity=0.50,line width= 0.4pt,line join=round,line cap=round] (307.45,124.05) -- (311.48,124.05);

\path[draw=drawColor,draw opacity=0.50,line width= 0.4pt,line join=round,line cap=round] (309.47,122.03) -- (309.47,126.07);

\path[draw=drawColor,draw opacity=0.50,line width= 0.4pt,line join=round,line cap=round] (317.56,124.94) -- (321.59,124.94);

\path[draw=drawColor,draw opacity=0.50,line width= 0.4pt,line join=round,line cap=round] (319.58,122.93) -- (319.58,126.96);

\path[draw=drawColor,draw opacity=0.50,line width= 0.4pt,line join=round,line cap=round] (304.64,116.88) -- (308.67,116.88);

\path[draw=drawColor,draw opacity=0.50,line width= 0.4pt,line join=round,line cap=round] (306.65,114.87) -- (306.65,118.90);

\path[draw=drawColor,draw opacity=0.50,line width= 0.4pt,line join=round,line cap=round] (307.28,125.26) -- (311.31,125.26);

\path[draw=drawColor,draw opacity=0.50,line width= 0.4pt,line join=round,line cap=round] (309.29,123.25) -- (309.29,127.28);
\definecolor{drawColor}{RGB}{102,166,30}

\path[draw=drawColor,draw opacity=0.50,line width= 0.4pt,line join=round,line cap=round] (292.67,104.47) rectangle (295.52,107.33);

\path[draw=drawColor,draw opacity=0.50,line width= 0.4pt,line join=round,line cap=round] (292.67,104.47) -- (295.52,107.33);

\path[draw=drawColor,draw opacity=0.50,line width= 0.4pt,line join=round,line cap=round] (292.67,107.33) -- (295.52,104.47);

\path[draw=drawColor,draw opacity=0.50,line width= 0.4pt,line join=round,line cap=round] (293.89,106.50) rectangle (296.75,109.35);

\path[draw=drawColor,draw opacity=0.50,line width= 0.4pt,line join=round,line cap=round] (293.89,106.50) -- (296.75,109.35);

\path[draw=drawColor,draw opacity=0.50,line width= 0.4pt,line join=round,line cap=round] (293.89,109.35) -- (296.75,106.50);

\path[draw=drawColor,draw opacity=0.50,line width= 0.4pt,line join=round,line cap=round] (293.60,105.53) rectangle (296.45,108.38);

\path[draw=drawColor,draw opacity=0.50,line width= 0.4pt,line join=round,line cap=round] (293.60,105.53) -- (296.45,108.38);

\path[draw=drawColor,draw opacity=0.50,line width= 0.4pt,line join=round,line cap=round] (293.60,108.38) -- (296.45,105.53);

\path[draw=drawColor,draw opacity=0.50,line width= 0.4pt,line join=round,line cap=round] (292.67,111.37) rectangle (295.52,114.22);

\path[draw=drawColor,draw opacity=0.50,line width= 0.4pt,line join=round,line cap=round] (292.67,111.37) -- (295.52,114.22);

\path[draw=drawColor,draw opacity=0.50,line width= 0.4pt,line join=round,line cap=round] (292.67,114.22) -- (295.52,111.37);

\path[draw=drawColor,draw opacity=0.50,line width= 0.4pt,line join=round,line cap=round] (294.46,105.01) rectangle (297.32,107.87);

\path[draw=drawColor,draw opacity=0.50,line width= 0.4pt,line join=round,line cap=round] (294.46,105.01) -- (297.32,107.87);

\path[draw=drawColor,draw opacity=0.50,line width= 0.4pt,line join=round,line cap=round] (294.46,107.87) -- (297.32,105.01);

\path[draw=drawColor,draw opacity=0.50,line width= 0.4pt,line join=round,line cap=round] (294.46,105.01) rectangle (297.32,107.87);

\path[draw=drawColor,draw opacity=0.50,line width= 0.4pt,line join=round,line cap=round] (294.46,105.01) -- (297.32,107.87);

\path[draw=drawColor,draw opacity=0.50,line width= 0.4pt,line join=round,line cap=round] (294.46,107.87) -- (297.32,105.01);

\path[draw=drawColor,draw opacity=0.50,line width= 0.4pt,line join=round,line cap=round] (292.99,104.47) rectangle (295.84,107.33);

\path[draw=drawColor,draw opacity=0.50,line width= 0.4pt,line join=round,line cap=round] (292.99,104.47) -- (295.84,107.33);

\path[draw=drawColor,draw opacity=0.50,line width= 0.4pt,line join=round,line cap=round] (292.99,107.33) -- (295.84,104.47);

\path[draw=drawColor,draw opacity=0.50,line width= 0.4pt,line join=round,line cap=round] (291.30,103.91) rectangle (294.15,106.76);

\path[draw=drawColor,draw opacity=0.50,line width= 0.4pt,line join=round,line cap=round] (291.30,103.91) -- (294.15,106.76);

\path[draw=drawColor,draw opacity=0.50,line width= 0.4pt,line join=round,line cap=round] (291.30,106.76) -- (294.15,103.91);

\path[draw=drawColor,draw opacity=0.50,line width= 0.4pt,line join=round,line cap=round] (293.60,105.01) rectangle (296.45,107.87);

\path[draw=drawColor,draw opacity=0.50,line width= 0.4pt,line join=round,line cap=round] (293.60,105.01) -- (296.45,107.87);

\path[draw=drawColor,draw opacity=0.50,line width= 0.4pt,line join=round,line cap=round] (293.60,107.87) -- (296.45,105.01);

\path[draw=drawColor,draw opacity=0.50,line width= 0.4pt,line join=round,line cap=round] (294.18,105.53) rectangle (297.04,108.38);

\path[draw=drawColor,draw opacity=0.50,line width= 0.4pt,line join=round,line cap=round] (294.18,105.53) -- (297.04,108.38);

\path[draw=drawColor,draw opacity=0.50,line width= 0.4pt,line join=round,line cap=round] (294.18,108.38) -- (297.04,105.53);

\path[draw=drawColor,draw opacity=0.50,line width= 0.4pt,line join=round,line cap=round] (293.89,106.03) rectangle (296.75,108.88);

\path[draw=drawColor,draw opacity=0.50,line width= 0.4pt,line join=round,line cap=round] (293.89,106.03) -- (296.75,108.88);

\path[draw=drawColor,draw opacity=0.50,line width= 0.4pt,line join=round,line cap=round] (293.89,108.88) -- (296.75,106.03);

\path[draw=drawColor,draw opacity=0.50,line width= 0.4pt,line join=round,line cap=round] (293.60,106.03) rectangle (296.45,108.88);

\path[draw=drawColor,draw opacity=0.50,line width= 0.4pt,line join=round,line cap=round] (293.60,106.03) -- (296.45,108.88);

\path[draw=drawColor,draw opacity=0.50,line width= 0.4pt,line join=round,line cap=round] (293.60,108.88) -- (296.45,106.03);

\path[draw=drawColor,draw opacity=0.50,line width= 0.4pt,line join=round,line cap=round] (293.30,108.23) rectangle (296.15,111.08);

\path[draw=drawColor,draw opacity=0.50,line width= 0.4pt,line join=round,line cap=round] (293.30,108.23) -- (296.15,111.08);

\path[draw=drawColor,draw opacity=0.50,line width= 0.4pt,line join=round,line cap=round] (293.30,111.08) -- (296.15,108.23);

\path[draw=drawColor,draw opacity=0.50,line width= 0.4pt,line join=round,line cap=round] (294.46,106.96) rectangle (297.32,109.81);

\path[draw=drawColor,draw opacity=0.50,line width= 0.4pt,line join=round,line cap=round] (294.46,106.96) -- (297.32,109.81);

\path[draw=drawColor,draw opacity=0.50,line width= 0.4pt,line join=round,line cap=round] (294.46,109.81) -- (297.32,106.96);

\path[draw=drawColor,draw opacity=0.50,line width= 0.4pt,line join=round,line cap=round] (293.30,105.53) rectangle (296.15,108.38);

\path[draw=drawColor,draw opacity=0.50,line width= 0.4pt,line join=round,line cap=round] (293.30,105.53) -- (296.15,108.38);

\path[draw=drawColor,draw opacity=0.50,line width= 0.4pt,line join=round,line cap=round] (293.30,108.38) -- (296.15,105.53);

\path[draw=drawColor,draw opacity=0.50,line width= 0.4pt,line join=round,line cap=round] (292.67,106.96) rectangle (295.52,109.81);

\path[draw=drawColor,draw opacity=0.50,line width= 0.4pt,line join=round,line cap=round] (292.67,106.96) -- (295.52,109.81);

\path[draw=drawColor,draw opacity=0.50,line width= 0.4pt,line join=round,line cap=round] (292.67,109.81) -- (295.52,106.96);

\path[draw=drawColor,draw opacity=0.50,line width= 0.4pt,line join=round,line cap=round] (299.18,110.08) rectangle (302.03,112.93);

\path[draw=drawColor,draw opacity=0.50,line width= 0.4pt,line join=round,line cap=round] (299.18,110.08) -- (302.03,112.93);

\path[draw=drawColor,draw opacity=0.50,line width= 0.4pt,line join=round,line cap=round] (299.18,112.93) -- (302.03,110.08);

\path[draw=drawColor,draw opacity=0.50,line width= 0.4pt,line join=round,line cap=round] (299.18,110.08) rectangle (302.03,112.93);

\path[draw=drawColor,draw opacity=0.50,line width= 0.4pt,line join=round,line cap=round] (299.18,110.08) -- (302.03,112.93);

\path[draw=drawColor,draw opacity=0.50,line width= 0.4pt,line join=round,line cap=round] (299.18,112.93) -- (302.03,110.08);

\path[draw=drawColor,draw opacity=0.50,line width= 0.4pt,line join=round,line cap=round] (299.89,110.74) rectangle (302.75,113.59);

\path[draw=drawColor,draw opacity=0.50,line width= 0.4pt,line join=round,line cap=round] (299.89,110.74) -- (302.75,113.59);

\path[draw=drawColor,draw opacity=0.50,line width= 0.4pt,line join=round,line cap=round] (299.89,113.59) -- (302.75,110.74);

\path[draw=drawColor,draw opacity=0.50,line width= 0.4pt,line join=round,line cap=round] (300.24,110.08) rectangle (303.09,112.93);

\path[draw=drawColor,draw opacity=0.50,line width= 0.4pt,line join=round,line cap=round] (300.24,110.08) -- (303.09,112.93);

\path[draw=drawColor,draw opacity=0.50,line width= 0.4pt,line join=round,line cap=round] (300.24,112.93) -- (303.09,110.08);

\path[draw=drawColor,draw opacity=0.50,line width= 0.4pt,line join=round,line cap=round] (299.18,109.73) rectangle (302.03,112.58);

\path[draw=drawColor,draw opacity=0.50,line width= 0.4pt,line join=round,line cap=round] (299.18,109.73) -- (302.03,112.58);

\path[draw=drawColor,draw opacity=0.50,line width= 0.4pt,line join=round,line cap=round] (299.18,112.58) -- (302.03,109.73);

\path[draw=drawColor,draw opacity=0.50,line width= 0.4pt,line join=round,line cap=round] (299.72,109.73) rectangle (302.57,112.58);

\path[draw=drawColor,draw opacity=0.50,line width= 0.4pt,line join=round,line cap=round] (299.72,109.73) -- (302.57,112.58);

\path[draw=drawColor,draw opacity=0.50,line width= 0.4pt,line join=round,line cap=round] (299.72,112.58) -- (302.57,109.73);

\path[draw=drawColor,draw opacity=0.50,line width= 0.4pt,line join=round,line cap=round] (299.89,109.37) rectangle (302.75,112.23);

\path[draw=drawColor,draw opacity=0.50,line width= 0.4pt,line join=round,line cap=round] (299.89,109.37) -- (302.75,112.23);

\path[draw=drawColor,draw opacity=0.50,line width= 0.4pt,line join=round,line cap=round] (299.89,112.23) -- (302.75,109.37);

\path[draw=drawColor,draw opacity=0.50,line width= 0.4pt,line join=round,line cap=round] (299.18,109.37) rectangle (302.03,112.23);

\path[draw=drawColor,draw opacity=0.50,line width= 0.4pt,line join=round,line cap=round] (299.18,109.37) -- (302.03,112.23);

\path[draw=drawColor,draw opacity=0.50,line width= 0.4pt,line join=round,line cap=round] (299.18,112.23) -- (302.03,109.37);

\path[draw=drawColor,draw opacity=0.50,line width= 0.4pt,line join=round,line cap=round] (298.22,108.23) rectangle (301.07,111.08);

\path[draw=drawColor,draw opacity=0.50,line width= 0.4pt,line join=round,line cap=round] (298.22,108.23) -- (301.07,111.08);

\path[draw=drawColor,draw opacity=0.50,line width= 0.4pt,line join=round,line cap=round] (298.22,111.08) -- (301.07,108.23);

\path[draw=drawColor,draw opacity=0.50,line width= 0.4pt,line join=round,line cap=round] (297.17,106.96) rectangle (300.02,109.81);

\path[draw=drawColor,draw opacity=0.50,line width= 0.4pt,line join=round,line cap=round] (297.17,106.96) -- (300.02,109.81);

\path[draw=drawColor,draw opacity=0.50,line width= 0.4pt,line join=round,line cap=round] (297.17,109.81) -- (300.02,106.96);

\path[draw=drawColor,draw opacity=0.50,line width= 0.4pt,line join=round,line cap=round] (300.07,109.37) rectangle (302.92,112.23);

\path[draw=drawColor,draw opacity=0.50,line width= 0.4pt,line join=round,line cap=round] (300.07,109.37) -- (302.92,112.23);

\path[draw=drawColor,draw opacity=0.50,line width= 0.4pt,line join=round,line cap=round] (300.07,112.23) -- (302.92,109.37);

\path[draw=drawColor,draw opacity=0.50,line width= 0.4pt,line join=round,line cap=round] (299.72,110.08) rectangle (302.57,112.93);

\path[draw=drawColor,draw opacity=0.50,line width= 0.4pt,line join=round,line cap=round] (299.72,110.08) -- (302.57,112.93);

\path[draw=drawColor,draw opacity=0.50,line width= 0.4pt,line join=round,line cap=round] (299.72,112.93) -- (302.57,110.08);

\path[draw=drawColor,draw opacity=0.50,line width= 0.4pt,line join=round,line cap=round] (298.61,109.37) rectangle (301.46,112.23);

\path[draw=drawColor,draw opacity=0.50,line width= 0.4pt,line join=round,line cap=round] (298.61,109.37) -- (301.46,112.23);

\path[draw=drawColor,draw opacity=0.50,line width= 0.4pt,line join=round,line cap=round] (298.61,112.23) -- (301.46,109.37);

\path[draw=drawColor,draw opacity=0.50,line width= 0.4pt,line join=round,line cap=round] (301.20,117.97) rectangle (304.06,120.82);

\path[draw=drawColor,draw opacity=0.50,line width= 0.4pt,line join=round,line cap=round] (301.20,117.97) -- (304.06,120.82);

\path[draw=drawColor,draw opacity=0.50,line width= 0.4pt,line join=round,line cap=round] (301.20,120.82) -- (304.06,117.97);

\path[draw=drawColor,draw opacity=0.50,line width= 0.4pt,line join=round,line cap=round] (297.17,108.62) rectangle (300.02,111.47);

\path[draw=drawColor,draw opacity=0.50,line width= 0.4pt,line join=round,line cap=round] (297.17,108.62) -- (300.02,111.47);

\path[draw=drawColor,draw opacity=0.50,line width= 0.4pt,line join=round,line cap=round] (297.17,111.47) -- (300.02,108.62);

\path[draw=drawColor,draw opacity=0.50,line width= 0.4pt,line join=round,line cap=round] (298.80,109.73) rectangle (301.66,112.58);

\path[draw=drawColor,draw opacity=0.50,line width= 0.4pt,line join=round,line cap=round] (298.80,109.73) -- (301.66,112.58);

\path[draw=drawColor,draw opacity=0.50,line width= 0.4pt,line join=round,line cap=round] (298.80,112.58) -- (301.66,109.73);
\definecolor{drawColor}{RGB}{231,41,138}

\path[draw=drawColor,draw opacity=0.50,line width= 0.4pt,line join=round,line cap=round] (324.50,125.47) -- (328.54,125.47);

\path[draw=drawColor,draw opacity=0.50,line width= 0.4pt,line join=round,line cap=round] (326.52,123.45) -- (326.52,127.49);

\path[draw=drawColor,draw opacity=0.50,line width= 0.4pt,line join=round,line cap=round] (324.81,126.27) -- (328.85,126.27);

\path[draw=drawColor,draw opacity=0.50,line width= 0.4pt,line join=round,line cap=round] (326.83,124.25) -- (326.83,128.28);

\path[draw=drawColor,draw opacity=0.50,line width= 0.4pt,line join=round,line cap=round] (325.29,127.88) -- (329.32,127.88);

\path[draw=drawColor,draw opacity=0.50,line width= 0.4pt,line join=round,line cap=round] (327.31,125.86) -- (327.31,129.90);

\path[draw=drawColor,draw opacity=0.50,line width= 0.4pt,line join=round,line cap=round] (325.34,126.36) -- (329.38,126.36);

\path[draw=drawColor,draw opacity=0.50,line width= 0.4pt,line join=round,line cap=round] (327.36,124.34) -- (327.36,128.38);

\path[draw=drawColor,draw opacity=0.50,line width= 0.4pt,line join=round,line cap=round] (324.80,127.28) -- (328.83,127.28);

\path[draw=drawColor,draw opacity=0.50,line width= 0.4pt,line join=round,line cap=round] (326.81,125.26) -- (326.81,129.30);

\path[draw=drawColor,draw opacity=0.50,line width= 0.4pt,line join=round,line cap=round] (325.44,125.78) -- (329.47,125.78);

\path[draw=drawColor,draw opacity=0.50,line width= 0.4pt,line join=round,line cap=round] (327.45,123.76) -- (327.45,127.79);

\path[draw=drawColor,draw opacity=0.50,line width= 0.4pt,line join=round,line cap=round] (323.74,124.84) -- (327.78,124.84);

\path[draw=drawColor,draw opacity=0.50,line width= 0.4pt,line join=round,line cap=round] (325.76,122.82) -- (325.76,126.85);

\path[draw=drawColor,draw opacity=0.50,line width= 0.4pt,line join=round,line cap=round] (323.20,124.62) -- (327.23,124.62);

\path[draw=drawColor,draw opacity=0.50,line width= 0.4pt,line join=round,line cap=round] (325.22,122.60) -- (325.22,126.63);

\path[draw=drawColor,draw opacity=0.50,line width= 0.4pt,line join=round,line cap=round] (325.08,126.46) -- (329.12,126.46);

\path[draw=drawColor,draw opacity=0.50,line width= 0.4pt,line join=round,line cap=round] (327.10,124.44) -- (327.10,128.47);

\path[draw=drawColor,draw opacity=0.50,line width= 0.4pt,line join=round,line cap=round] (324.22,124.51) -- (328.25,124.51);

\path[draw=drawColor,draw opacity=0.50,line width= 0.4pt,line join=round,line cap=round] (326.24,122.49) -- (326.24,126.52);

\path[draw=drawColor,draw opacity=0.50,line width= 0.4pt,line join=round,line cap=round] (324.99,125.26) -- (329.02,125.26);

\path[draw=drawColor,draw opacity=0.50,line width= 0.4pt,line join=round,line cap=round] (327.01,123.25) -- (327.01,127.28);

\path[draw=drawColor,draw opacity=0.50,line width= 0.4pt,line join=round,line cap=round] (326.23,128.45) -- (330.27,128.45);

\path[draw=drawColor,draw opacity=0.50,line width= 0.4pt,line join=round,line cap=round] (328.25,126.43) -- (328.25,130.47);

\path[draw=drawColor,draw opacity=0.50,line width= 0.4pt,line join=round,line cap=round] (324.16,127.01) -- (328.19,127.01);

\path[draw=drawColor,draw opacity=0.50,line width= 0.4pt,line join=round,line cap=round] (326.18,124.99) -- (326.18,129.03);

\path[draw=drawColor,draw opacity=0.50,line width= 0.4pt,line join=round,line cap=round] (325.65,131.87) -- (329.69,131.87);

\path[draw=drawColor,draw opacity=0.50,line width= 0.4pt,line join=round,line cap=round] (327.67,129.86) -- (327.67,133.89);

\path[draw=drawColor,draw opacity=0.50,line width= 0.4pt,line join=round,line cap=round] (325.33,128.92) -- (329.36,128.92);

\path[draw=drawColor,draw opacity=0.50,line width= 0.4pt,line join=round,line cap=round] (327.34,126.90) -- (327.34,130.94);

\path[draw=drawColor,draw opacity=0.50,line width= 0.4pt,line join=round,line cap=round] (325.87,126.46) -- (329.90,126.46);

\path[draw=drawColor,draw opacity=0.50,line width= 0.4pt,line join=round,line cap=round] (327.89,124.44) -- (327.89,128.47);
\definecolor{drawColor}{RGB}{102,166,30}

\path[draw=drawColor,draw opacity=0.50,line width= 0.4pt,line join=round,line cap=round] (276.55, 95.85) rectangle (279.40, 98.70);

\path[draw=drawColor,draw opacity=0.50,line width= 0.4pt,line join=round,line cap=round] (276.55, 95.85) -- (279.40, 98.70);

\path[draw=drawColor,draw opacity=0.50,line width= 0.4pt,line join=round,line cap=round] (276.55, 98.70) -- (279.40, 95.85);

\path[draw=drawColor,draw opacity=0.50,line width= 0.4pt,line join=round,line cap=round] (286.94,100.56) rectangle (289.80,103.41);

\path[draw=drawColor,draw opacity=0.50,line width= 0.4pt,line join=round,line cap=round] (286.94,100.56) -- (289.80,103.41);

\path[draw=drawColor,draw opacity=0.50,line width= 0.4pt,line join=round,line cap=round] (286.94,103.41) -- (289.80,100.56);

\path[draw=drawColor,draw opacity=0.50,line width= 0.4pt,line join=round,line cap=round] (310.79,116.09) rectangle (313.64,118.94);

\path[draw=drawColor,draw opacity=0.50,line width= 0.4pt,line join=round,line cap=round] (310.79,116.09) -- (313.64,118.94);

\path[draw=drawColor,draw opacity=0.50,line width= 0.4pt,line join=round,line cap=round] (310.79,118.94) -- (313.64,116.09);

\path[draw=drawColor,draw opacity=0.50,line width= 0.4pt,line join=round,line cap=round] (333.14,135.12) rectangle (336.00,137.98);

\path[draw=drawColor,draw opacity=0.50,line width= 0.4pt,line join=round,line cap=round] (333.14,135.12) -- (336.00,137.98);

\path[draw=drawColor,draw opacity=0.50,line width= 0.4pt,line join=round,line cap=round] (333.14,137.98) -- (336.00,135.12);

\path[draw=drawColor,draw opacity=0.50,line width= 0.4pt,line join=round,line cap=round] (349.28,148.33) rectangle (352.13,151.18);

\path[draw=drawColor,draw opacity=0.50,line width= 0.4pt,line join=round,line cap=round] (349.28,148.33) -- (352.13,151.18);

\path[draw=drawColor,draw opacity=0.50,line width= 0.4pt,line join=round,line cap=round] (349.28,151.18) -- (352.13,148.33);

\path[draw=drawColor,draw opacity=0.50,line width= 0.4pt,line join=round,line cap=round] (363.27,162.40) rectangle (366.12,165.25);

\path[draw=drawColor,draw opacity=0.50,line width= 0.4pt,line join=round,line cap=round] (363.27,162.40) -- (366.12,165.25);

\path[draw=drawColor,draw opacity=0.50,line width= 0.4pt,line join=round,line cap=round] (363.27,165.25) -- (366.12,162.40);

\path[draw=drawColor,draw opacity=0.50,line width= 0.4pt,line join=round,line cap=round] (374.26,172.10) rectangle (377.12,174.95);

\path[draw=drawColor,draw opacity=0.50,line width= 0.4pt,line join=round,line cap=round] (374.26,172.10) -- (377.12,174.95);

\path[draw=drawColor,draw opacity=0.50,line width= 0.4pt,line join=round,line cap=round] (374.26,174.95) -- (377.12,172.10);

\path[draw=drawColor,draw opacity=0.50,line width= 0.4pt,line join=round,line cap=round] (392.94,184.58) rectangle (395.79,187.43);

\path[draw=drawColor,draw opacity=0.50,line width= 0.4pt,line join=round,line cap=round] (392.94,184.58) -- (395.79,187.43);

\path[draw=drawColor,draw opacity=0.50,line width= 0.4pt,line join=round,line cap=round] (392.94,187.43) -- (395.79,184.58);

\path[draw=drawColor,draw opacity=0.50,line width= 0.4pt,line join=round,line cap=round] (392.94,194.70) rectangle (395.79,197.55);

\path[draw=drawColor,draw opacity=0.50,line width= 0.4pt,line join=round,line cap=round] (392.94,194.70) -- (395.79,197.55);

\path[draw=drawColor,draw opacity=0.50,line width= 0.4pt,line join=round,line cap=round] (392.94,197.55) -- (395.79,194.70);

\path[draw=drawColor,draw opacity=0.50,line width= 0.4pt,line join=round,line cap=round] (392.94,202.26) rectangle (395.79,205.11);

\path[draw=drawColor,draw opacity=0.50,line width= 0.4pt,line join=round,line cap=round] (392.94,202.26) -- (395.79,205.11);

\path[draw=drawColor,draw opacity=0.50,line width= 0.4pt,line join=round,line cap=round] (392.94,205.11) -- (395.79,202.26);

\path[draw=drawColor,draw opacity=0.50,line width= 0.4pt,line join=round,line cap=round] (392.94,203.31) rectangle (395.79,206.16);

\path[draw=drawColor,draw opacity=0.50,line width= 0.4pt,line join=round,line cap=round] (392.94,203.31) -- (395.79,206.16);

\path[draw=drawColor,draw opacity=0.50,line width= 0.4pt,line join=round,line cap=round] (392.94,206.16) -- (395.79,203.31);
\definecolor{fillColor}{RGB}{27,158,119}

\path[fill=fillColor,fill opacity=0.50] (394.37,170.98) circle (  1.43);

\path[fill=fillColor,fill opacity=0.50] (394.37,152.93) circle (  1.43);

\path[fill=fillColor,fill opacity=0.50] (394.37,155.45) circle (  1.43);

\path[fill=fillColor,fill opacity=0.50] (394.37,175.70) circle (  1.43);

\path[fill=fillColor,fill opacity=0.50] (394.37,178.18) circle (  1.43);

\path[fill=fillColor,fill opacity=0.50] (394.37,170.52) circle (  1.43);

\path[fill=fillColor,fill opacity=0.50] (394.37,174.79) circle (  1.43);

\path[fill=fillColor,fill opacity=0.50] (394.37,157.29) circle (  1.43);

\path[fill=fillColor,fill opacity=0.50] (394.37,156.58) circle (  1.43);

\path[fill=fillColor,fill opacity=0.50] (394.37,165.25) circle (  1.43);

\path[fill=fillColor,fill opacity=0.50] (394.37,171.09) circle (  1.43);

\path[fill=fillColor,fill opacity=0.50] (394.37,188.10) circle (  1.43);

\path[fill=fillColor,fill opacity=0.50] (394.37,172.46) circle (  1.43);

\path[fill=fillColor,fill opacity=0.50] (394.37,177.05) circle (  1.43);

\path[fill=fillColor,fill opacity=0.50] (394.37,172.08) circle (  1.43);

\path[fill=fillColor,fill opacity=0.50] (394.37,179.03) circle (  1.43);

\path[fill=fillColor,fill opacity=0.50] (394.37,212.44) circle (  1.43);

\path[fill=fillColor,fill opacity=0.50] (394.37,212.44) circle (  1.43);

\path[fill=fillColor,fill opacity=0.50] (394.37,212.44) circle (  1.43);

\path[fill=fillColor,fill opacity=0.50] (394.37,196.61) circle (  1.43);

\path[fill=fillColor,fill opacity=0.50] (394.37,167.05) circle (  1.43);

\path[fill=fillColor,fill opacity=0.50] (394.37,138.07) circle (  1.43);

\path[fill=fillColor,fill opacity=0.50] (341.43,156.84) circle (  1.43);

\path[fill=fillColor,fill opacity=0.50] (394.37,135.64) circle (  1.43);

\path[fill=fillColor,fill opacity=0.50] (382.07,157.83) circle (  1.43);

\path[fill=fillColor,fill opacity=0.50] (375.82,149.56) circle (  1.43);

\path[fill=fillColor,fill opacity=0.50] (394.37,172.08) circle (  1.43);

\path[fill=fillColor,fill opacity=0.50] (394.37,139.82) circle (  1.43);

\path[fill=fillColor,fill opacity=0.50] (355.54,150.41) circle (  1.43);

\path[fill=fillColor,fill opacity=0.50] (394.37,146.22) circle (  1.43);

\path[fill=fillColor,fill opacity=0.50] (283.11,102.74) circle (  1.43);

\path[fill=fillColor,fill opacity=0.50] (280.31, 98.38) circle (  1.43);

\path[fill=fillColor,fill opacity=0.50] (280.31, 98.38) circle (  1.43);

\path[fill=fillColor,fill opacity=0.50] (280.31, 98.38) circle (  1.43);

\path[fill=fillColor,fill opacity=0.50] (281.32, 98.38) circle (  1.43);

\path[fill=fillColor,fill opacity=0.50] (281.32, 99.39) circle (  1.43);

\path[fill=fillColor,fill opacity=0.50] (283.92,102.74) circle (  1.43);

\path[fill=fillColor,fill opacity=0.50] (281.32, 99.39) circle (  1.43);

\path[fill=fillColor,fill opacity=0.50] (280.31, 98.38) circle (  1.43);

\path[fill=fillColor,fill opacity=0.50] (280.31, 98.38) circle (  1.43);

\path[fill=fillColor,fill opacity=0.50] (283.92,100.32) circle (  1.43);

\path[fill=fillColor,fill opacity=0.50] (283.11,100.32) circle (  1.43);

\path[fill=fillColor,fill opacity=0.50] (283.92,101.19) circle (  1.43);

\path[fill=fillColor,fill opacity=0.50] (283.11, 98.38) circle (  1.43);

\path[fill=fillColor,fill opacity=0.50] (280.31, 99.39) circle (  1.43);

\path[fill=fillColor,fill opacity=0.50] (280.31, 98.38) circle (  1.43);

\path[fill=fillColor,fill opacity=0.50] (281.32,101.19) circle (  1.43);

\path[fill=fillColor,fill opacity=0.50] (281.32, 98.38) circle (  1.43);

\path[fill=fillColor,fill opacity=0.50] (281.32, 98.38) circle (  1.43);

\path[fill=fillColor,fill opacity=0.50] (283.92,101.19) circle (  1.43);

\path[fill=fillColor,fill opacity=0.50] (282.25, 98.38) circle (  1.43);

\path[fill=fillColor,fill opacity=0.50] (283.11,100.32) circle (  1.43);

\path[fill=fillColor,fill opacity=0.50] (281.32, 98.38) circle (  1.43);

\path[fill=fillColor,fill opacity=0.50] (280.31,101.19) circle (  1.43);

\path[fill=fillColor,fill opacity=0.50] (283.92,100.32) circle (  1.43);

\path[fill=fillColor,fill opacity=0.50] (281.32, 98.38) circle (  1.43);

\path[fill=fillColor,fill opacity=0.50] (283.92,101.99) circle (  1.43);

\path[fill=fillColor,fill opacity=0.50] (282.25, 99.39) circle (  1.43);

\path[fill=fillColor,fill opacity=0.50] (306.54,101.19) circle (  1.43);

\path[fill=fillColor,fill opacity=0.50] (281.32, 98.38) circle (  1.43);

\path[fill=fillColor,fill opacity=0.50] (280.31, 99.39) circle (  1.43);

\path[fill=fillColor,fill opacity=0.50] (280.31, 99.39) circle (  1.43);

\path[fill=fillColor,fill opacity=0.50] (282.25, 98.38) circle (  1.43);

\path[fill=fillColor,fill opacity=0.50] (280.31, 99.39) circle (  1.43);

\path[fill=fillColor,fill opacity=0.50] (283.11,101.19) circle (  1.43);

\path[fill=fillColor,fill opacity=0.50] (280.31,101.99) circle (  1.43);

\path[fill=fillColor,fill opacity=0.50] (289.38,103.44) circle (  1.43);

\path[fill=fillColor,fill opacity=0.50] (283.11,101.99) circle (  1.43);

\path[fill=fillColor,fill opacity=0.50] (280.31, 98.38) circle (  1.43);

\path[fill=fillColor,fill opacity=0.50] (280.31, 98.38) circle (  1.43);

\path[fill=fillColor,fill opacity=0.50] (294.41,101.19) circle (  1.43);

\path[fill=fillColor,fill opacity=0.50] (283.11, 98.38) circle (  1.43);

\path[fill=fillColor,fill opacity=0.50] (283.92,101.19) circle (  1.43);

\path[fill=fillColor,fill opacity=0.50] (281.32, 98.38) circle (  1.43);

\path[fill=fillColor,fill opacity=0.50] (287.83,102.74) circle (  1.43);

\path[fill=fillColor,fill opacity=0.50] (286.66,112.17) circle (  1.43);

\path[fill=fillColor,fill opacity=0.50] (282.25, 98.38) circle (  1.43);

\path[fill=fillColor,fill opacity=0.50] (283.11,101.19) circle (  1.43);

\path[fill=fillColor,fill opacity=0.50] (291.17,104.11) circle (  1.43);

\path[fill=fillColor,fill opacity=0.50] (283.92,100.32) circle (  1.43);

\path[fill=fillColor,fill opacity=0.50] (280.31, 97.27) circle (  1.43);

\path[fill=fillColor,fill opacity=0.50] (281.32, 98.38) circle (  1.43);

\path[fill=fillColor,fill opacity=0.50] (293.43,107.45) circle (  1.43);

\path[fill=fillColor,fill opacity=0.50] (281.32, 98.38) circle (  1.43);

\path[fill=fillColor,fill opacity=0.50] (280.31, 98.38) circle (  1.43);

\path[fill=fillColor,fill opacity=0.50] (284.67,101.19) circle (  1.43);

\path[fill=fillColor,fill opacity=0.50] (282.25,100.32) circle (  1.43);

\path[fill=fillColor,fill opacity=0.50] (281.32, 99.39) circle (  1.43);

\path[fill=fillColor,fill opacity=0.50] (284.67, 98.38) circle (  1.43);

\path[fill=fillColor,fill opacity=0.50] (280.31, 98.38) circle (  1.43);

\path[fill=fillColor,fill opacity=0.50] (283.11, 99.39) circle (  1.43);

\path[fill=fillColor,fill opacity=0.50] (281.32,108.38) circle (  1.43);

\path[fill=fillColor,fill opacity=0.50] (281.32, 99.39) circle (  1.43);

\path[fill=fillColor,fill opacity=0.50] (281.32, 98.38) circle (  1.43);

\path[fill=fillColor,fill opacity=0.50] (283.11,101.99) circle (  1.43);

\path[fill=fillColor,fill opacity=0.50] (281.32,108.82) circle (  1.43);

\path[fill=fillColor,fill opacity=0.50] (281.32, 97.27) circle (  1.43);

\path[fill=fillColor,fill opacity=0.50] (280.31, 98.38) circle (  1.43);

\path[fill=fillColor,fill opacity=0.50] (283.92, 99.39) circle (  1.43);

\path[fill=fillColor,fill opacity=0.50] (282.25,101.99) circle (  1.43);

\path[fill=fillColor,fill opacity=0.50] (281.32,111.50) circle (  1.43);

\path[fill=fillColor,fill opacity=0.50] (282.25, 99.39) circle (  1.43);

\path[fill=fillColor,fill opacity=0.50] (281.32, 98.38) circle (  1.43);

\path[fill=fillColor,fill opacity=0.50] (283.11,101.19) circle (  1.43);

\path[fill=fillColor,fill opacity=0.50] (280.31, 98.38) circle (  1.43);

\path[fill=fillColor,fill opacity=0.50] (282.25, 99.39) circle (  1.43);

\path[fill=fillColor,fill opacity=0.50] (303.95,111.16) circle (  1.43);

\path[fill=fillColor,fill opacity=0.50] (283.92,101.99) circle (  1.43);

\path[fill=fillColor,fill opacity=0.50] (283.11,101.19) circle (  1.43);

\path[fill=fillColor,fill opacity=0.50] (283.92,101.19) circle (  1.43);

\path[fill=fillColor,fill opacity=0.50] (283.92,101.99) circle (  1.43);

\path[fill=fillColor,fill opacity=0.50] (281.32, 98.38) circle (  1.43);

\path[fill=fillColor,fill opacity=0.50] (280.31, 99.39) circle (  1.43);

\path[fill=fillColor,fill opacity=0.50] (280.31, 99.39) circle (  1.43);

\path[fill=fillColor,fill opacity=0.50] (281.32, 98.38) circle (  1.43);

\path[fill=fillColor,fill opacity=0.50] (283.92,101.19) circle (  1.43);

\path[fill=fillColor,fill opacity=0.50] (280.31, 98.38) circle (  1.43);

\path[fill=fillColor,fill opacity=0.50] (280.31, 99.39) circle (  1.43);

\path[fill=fillColor,fill opacity=0.50] (289.86,104.74) circle (  1.43);

\path[fill=fillColor,fill opacity=0.50] (281.32, 98.38) circle (  1.43);

\path[fill=fillColor,fill opacity=0.50] (281.32, 98.38) circle (  1.43);

\path[fill=fillColor,fill opacity=0.50] (283.92, 99.39) circle (  1.43);

\path[fill=fillColor,fill opacity=0.50] (280.31, 98.38) circle (  1.43);

\path[fill=fillColor,fill opacity=0.50] (280.31, 98.38) circle (  1.43);

\path[fill=fillColor,fill opacity=0.50] (280.31, 98.38) circle (  1.43);

\path[fill=fillColor,fill opacity=0.50] (280.31, 98.38) circle (  1.43);

\path[fill=fillColor,fill opacity=0.50] (286.04,100.32) circle (  1.43);

\path[fill=fillColor,fill opacity=0.50] (284.67,105.33) circle (  1.43);

\path[fill=fillColor,fill opacity=0.50] (281.32, 98.38) circle (  1.43);

\path[fill=fillColor,fill opacity=0.50] (282.25,101.99) circle (  1.43);

\path[fill=fillColor,fill opacity=0.50] (292.36,101.19) circle (  1.43);

\path[fill=fillColor,fill opacity=0.50] (281.32,111.84) circle (  1.43);

\path[fill=fillColor,fill opacity=0.50] (283.92,101.19) circle (  1.43);

\path[fill=fillColor,fill opacity=0.50] (282.25,100.32) circle (  1.43);

\path[fill=fillColor,fill opacity=0.50] (282.25,100.32) circle (  1.43);

\path[fill=fillColor,fill opacity=0.50] (281.32, 98.38) circle (  1.43);

\path[fill=fillColor,fill opacity=0.50] (282.25,101.19) circle (  1.43);

\path[fill=fillColor,fill opacity=0.50] (280.31, 98.38) circle (  1.43);

\path[fill=fillColor,fill opacity=0.50] (283.92,101.99) circle (  1.43);

\path[fill=fillColor,fill opacity=0.50] (280.31,112.49) circle (  1.43);

\path[fill=fillColor,fill opacity=0.50] (280.31, 98.38) circle (  1.43);

\path[fill=fillColor,fill opacity=0.50] (281.32,100.32) circle (  1.43);

\path[fill=fillColor,fill opacity=0.50] (283.92,100.32) circle (  1.43);

\path[fill=fillColor,fill opacity=0.50] (281.32,101.19) circle (  1.43);

\path[fill=fillColor,fill opacity=0.50] (280.31, 98.38) circle (  1.43);

\path[fill=fillColor,fill opacity=0.50] (280.31, 97.27) circle (  1.43);

\path[fill=fillColor,fill opacity=0.50] (282.25,117.31) circle (  1.43);

\path[fill=fillColor,fill opacity=0.50] (281.32,101.19) circle (  1.43);

\path[fill=fillColor,fill opacity=0.50] (280.31, 99.39) circle (  1.43);

\path[fill=fillColor,fill opacity=0.50] (281.32, 99.39) circle (  1.43);

\path[fill=fillColor,fill opacity=0.50] (310.77,111.16) circle (  1.43);

\path[fill=fillColor,fill opacity=0.50] (287.26,107.45) circle (  1.43);

\path[fill=fillColor,fill opacity=0.50] (286.04,101.19) circle (  1.43);

\path[fill=fillColor,fill opacity=0.50] (284.67,101.99) circle (  1.43);

\path[fill=fillColor,fill opacity=0.50] (293.08,108.38) circle (  1.43);

\path[fill=fillColor,fill opacity=0.50] (284.67,101.19) circle (  1.43);

\path[fill=fillColor,fill opacity=0.50] (283.11,100.32) circle (  1.43);

\path[fill=fillColor,fill opacity=0.50] (285.37,101.99) circle (  1.43);

\path[fill=fillColor,fill opacity=0.50] (289.86,101.99) circle (  1.43);

\path[fill=fillColor,fill opacity=0.50] (284.67,100.32) circle (  1.43);

\path[fill=fillColor,fill opacity=0.50] (283.92,101.19) circle (  1.43);

\path[fill=fillColor,fill opacity=0.50] (285.37,101.99) circle (  1.43);

\path[fill=fillColor,fill opacity=0.50] (309.03,111.84) circle (  1.43);

\path[fill=fillColor,fill opacity=0.50] (290.75,103.44) circle (  1.43);

\path[fill=fillColor,fill opacity=0.50] (283.11, 99.39) circle (  1.43);

\path[fill=fillColor,fill opacity=0.50] (282.25, 99.39) circle (  1.43);

\path[fill=fillColor,fill opacity=0.50] (291.17,106.96) circle (  1.43);

\path[fill=fillColor,fill opacity=0.50] (281.32,101.99) circle (  1.43);

\path[fill=fillColor,fill opacity=0.50] (283.11,101.19) circle (  1.43);

\path[fill=fillColor,fill opacity=0.50] (282.25, 98.38) circle (  1.43);

\path[fill=fillColor,fill opacity=0.50] (284.67,100.32) circle (  1.43);

\path[fill=fillColor,fill opacity=0.50] (284.67,101.99) circle (  1.43);

\path[fill=fillColor,fill opacity=0.50] (283.92,101.19) circle (  1.43);

\path[fill=fillColor,fill opacity=0.50] (282.25, 99.39) circle (  1.43);

\path[fill=fillColor,fill opacity=0.50] (318.83,109.25) circle (  1.43);

\path[fill=fillColor,fill opacity=0.50] (280.31, 98.38) circle (  1.43);

\path[fill=fillColor,fill opacity=0.50] (281.32, 98.38) circle (  1.43);

\path[fill=fillColor,fill opacity=0.50] (283.92,101.99) circle (  1.43);

\path[fill=fillColor,fill opacity=0.50] (283.11,100.32) circle (  1.43);

\path[fill=fillColor,fill opacity=0.50] (285.37, 99.39) circle (  1.43);

\path[fill=fillColor,fill opacity=0.50] (281.32, 98.38) circle (  1.43);

\path[fill=fillColor,fill opacity=0.50] (283.11,101.99) circle (  1.43);

\path[fill=fillColor,fill opacity=0.50] (283.92,101.99) circle (  1.43);

\path[fill=fillColor,fill opacity=0.50] (282.25, 99.39) circle (  1.43);

\path[fill=fillColor,fill opacity=0.50] (280.31, 98.38) circle (  1.43);

\path[fill=fillColor,fill opacity=0.50] (281.32, 99.39) circle (  1.43);

\path[fill=fillColor,fill opacity=0.50] (288.89,101.19) circle (  1.43);

\path[fill=fillColor,fill opacity=0.50] (283.92,101.19) circle (  1.43);

\path[fill=fillColor,fill opacity=0.50] (281.32, 99.39) circle (  1.43);

\path[fill=fillColor,fill opacity=0.50] (281.32, 98.38) circle (  1.43);

\path[fill=fillColor,fill opacity=0.50] (287.26,115.75) circle (  1.43);

\path[fill=fillColor,fill opacity=0.50] (281.32, 98.38) circle (  1.43);

\path[fill=fillColor,fill opacity=0.50] (283.11,101.19) circle (  1.43);

\path[fill=fillColor,fill opacity=0.50] (280.31, 98.38) circle (  1.43);

\path[fill=fillColor,fill opacity=0.50] (299.64,102.74) circle (  1.43);

\path[fill=fillColor,fill opacity=0.50] (283.92, 99.39) circle (  1.43);

\path[fill=fillColor,fill opacity=0.50] (284.67,101.19) circle (  1.43);

\path[fill=fillColor,fill opacity=0.50] (283.92,100.32) circle (  1.43);

\path[fill=fillColor,fill opacity=0.50] (309.89,114.23) circle (  1.43);

\path[fill=fillColor,fill opacity=0.50] (283.92,101.99) circle (  1.43);

\path[fill=fillColor,fill opacity=0.50] (281.32, 99.39) circle (  1.43);

\path[fill=fillColor,fill opacity=0.50] (281.32, 99.39) circle (  1.43);

\path[fill=fillColor,fill opacity=0.50] (282.25,101.99) circle (  1.43);

\path[fill=fillColor,fill opacity=0.50] (282.25, 99.39) circle (  1.43);

\path[fill=fillColor,fill opacity=0.50] (281.32, 99.39) circle (  1.43);

\path[fill=fillColor,fill opacity=0.50] (283.11,101.19) circle (  1.43);

\path[fill=fillColor,fill opacity=0.50] (302.79,116.67) circle (  1.43);

\path[fill=fillColor,fill opacity=0.50] (286.66,101.99) circle (  1.43);

\path[fill=fillColor,fill opacity=0.50] (284.67,102.74) circle (  1.43);

\path[fill=fillColor,fill opacity=0.50] (291.17,101.99) circle (  1.43);

\path[fill=fillColor,fill opacity=0.50] (318.12,120.07) circle (  1.43);

\path[fill=fillColor,fill opacity=0.50] (319.06,131.08) circle (  1.43);

\path[fill=fillColor,fill opacity=0.50] (290.75,107.93) circle (  1.43);

\path[fill=fillColor,fill opacity=0.50] (286.04,101.99) circle (  1.43);

\path[fill=fillColor,fill opacity=0.50] (333.51,135.60) circle (  1.43);

\path[fill=fillColor,fill opacity=0.50] (304.22,115.51) circle (  1.43);

\path[fill=fillColor,fill opacity=0.50] (290.31,105.90) circle (  1.43);

\path[fill=fillColor,fill opacity=0.50] (283.11,102.74) circle (  1.43);

\path[fill=fillColor,fill opacity=0.50] (303.24,114.23) circle (  1.43);

\path[fill=fillColor,fill opacity=0.50] (307.19,118.30) circle (  1.43);

\path[fill=fillColor,fill opacity=0.50] (285.37,101.19) circle (  1.43);

\path[fill=fillColor,fill opacity=0.50] (280.31, 98.38) circle (  1.43);

\path[fill=fillColor,fill opacity=0.50] (296.69,103.44) circle (  1.43);

\path[fill=fillColor,fill opacity=0.50] (281.32, 98.38) circle (  1.43);

\path[fill=fillColor,fill opacity=0.50] (282.25, 99.39) circle (  1.43);

\path[fill=fillColor,fill opacity=0.50] (281.32, 98.38) circle (  1.43);

\path[fill=fillColor,fill opacity=0.50] (282.25,111.16) circle (  1.43);

\path[fill=fillColor,fill opacity=0.50] (284.67,101.19) circle (  1.43);

\path[fill=fillColor,fill opacity=0.50] (280.31,100.32) circle (  1.43);

\path[fill=fillColor,fill opacity=0.50] (281.32, 99.39) circle (  1.43);

\path[fill=fillColor,fill opacity=0.50] (283.92, 97.27) circle (  1.43);

\path[fill=fillColor,fill opacity=0.50] (283.92,101.99) circle (  1.43);

\path[fill=fillColor,fill opacity=0.50] (281.32, 98.38) circle (  1.43);

\path[fill=fillColor,fill opacity=0.50] (283.92,101.99) circle (  1.43);

\path[fill=fillColor,fill opacity=0.50] (283.92,104.74) circle (  1.43);

\path[fill=fillColor,fill opacity=0.50] (284.67,101.19) circle (  1.43);

\path[fill=fillColor,fill opacity=0.50] (285.37,101.19) circle (  1.43);

\path[fill=fillColor,fill opacity=0.50] (284.67,101.99) circle (  1.43);

\path[fill=fillColor,fill opacity=0.50] (284.67,107.45) circle (  1.43);

\path[fill=fillColor,fill opacity=0.50] (285.37, 99.39) circle (  1.43);

\path[fill=fillColor,fill opacity=0.50] (281.32, 98.38) circle (  1.43);

\path[fill=fillColor,fill opacity=0.50] (281.32, 98.38) circle (  1.43);

\path[fill=fillColor,fill opacity=0.50] (303.53,109.65) circle (  1.43);

\path[fill=fillColor,fill opacity=0.50] (283.11,100.32) circle (  1.43);

\path[fill=fillColor,fill opacity=0.50] (281.32, 98.38) circle (  1.43);

\path[fill=fillColor,fill opacity=0.50] (281.32, 98.38) circle (  1.43);

\path[fill=fillColor,fill opacity=0.50] (282.25, 99.39) circle (  1.43);

\path[fill=fillColor,fill opacity=0.50] (282.25,101.19) circle (  1.43);

\path[fill=fillColor,fill opacity=0.50] (284.67,101.99) circle (  1.43);

\path[fill=fillColor,fill opacity=0.50] (285.37,101.99) circle (  1.43);

\path[fill=fillColor,fill opacity=0.50] (287.26,121.74) circle (  1.43);

\path[fill=fillColor,fill opacity=0.50] (283.92,101.19) circle (  1.43);

\path[fill=fillColor,fill opacity=0.50] (285.37,101.19) circle (  1.43);

\path[fill=fillColor,fill opacity=0.50] (285.37,101.99) circle (  1.43);

\path[fill=fillColor,fill opacity=0.50] (282.25, 99.39) circle (  1.43);

\path[fill=fillColor,fill opacity=0.50] (284.67,101.99) circle (  1.43);

\path[fill=fillColor,fill opacity=0.50] (285.37, 99.39) circle (  1.43);

\path[fill=fillColor,fill opacity=0.50] (282.25, 98.38) circle (  1.43);

\path[fill=fillColor,fill opacity=0.50] (283.11, 99.39) circle (  1.43);

\path[fill=fillColor,fill opacity=0.50] (281.32, 98.38) circle (  1.43);

\path[fill=fillColor,fill opacity=0.50] (285.37,103.44) circle (  1.43);

\path[fill=fillColor,fill opacity=0.50] (282.25, 99.39) circle (  1.43);

\path[fill=fillColor,fill opacity=0.50] (306.76,108.82) circle (  1.43);

\path[fill=fillColor,fill opacity=0.50] (283.11, 99.39) circle (  1.43);

\path[fill=fillColor,fill opacity=0.50] (284.67,101.99) circle (  1.43);

\path[fill=fillColor,fill opacity=0.50] (280.31, 98.38) circle (  1.43);

\path[fill=fillColor,fill opacity=0.50] (283.11, 99.39) circle (  1.43);

\path[fill=fillColor,fill opacity=0.50] (281.32,101.99) circle (  1.43);

\path[fill=fillColor,fill opacity=0.50] (282.25, 99.39) circle (  1.43);

\path[fill=fillColor,fill opacity=0.50] (283.92,101.19) circle (  1.43);

\path[fill=fillColor,fill opacity=0.50] (283.92, 98.38) circle (  1.43);

\path[fill=fillColor,fill opacity=0.50] (282.25,101.99) circle (  1.43);

\path[fill=fillColor,fill opacity=0.50] (284.67,101.99) circle (  1.43);

\path[fill=fillColor,fill opacity=0.50] (281.32, 98.38) circle (  1.43);

\path[fill=fillColor,fill opacity=0.50] (310.61,105.33) circle (  1.43);

\path[fill=fillColor,fill opacity=0.50] (286.04,101.19) circle (  1.43);

\path[fill=fillColor,fill opacity=0.50] (284.67,101.19) circle (  1.43);

\path[fill=fillColor,fill opacity=0.50] (284.67,101.99) circle (  1.43);

\path[fill=fillColor,fill opacity=0.50] (283.11,120.70) circle (  1.43);

\path[fill=fillColor,fill opacity=0.50] (281.32,102.74) circle (  1.43);

\path[fill=fillColor,fill opacity=0.50] (284.67,101.19) circle (  1.43);

\path[fill=fillColor,fill opacity=0.50] (283.92,101.19) circle (  1.43);

\path[fill=fillColor,fill opacity=0.50] (286.04,101.19) circle (  1.43);

\path[fill=fillColor,fill opacity=0.50] (283.11, 99.39) circle (  1.43);

\path[fill=fillColor,fill opacity=0.50] (285.37,101.99) circle (  1.43);

\path[fill=fillColor,fill opacity=0.50] (283.92,100.32) circle (  1.43);

\path[fill=fillColor,fill opacity=0.50] (309.97,118.68) circle (  1.43);

\path[fill=fillColor,fill opacity=0.50] (306.65,105.33) circle (  1.43);

\path[fill=fillColor,fill opacity=0.50] (288.37,104.11) circle (  1.43);

\path[fill=fillColor,fill opacity=0.50] (281.32, 99.39) circle (  1.43);

\path[fill=fillColor,fill opacity=0.50] (305.62,101.99) circle (  1.43);

\path[fill=fillColor,fill opacity=0.50] (284.67,101.19) circle (  1.43);

\path[fill=fillColor,fill opacity=0.50] (282.25, 98.38) circle (  1.43);

\path[fill=fillColor,fill opacity=0.50] (282.25, 98.38) circle (  1.43);

\path[fill=fillColor,fill opacity=0.50] (299.44,105.33) circle (  1.43);

\path[fill=fillColor,fill opacity=0.50] (284.67, 99.39) circle (  1.43);

\path[fill=fillColor,fill opacity=0.50] (285.37, 99.39) circle (  1.43);

\path[fill=fillColor,fill opacity=0.50] (282.25, 98.38) circle (  1.43);

\path[fill=fillColor,fill opacity=0.50] (295.03,111.50) circle (  1.43);

\path[fill=fillColor,fill opacity=0.50] (286.66,103.44) circle (  1.43);

\path[fill=fillColor,fill opacity=0.50] (282.25, 99.39) circle (  1.43);

\path[fill=fillColor,fill opacity=0.50] (286.66,101.19) circle (  1.43);

\path[fill=fillColor,fill opacity=0.50] (288.89,101.19) circle (  1.43);

\path[fill=fillColor,fill opacity=0.50] (282.25, 99.39) circle (  1.43);

\path[fill=fillColor,fill opacity=0.50] (281.32, 99.39) circle (  1.43);

\path[fill=fillColor,fill opacity=0.50] (283.92,101.19) circle (  1.43);

\path[fill=fillColor,fill opacity=0.50] (285.37,101.99) circle (  1.43);

\path[fill=fillColor,fill opacity=0.50] (285.37,102.74) circle (  1.43);

\path[fill=fillColor,fill opacity=0.50] (282.25, 98.38) circle (  1.43);

\path[fill=fillColor,fill opacity=0.50] (284.67,101.19) circle (  1.43);

\path[fill=fillColor,fill opacity=0.50] (288.37,102.74) circle (  1.43);

\path[fill=fillColor,fill opacity=0.50] (283.11,102.74) circle (  1.43);

\path[fill=fillColor,fill opacity=0.50] (283.11,101.19) circle (  1.43);

\path[fill=fillColor,fill opacity=0.50] (286.04,102.74) circle (  1.43);

\path[fill=fillColor,fill opacity=0.50] (300.60,114.23) circle (  1.43);

\path[fill=fillColor,fill opacity=0.50] (333.05,133.68) circle (  1.43);

\path[fill=fillColor,fill opacity=0.50] (313.50,113.39) circle (  1.43);

\path[fill=fillColor,fill opacity=0.50] (284.67, 99.39) circle (  1.43);

\path[fill=fillColor,fill opacity=0.50] (300.79,114.76) circle (  1.43);

\path[fill=fillColor,fill opacity=0.50] (298.59,111.16) circle (  1.43);

\path[fill=fillColor,fill opacity=0.50] (290.31,106.96) circle (  1.43);

\path[fill=fillColor,fill opacity=0.50] (281.32,100.32) circle (  1.43);

\path[fill=fillColor,fill opacity=0.50] (339.26,142.31) circle (  1.43);

\path[fill=fillColor,fill opacity=0.50] (312.94,113.10) circle (  1.43);

\path[fill=fillColor,fill opacity=0.50] (308.19,113.10) circle (  1.43);

\path[fill=fillColor,fill opacity=0.50] (283.11,101.19) circle (  1.43);

\path[fill=fillColor,fill opacity=0.50] (323.77,127.19) circle (  1.43);

\path[fill=fillColor,fill opacity=0.50] (312.62,113.68) circle (  1.43);

\path[fill=fillColor,fill opacity=0.50] (283.92,101.99) circle (  1.43);

\path[fill=fillColor,fill opacity=0.50] (285.37,101.99) circle (  1.43);

\path[fill=fillColor,fill opacity=0.50] (319.54,126.83) circle (  1.43);

\path[fill=fillColor,fill opacity=0.50] (305.01,129.80) circle (  1.43);

\path[fill=fillColor,fill opacity=0.50] (288.89,117.31) circle (  1.43);

\path[fill=fillColor,fill opacity=0.50] (282.25,102.74) circle (  1.43);

\path[fill=fillColor,fill opacity=0.50] (318.16, 99.39) circle (  1.43);

\path[fill=fillColor,fill opacity=0.50] (289.38,100.32) circle (  1.43);

\path[fill=fillColor,fill opacity=0.50] (281.32,101.99) circle (  1.43);

\path[fill=fillColor,fill opacity=0.50] (282.25, 99.39) circle (  1.43);

\path[fill=fillColor,fill opacity=0.50] (316.68,132.05) circle (  1.43);

\path[fill=fillColor,fill opacity=0.50] (308.10,116.67) circle (  1.43);

\path[fill=fillColor,fill opacity=0.50] (290.31,103.44) circle (  1.43);

\path[fill=fillColor,fill opacity=0.50] (284.67,101.99) circle (  1.43);

\path[fill=fillColor,fill opacity=0.50] (317.91,128.61) circle (  1.43);

\path[fill=fillColor,fill opacity=0.50] (311.87,125.05) circle (  1.43);

\path[fill=fillColor,fill opacity=0.50] (288.89,103.44) circle (  1.43);

\path[fill=fillColor,fill opacity=0.50] (287.83,101.99) circle (  1.43);

\path[fill=fillColor,fill opacity=0.50] (282.25,101.19) circle (  1.43);

\path[fill=fillColor,fill opacity=0.50] (281.32, 99.39) circle (  1.43);

\path[fill=fillColor,fill opacity=0.50] (283.11,101.99) circle (  1.43);

\path[fill=fillColor,fill opacity=0.50] (282.25, 98.38) circle (  1.43);

\path[fill=fillColor,fill opacity=0.50] (286.66,102.74) circle (  1.43);

\path[fill=fillColor,fill opacity=0.50] (282.25, 98.38) circle (  1.43);

\path[fill=fillColor,fill opacity=0.50] (283.92,101.99) circle (  1.43);

\path[fill=fillColor,fill opacity=0.50] (284.67,101.19) circle (  1.43);

\path[fill=fillColor,fill opacity=0.50] (293.43,100.32) circle (  1.43);

\path[fill=fillColor,fill opacity=0.50] (282.25, 98.38) circle (  1.43);

\path[fill=fillColor,fill opacity=0.50] (284.67, 98.38) circle (  1.43);

\path[fill=fillColor,fill opacity=0.50] (281.32, 99.39) circle (  1.43);

\path[fill=fillColor,fill opacity=0.50] (293.43,101.99) circle (  1.43);

\path[fill=fillColor,fill opacity=0.50] (284.67,100.32) circle (  1.43);

\path[fill=fillColor,fill opacity=0.50] (281.32, 98.38) circle (  1.43);

\path[fill=fillColor,fill opacity=0.50] (284.67,101.19) circle (  1.43);

\path[fill=fillColor,fill opacity=0.50] (283.11, 99.39) circle (  1.43);

\path[fill=fillColor,fill opacity=0.50] (283.11, 99.39) circle (  1.43);

\path[fill=fillColor,fill opacity=0.50] (281.32,100.32) circle (  1.43);

\path[fill=fillColor,fill opacity=0.50] (283.11,101.19) circle (  1.43);

\path[fill=fillColor,fill opacity=0.50] (300.23,105.90) circle (  1.43);

\path[fill=fillColor,fill opacity=0.50] (283.11,117.51) circle (  1.43);

\path[fill=fillColor,fill opacity=0.50] (284.67,101.99) circle (  1.43);

\path[fill=fillColor,fill opacity=0.50] (282.25,107.45) circle (  1.43);

\path[fill=fillColor,fill opacity=0.50] (286.04,100.32) circle (  1.43);

\path[fill=fillColor,fill opacity=0.50] (284.67, 98.38) circle (  1.43);

\path[fill=fillColor,fill opacity=0.50] (281.32, 98.38) circle (  1.43);

\path[fill=fillColor,fill opacity=0.50] (284.67,101.99) circle (  1.43);

\path[fill=fillColor,fill opacity=0.50] (314.04,106.44) circle (  1.43);

\path[fill=fillColor,fill opacity=0.50] (283.11, 99.39) circle (  1.43);

\path[fill=fillColor,fill opacity=0.50] (282.25, 99.39) circle (  1.43);

\path[fill=fillColor,fill opacity=0.50] (281.32, 99.39) circle (  1.43);

\path[fill=fillColor,fill opacity=0.50] (291.98,103.44) circle (  1.43);

\path[fill=fillColor,fill opacity=0.50] (282.25, 99.39) circle (  1.43);

\path[fill=fillColor,fill opacity=0.50] (284.67,101.19) circle (  1.43);

\path[fill=fillColor,fill opacity=0.50] (281.32, 98.38) circle (  1.43);

\path[fill=fillColor,fill opacity=0.50] (286.04,100.32) circle (  1.43);

\path[fill=fillColor,fill opacity=0.50] (284.67,101.99) circle (  1.43);

\path[fill=fillColor,fill opacity=0.50] (281.32, 98.38) circle (  1.43);

\path[fill=fillColor,fill opacity=0.50] (281.32, 98.38) circle (  1.43);

\path[fill=fillColor,fill opacity=0.50] (281.32,102.74) circle (  1.43);

\path[fill=fillColor,fill opacity=0.50] (281.32,101.19) circle (  1.43);

\path[fill=fillColor,fill opacity=0.50] (282.25,101.19) circle (  1.43);

\path[fill=fillColor,fill opacity=0.50] (281.32,100.32) circle (  1.43);
\definecolor{fillColor}{RGB}{217,95,2}

\path[fill=fillColor,fill opacity=0.50] (283.92,100.60) --
	(285.84, 97.27) --
	(281.99, 97.27) --
	cycle;

\path[fill=fillColor,fill opacity=0.50] (286.66,104.21) --
	(288.59,100.88) --
	(284.74,100.88) --
	cycle;

\path[fill=fillColor,fill opacity=0.50] (284.67,100.60) --
	(286.59, 97.27) --
	(282.74, 97.27) --
	cycle;

\path[fill=fillColor,fill opacity=0.50] (287.83,102.54) --
	(289.75, 99.21) --
	(285.91, 99.21) --
	cycle;

\path[fill=fillColor,fill opacity=0.50] (286.04,102.54) --
	(287.96, 99.21) --
	(284.11, 99.21) --
	cycle;

\path[fill=fillColor,fill opacity=0.50] (282.25, 99.49) --
	(284.17, 96.16) --
	(280.33, 96.16) --
	cycle;

\path[fill=fillColor,fill opacity=0.50] (283.11, 99.49) --
	(285.03, 96.16) --
	(281.19, 96.16) --
	cycle;

\path[fill=fillColor,fill opacity=0.50] (286.66,103.40) --
	(288.59,100.08) --
	(284.74,100.08) --
	cycle;

\path[fill=fillColor,fill opacity=0.50] (283.92,101.61) --
	(285.84, 98.28) --
	(281.99, 98.28) --
	cycle;

\path[fill=fillColor,fill opacity=0.50] (287.26,104.21) --
	(289.18,100.88) --
	(285.34,100.88) --
	cycle;

\path[fill=fillColor,fill opacity=0.50] (289.38,105.66) --
	(291.30,102.33) --
	(287.46,102.33) --
	cycle;

\path[fill=fillColor,fill opacity=0.50] (287.26,101.61) --
	(289.18, 98.28) --
	(285.34, 98.28) --
	cycle;

\path[fill=fillColor,fill opacity=0.50] (289.86,104.96) --
	(291.78,101.63) --
	(287.93,101.63) --
	cycle;

\path[fill=fillColor,fill opacity=0.50] (293.77,105.66) --
	(295.69,102.33) --
	(291.85,102.33) --
	cycle;

\path[fill=fillColor,fill opacity=0.50] (291.58,106.96) --
	(293.50,103.63) --
	(289.66,103.63) --
	cycle;

\path[fill=fillColor,fill opacity=0.50] (288.37,102.54) --
	(290.29, 99.21) --
	(286.45, 99.21) --
	cycle;

\path[fill=fillColor,fill opacity=0.50] (287.26,103.40) --
	(289.18,100.08) --
	(285.34,100.08) --
	cycle;

\path[fill=fillColor,fill opacity=0.50] (290.31,106.96) --
	(292.23,103.63) --
	(288.39,103.63) --
	cycle;

\path[fill=fillColor,fill opacity=0.50] (302.16,117.97) --
	(304.08,114.64) --
	(300.24,114.64) --
	cycle;

\path[fill=fillColor,fill opacity=0.50] (288.89,104.96) --
	(290.81,101.63) --
	(286.96,101.63) --
	cycle;

\path[fill=fillColor,fill opacity=0.50] (296.43,107.55) --
	(298.35,104.22) --
	(294.51,104.22) --
	cycle;

\path[fill=fillColor,fill opacity=0.50] (308.94,113.72) --
	(310.86,110.39) --
	(307.02,110.39) --
	cycle;

\path[fill=fillColor,fill opacity=0.50] (303.53,119.93) --
	(305.45,116.61) --
	(301.60,116.61) --
	cycle;

\path[fill=fillColor,fill opacity=0.50] (308.66,127.69) --
	(310.59,124.36) --
	(306.74,124.36) --
	cycle;

\path[fill=fillColor,fill opacity=0.50] (291.17,104.21) --
	(293.09,100.88) --
	(289.25,100.88) --
	cycle;

\path[fill=fillColor,fill opacity=0.50] (300.04,112.27) --
	(301.96,108.94) --
	(298.12,108.94) --
	cycle;

\path[fill=fillColor,fill opacity=0.50] (306.54,111.04) --
	(308.47,107.71) --
	(304.62,107.71) --
	cycle;

\path[fill=fillColor,fill opacity=0.50] (305.50,113.72) --
	(307.42,110.39) --
	(303.58,110.39) --
	cycle;

\path[fill=fillColor,fill opacity=0.50] (295.61,106.96) --
	(297.53,103.63) --
	(293.69,103.63) --
	cycle;

\path[fill=fillColor,fill opacity=0.50] (298.59,111.87) --
	(300.52,108.54) --
	(296.67,108.54) --
	cycle;

\path[fill=fillColor,fill opacity=0.50] (300.04,117.24) --
	(301.96,113.91) --
	(298.12,113.91) --
	cycle;

\path[fill=fillColor,fill opacity=0.50] (298.81,117.24) --
	(300.73,113.91) --
	(296.89,113.91) --
	cycle;

\path[fill=fillColor,fill opacity=0.50] (314.27,115.32) --
	(316.19,111.99) --
	(312.35,111.99) --
	cycle;

\path[fill=fillColor,fill opacity=0.50] (306.32,126.95) --
	(308.24,123.62) --
	(304.40,123.62) --
	cycle;

\path[fill=fillColor,fill opacity=0.50] (293.08,105.66) --
	(295.01,102.33) --
	(291.16,102.33) --
	cycle;

\path[fill=fillColor,fill opacity=0.50] (307.29,115.90) --
	(309.22,112.57) --
	(305.37,112.57) --
	cycle;

\path[fill=fillColor,fill opacity=0.50] (301.49,117.97) --
	(303.41,114.64) --
	(299.57,114.64) --
	cycle;

\path[fill=fillColor,fill opacity=0.50] (302.48,132.51) --
	(304.40,129.18) --
	(300.55,129.18) --
	cycle;

\path[fill=fillColor,fill opacity=0.50] (308.10,135.85) --
	(310.02,132.52) --
	(306.18,132.52) --
	cycle;

\path[fill=fillColor,fill opacity=0.50] (306.43,119.93) --
	(308.35,116.61) --
	(304.51,116.61) --
	cycle;

\path[fill=fillColor,fill opacity=0.50] (308.19,120.71) --
	(310.11,117.38) --
	(306.27,117.38) --
	cycle;

\path[fill=fillColor,fill opacity=0.50] (309.47,139.08) --
	(311.39,135.75) --
	(307.55,135.75) --
	cycle;

\path[fill=fillColor,fill opacity=0.50] (324.12,139.16) --
	(326.04,135.83) --
	(322.20,135.83) --
	cycle;

\path[fill=fillColor,fill opacity=0.50] (308.85,129.14) --
	(310.77,125.81) --
	(306.93,125.81) --
	cycle;

\path[fill=fillColor,fill opacity=0.50] (324.46,156.48) --
	(326.38,153.15) --
	(322.54,153.15) --
	cycle;

\path[fill=fillColor,fill opacity=0.50] (318.67,139.24) --
	(320.60,135.91) --
	(316.75,135.91) --
	cycle;

\path[fill=fillColor,fill opacity=0.50] (317.09,125.79) --
	(319.01,122.47) --
	(315.17,122.47) --
	cycle;

\path[fill=fillColor,fill opacity=0.50] (301.32,114.06) --
	(303.24,110.73) --
	(299.40,110.73) --
	cycle;

\path[fill=fillColor,fill opacity=0.50] (304.36,127.16) --
	(306.28,123.83) --
	(302.44,123.83) --
	cycle;

\path[fill=fillColor,fill opacity=0.50] (312.42,128.68) --
	(314.34,125.35) --
	(310.50,125.35) --
	cycle;

\path[fill=fillColor,fill opacity=0.50] (310.85,141.70) --
	(312.77,138.37) --
	(308.92,138.37) --
	cycle;

\path[fill=fillColor,fill opacity=0.50] (315.81,138.28) --
	(317.73,134.95) --
	(313.89,134.95) --
	cycle;

\path[fill=fillColor,fill opacity=0.50] (310.85,126.38) --
	(312.77,123.06) --
	(308.92,123.06) --
	cycle;

\path[fill=fillColor,fill opacity=0.50] (321.80,146.95) --
	(323.72,143.62) --
	(319.88,143.62) --
	cycle;

\path[fill=fillColor,fill opacity=0.50] (319.06,119.73) --
	(320.98,116.40) --
	(317.14,116.40) --
	cycle;

\path[fill=fillColor,fill opacity=0.50] (312.28,139.68) --
	(314.20,136.36) --
	(310.36,136.36) --
	cycle;

\path[fill=fillColor,fill opacity=0.50] (316.49,131.44) --
	(318.41,128.11) --
	(314.57,128.11) --
	cycle;

\path[fill=fillColor,fill opacity=0.50] (321.29,141.26) --
	(323.21,137.93) --
	(319.36,137.93) --
	cycle;

\path[fill=fillColor,fill opacity=0.50] (320.21,141.54) --
	(322.13,138.22) --
	(318.29,138.22) --
	cycle;

\path[fill=fillColor,fill opacity=0.50] (339.45,144.04) --
	(341.37,140.71) --
	(337.53,140.71) --
	cycle;

\path[fill=fillColor,fill opacity=0.50] (322.67,164.83) --
	(324.59,161.50) --
	(320.74,161.50) --
	cycle;

\path[fill=fillColor,fill opacity=0.50] (325.19,144.06) --
	(327.11,140.74) --
	(323.27,140.74) --
	cycle;

\path[fill=fillColor,fill opacity=0.50] (320.28,137.60) --
	(322.20,134.27) --
	(318.35,134.27) --
	cycle;

\path[fill=fillColor,fill opacity=0.50] (326.13,135.12) --
	(328.06,131.79) --
	(324.21,131.79) --
	cycle;

\path[fill=fillColor,fill opacity=0.50] (332.87,144.65) --
	(334.79,141.33) --
	(330.95,141.33) --
	cycle;

\path[fill=fillColor,fill opacity=0.50] (340.28,143.01) --
	(342.21,139.68) --
	(338.36,139.68) --
	cycle;

\path[fill=fillColor,fill opacity=0.50] (316.44,140.63) --
	(318.37,137.31) --
	(314.52,137.31) --
	cycle;

\path[fill=fillColor,fill opacity=0.50] (325.86,141.76) --
	(327.79,138.43) --
	(323.94,138.43) --
	cycle;

\path[fill=fillColor,fill opacity=0.50] (312.55,134.33) --
	(314.47,131.00) --
	(310.63,131.00) --
	cycle;

\path[fill=fillColor,fill opacity=0.50] (315.76,147.28) --
	(317.68,143.95) --
	(313.84,143.95) --
	cycle;

\path[fill=fillColor,fill opacity=0.50] (325.10,140.87) --
	(327.03,137.54) --
	(323.18,137.54) --
	cycle;

\path[fill=fillColor,fill opacity=0.50] (328.64,148.16) --
	(330.56,144.83) --
	(326.72,144.83) --
	cycle;

\path[fill=fillColor,fill opacity=0.50] (325.99,151.83) --
	(327.91,148.50) --
	(324.07,148.50) --
	cycle;

\path[fill=fillColor,fill opacity=0.50] (326.16,141.07) --
	(328.08,137.74) --
	(324.23,137.74) --
	cycle;

\path[fill=fillColor,fill opacity=0.50] (335.43,163.37) --
	(337.35,160.05) --
	(333.51,160.05) --
	cycle;

\path[fill=fillColor,fill opacity=0.50] (357.99,154.11) --
	(359.91,150.78) --
	(356.07,150.78) --
	cycle;

\path[fill=fillColor,fill opacity=0.50] (321.98,148.42) --
	(323.90,145.09) --
	(320.06,145.09) --
	cycle;

\path[fill=fillColor,fill opacity=0.50] (335.51,156.27) --
	(337.43,152.95) --
	(333.58,152.95) --
	cycle;

\path[fill=fillColor,fill opacity=0.50] (323.39,136.50) --
	(325.31,133.17) --
	(321.47,133.17) --
	cycle;

\path[fill=fillColor,fill opacity=0.50] (335.37,148.90) --
	(337.29,145.57) --
	(333.45,145.57) --
	cycle;

\path[fill=fillColor,fill opacity=0.50] (338.22,161.73) --
	(340.14,158.40) --
	(336.30,158.40) --
	cycle;

\path[fill=fillColor,fill opacity=0.50] (338.46,157.24) --
	(340.39,153.92) --
	(336.54,153.92) --
	cycle;

\path[fill=fillColor,fill opacity=0.50] (363.65,164.34) --
	(365.57,161.01) --
	(361.73,161.01) --
	cycle;

\path[fill=fillColor,fill opacity=0.50] (342.74,152.50) --
	(344.66,149.17) --
	(340.82,149.17) --
	cycle;

\path[fill=fillColor,fill opacity=0.50] (338.74,170.80) --
	(340.66,167.47) --
	(336.82,167.47) --
	cycle;

\path[fill=fillColor,fill opacity=0.50] (350.18,179.36) --
	(352.10,176.04) --
	(348.26,176.04) --
	cycle;

\path[fill=fillColor,fill opacity=0.50] (348.01,152.63) --
	(349.93,149.30) --
	(346.08,149.30) --
	cycle;

\path[fill=fillColor,fill opacity=0.50] (338.39,163.82) --
	(340.31,160.50) --
	(336.47,160.50) --
	cycle;

\path[fill=fillColor,fill opacity=0.50] (346.16,154.08) --
	(348.08,150.75) --
	(344.23,150.75) --
	cycle;

\path[fill=fillColor,fill opacity=0.50] (365.72,143.71) --
	(367.64,140.38) --
	(363.80,140.38) --
	cycle;

\path[fill=fillColor,fill opacity=0.50] (283.92,101.61) --
	(285.84, 98.28) --
	(281.99, 98.28) --
	cycle;

\path[fill=fillColor,fill opacity=0.50] (282.25,100.60) --
	(284.17, 97.27) --
	(280.33, 97.27) --
	cycle;

\path[fill=fillColor,fill opacity=0.50] (285.37,104.21) --
	(287.29,100.88) --
	(283.45,100.88) --
	cycle;

\path[fill=fillColor,fill opacity=0.50] (283.11,100.60) --
	(285.03, 97.27) --
	(281.19, 97.27) --
	cycle;

\path[fill=fillColor,fill opacity=0.50] (284.67,106.96) --
	(286.59,103.63) --
	(282.74,103.63) --
	cycle;

\path[fill=fillColor,fill opacity=0.50] (282.25,100.60) --
	(284.17, 97.27) --
	(280.33, 97.27) --
	cycle;

\path[fill=fillColor,fill opacity=0.50] (285.37,104.21) --
	(287.29,100.88) --
	(283.45,100.88) --
	cycle;

\path[fill=fillColor,fill opacity=0.50] (281.32,103.40) --
	(283.24,100.08) --
	(279.40,100.08) --
	cycle;

\path[fill=fillColor,fill opacity=0.50] (281.32,101.61) --
	(283.24, 98.28) --
	(279.40, 98.28) --
	cycle;

\path[fill=fillColor,fill opacity=0.50] (284.67,101.61) --
	(286.59, 98.28) --
	(282.74, 98.28) --
	cycle;

\path[fill=fillColor,fill opacity=0.50] (286.66,104.96) --
	(288.59,101.63) --
	(284.74,101.63) --
	cycle;

\path[fill=fillColor,fill opacity=0.50] (283.11,102.54) --
	(285.03, 99.21) --
	(281.19, 99.21) --
	cycle;

\path[fill=fillColor,fill opacity=0.50] (287.83,104.96) --
	(289.75,101.63) --
	(285.91,101.63) --
	cycle;

\path[fill=fillColor,fill opacity=0.50] (288.89,107.55) --
	(290.81,104.22) --
	(286.96,104.22) --
	cycle;

\path[fill=fillColor,fill opacity=0.50] (287.26,104.96) --
	(289.18,101.63) --
	(285.34,101.63) --
	cycle;

\path[fill=fillColor,fill opacity=0.50] (284.67,101.61) --
	(286.59, 98.28) --
	(282.74, 98.28) --
	cycle;

\path[fill=fillColor,fill opacity=0.50] (287.83,104.21) --
	(289.75,100.88) --
	(285.91,100.88) --
	cycle;

\path[fill=fillColor,fill opacity=0.50] (284.67,102.54) --
	(286.59, 99.21) --
	(282.74, 99.21) --
	cycle;

\path[fill=fillColor,fill opacity=0.50] (287.83,104.21) --
	(289.75,100.88) --
	(285.91,100.88) --
	cycle;

\path[fill=fillColor,fill opacity=0.50] (286.66,104.96) --
	(288.59,101.63) --
	(284.74,101.63) --
	cycle;

\path[fill=fillColor,fill opacity=0.50] (286.04,103.40) --
	(287.96,100.08) --
	(284.11,100.08) --
	cycle;

\path[fill=fillColor,fill opacity=0.50] (288.37,104.96) --
	(290.29,101.63) --
	(286.45,101.63) --
	cycle;

\path[fill=fillColor,fill opacity=0.50] (287.83,103.40) --
	(289.75,100.08) --
	(285.91,100.08) --
	cycle;

\path[fill=fillColor,fill opacity=0.50] (290.75,106.33) --
	(292.67,103.00) --
	(288.83,103.00) --
	cycle;

\path[fill=fillColor,fill opacity=0.50] (291.17,106.96) --
	(293.09,103.63) --
	(289.25,103.63) --
	cycle;

\path[fill=fillColor,fill opacity=0.50] (290.31,107.55) --
	(292.23,104.22) --
	(288.39,104.22) --
	cycle;

\path[fill=fillColor,fill opacity=0.50] (290.31,110.15) --
	(292.23,106.82) --
	(288.39,106.82) --
	cycle;

\path[fill=fillColor,fill opacity=0.50] (289.86,117.97) --
	(291.78,114.64) --
	(287.93,114.64) --
	cycle;

\path[fill=fillColor,fill opacity=0.50] (292.73,108.66) --
	(294.65,105.33) --
	(290.80,105.33) --
	cycle;

\path[fill=fillColor,fill opacity=0.50] (292.36,112.65) --
	(294.28,109.32) --
	(290.44,109.32) --
	cycle;

\path[fill=fillColor,fill opacity=0.50] (289.86,108.12) --
	(291.78,104.79) --
	(287.93,104.79) --
	cycle;

\path[fill=fillColor,fill opacity=0.50] (294.10,104.96) --
	(296.02,101.63) --
	(292.17,101.63) --
	cycle;

\path[fill=fillColor,fill opacity=0.50] (291.58,105.66) --
	(293.50,102.33) --
	(289.66,102.33) --
	cycle;

\path[fill=fillColor,fill opacity=0.50] (288.89,103.40) --
	(290.81,100.08) --
	(286.96,100.08) --
	cycle;

\path[fill=fillColor,fill opacity=0.50] (288.89,104.21) --
	(290.81,100.88) --
	(286.96,100.88) --
	cycle;

\path[fill=fillColor,fill opacity=0.50] (291.58,108.12) --
	(293.50,104.79) --
	(289.66,104.79) --
	cycle;

\path[fill=fillColor,fill opacity=0.50] (288.89,105.66) --
	(290.81,102.33) --
	(286.96,102.33) --
	cycle;

\path[fill=fillColor,fill opacity=0.50] (291.58,110.15) --
	(293.50,106.82) --
	(289.66,106.82) --
	cycle;

\path[fill=fillColor,fill opacity=0.50] (293.77,107.55) --
	(295.69,104.22) --
	(291.85,104.22) --
	cycle;

\path[fill=fillColor,fill opacity=0.50] (294.41,107.55) --
	(296.34,104.22) --
	(292.49,104.22) --
	cycle;

\path[fill=fillColor,fill opacity=0.50] (291.98,106.33) --
	(293.90,103.00) --
	(290.05,103.00) --
	cycle;

\path[fill=fillColor,fill opacity=0.50] (294.10,106.96) --
	(296.02,103.63) --
	(292.17,103.63) --
	cycle;

\path[fill=fillColor,fill opacity=0.50] (295.89,108.66) --
	(297.81,105.33) --
	(293.97,105.33) --
	cycle;

\path[fill=fillColor,fill opacity=0.50] (294.41,114.39) --
	(296.34,111.06) --
	(292.49,111.06) --
	cycle;

\path[fill=fillColor,fill opacity=0.50] (296.69,109.18) --
	(298.61,105.85) --
	(294.77,105.85) --
	cycle;

\path[fill=fillColor,fill opacity=0.50] (298.81,105.66) --
	(300.73,102.33) --
	(296.89,102.33) --
	cycle;

\path[fill=fillColor,fill opacity=0.50] (294.10,109.67) --
	(296.02,106.34) --
	(292.17,106.34) --
	cycle;

\path[fill=fillColor,fill opacity=0.50] (297.20,109.18) --
	(299.12,105.85) --
	(295.28,105.85) --
	cycle;

\path[fill=fillColor,fill opacity=0.50] (295.03,113.72) --
	(296.95,110.39) --
	(293.11,110.39) --
	cycle;

\path[fill=fillColor,fill opacity=0.50] (294.10,109.18) --
	(296.02,105.85) --
	(292.17,105.85) --
	cycle;

\path[fill=fillColor,fill opacity=0.50] (295.89,109.18) --
	(297.81,105.85) --
	(293.97,105.85) --
	cycle;

\path[fill=fillColor,fill opacity=0.50] (295.03,110.15) --
	(296.95,106.82) --
	(293.11,106.82) --
	cycle;

\path[fill=fillColor,fill opacity=0.50] (295.03,112.27) --
	(296.95,108.94) --
	(293.11,108.94) --
	cycle;

\path[fill=fillColor,fill opacity=0.50] (299.64,111.87) --
	(301.56,108.54) --
	(297.72,108.54) --
	cycle;

\path[fill=fillColor,fill opacity=0.50] (295.61,106.96) --
	(297.53,103.63) --
	(293.69,103.63) --
	cycle;

\path[fill=fillColor,fill opacity=0.50] (293.77,106.33) --
	(295.69,103.00) --
	(291.85,103.00) --
	cycle;

\path[fill=fillColor,fill opacity=0.50] (293.43,110.15) --
	(295.35,106.82) --
	(291.51,106.82) --
	cycle;

\path[fill=fillColor,fill opacity=0.50] (297.68,111.04) --
	(299.60,107.71) --
	(295.76,107.71) --
	cycle;

\path[fill=fillColor,fill opacity=0.50] (295.61,110.15) --
	(297.53,106.82) --
	(293.69,106.82) --
	cycle;

\path[fill=fillColor,fill opacity=0.50] (297.92,115.61) --
	(299.84,112.28) --
	(295.99,112.28) --
	cycle;

\path[fill=fillColor,fill opacity=0.50] (299.02,116.72) --
	(300.95,113.39) --
	(297.10,113.39) --
	cycle;

\path[fill=fillColor,fill opacity=0.50] (299.23,112.27) --
	(301.16,108.94) --
	(297.31,108.94) --
	cycle;

\path[fill=fillColor,fill opacity=0.50] (303.09,117.49) --
	(305.01,114.16) --
	(301.17,114.16) --
	cycle;

\path[fill=fillColor,fill opacity=0.50] (308.38,114.71) --
	(310.31,111.38) --
	(306.46,111.38) --
	cycle;

\path[fill=fillColor,fill opacity=0.50] (296.43,110.15) --
	(298.35,106.82) --
	(294.51,106.82) --
	cycle;

\path[fill=fillColor,fill opacity=0.50] (297.44,110.15) --
	(299.36,106.82) --
	(295.52,106.82) --
	cycle;

\path[fill=fillColor,fill opacity=0.50] (298.59,124.91) --
	(300.52,121.58) --
	(296.67,121.58) --
	cycle;

\path[fill=fillColor,fill opacity=0.50] (298.37,111.87) --
	(300.29,108.54) --
	(296.45,108.54) --
	cycle;

\path[fill=fillColor,fill opacity=0.50] (303.53,120.13) --
	(305.45,116.80) --
	(301.60,116.80) --
	cycle;

\path[fill=fillColor,fill opacity=0.50] (307.09,111.04) --
	(309.01,107.71) --
	(305.16,107.71) --
	cycle;

\path[fill=fillColor,fill opacity=0.50] (302.63,131.36) --
	(304.55,128.04) --
	(300.71,128.04) --
	cycle;

\path[fill=fillColor,fill opacity=0.50] (298.81,124.91) --
	(300.73,121.58) --
	(296.89,121.58) --
	cycle;

\path[fill=fillColor,fill opacity=0.50] (313.26,118.88) --
	(315.18,115.56) --
	(311.33,115.56) --
	cycle;

\path[fill=fillColor,fill opacity=0.50] (301.66,123.23) --
	(303.58,119.90) --
	(299.74,119.90) --
	cycle;

\path[fill=fillColor,fill opacity=0.50] (299.44,117.24) --
	(301.36,113.91) --
	(297.52,113.91) --
	cycle;

\path[fill=fillColor,fill opacity=0.50] (301.14,111.87) --
	(303.07,108.54) --
	(299.22,108.54) --
	cycle;

\path[fill=fillColor,fill opacity=0.50] (300.04,117.49) --
	(301.96,114.16) --
	(298.12,114.16) --
	cycle;

\path[fill=fillColor,fill opacity=0.50] (300.97,127.27) --
	(302.89,123.94) --
	(299.05,123.94) --
	cycle;

\path[fill=fillColor,fill opacity=0.50] (311.44,125.91) --
	(313.36,122.59) --
	(309.52,122.59) --
	cycle;

\path[fill=fillColor,fill opacity=0.50] (299.64,111.87) --
	(301.56,108.54) --
	(297.72,108.54) --
	cycle;

\path[fill=fillColor,fill opacity=0.50] (320.55,141.58) --
	(322.47,138.25) --
	(318.62,138.25) --
	cycle;

\path[fill=fillColor,fill opacity=0.50] (342.89,139.65) --
	(344.81,136.32) --
	(340.96,136.32) --
	cycle;

\path[fill=fillColor,fill opacity=0.50] (303.09,141.29) --
	(305.01,137.96) --
	(301.17,137.96) --
	cycle;

\path[fill=fillColor,fill opacity=0.50] (300.42,113.38) --
	(302.34,110.05) --
	(298.50,110.05) --
	cycle;

\path[fill=fillColor,fill opacity=0.50] (324.67,114.71) --
	(326.59,111.38) --
	(322.75,111.38) --
	cycle;

\path[fill=fillColor,fill opacity=0.50] (305.74,121.44) --
	(307.66,118.11) --
	(303.82,118.11) --
	cycle;

\path[fill=fillColor,fill opacity=0.50] (301.99,124.24) --
	(303.92,120.91) --
	(300.07,120.91) --
	cycle;

\path[fill=fillColor,fill opacity=0.50] (305.62,119.10) --
	(307.54,115.77) --
	(303.70,115.77) --
	cycle;

\path[fill=fillColor,fill opacity=0.50] (315.61,120.52) --
	(317.53,117.19) --
	(313.69,117.19) --
	cycle;

\path[fill=fillColor,fill opacity=0.50] (308.29,127.05) --
	(310.21,123.73) --
	(306.37,123.73) --
	cycle;

\path[fill=fillColor,fill opacity=0.50] (315.51,127.05) --
	(317.43,123.73) --
	(313.59,123.73) --
	cycle;

\path[fill=fillColor,fill opacity=0.50] (316.91,118.44) --
	(318.83,115.11) --
	(314.99,115.11) --
	cycle;

\path[fill=fillColor,fill opacity=0.50] (313.26,144.65) --
	(315.18,141.33) --
	(311.33,141.33) --
	cycle;

\path[fill=fillColor,fill opacity=0.50] (325.72,127.59) --
	(327.64,124.26) --
	(323.80,124.26) --
	cycle;

\path[fill=fillColor,fill opacity=0.50] (310.14,124.78) --
	(312.06,121.45) --
	(308.21,121.45) --
	cycle;

\path[fill=fillColor,fill opacity=0.50] (302.16,124.78) --
	(304.08,121.45) --
	(300.24,121.45) --
	cycle;

\path[fill=fillColor,fill opacity=0.50] (328.67,148.28) --
	(330.59,144.95) --
	(326.75,144.95) --
	cycle;

\path[fill=fillColor,fill opacity=0.50] (314.82,142.33) --
	(316.75,139.00) --
	(312.90,139.00) --
	cycle;

\path[fill=fillColor,fill opacity=0.50] (308.00,133.85) --
	(309.92,130.53) --
	(306.08,130.53) --
	cycle;

\path[fill=fillColor,fill opacity=0.50] (301.99,143.31) --
	(303.92,139.98) --
	(300.07,139.98) --
	cycle;

\path[fill=fillColor,fill opacity=0.50] (305.62,139.08) --
	(307.54,135.75) --
	(303.70,135.75) --
	cycle;

\path[fill=fillColor,fill opacity=0.50] (319.58,138.93) --
	(321.50,135.60) --
	(317.65,135.60) --
	cycle;

\path[fill=fillColor,fill opacity=0.50] (319.10,150.00) --
	(321.02,146.68) --
	(317.17,146.68) --
	cycle;

\path[fill=fillColor,fill opacity=0.50] (329.49,146.91) --
	(331.41,143.58) --
	(327.56,143.58) --
	cycle;

\path[fill=fillColor,fill opacity=0.50] (335.55,159.09) --
	(337.47,155.76) --
	(333.63,155.76) --
	cycle;

\path[fill=fillColor,fill opacity=0.50] (308.76,138.24) --
	(310.68,134.91) --
	(306.83,134.91) --
	cycle;

\path[fill=fillColor,fill opacity=0.50] (326.38,151.21) --
	(328.30,147.88) --
	(324.46,147.88) --
	cycle;

\path[fill=fillColor,fill opacity=0.50] (323.97,140.12) --
	(325.89,136.79) --
	(322.05,136.79) --
	cycle;

\path[fill=fillColor,fill opacity=0.50] (326.64,135.44) --
	(328.56,132.11) --
	(324.72,132.11) --
	cycle;

\path[fill=fillColor,fill opacity=0.50] (319.17,135.22) --
	(321.09,131.90) --
	(317.25,131.90) --
	cycle;

\path[fill=fillColor,fill opacity=0.50] (321.41,138.49) --
	(323.33,135.16) --
	(319.49,135.16) --
	cycle;

\path[fill=fillColor,fill opacity=0.50] (317.78,142.30) --
	(319.70,138.97) --
	(315.86,138.97) --
	cycle;

\path[fill=fillColor,fill opacity=0.50] (337.71,131.21) --
	(339.63,127.89) --
	(335.79,127.89) --
	cycle;

\path[fill=fillColor,fill opacity=0.50] (324.36,154.70) --
	(326.28,151.37) --
	(322.44,151.37) --
	cycle;

\path[fill=fillColor,fill opacity=0.50] (328.98,141.29) --
	(330.90,137.96) --
	(327.06,137.96) --
	cycle;

\path[fill=fillColor,fill opacity=0.50] (319.21,149.71) --
	(321.13,146.39) --
	(317.29,146.39) --
	cycle;

\path[fill=fillColor,fill opacity=0.50] (314.66,137.47) --
	(316.58,134.14) --
	(312.74,134.14) --
	cycle;

\path[fill=fillColor,fill opacity=0.50] (349.49,161.98) --
	(351.42,158.66) --
	(347.57,158.66) --
	cycle;

\path[fill=fillColor,fill opacity=0.50] (329.62,145.94) --
	(331.54,142.62) --
	(327.70,142.62) --
	cycle;

\path[fill=fillColor,fill opacity=0.50] (313.19,130.10) --
	(315.11,126.77) --
	(311.27,126.77) --
	cycle;

\path[fill=fillColor,fill opacity=0.50] (351.61,160.01) --
	(353.53,156.68) --
	(349.69,156.68) --
	cycle;

\path[fill=fillColor,fill opacity=0.50] (334.64,176.16) --
	(336.56,172.83) --
	(332.72,172.83) --
	cycle;

\path[fill=fillColor,fill opacity=0.50] (338.23,131.81) --
	(340.15,128.48) --
	(336.31,128.48) --
	cycle;

\path[fill=fillColor,fill opacity=0.50] (353.07,149.09) --
	(354.99,145.77) --
	(351.15,145.77) --
	cycle;

\path[fill=fillColor,fill opacity=0.50] (318.48,143.39) --
	(320.40,140.06) --
	(316.56,140.06) --
	cycle;

\path[fill=fillColor,fill opacity=0.50] (336.18,173.13) --
	(338.10,169.80) --
	(334.26,169.80) --
	cycle;

\path[fill=fillColor,fill opacity=0.50] (349.93,174.33) --
	(351.85,171.01) --
	(348.01,171.01) --
	cycle;

\path[fill=fillColor,fill opacity=0.50] (332.24,137.33) --
	(334.16,134.01) --
	(330.32,134.01) --
	cycle;

\path[fill=fillColor,fill opacity=0.50] (324.24,146.43) --
	(326.16,143.10) --
	(322.32,143.10) --
	cycle;

\path[fill=fillColor,fill opacity=0.50] (324.29,133.55) --
	(326.21,130.22) --
	(322.37,130.22) --
	cycle;

\path[fill=fillColor,fill opacity=0.50] (322.21,138.89) --
	(324.14,135.56) --
	(320.29,135.56) --
	cycle;

\path[fill=fillColor,fill opacity=0.50] (358.90,171.04) --
	(360.82,167.71) --
	(356.98,167.71) --
	cycle;

\path[fill=fillColor,fill opacity=0.50] (322.58,145.35) --
	(324.50,142.02) --
	(320.66,142.02) --
	cycle;

\path[fill=fillColor,fill opacity=0.50] (350.05,155.11) --
	(351.97,151.78) --
	(348.13,151.78) --
	cycle;

\path[fill=fillColor,fill opacity=0.50] (351.75,144.46) --
	(353.67,141.13) --
	(349.83,141.13) --
	cycle;

\path[fill=fillColor,fill opacity=0.50] (347.62,175.94) --
	(349.54,172.62) --
	(345.70,172.62) --
	cycle;

\path[fill=fillColor,fill opacity=0.50] (351.26,186.05) --
	(353.18,182.72) --
	(349.34,182.72) --
	cycle;

\path[fill=fillColor,fill opacity=0.50] (335.71,161.77) --
	(337.64,158.44) --
	(333.79,158.44) --
	cycle;

\path[fill=fillColor,fill opacity=0.50] (340.40,186.08) --
	(342.32,182.75) --
	(338.48,182.75) --
	cycle;

\path[fill=fillColor,fill opacity=0.50] (334.97,154.49) --
	(336.89,151.16) --
	(333.05,151.16) --
	cycle;

\path[fill=fillColor,fill opacity=0.50] (358.59,152.63) --
	(360.51,149.30) --
	(356.67,149.30) --
	cycle;

\path[fill=fillColor,fill opacity=0.50] (341.74,168.92) --
	(343.66,165.59) --
	(339.82,165.59) --
	cycle;

\path[fill=fillColor,fill opacity=0.50] (359.23,174.37) --
	(361.15,171.04) --
	(357.31,171.04) --
	cycle;

\path[fill=fillColor,fill opacity=0.50] (362.46,171.27) --
	(364.39,167.94) --
	(360.54,167.94) --
	cycle;

\path[fill=fillColor,fill opacity=0.50] (339.29,152.44) --
	(341.21,149.11) --
	(337.37,149.11) --
	cycle;

\path[fill=fillColor,fill opacity=0.50] (352.26,169.79) --
	(354.18,166.46) --
	(350.34,166.46) --
	cycle;

\path[fill=fillColor,fill opacity=0.50] (345.27,182.26) --
	(347.20,178.93) --
	(343.35,178.93) --
	cycle;

\path[fill=fillColor,fill opacity=0.50] (327.49,142.39) --
	(329.41,139.06) --
	(325.57,139.06) --
	cycle;

\path[fill=fillColor,fill opacity=0.50] (356.66,156.76) --
	(358.58,153.44) --
	(354.74,153.44) --
	cycle;

\path[fill=fillColor,fill opacity=0.50] (339.86,166.10) --
	(341.78,162.77) --
	(337.94,162.77) --
	cycle;

\path[fill=fillColor,fill opacity=0.50] (284.67,103.40) --
	(286.59,100.08) --
	(282.74,100.08) --
	cycle;

\path[fill=fillColor,fill opacity=0.50] (283.11,102.54) --
	(285.03, 99.21) --
	(281.19, 99.21) --
	cycle;

\path[fill=fillColor,fill opacity=0.50] (281.32,102.54) --
	(283.24, 99.21) --
	(279.40, 99.21) --
	cycle;

\path[fill=fillColor,fill opacity=0.50] (283.92,109.67) --
	(285.84,106.34) --
	(281.99,106.34) --
	cycle;

\path[fill=fillColor,fill opacity=0.50] (282.25,102.54) --
	(284.17, 99.21) --
	(280.33, 99.21) --
	cycle;

\path[fill=fillColor,fill opacity=0.50] (283.11,102.54) --
	(285.03, 99.21) --
	(281.19, 99.21) --
	cycle;

\path[fill=fillColor,fill opacity=0.50] (283.11,102.54) --
	(285.03, 99.21) --
	(281.19, 99.21) --
	cycle;

\path[fill=fillColor,fill opacity=0.50] (283.11,101.61) --
	(285.03, 98.28) --
	(281.19, 98.28) --
	cycle;

\path[fill=fillColor,fill opacity=0.50] (283.92,102.54) --
	(285.84, 99.21) --
	(281.99, 99.21) --
	cycle;

\path[fill=fillColor,fill opacity=0.50] (288.89,102.54) --
	(290.81, 99.21) --
	(286.96, 99.21) --
	cycle;

\path[fill=fillColor,fill opacity=0.50] (285.37,104.21) --
	(287.29,100.88) --
	(283.45,100.88) --
	cycle;

\path[fill=fillColor,fill opacity=0.50] (286.04,103.40) --
	(287.96,100.08) --
	(284.11,100.08) --
	cycle;

\path[fill=fillColor,fill opacity=0.50] (284.67,103.40) --
	(286.59,100.08) --
	(282.74,100.08) --
	cycle;

\path[fill=fillColor,fill opacity=0.50] (286.04,111.46) --
	(287.96,108.14) --
	(284.11,108.14) --
	cycle;

\path[fill=fillColor,fill opacity=0.50] (285.37,115.90) --
	(287.29,112.57) --
	(283.45,112.57) --
	cycle;

\path[fill=fillColor,fill opacity=0.50] (283.11,103.40) --
	(285.03,100.08) --
	(281.19,100.08) --
	cycle;

\path[fill=fillColor,fill opacity=0.50] (285.37,103.40) --
	(287.29,100.08) --
	(283.45,100.08) --
	cycle;

\path[fill=fillColor,fill opacity=0.50] (286.66,103.40) --
	(288.59,100.08) --
	(284.74,100.08) --
	cycle;

\path[fill=fillColor,fill opacity=0.50] (286.04,104.21) --
	(287.96,100.88) --
	(284.11,100.88) --
	cycle;

\path[fill=fillColor,fill opacity=0.50] (286.04,104.21) --
	(287.96,100.88) --
	(284.11,100.88) --
	cycle;

\path[fill=fillColor,fill opacity=0.50] (288.89,105.66) --
	(290.81,102.33) --
	(286.96,102.33) --
	cycle;

\path[fill=fillColor,fill opacity=0.50] (291.98,110.15) --
	(293.90,106.82) --
	(290.05,106.82) --
	cycle;

\path[fill=fillColor,fill opacity=0.50] (295.03,104.96) --
	(296.95,101.63) --
	(293.11,101.63) --
	cycle;

\path[fill=fillColor,fill opacity=0.50] (287.26,105.66) --
	(289.18,102.33) --
	(285.34,102.33) --
	cycle;

\path[fill=fillColor,fill opacity=0.50] (286.66,104.21) --
	(288.59,100.88) --
	(284.74,100.88) --
	cycle;

\path[fill=fillColor,fill opacity=0.50] (288.89,107.55) --
	(290.81,104.22) --
	(286.96,104.22) --
	cycle;

\path[fill=fillColor,fill opacity=0.50] (287.83,104.96) --
	(289.75,101.63) --
	(285.91,101.63) --
	cycle;

\path[fill=fillColor,fill opacity=0.50] (289.86,105.66) --
	(291.78,102.33) --
	(287.93,102.33) --
	cycle;

\path[fill=fillColor,fill opacity=0.50] (287.26,105.66) --
	(289.18,102.33) --
	(285.34,102.33) --
	cycle;

\path[fill=fillColor,fill opacity=0.50] (287.26,104.21) --
	(289.18,100.88) --
	(285.34,100.88) --
	cycle;

\path[fill=fillColor,fill opacity=0.50] (289.86,104.96) --
	(291.78,101.63) --
	(287.93,101.63) --
	cycle;

\path[fill=fillColor,fill opacity=0.50] (291.17,108.66) --
	(293.09,105.33) --
	(289.25,105.33) --
	cycle;

\path[fill=fillColor,fill opacity=0.50] (290.31,106.33) --
	(292.23,103.00) --
	(288.39,103.00) --
	cycle;

\path[fill=fillColor,fill opacity=0.50] (291.17,106.33) --
	(293.09,103.00) --
	(289.25,103.00) --
	cycle;

\path[fill=fillColor,fill opacity=0.50] (288.89,105.66) --
	(290.81,102.33) --
	(286.96,102.33) --
	cycle;

\path[fill=fillColor,fill opacity=0.50] (288.89,105.66) --
	(290.81,102.33) --
	(286.96,102.33) --
	cycle;

\path[fill=fillColor,fill opacity=0.50] (289.86,106.96) --
	(291.78,103.63) --
	(287.93,103.63) --
	cycle;

\path[fill=fillColor,fill opacity=0.50] (295.03,108.66) --
	(296.95,105.33) --
	(293.11,105.33) --
	cycle;

\path[fill=fillColor,fill opacity=0.50] (289.38,108.66) --
	(291.30,105.33) --
	(287.46,105.33) --
	cycle;

\path[fill=fillColor,fill opacity=0.50] (287.26,104.96) --
	(289.18,101.63) --
	(285.34,101.63) --
	cycle;

\path[fill=fillColor,fill opacity=0.50] (293.08,109.18) --
	(295.01,105.85) --
	(291.16,105.85) --
	cycle;

\path[fill=fillColor,fill opacity=0.50] (290.31,106.33) --
	(292.23,103.00) --
	(288.39,103.00) --
	cycle;

\path[fill=fillColor,fill opacity=0.50] (289.38,106.96) --
	(291.30,103.63) --
	(287.46,103.63) --
	cycle;

\path[fill=fillColor,fill opacity=0.50] (294.10,111.46) --
	(296.02,108.14) --
	(292.17,108.14) --
	cycle;

\path[fill=fillColor,fill opacity=0.50] (290.31,105.66) --
	(292.23,102.33) --
	(288.39,102.33) --
	cycle;

\path[fill=fillColor,fill opacity=0.50] (291.58,106.96) --
	(293.50,103.63) --
	(289.66,103.63) --
	cycle;

\path[fill=fillColor,fill opacity=0.50] (290.31,105.66) --
	(292.23,102.33) --
	(288.39,102.33) --
	cycle;

\path[fill=fillColor,fill opacity=0.50] (296.43,117.49) --
	(298.35,114.16) --
	(294.51,114.16) --
	cycle;

\path[fill=fillColor,fill opacity=0.50] (290.75,105.66) --
	(292.67,102.33) --
	(288.83,102.33) --
	cycle;

\path[fill=fillColor,fill opacity=0.50] (291.58,105.66) --
	(293.50,102.33) --
	(289.66,102.33) --
	cycle;

\path[fill=fillColor,fill opacity=0.50] (291.58,115.61) --
	(293.50,112.28) --
	(289.66,112.28) --
	cycle;

\path[fill=fillColor,fill opacity=0.50] (291.17,106.96) --
	(293.09,103.63) --
	(289.25,103.63) --
	cycle;

\path[fill=fillColor,fill opacity=0.50] (292.36,106.96) --
	(294.28,103.63) --
	(290.44,103.63) --
	cycle;

\path[fill=fillColor,fill opacity=0.50] (292.73,111.04) --
	(294.65,107.71) --
	(290.80,107.71) --
	cycle;

\path[fill=fillColor,fill opacity=0.50] (291.98,111.04) --
	(293.90,107.71) --
	(290.05,107.71) --
	cycle;

\path[fill=fillColor,fill opacity=0.50] (291.98,106.96) --
	(293.90,103.63) --
	(290.05,103.63) --
	cycle;

\path[fill=fillColor,fill opacity=0.50] (292.36,107.55) --
	(294.28,104.22) --
	(290.44,104.22) --
	cycle;

\path[fill=fillColor,fill opacity=0.50] (297.20,108.12) --
	(299.12,104.79) --
	(295.28,104.79) --
	cycle;

\path[fill=fillColor,fill opacity=0.50] (292.73,106.96) --
	(294.65,103.63) --
	(290.80,103.63) --
	cycle;

\path[fill=fillColor,fill opacity=0.50] (293.08,108.12) --
	(295.01,104.79) --
	(291.16,104.79) --
	cycle;

\path[fill=fillColor,fill opacity=0.50] (296.16,108.12) --
	(298.08,104.79) --
	(294.24,104.79) --
	cycle;

\path[fill=fillColor,fill opacity=0.50] (293.43,108.66) --
	(295.35,105.33) --
	(291.51,105.33) --
	cycle;

\path[fill=fillColor,fill opacity=0.50] (292.73,107.55) --
	(294.65,104.22) --
	(290.80,104.22) --
	cycle;

\path[fill=fillColor,fill opacity=0.50] (294.72,121.61) --
	(296.65,118.28) --
	(292.80,118.28) --
	cycle;

\path[fill=fillColor,fill opacity=0.50] (292.73,110.15) --
	(294.65,106.82) --
	(290.80,106.82) --
	cycle;

\path[fill=fillColor,fill opacity=0.50] (293.43,108.66) --
	(295.35,105.33) --
	(291.51,105.33) --
	cycle;

\path[fill=fillColor,fill opacity=0.50] (293.77,108.12) --
	(295.69,104.79) --
	(291.85,104.79) --
	cycle;

\path[fill=fillColor,fill opacity=0.50] (292.36,108.66) --
	(294.28,105.33) --
	(290.44,105.33) --
	cycle;

\path[fill=fillColor,fill opacity=0.50] (292.36,108.12) --
	(294.28,104.79) --
	(290.44,104.79) --
	cycle;

\path[fill=fillColor,fill opacity=0.50] (300.42,108.12) --
	(302.34,104.79) --
	(298.50,104.79) --
	cycle;

\path[fill=fillColor,fill opacity=0.50] (295.61,109.67) --
	(297.53,106.34) --
	(293.69,106.34) --
	cycle;

\path[fill=fillColor,fill opacity=0.50] (295.89,109.67) --
	(297.81,106.34) --
	(293.97,106.34) --
	cycle;

\path[fill=fillColor,fill opacity=0.50] (295.89,110.15) --
	(297.81,106.82) --
	(293.97,106.82) --
	cycle;

\path[fill=fillColor,fill opacity=0.50] (294.10,108.12) --
	(296.02,104.79) --
	(292.17,104.79) --
	cycle;

\path[fill=fillColor,fill opacity=0.50] (293.08,108.66) --
	(295.01,105.33) --
	(291.16,105.33) --
	cycle;

\path[fill=fillColor,fill opacity=0.50] (296.69,109.67) --
	(298.61,106.34) --
	(294.77,106.34) --
	cycle;

\path[fill=fillColor,fill opacity=0.50] (295.03,109.18) --
	(296.95,105.85) --
	(293.11,105.85) --
	cycle;

\path[fill=fillColor,fill opacity=0.50] (295.03,109.18) --
	(296.95,105.85) --
	(293.11,105.85) --
	cycle;

\path[fill=fillColor,fill opacity=0.50] (294.10,110.15) --
	(296.02,106.82) --
	(292.17,106.82) --
	cycle;

\path[fill=fillColor,fill opacity=0.50] (295.32,110.60) --
	(297.24,107.27) --
	(293.40,107.27) --
	cycle;

\path[fill=fillColor,fill opacity=0.50] (295.89,109.67) --
	(297.81,106.34) --
	(293.97,106.34) --
	cycle;

\path[fill=fillColor,fill opacity=0.50] (296.69,117.24) --
	(298.61,113.91) --
	(294.77,113.91) --
	cycle;

\path[fill=fillColor,fill opacity=0.50] (297.68,120.89) --
	(299.60,117.57) --
	(295.76,117.57) --
	cycle;

\path[fill=fillColor,fill opacity=0.50] (302.48,109.67) --
	(304.40,106.34) --
	(300.55,106.34) --
	cycle;

\path[fill=fillColor,fill opacity=0.50] (295.61,110.60) --
	(297.53,107.27) --
	(293.69,107.27) --
	cycle;

\path[fill=fillColor,fill opacity=0.50] (297.20,111.04) --
	(299.12,107.71) --
	(295.28,107.71) --
	cycle;

\path[fill=fillColor,fill opacity=0.50] (295.61,109.67) --
	(297.53,106.34) --
	(293.69,106.34) --
	cycle;

\path[fill=fillColor,fill opacity=0.50] (296.43,109.67) --
	(298.35,106.34) --
	(294.51,106.34) --
	cycle;

\path[fill=fillColor,fill opacity=0.50] (296.43,110.60) --
	(298.35,107.27) --
	(294.51,107.27) --
	cycle;

\path[fill=fillColor,fill opacity=0.50] (297.68,110.60) --
	(299.60,107.27) --
	(295.76,107.27) --
	cycle;

\path[fill=fillColor,fill opacity=0.50] (298.37,111.46) --
	(300.29,108.14) --
	(296.45,108.14) --
	cycle;

\path[fill=fillColor,fill opacity=0.50] (298.59,111.87) --
	(300.52,108.54) --
	(296.67,108.54) --
	cycle;

\path[fill=fillColor,fill opacity=0.50] (299.64,111.46) --
	(301.56,108.14) --
	(297.72,108.14) --
	cycle;

\path[fill=fillColor,fill opacity=0.50] (299.44,111.04) --
	(301.36,107.71) --
	(297.52,107.71) --
	cycle;

\path[fill=fillColor,fill opacity=0.50] (297.20,110.15) --
	(299.12,106.82) --
	(295.28,106.82) --
	cycle;

\path[fill=fillColor,fill opacity=0.50] (299.84,119.32) --
	(301.76,115.99) --
	(297.92,115.99) --
	cycle;

\path[fill=fillColor,fill opacity=0.50] (296.95,110.15) --
	(298.87,106.82) --
	(295.03,106.82) --
	cycle;

\path[fill=fillColor,fill opacity=0.50] (299.64,111.46) --
	(301.56,108.14) --
	(297.72,108.14) --
	cycle;

\path[fill=fillColor,fill opacity=0.50] (297.20,109.67) --
	(299.12,106.34) --
	(295.28,106.34) --
	cycle;

\path[fill=fillColor,fill opacity=0.50] (299.23,110.60) --
	(301.16,107.27) --
	(297.31,107.27) --
	cycle;

\path[fill=fillColor,fill opacity=0.50] (297.68,112.27) --
	(299.60,108.94) --
	(295.76,108.94) --
	cycle;

\path[fill=fillColor,fill opacity=0.50] (301.32,111.46) --
	(303.24,108.14) --
	(299.40,108.14) --
	cycle;

\path[fill=fillColor,fill opacity=0.50] (299.64,112.27) --
	(301.56,108.94) --
	(297.72,108.94) --
	cycle;

\path[fill=fillColor,fill opacity=0.50] (300.23,112.65) --
	(302.15,109.32) --
	(298.31,109.32) --
	cycle;

\path[fill=fillColor,fill opacity=0.50] (299.23,111.04) --
	(301.16,107.71) --
	(297.31,107.71) --
	cycle;

\path[fill=fillColor,fill opacity=0.50] (303.24,113.02) --
	(305.16,109.69) --
	(301.31,109.69) --
	cycle;

\path[fill=fillColor,fill opacity=0.50] (299.64,110.60) --
	(301.56,107.27) --
	(297.72,107.27) --
	cycle;

\path[fill=fillColor,fill opacity=0.50] (300.42,114.06) --
	(302.34,110.73) --
	(298.50,110.73) --
	cycle;

\path[fill=fillColor,fill opacity=0.50] (301.49,111.46) --
	(303.41,108.14) --
	(299.57,108.14) --
	cycle;

\path[fill=fillColor,fill opacity=0.50] (300.04,111.46) --
	(301.96,108.14) --
	(298.12,108.14) --
	cycle;

\path[fill=fillColor,fill opacity=0.50] (300.97,113.38) --
	(302.89,110.05) --
	(299.05,110.05) --
	cycle;

\path[fill=fillColor,fill opacity=0.50] (301.14,113.72) --
	(303.07,110.39) --
	(299.22,110.39) --
	cycle;

\path[fill=fillColor,fill opacity=0.50] (308.85,122.77) --
	(310.77,119.44) --
	(306.93,119.44) --
	cycle;

\path[fill=fillColor,fill opacity=0.50] (300.23,111.87) --
	(302.15,108.54) --
	(298.31,108.54) --
	cycle;

\path[fill=fillColor,fill opacity=0.50] (300.42,112.27) --
	(302.34,108.94) --
	(298.50,108.94) --
	cycle;

\path[fill=fillColor,fill opacity=0.50] (301.14,113.38) --
	(303.07,110.05) --
	(299.22,110.05) --
	cycle;

\path[fill=fillColor,fill opacity=0.50] (300.04,115.02) --
	(301.96,111.69) --
	(298.12,111.69) --
	cycle;

\path[fill=fillColor,fill opacity=0.50] (301.99,112.65) --
	(303.92,109.32) --
	(300.07,109.32) --
	cycle;

\path[fill=fillColor,fill opacity=0.50] (302.48,112.27) --
	(304.40,108.94) --
	(300.55,108.94) --
	cycle;

\path[fill=fillColor,fill opacity=0.50] (299.84,117.24) --
	(301.76,113.91) --
	(297.92,113.91) --
	cycle;

\path[fill=fillColor,fill opacity=0.50] (303.09,112.27) --
	(305.01,108.94) --
	(301.17,108.94) --
	cycle;

\path[fill=fillColor,fill opacity=0.50] (302.16,113.38) --
	(304.08,110.05) --
	(300.24,110.05) --
	cycle;

\path[fill=fillColor,fill opacity=0.50] (301.49,113.72) --
	(303.41,110.39) --
	(299.57,110.39) --
	cycle;

\path[fill=fillColor,fill opacity=0.50] (302.94,114.06) --
	(304.86,110.73) --
	(301.02,110.73) --
	cycle;

\path[fill=fillColor,fill opacity=0.50] (302.63,116.72) --
	(304.55,113.39) --
	(300.71,113.39) --
	cycle;

\path[fill=fillColor,fill opacity=0.50] (302.32,114.39) --
	(304.24,111.06) --
	(300.40,111.06) --
	cycle;

\path[fill=fillColor,fill opacity=0.50] (302.63,115.90) --
	(304.55,112.57) --
	(300.71,112.57) --
	cycle;

\path[fill=fillColor,fill opacity=0.50] (302.48,112.27) --
	(304.40,108.94) --
	(300.55,108.94) --
	cycle;

\path[fill=fillColor,fill opacity=0.50] (302.32,114.39) --
	(304.24,111.06) --
	(300.40,111.06) --
	cycle;

\path[fill=fillColor,fill opacity=0.50] (302.48,113.72) --
	(304.40,110.39) --
	(300.55,110.39) --
	cycle;

\path[fill=fillColor,fill opacity=0.50] (303.95,114.39) --
	(305.87,111.06) --
	(302.03,111.06) --
	cycle;

\path[fill=fillColor,fill opacity=0.50] (306.21,115.90) --
	(308.13,112.57) --
	(304.29,112.57) --
	cycle;

\path[fill=fillColor,fill opacity=0.50] (301.99,113.72) --
	(303.92,110.39) --
	(300.07,110.39) --
	cycle;

\path[fill=fillColor,fill opacity=0.50] (305.62,117.24) --
	(307.54,113.91) --
	(303.70,113.91) --
	cycle;

\path[fill=fillColor,fill opacity=0.50] (304.36,115.90) --
	(306.28,112.57) --
	(302.44,112.57) --
	cycle;

\path[fill=fillColor,fill opacity=0.50] (305.26,120.89) --
	(307.18,117.57) --
	(303.34,117.57) --
	cycle;

\path[fill=fillColor,fill opacity=0.50] (304.09,114.39) --
	(306.01,111.06) --
	(302.17,111.06) --
	cycle;

\path[fill=fillColor,fill opacity=0.50] (308.19,134.56) --
	(310.11,131.23) --
	(306.27,131.23) --
	cycle;

\path[fill=fillColor,fill opacity=0.50] (305.38,118.88) --
	(307.30,115.56) --
	(303.46,115.56) --
	cycle;

\path[fill=fillColor,fill opacity=0.50] (309.64,148.19) --
	(311.56,144.87) --
	(307.72,144.87) --
	cycle;

\path[fill=fillColor,fill opacity=0.50] (305.74,116.18) --
	(307.66,112.85) --
	(303.82,112.85) --
	cycle;

\path[fill=fillColor,fill opacity=0.50] (319.79,125.91) --
	(321.71,122.59) --
	(317.87,122.59) --
	cycle;

\path[fill=fillColor,fill opacity=0.50] (307.09,116.18) --
	(309.01,112.85) --
	(305.16,112.85) --
	cycle;

\path[fill=fillColor,fill opacity=0.50] (305.74,118.88) --
	(307.66,115.56) --
	(303.82,115.56) --
	cycle;

\path[fill=fillColor,fill opacity=0.50] (309.03,120.52) --
	(310.95,117.19) --
	(307.11,117.19) --
	cycle;

\path[fill=fillColor,fill opacity=0.50] (306.76,118.21) --
	(308.68,114.88) --
	(304.84,114.88) --
	cycle;

\path[fill=fillColor,fill opacity=0.50] (306.98,124.91) --
	(308.90,121.58) --
	(305.06,121.58) --
	cycle;

\path[fill=fillColor,fill opacity=0.50] (311.44,116.18) --
	(313.36,112.85) --
	(309.52,112.85) --
	cycle;

\path[fill=fillColor,fill opacity=0.50] (307.50,124.38) --
	(309.42,121.05) --
	(305.58,121.05) --
	cycle;

\path[fill=fillColor,fill opacity=0.50] (305.26,116.72) --
	(307.18,113.39) --
	(303.34,113.39) --
	cycle;

\path[fill=fillColor,fill opacity=0.50] (312.62,133.23) --
	(314.54,129.90) --
	(310.70,129.90) --
	cycle;

\path[fill=fillColor,fill opacity=0.50] (331.03,129.32) --
	(332.95,125.99) --
	(329.11,125.99) --
	cycle;

\path[fill=fillColor,fill opacity=0.50] (330.90,124.10) --
	(332.83,120.77) --
	(328.98,120.77) --
	cycle;

\path[fill=fillColor,fill opacity=0.50] (334.30,140.29) --
	(336.22,136.96) --
	(332.38,136.96) --
	cycle;

\path[fill=fillColor,fill opacity=0.50] (309.97,122.28) --
	(311.89,118.96) --
	(308.05,118.96) --
	cycle;

\path[fill=fillColor,fill opacity=0.50] (335.47,156.19) --
	(337.39,152.87) --
	(333.55,152.87) --
	cycle;

\path[fill=fillColor,fill opacity=0.50] (313.00,126.03) --
	(314.93,122.70) --
	(311.08,122.70) --
	cycle;

\path[fill=fillColor,fill opacity=0.50] (314.32,122.77) --
	(316.25,119.44) --
	(312.40,119.44) --
	cycle;

\path[fill=fillColor,fill opacity=0.50] (328.32,154.57) --
	(330.24,151.24) --
	(326.40,151.24) --
	cycle;

\path[fill=fillColor,fill opacity=0.50] (312.01,149.45) --
	(313.93,146.12) --
	(310.09,146.12) --
	cycle;

\path[fill=fillColor,fill opacity=0.50] (333.66,160.10) --
	(335.58,156.78) --
	(331.74,156.78) --
	cycle;

\path[fill=fillColor,fill opacity=0.50] (319.72,128.86) --
	(321.64,125.53) --
	(317.80,125.53) --
	cycle;

\path[fill=fillColor,fill opacity=0.50] (320.28,139.61) --
	(322.20,136.28) --
	(318.35,136.28) --
	cycle;

\path[fill=fillColor,fill opacity=0.50] (317.99,127.05) --
	(319.91,123.73) --
	(316.07,123.73) --
	cycle;

\path[fill=fillColor,fill opacity=0.50] (315.81,125.55) --
	(317.73,122.22) --
	(313.89,122.22) --
	cycle;

\path[fill=fillColor,fill opacity=0.50] (315.25,135.95) --
	(317.17,132.63) --
	(313.33,132.63) --
	cycle;

\path[fill=fillColor,fill opacity=0.50] (322.30,149.11) --
	(324.22,145.78) --
	(320.38,145.78) --
	cycle;

\path[fill=fillColor,fill opacity=0.50] (328.67,167.74) --
	(330.59,164.41) --
	(326.75,164.41) --
	cycle;

\path[fill=fillColor,fill opacity=0.50] (317.82,135.33) --
	(319.75,132.00) --
	(315.90,132.00) --
	cycle;

\path[fill=fillColor,fill opacity=0.50] (320.14,131.66) --
	(322.06,128.33) --
	(318.22,128.33) --
	cycle;

\path[fill=fillColor,fill opacity=0.50] (342.38,164.33) --
	(344.30,161.00) --
	(340.46,161.00) --
	cycle;

\path[fill=fillColor,fill opacity=0.50] (394.37,177.24) --
	(396.29,173.91) --
	(392.45,173.91) --
	cycle;

\path[fill=fillColor,fill opacity=0.50] (336.74,141.61) --
	(338.66,138.28) --
	(334.82,138.28) --
	cycle;

\path[fill=fillColor,fill opacity=0.50] (349.79,170.28) --
	(351.71,166.96) --
	(347.87,166.96) --
	cycle;

\path[fill=fillColor,fill opacity=0.50] (338.89,148.71) --
	(340.81,145.38) --
	(336.97,145.38) --
	cycle;

\path[fill=fillColor,fill opacity=0.50] (333.76,158.14) --
	(335.68,154.81) --
	(331.83,154.81) --
	cycle;

\path[fill=fillColor,fill opacity=0.50] (377.51,199.36) --
	(379.43,196.03) --
	(375.59,196.03) --
	cycle;

\path[fill=fillColor,fill opacity=0.50] (330.40,158.68) --
	(332.32,155.35) --
	(328.48,155.35) --
	cycle;

\path[fill=fillColor,fill opacity=0.50] (328.29,160.64) --
	(330.21,157.31) --
	(326.36,157.31) --
	cycle;

\path[fill=fillColor,fill opacity=0.50] (330.47,134.56) --
	(332.40,131.23) --
	(328.55,131.23) --
	cycle;

\path[fill=fillColor,fill opacity=0.50] (355.47,183.14) --
	(357.39,179.81) --
	(353.55,179.81) --
	cycle;

\path[fill=fillColor,fill opacity=0.50] (372.77,197.84) --
	(374.69,194.52) --
	(370.85,194.52) --
	cycle;

\path[fill=fillColor,fill opacity=0.50] (334.87,144.41) --
	(336.80,141.08) --
	(332.95,141.08) --
	cycle;

\path[fill=fillColor,fill opacity=0.50] (356.78,149.53) --
	(358.70,146.20) --
	(354.86,146.20) --
	cycle;

\path[fill=fillColor,fill opacity=0.50] (394.37,214.66) --
	(396.29,211.33) --
	(392.45,211.33) --
	cycle;

\path[fill=fillColor,fill opacity=0.50] (355.60,166.98) --
	(357.53,163.65) --
	(353.68,163.65) --
	cycle;

\path[fill=fillColor,fill opacity=0.50] (356.60,162.89) --
	(358.52,159.56) --
	(354.67,159.56) --
	cycle;

\path[fill=fillColor,fill opacity=0.50] (394.37,161.38) --
	(396.29,158.05) --
	(392.45,158.05) --
	cycle;

\path[fill=fillColor,fill opacity=0.50] (366.06,214.66) --
	(367.98,211.33) --
	(364.14,211.33) --
	cycle;

\path[fill=fillColor,fill opacity=0.50] (377.17,186.67) --
	(379.09,183.35) --
	(375.25,183.35) --
	cycle;

\path[fill=fillColor,fill opacity=0.50] (351.09,193.84) --
	(353.01,190.51) --
	(349.17,190.51) --
	cycle;

\path[fill=fillColor,fill opacity=0.50] (375.20,155.63) --
	(377.12,152.30) --
	(373.28,152.30) --
	cycle;

\path[fill=fillColor,fill opacity=0.50] (359.51,158.42) --
	(361.43,155.09) --
	(357.59,155.09) --
	cycle;

\path[fill=fillColor,fill opacity=0.50] (378.34,192.42) --
	(380.26,189.09) --
	(376.42,189.09) --
	cycle;

\path[fill=fillColor,fill opacity=0.50] (360.14,180.15) --
	(362.06,176.82) --
	(358.22,176.82) --
	cycle;

\path[fill=fillColor,fill opacity=0.50] (394.37,214.66) --
	(396.29,211.33) --
	(392.45,211.33) --
	cycle;

\path[fill=fillColor,fill opacity=0.50] (384.09,198.28) --
	(386.01,194.95) --
	(382.17,194.95) --
	cycle;

\path[fill=fillColor,fill opacity=0.50] (394.37,214.66) --
	(396.29,211.33) --
	(392.45,211.33) --
	cycle;

\path[fill=fillColor,fill opacity=0.50] (369.56,188.41) --
	(371.48,185.08) --
	(367.64,185.08) --
	cycle;

\path[fill=fillColor,fill opacity=0.50] (378.04,214.66) --
	(379.96,211.33) --
	(376.12,211.33) --
	cycle;

\path[fill=fillColor,fill opacity=0.50] (394.37,214.66) --
	(396.29,211.33) --
	(392.45,211.33) --
	cycle;

\path[fill=fillColor,fill opacity=0.50] (394.37,205.26) --
	(396.29,201.93) --
	(392.45,201.93) --
	cycle;

\path[fill=fillColor,fill opacity=0.50] (394.37,214.66) --
	(396.29,211.33) --
	(392.45,211.33) --
	cycle;

\path[fill=fillColor,fill opacity=0.50] (342.50,196.18) --
	(344.43,192.85) --
	(340.58,192.85) --
	cycle;

\path[fill=fillColor,fill opacity=0.50] (365.44,214.66) --
	(367.37,211.33) --
	(363.52,211.33) --
	cycle;

\path[fill=fillColor,fill opacity=0.50] (394.37,200.75) --
	(396.29,197.42) --
	(392.45,197.42) --
	cycle;

\path[fill=fillColor,fill opacity=0.50] (394.37,214.66) --
	(396.29,211.33) --
	(392.45,211.33) --
	cycle;

\path[fill=fillColor,fill opacity=0.50] (345.85,175.72) --
	(347.77,172.40) --
	(343.93,172.40) --
	cycle;

\path[fill=fillColor,fill opacity=0.50] (362.95,214.66) --
	(364.87,211.33) --
	(361.03,211.33) --
	cycle;

\path[fill=fillColor,fill opacity=0.50] (394.37,214.66) --
	(396.29,211.33) --
	(392.45,211.33) --
	cycle;

\path[draw=drawColor,draw opacity=0.50,line width= 0.4pt,line join=round,line cap=round] (296.01,111.06) rectangle (298.87,113.91);

\path[draw=drawColor,draw opacity=0.50,line width= 0.4pt,line join=round,line cap=round] (296.01,111.06) -- (298.87,113.91);

\path[draw=drawColor,draw opacity=0.50,line width= 0.4pt,line join=round,line cap=round] (296.01,113.91) -- (298.87,111.06);

\path[draw=drawColor,draw opacity=0.50,line width= 0.4pt,line join=round,line cap=round] (297.60,112.25) rectangle (300.45,115.11);

\path[draw=drawColor,draw opacity=0.50,line width= 0.4pt,line join=round,line cap=round] (297.60,112.25) -- (300.45,115.11);

\path[draw=drawColor,draw opacity=0.50,line width= 0.4pt,line join=round,line cap=round] (297.60,115.11) -- (300.45,112.25);

\path[draw=drawColor,draw opacity=0.50,line width= 0.4pt,line join=round,line cap=round] (344.13,167.88) rectangle (346.98,170.73);

\path[draw=drawColor,draw opacity=0.50,line width= 0.4pt,line join=round,line cap=round] (344.13,167.88) -- (346.98,170.73);

\path[draw=drawColor,draw opacity=0.50,line width= 0.4pt,line join=round,line cap=round] (344.13,170.73) -- (346.98,167.88);

\path[draw=drawColor,draw opacity=0.50,line width= 0.4pt,line join=round,line cap=round] (315.97,135.93) rectangle (318.82,138.78);

\path[draw=drawColor,draw opacity=0.50,line width= 0.4pt,line join=round,line cap=round] (315.97,135.93) -- (318.82,138.78);

\path[draw=drawColor,draw opacity=0.50,line width= 0.4pt,line join=round,line cap=round] (315.97,138.78) -- (318.82,135.93);

\path[draw=drawColor,draw opacity=0.50,line width= 0.4pt,line join=round,line cap=round] (323.88,178.59) rectangle (326.73,181.44);

\path[draw=drawColor,draw opacity=0.50,line width= 0.4pt,line join=round,line cap=round] (323.88,178.59) -- (326.73,181.44);

\path[draw=drawColor,draw opacity=0.50,line width= 0.4pt,line join=round,line cap=round] (323.88,181.44) -- (326.73,178.59);

\path[draw=drawColor,draw opacity=0.50,line width= 0.4pt,line join=round,line cap=round] (324.60,132.00) rectangle (327.46,134.85);

\path[draw=drawColor,draw opacity=0.50,line width= 0.4pt,line join=round,line cap=round] (324.60,132.00) -- (327.46,134.85);

\path[draw=drawColor,draw opacity=0.50,line width= 0.4pt,line join=round,line cap=round] (324.60,134.85) -- (327.46,132.00);

\path[draw=drawColor,draw opacity=0.50,line width= 0.4pt,line join=round,line cap=round] (332.48,139.34) rectangle (335.33,142.19);

\path[draw=drawColor,draw opacity=0.50,line width= 0.4pt,line join=round,line cap=round] (332.48,139.34) -- (335.33,142.19);

\path[draw=drawColor,draw opacity=0.50,line width= 0.4pt,line join=round,line cap=round] (332.48,142.19) -- (335.33,139.34);
\definecolor{drawColor}{RGB}{152,152,152}

\path[draw=drawColor,line width= 0.6pt,line join=round] ( 89.19,-92.74) -- (557.60,375.68);
\definecolor{drawColor}{gray}{0.70}

\path[draw=drawColor,line width= 0.6pt,line join=round,line cap=round] (246.48, 63.40) -- (246.48, 66.24);

\path[draw=drawColor,line width= 0.6pt,line join=round,line cap=round] (248.28, 63.40) -- (248.28, 66.24);

\path[draw=drawColor,line width= 0.6pt,line join=round,line cap=round] (249.83, 63.40) -- (249.83, 66.24);

\path[draw=drawColor,line width= 0.6pt,line join=round,line cap=round] (251.20, 63.40) -- (251.20, 66.24);

\path[draw=drawColor,line width= 0.6pt,line join=round,line cap=round] (252.42, 63.40) -- (252.42, 71.93);

\path[draw=drawColor,line width= 0.6pt,line join=round,line cap=round] (260.48, 63.40) -- (260.48, 66.24);

\path[draw=drawColor,line width= 0.6pt,line join=round,line cap=round] (265.20, 63.40) -- (265.20, 66.24);

\path[draw=drawColor,line width= 0.6pt,line join=round,line cap=round] (268.54, 63.40) -- (268.54, 66.24);

\path[draw=drawColor,line width= 0.6pt,line join=round,line cap=round] (271.14, 63.40) -- (271.14, 69.09);

\path[draw=drawColor,line width= 0.6pt,line join=round,line cap=round] (273.26, 63.40) -- (273.26, 66.24);

\path[draw=drawColor,line width= 0.6pt,line join=round,line cap=round] (275.05, 63.40) -- (275.05, 66.24);

\path[draw=drawColor,line width= 0.6pt,line join=round,line cap=round] (276.61, 63.40) -- (276.61, 66.24);

\path[draw=drawColor,line width= 0.6pt,line join=round,line cap=round] (277.97, 63.40) -- (277.97, 66.24);

\path[draw=drawColor,line width= 0.6pt,line join=round,line cap=round] (279.20, 63.40) -- (279.20, 71.93);

\path[draw=drawColor,line width= 0.6pt,line join=round,line cap=round] (287.26, 63.40) -- (287.26, 66.24);

\path[draw=drawColor,line width= 0.6pt,line join=round,line cap=round] (291.98, 63.40) -- (291.98, 66.24);

\path[draw=drawColor,line width= 0.6pt,line join=round,line cap=round] (295.32, 63.40) -- (295.32, 66.24);

\path[draw=drawColor,line width= 0.6pt,line join=round,line cap=round] (297.92, 63.40) -- (297.92, 69.09);

\path[draw=drawColor,line width= 0.6pt,line join=round,line cap=round] (300.04, 63.40) -- (300.04, 66.24);

\path[draw=drawColor,line width= 0.6pt,line join=round,line cap=round] (301.83, 63.40) -- (301.83, 66.24);

\path[draw=drawColor,line width= 0.6pt,line join=round,line cap=round] (303.38, 63.40) -- (303.38, 66.24);

\path[draw=drawColor,line width= 0.6pt,line join=round,line cap=round] (304.75, 63.40) -- (304.75, 66.24);

\path[draw=drawColor,line width= 0.6pt,line join=round,line cap=round] (305.98, 63.40) -- (305.98, 71.93);

\path[draw=drawColor,line width= 0.6pt,line join=round,line cap=round] (314.04, 63.40) -- (314.04, 66.24);

\path[draw=drawColor,line width= 0.6pt,line join=round,line cap=round] (318.75, 63.40) -- (318.75, 66.24);

\path[draw=drawColor,line width= 0.6pt,line join=round,line cap=round] (322.10, 63.40) -- (322.10, 66.24);

\path[draw=drawColor,line width= 0.6pt,line join=round,line cap=round] (324.69, 63.40) -- (324.69, 69.09);

\path[draw=drawColor,line width= 0.6pt,line join=round,line cap=round] (326.81, 63.40) -- (326.81, 66.24);

\path[draw=drawColor,line width= 0.6pt,line join=round,line cap=round] (328.61, 63.40) -- (328.61, 66.24);

\path[draw=drawColor,line width= 0.6pt,line join=round,line cap=round] (330.16, 63.40) -- (330.16, 66.24);

\path[draw=drawColor,line width= 0.6pt,line join=round,line cap=round] (331.53, 63.40) -- (331.53, 66.24);

\path[draw=drawColor,line width= 0.6pt,line join=round,line cap=round] (332.75, 63.40) -- (332.75, 71.93);

\path[draw=drawColor,line width= 0.6pt,line join=round,line cap=round] (340.81, 63.40) -- (340.81, 66.24);

\path[draw=drawColor,line width= 0.6pt,line join=round,line cap=round] (345.53, 63.40) -- (345.53, 66.24);

\path[draw=drawColor,line width= 0.6pt,line join=round,line cap=round] (348.87, 63.40) -- (348.87, 66.24);

\path[draw=drawColor,line width= 0.6pt,line join=round,line cap=round] (351.47, 63.40) -- (351.47, 69.09);

\path[draw=drawColor,line width= 0.6pt,line join=round,line cap=round] (353.59, 63.40) -- (353.59, 66.24);

\path[draw=drawColor,line width= 0.6pt,line join=round,line cap=round] (355.38, 63.40) -- (355.38, 66.24);

\path[draw=drawColor,line width= 0.6pt,line join=round,line cap=round] (356.94, 63.40) -- (356.94, 66.24);

\path[draw=drawColor,line width= 0.6pt,line join=round,line cap=round] (358.30, 63.40) -- (358.30, 66.24);

\path[draw=drawColor,line width= 0.6pt,line join=round,line cap=round] (359.53, 63.40) -- (359.53, 71.93);

\path[draw=drawColor,line width= 0.6pt,line join=round,line cap=round] (367.59, 63.40) -- (367.59, 66.24);

\path[draw=drawColor,line width= 0.6pt,line join=round,line cap=round] (372.31, 63.40) -- (372.31, 66.24);

\path[draw=drawColor,line width= 0.6pt,line join=round,line cap=round] (375.65, 63.40) -- (375.65, 66.24);

\path[draw=drawColor,line width= 0.6pt,line join=round,line cap=round] (378.25, 63.40) -- (378.25, 69.09);

\path[draw=drawColor,line width= 0.6pt,line join=round,line cap=round] (380.37, 63.40) -- (380.37, 66.24);

\path[draw=drawColor,line width= 0.6pt,line join=round,line cap=round] (382.16, 63.40) -- (382.16, 66.24);

\path[draw=drawColor,line width= 0.6pt,line join=round,line cap=round] (383.71, 63.40) -- (383.71, 66.24);

\path[draw=drawColor,line width= 0.6pt,line join=round,line cap=round] (385.08, 63.40) -- (385.08, 66.24);

\path[draw=drawColor,line width= 0.6pt,line join=round,line cap=round] (386.31, 63.40) -- (386.31, 71.93);

\path[draw=drawColor,line width= 0.6pt,line join=round,line cap=round] (394.37, 63.40) -- (394.37, 66.24);

\path[draw=drawColor,line width= 0.6pt,line join=round,line cap=round] (399.08, 63.40) -- (399.08, 66.24);

\path[draw=drawColor,line width= 0.6pt,line join=round,line cap=round] (245.33, 64.56) -- (248.17, 64.56);

\path[draw=drawColor,line width= 0.6pt,line join=round,line cap=round] (245.33, 66.35) -- (248.17, 66.35);

\path[draw=drawColor,line width= 0.6pt,line join=round,line cap=round] (245.33, 67.90) -- (248.17, 67.90);

\path[draw=drawColor,line width= 0.6pt,line join=round,line cap=round] (245.33, 69.27) -- (248.17, 69.27);

\path[draw=drawColor,line width= 0.6pt,line join=round,line cap=round] (245.33, 70.50) -- (253.86, 70.50);

\path[draw=drawColor,line width= 0.6pt,line join=round,line cap=round] (245.33, 78.56) -- (248.17, 78.56);

\path[draw=drawColor,line width= 0.6pt,line join=round,line cap=round] (245.33, 83.27) -- (248.17, 83.27);

\path[draw=drawColor,line width= 0.6pt,line join=round,line cap=round] (245.33, 86.62) -- (248.17, 86.62);

\path[draw=drawColor,line width= 0.6pt,line join=round,line cap=round] (245.33, 89.21) -- (251.02, 89.21);

\path[draw=drawColor,line width= 0.6pt,line join=round,line cap=round] (245.33, 91.33) -- (248.17, 91.33);

\path[draw=drawColor,line width= 0.6pt,line join=round,line cap=round] (245.33, 93.12) -- (248.17, 93.12);

\path[draw=drawColor,line width= 0.6pt,line join=round,line cap=round] (245.33, 94.68) -- (248.17, 94.68);

\path[draw=drawColor,line width= 0.6pt,line join=round,line cap=round] (245.33, 96.05) -- (248.17, 96.05);

\path[draw=drawColor,line width= 0.6pt,line join=round,line cap=round] (245.33, 97.27) -- (253.86, 97.27);

\path[draw=drawColor,line width= 0.6pt,line join=round,line cap=round] (245.33,105.33) -- (248.17,105.33);

\path[draw=drawColor,line width= 0.6pt,line join=round,line cap=round] (245.33,110.05) -- (248.17,110.05);

\path[draw=drawColor,line width= 0.6pt,line join=round,line cap=round] (245.33,113.39) -- (248.17,113.39);

\path[draw=drawColor,line width= 0.6pt,line join=round,line cap=round] (245.33,115.99) -- (251.02,115.99);

\path[draw=drawColor,line width= 0.6pt,line join=round,line cap=round] (245.33,118.11) -- (248.17,118.11);

\path[draw=drawColor,line width= 0.6pt,line join=round,line cap=round] (245.33,119.90) -- (248.17,119.90);

\path[draw=drawColor,line width= 0.6pt,line join=round,line cap=round] (245.33,121.45) -- (248.17,121.45);

\path[draw=drawColor,line width= 0.6pt,line join=round,line cap=round] (245.33,122.82) -- (248.17,122.82);

\path[draw=drawColor,line width= 0.6pt,line join=round,line cap=round] (245.33,124.05) -- (253.86,124.05);

\path[draw=drawColor,line width= 0.6pt,line join=round,line cap=round] (245.33,132.11) -- (248.17,132.11);

\path[draw=drawColor,line width= 0.6pt,line join=round,line cap=round] (245.33,136.82) -- (248.17,136.82);

\path[draw=drawColor,line width= 0.6pt,line join=round,line cap=round] (245.33,140.17) -- (248.17,140.17);

\path[draw=drawColor,line width= 0.6pt,line join=round,line cap=round] (245.33,142.77) -- (251.02,142.77);

\path[draw=drawColor,line width= 0.6pt,line join=round,line cap=round] (245.33,144.89) -- (248.17,144.89);

\path[draw=drawColor,line width= 0.6pt,line join=round,line cap=round] (245.33,146.68) -- (248.17,146.68);

\path[draw=drawColor,line width= 0.6pt,line join=round,line cap=round] (245.33,148.23) -- (248.17,148.23);

\path[draw=drawColor,line width= 0.6pt,line join=round,line cap=round] (245.33,149.60) -- (248.17,149.60);

\path[draw=drawColor,line width= 0.6pt,line join=round,line cap=round] (245.33,150.83) -- (253.86,150.83);

\path[draw=drawColor,line width= 0.6pt,line join=round,line cap=round] (245.33,158.89) -- (248.17,158.89);

\path[draw=drawColor,line width= 0.6pt,line join=round,line cap=round] (245.33,163.60) -- (248.17,163.60);

\path[draw=drawColor,line width= 0.6pt,line join=round,line cap=round] (245.33,166.95) -- (248.17,166.95);

\path[draw=drawColor,line width= 0.6pt,line join=round,line cap=round] (245.33,169.54) -- (251.02,169.54);

\path[draw=drawColor,line width= 0.6pt,line join=round,line cap=round] (245.33,171.66) -- (248.17,171.66);

\path[draw=drawColor,line width= 0.6pt,line join=round,line cap=round] (245.33,173.45) -- (248.17,173.45);

\path[draw=drawColor,line width= 0.6pt,line join=round,line cap=round] (245.33,175.01) -- (248.17,175.01);

\path[draw=drawColor,line width= 0.6pt,line join=round,line cap=round] (245.33,176.38) -- (248.17,176.38);

\path[draw=drawColor,line width= 0.6pt,line join=round,line cap=round] (245.33,177.60) -- (253.86,177.60);

\path[draw=drawColor,line width= 0.6pt,line join=round,line cap=round] (245.33,185.66) -- (248.17,185.66);

\path[draw=drawColor,line width= 0.6pt,line join=round,line cap=round] (245.33,190.38) -- (248.17,190.38);

\path[draw=drawColor,line width= 0.6pt,line join=round,line cap=round] (245.33,193.72) -- (248.17,193.72);

\path[draw=drawColor,line width= 0.6pt,line join=round,line cap=round] (245.33,196.32) -- (251.02,196.32);

\path[draw=drawColor,line width= 0.6pt,line join=round,line cap=round] (245.33,198.44) -- (248.17,198.44);

\path[draw=drawColor,line width= 0.6pt,line join=round,line cap=round] (245.33,200.23) -- (248.17,200.23);

\path[draw=drawColor,line width= 0.6pt,line join=round,line cap=round] (245.33,201.78) -- (248.17,201.78);

\path[draw=drawColor,line width= 0.6pt,line join=round,line cap=round] (245.33,203.15) -- (248.17,203.15);

\path[draw=drawColor,line width= 0.6pt,line join=round,line cap=round] (245.33,204.38) -- (253.86,204.38);

\path[draw=drawColor,line width= 0.6pt,line join=round,line cap=round] (245.33,212.44) -- (248.17,212.44);

\path[draw=drawColor,line width= 0.6pt,line join=round,line cap=round] (245.33,217.15) -- (248.17,217.15);

\path[draw=drawColor,line width= 0.5pt,line join=round,line cap=round] (245.33, 63.40) rectangle (401.46,219.54);
\end{scope}
\begin{scope}
\path[clip] (  0.00,  0.00) rectangle (411.94,224.04);
\definecolor{drawColor}{gray}{0.30}

\node[text=drawColor,anchor=base east,inner sep=0pt, outer sep=0pt, scale=  0.72] at (241.28, 94.79) {0.1};

\node[text=drawColor,anchor=base east,inner sep=0pt, outer sep=0pt, scale=  0.72] at (241.28,148.35) {10};

\node[text=drawColor,anchor=base east,inner sep=0pt, outer sep=0pt, scale=  0.72] at (241.28,201.90) {1000};
\end{scope}
\begin{scope}
\path[clip] (  0.00,  0.00) rectangle (411.94,224.04);
\definecolor{drawColor}{gray}{0.70}

\path[draw=drawColor,line width= 0.2pt,line join=round] (243.08, 97.27) --
	(245.33, 97.27);

\path[draw=drawColor,line width= 0.2pt,line join=round] (243.08,150.83) --
	(245.33,150.83);

\path[draw=drawColor,line width= 0.2pt,line join=round] (243.08,204.38) --
	(245.33,204.38);
\end{scope}
\begin{scope}
\path[clip] (  0.00,  0.00) rectangle (411.94,224.04);
\definecolor{drawColor}{gray}{0.70}

\path[draw=drawColor,line width= 0.2pt,line join=round] (279.20, 61.15) --
	(279.20, 63.40);

\path[draw=drawColor,line width= 0.2pt,line join=round] (332.75, 61.15) --
	(332.75, 63.40);

\path[draw=drawColor,line width= 0.2pt,line join=round] (386.31, 61.15) --
	(386.31, 63.40);
\end{scope}
\begin{scope}
\path[clip] (  0.00,  0.00) rectangle (411.94,224.04);
\definecolor{drawColor}{gray}{0.30}

\node[text=drawColor,anchor=base,inner sep=0pt, outer sep=0pt, scale=  0.72] at (279.20, 54.39) {0.1};

\node[text=drawColor,anchor=base,inner sep=0pt, outer sep=0pt, scale=  0.72] at (332.75, 54.39) {10};

\node[text=drawColor,anchor=base,inner sep=0pt, outer sep=0pt, scale=  0.72] at (386.31, 54.39) {1000};
\end{scope}
\begin{scope}
\path[clip] (  0.00,  0.00) rectangle (411.94,224.04);
\definecolor{drawColor}{RGB}{0,0,0}

\node[text=drawColor,anchor=base,inner sep=0pt, outer sep=0pt, scale=  0.90] at (323.40, 44.40) {\textsc{DPMC} + \texttt{bklm16} time (s)};
\end{scope}
\begin{scope}
\path[clip] (  0.00,  0.00) rectangle (411.94,224.04);
\definecolor{drawColor}{RGB}{0,0,0}

\node[text=drawColor,rotate= 90.00,anchor=base,inner sep=0pt, outer sep=0pt, scale=  0.90] at (222.64,141.47) {\textsc{DPMC} + \texttt{bklm16++} time (s)};
\end{scope}
\begin{scope}
\path[clip] (  0.00,  0.00) rectangle (411.94,224.04);
\definecolor{fillColor}{RGB}{255,255,255}

\path[fill=fillColor] (111.15,  0.00) rectangle (300.79, 37.91);
\end{scope}
\begin{scope}
\path[clip] (  0.00,  0.00) rectangle (411.94,224.04);
\definecolor{fillColor}{RGB}{255,255,255}

\path[fill=fillColor] (120.15, 18.95) rectangle (134.61, 33.41);
\end{scope}
\begin{scope}
\path[clip] (  0.00,  0.00) rectangle (411.94,224.04);
\definecolor{fillColor}{RGB}{27,158,119}

\path[fill=fillColor,fill opacity=0.50] (127.38, 26.18) circle (  1.43);
\end{scope}
\begin{scope}
\path[clip] (  0.00,  0.00) rectangle (411.94,224.04);
\definecolor{fillColor}{RGB}{255,255,255}

\path[fill=fillColor] (120.15,  4.50) rectangle (134.61, 18.95);
\end{scope}
\begin{scope}
\path[clip] (  0.00,  0.00) rectangle (411.94,224.04);
\definecolor{fillColor}{RGB}{217,95,2}

\path[fill=fillColor,fill opacity=0.50] (127.38, 13.95) --
	(129.30, 10.62) --
	(125.46, 10.62) --
	cycle;
\end{scope}
\begin{scope}
\path[clip] (  0.00,  0.00) rectangle (411.94,224.04);
\definecolor{fillColor}{RGB}{255,255,255}

\path[fill=fillColor] (166.60, 18.95) rectangle (181.05, 33.41);
\end{scope}
\begin{scope}
\path[clip] (  0.00,  0.00) rectangle (411.94,224.04);
\definecolor{fillColor}{RGB}{117,112,179}

\path[fill=fillColor,fill opacity=0.50] (172.40, 24.75) --
	(175.25, 24.75) --
	(175.25, 27.61) --
	(172.40, 27.61) --
	cycle;
\end{scope}
\begin{scope}
\path[clip] (  0.00,  0.00) rectangle (411.94,224.04);
\definecolor{fillColor}{RGB}{255,255,255}

\path[fill=fillColor] (166.60,  4.50) rectangle (181.05, 18.95);
\end{scope}
\begin{scope}
\path[clip] (  0.00,  0.00) rectangle (411.94,224.04);
\definecolor{drawColor}{RGB}{231,41,138}

\path[draw=drawColor,draw opacity=0.50,line width= 0.4pt,line join=round,line cap=round] (171.81, 11.73) -- (175.84, 11.73);

\path[draw=drawColor,draw opacity=0.50,line width= 0.4pt,line join=round,line cap=round] (173.83,  9.71) -- (173.83, 13.74);
\end{scope}
\begin{scope}
\path[clip] (  0.00,  0.00) rectangle (411.94,224.04);
\definecolor{fillColor}{RGB}{255,255,255}

\path[fill=fillColor] (227.91, 18.95) rectangle (242.36, 33.41);
\end{scope}
\begin{scope}
\path[clip] (  0.00,  0.00) rectangle (411.94,224.04);
\definecolor{drawColor}{RGB}{102,166,30}

\path[draw=drawColor,draw opacity=0.50,line width= 0.4pt,line join=round,line cap=round] (233.71, 24.75) rectangle (236.56, 27.61);

\path[draw=drawColor,draw opacity=0.50,line width= 0.4pt,line join=round,line cap=round] (233.71, 24.75) -- (236.56, 27.61);

\path[draw=drawColor,draw opacity=0.50,line width= 0.4pt,line join=round,line cap=round] (233.71, 27.61) -- (236.56, 24.75);
\end{scope}
\begin{scope}
\path[clip] (  0.00,  0.00) rectangle (411.94,224.04);
\definecolor{fillColor}{RGB}{255,255,255}

\path[fill=fillColor] (227.91,  4.50) rectangle (242.36, 18.95);
\end{scope}
\begin{scope}
\path[clip] (  0.00,  0.00) rectangle (411.94,224.04);
\definecolor{drawColor}{RGB}{230,171,2}

\path[draw=drawColor,draw opacity=0.50,line width= 0.4pt,line join=round,line cap=round] (233.71, 10.30) -- (236.56, 13.15);

\path[draw=drawColor,draw opacity=0.50,line width= 0.4pt,line join=round,line cap=round] (233.71, 13.15) -- (236.56, 10.30);

\path[draw=drawColor,draw opacity=0.50,line width= 0.4pt,line join=round,line cap=round] (233.11, 11.73) -- (237.15, 11.73);

\path[draw=drawColor,draw opacity=0.50,line width= 0.4pt,line join=round,line cap=round] (235.13,  9.71) -- (235.13, 13.74);
\end{scope}
\begin{scope}
\path[clip] (  0.00,  0.00) rectangle (411.94,224.04);
\definecolor{drawColor}{RGB}{0,0,0}

\node[text=drawColor,anchor=base west,inner sep=0pt, outer sep=0pt, scale=  0.72] at (139.11, 23.70) {DQMR};
\end{scope}
\begin{scope}
\path[clip] (  0.00,  0.00) rectangle (411.94,224.04);
\definecolor{drawColor}{RGB}{0,0,0}

\node[text=drawColor,anchor=base west,inner sep=0pt, outer sep=0pt, scale=  0.72] at (139.11,  9.25) {Grid};
\end{scope}
\begin{scope}
\path[clip] (  0.00,  0.00) rectangle (411.94,224.04);
\definecolor{drawColor}{RGB}{0,0,0}

\node[text=drawColor,anchor=base west,inner sep=0pt, outer sep=0pt, scale=  0.72] at (185.55, 23.70) {Mastermind};
\end{scope}
\begin{scope}
\path[clip] (  0.00,  0.00) rectangle (411.94,224.04);
\definecolor{drawColor}{RGB}{0,0,0}

\node[text=drawColor,anchor=base west,inner sep=0pt, outer sep=0pt, scale=  0.72] at (185.55,  9.25) {Non-binary};
\end{scope}
\begin{scope}
\path[clip] (  0.00,  0.00) rectangle (411.94,224.04);
\definecolor{drawColor}{RGB}{0,0,0}

\node[text=drawColor,anchor=base west,inner sep=0pt, outer sep=0pt, scale=  0.72] at (246.86, 23.70) {Other binary};
\end{scope}
\begin{scope}
\path[clip] (  0.00,  0.00) rectangle (411.94,224.04);
\definecolor{drawColor}{RGB}{0,0,0}

\node[text=drawColor,anchor=base west,inner sep=0pt, outer sep=0pt, scale=  0.72] at (246.86,  9.25) {Random Blocks};
\end{scope}
\end{tikzpicture}

\end{document}
%
  \caption{An instance-by-instance comparison between
    $\textsf{ADDMC} + \texttt{cw}$ and the best overall combination of algorithm
    and encoding ($\textsf{Ace} + \texttt{cd06}$, on the left) as well as the
    second-best encoding for \textsf{ADDMC} (\texttt{sbk05}, on the
    right).}\label{fig:scatter}
\end{figure}

\Cref{fig:cumulative} shows that \texttt{cd05}
\citep{DBLP:conf/ijcai/ChaviraD05} and \texttt{cd06}
\citep{DBLP:conf/sat/ChaviraD06} (when run with \textsf{Ace}) are in the lead,
while \textsf{ADDMC} \citep{DBLP:conf/aaai/DudekPV20} significantly
underperforms when combined with any of the previous encodings. Our encoding
\texttt{cw} significantly improves the performance of \textsf{ADDMC}, making
$\textsf{ADDMC}+\texttt{cw}$ comparable to $\textsf{Ace}+\texttt{d02}$,
$\textsf{c2d}+\texttt{bklm16}$, and $\textsf{Cachet}+\texttt{sbk05}$.
Furthermore, \cref{tbl:tallies} shows that, while $\textsf{Ace}+\texttt{cd06}$
managed to solve the most instances, $\textsf{ADDMC}+\texttt{cw}$ was the
best-performing algorithm-encoding combination on the largest number of
instances. The scatter plot on the left-hand side of \cref{fig:scatter} add to
this by showing that \texttt{cw} is particularly promising on Grid networks and
tackles all DQMR instances in less than a second. The scatter plot on the
right-hand side of \cref{fig:scatter} shows that \texttt{cw} is better than
\texttt{sbk05} \citep{DBLP:conf/aaai/SangBK05} (i.e., the second-best encoding
for \textsf{ADDMC}) on the majority of instances. Seeing how, e.g., DQMR
instances are trivial for $\textsf{ADDMC}+\texttt{cw}$ but hard for
$\textsf{Ace}+\texttt{cd06}$, and vice versa for Mastermind instances, we
conclude that the best-performing algorithm-encoding combination depends
significantly on (as-of-yet unknown) properties of the Bayesian networks.

\begin{table}[t]
  \centering
  \begin{tabular}{m{0.4\linewidth} c c}
    \toprule
    Encoding(s) & Variables & Clauses/ADDs \\
    \midrule
    \texttt{bklm16}, \texttt{cd05}, \texttt{cd06}, \texttt{sbk05} & $O(nv^{d+1})$ & $O(nv^{d+1})$ \\
    \texttt{cw} & $O(nv)$ & $O(nv^2)$ \\
    \texttt{d02} & $O(nv^{d+1})$ & $O(ndv^{d+1})$ \\
    \bottomrule
  \end{tabular}
  \caption{Asymptotic upper bounds on the numbers of variables and clauses/ADDs
    for each encoding.}\label{tbl:asymptotes}
\end{table}

We can explain what makes \textsf{ADDMC} \citep{DBLP:conf/aaai/DudekPV20} run
significantly faster with \texttt{cw} than with any other encoding by
considering asymptotic upper bounds on the numbers of variables and ADDs based
on the size and structure of the Bayesian network. Let $n = |\mathcal{V}|$ be
the number of vertices in the Bayesian network, $d = \max_{X \in \mathcal{V}}
|\mathrm{pa}(X)|$ the maximum in-degree (i.e., the number of parents), and $v =
\max_{X \in \mathcal{V}} |\Imm X|$ the maximum number of values per variable.
\Cref{tbl:asymptotes} shows how \texttt{cw} has fewer variables and fewer ADDs
than any other encoding. We conjecture that it is primarily the reduced number
of variables that makes the \textsf{ADDMC} variable ordering heuristics much
more effective. Note that these are upper bounds and most encodings (including
\texttt{cw}) can be smaller in certain situations (e.g., with binary random
variables or when a CPT has repeating probabilities). We equate clauses and ADDs
(more specifically, factors of the function $\phi$ from \cref{alg:encoding})
here because \textsf{ADDMC} interprets each clause of any WMC encoding as a
multiplicative factor of the ADD that represents the entire WMC instance
\citep{DBLP:conf/aaai/DudekPV20}. For literal-weight encodings, each weight is
also a factor, but that does not affect our asymptotic bounds.

\section{Conclusions and Future Work}

WMC was originally motivated by an appeal to the success of SAT solvers in
efficiently tackling an $\NP{}$-complete problem
\citep{DBLP:conf/aaai/SangBK05}. \textsf{ADDMC} does not rely on SAT-based
algorithmic techniques \citep{DBLP:conf/aaai/DudekPV20}, and our proposed format
diverges even more from the DIMACS CNF format for Boolean formulas. To what
extent are SAT-based methods still applicable? The answer depends significantly
on the problem domain. For Bayesian networks, the rules describing that each
random variable can only be associated with exactly one value were still encoded
as clauses. As has been noted previously \citep{DBLP:conf/sat/ChaviraD06}, rows
in CPTs with probabilities equal to zero or one can be represented as clauses as
well. Therefore, our work can be seen as proposing a middle ground between \mc{}
and probabilistic inference.

While we chose \textsf{ADDMC} \citep{DBLP:conf/aaai/DudekPV20} as the WMC
algorithm and Bayesian networks as a canonical example of a probabilistic
inference task, these are only examples meant to illustrate the broader idea
that choosing a more expressive representation of weights can outperform
increasing the size of the problem to keep the weights simple. Indeed, in this
work, we have provided a new theoretical perspective on the expressive power of
WMC and illustrated the empirical benefits of that perspective. Perhaps the same
idea could be adapted to other inference problem domains such as probabilistic
programs
\citep{DBLP:journals/tplp/FierensBRSGTJR15,DBLP:journals/corr/abs-2005-09089} as
well as to search-based solvers such as \textsf{Cachet}
\citep{DBLP:conf/sat/SangBBKP04} and \textsf{DPMC} ---an extension to
\textsf{ADDMC} that adds support for computations based on tensors (rather than
ADDs) and planning based on tree decompositions \citep{DBLP:conf/cp/DudekPV20}.
