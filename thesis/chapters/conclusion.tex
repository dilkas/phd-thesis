% >= 5 pages
\chapter{Conclusion}\label{chapter:conclusion}

In \cref{sec:contributions} we review the contributions of this thesis and in
\cref{sec:future} we provide a perspective on how future work could develop,
either directly or indirectly building on the results of our work.

\section{Contributions}\label{sec:contributions}
% ================================= CONTRIBUTIONS (reiterating their importance) =================================
% how have I improved the state of the art?
% what do we understand better as a consequence of my work?
% recurring themes?
% claims?
% what broad areas of research are affected by my work? how might that effect propagate in the future?

% Recall that WMC is a fundamental computational task underlying many formalisms
% and applications.

The contributions of this thesis can be divided into two parts:
\begin{itemize}
  \item empirically-motivated contributions that make something new possible or
        something that already exists more efficient
  \item and (conceptual, theoretical, or experimental) contributions that help
        us understand something more fully or in a new way.
\end{itemize}

% ================================= EMPIRICAL STATE OF THE ART =================================

On the empirical front, most of our contributions focus on the efficiency and
tractability of the propositional and first-order variants of WMC\@. In
\cref{chapter:wmc1,chapter:wmc2}, we show how the efficiency of WMC can be
improved by generalising weights from their standard definition based on
literals to one capable of representing a richer subset of all possible
pseudo-Boolean functions. In \cref{chapter:wfomc}, we extend the capabilities of
\textsc{ForcLift} \citep{DBLP:conf/ijcai/BroeckTMDR11} so that it is able to
solve more instances in a lifted manner, e.g., instances with injective
mappings. The empirical contributions of
\cref{chapter:randomlps,chapter:comparison} are about making new things
possible, i.e., introducing novel tools and methods. In
\cref{chapter:randomlps}, we developed a constraint model for (probabilistic)
logic programs that can be used to generate random programs or enumerate all
small programs under some given constraints. The constraints include various
notions of size, the structure/complexity of a clause, and the independence of
random variables. Finally, in \cref{chapter:comparison}, we present a way to
generate propositional formulas in CNF with varying primal treewidth. As
treewidth is a well-known parameter commonly used to describe parameterised
complexity results \citep{DBLP:series/txcs/DowneyF13}, the same model (or a
variation thereof) can be used in experimental studies of many other logic-based
problems as well.

% ================================= UNDERSTANDING =================================

The experimental work in these last two chapters, i.e.,
\cref{chapter:randomlps,chapter:comparison}, also contains important
observations about WMC and probabilistic inference algorithms. First,
\cref{chapter:randomlps} demonstrates remarkable similarities among ProbLog
inference algorithms. This observation suggests that the bottleneck of ProbLog
inference (at least across our random instances) might be related to logic
programming more than WMC\@. Second, \cref{chapter:comparison} reveals, among
other things, that WMC algorithms based on algebraic decision diagrams (ADDs)
and dynamic programming scale worse with primal treewidth and work better with
instances that have fewer clauses (i.e., lower density) compared to other
algorithms. Understanding such differences among algorithms is important in the
development of new algorithms, algorithm portfolios, and hybrid approaches to
WMC\@. Back in \cref{chapter:wmc1}, we show how WMC can be seen as the problem
of computing the value of a measure on some element of a Boolean algebra. This
insight leads us to consider generalised weight functions that express, e.g.,
conditional probabilities more succinctly and can lead to improved probabilistic
inference speed for Bayesian networks. In \cref{chapter:wmc2}, we continue the
work on generalising WMC and formally define the generalisation as
pseudo-Boolean projection (PBP). Moreover, we show that previous work on WMC
encodings is not in vain and the benefits can (in most cases) be transferred to
PBP\@. Lastly, \cref{chapter:wfomc} contains two important lessons. First,
`circuits' with cycles can be more expressive than their acyclic predecessors.
Second, first-order model counting (and first-order knowledge compilation in
particular) can discover the definitions of recursive functions (including
recurrence relations) that capture the model count of a given sentence.

In summary, we
\begin{itemize}
  \item introduced new foundations for WMC based on measures on Boolean
        algebras,
  \item generalised WMC to PBP,
  \item introduced new encoding schemes and encoding transformation algorithms,
  \item introduced \textsc{Crane}, i.e., a more powerful version of
        \textsc{ForcLift} that works with graphs rather than circuits,
  \item and provided algorithms for random instance generation.
\end{itemize}

%% \section{Discussion}

%% While WMC and its counterpart for first-order logic WFOMC both take
%% logic-based input and compute a sum of products, algorithmically they are
%% quite different. The first difference is in the size of input: most WFOMC
%% benchmarks have at most three variables whereas most WMC instances have
%% thousands of variables. This difference motivates the definition of
%% liftability in lieu of tractability. Let us take DPMC
%% \citep{DBLP:conf/cp/DudekPV20} and \textsc{ForcLift} as representative
%% examples of algorithms for each problem. The way in which an algorithm solves
%% an instance can be explained as a sequence of operations: on ADDs in the
%% former case and on (a variation of) formulas in the latter. However, due to
%% the difference in input size, WMC algorithms are expected to perform a much
%% larger number of such operations. Hence, the efficiency of each individual
%% operation is more important in the case of WMC. Here, optimising not just the
%% asymptotic complexity of each operation but also the data structures in use
%% can yield significant performance gains. Finally, there is an important
%% difference regarding complexity. As is the case with all complete exact
%% solutions to \NP-hard problems, WMC algorithms have an exponential worst-case
%% complexity. A typical algorithm attempts to do much better on most realistic
%% instances, but provides no guarantees. In contrast, one can often express the
%% running time of a WFOMC algorithm on an instance $\mathscr{I}$ as
%% $\Theta(p(n_1, \dots, n_k))$, where $p$ is a polynomial, and
%% $\{\,n_i\,\}_{i=1}^k$ are the sizes of the domains in $\mathscr{I}$.

%% Algorithmically, W(FO)MC is interesting because of its connections to classic
%% decision problems such as SAT, counting problems like \mc{}, and other
%% function problems such as polynomial evaluation. The algorithms benefit from
%% many disparate techniques such as dynamic programming
%% \citep{DBLP:conf/aaai/DudekPV20,DBLP:conf/cp/DudekPV20}, knowledge
%% compilation \citep{DBLP:conf/ecai/Darwiche04,DBLP:conf/ijcai/BroeckTMDR11},
%% search \citep{DBLP:conf/sat/SangBBKP04,DBLP:conf/ijcai/BroeckTMDR11}, tree
%% decompositions \citep{DBLP:conf/cp/DudekPV20,DBLP:conf/cp/KorhonenJ21}, and
%% close collaboration with SAT algorithms
%% \citep{DBLP:conf/ecai/Darwiche04,DBLP:conf/ijcai/LagniezM17,DBLP:conf/sat/SangBBKP04}.

%% We hope that the contributions in this thesis will...

%% \paragraph*{Impact: What broad areas of research are affected by my work?}
%% \begin{itemize}
%% \item theory and algorithms: counting algorithms, arithmetic complexity
%% \item AI: PGMs, statistical relational learning
%% \end{itemize}

\section{Future Directions}\label{sec:future}
%% approximate DPMC: cite the APRICODD paper
%% \citep{DBLP:conf/nips/St-AubinHB00}. Also, if two sinks have very similar
%% values, an approximation could merge them

In this section, we present a broad overview of how our contributions and the
questions raised by this work could be taken forward and influence key areas of
research in computer science, artificial intelligence, and mathematics.

\subsection{Algorithms and Applications}

% You could mention the use of portfolio approaches to make SAT extremely fast
% \citep{DBLP:journals/jair/XuHHL08} (other domains
% \citep{DBLP:conf/lion/KotthoffMS16}), and whether something similar could be
% tried out with your work.

In this thesis, we contributed to the development and applicability of three WMC
algorithms: \textsc{ADDMC} \citep{DBLP:conf/aaai/DudekPV20}, \textsc{DPMC}
\citep{DBLP:conf/cp/DudekPV20}, and \textsc{ForcLift}
\citep{DBLP:conf/ijcai/BroeckTMDR11}. The first two are propositional WMC
algorithms based on ADDs whereas \textsc{ForcLift} is a WMC algorithm for
first-order logic based on knowledge compilation. While in this work we focused
exclusively on exact algorithms, all of them could be adapted to approximate
instead. An approximation technique called \emph{lifted relax, compensate and
  then recover} is already part of \textsc{ForcLift}
\citep{DBLP:conf/uai/BroeckCD12}, so it would only need to be adapted to the
generalised setting of \textsc{Crane}. Likewise, approximate computations using
ADDs have already been studied \citep{DBLP:conf/nips/St-AubinHB00}, so, e.g.,
\textsc{DPMC} could be extended to approximate as well.

Most weighted first-order model counting (WFOMC) algorithms try to solve each
instance in a lifted manner (i.e., run in polynomial time with respect to the
sizes of the domains involved) and fail if unsuccessful. \textsc{ForcLift} is an
exception as it supports using a (propositional) WMC algorithm for parts of the
problem that cannot be solved by other compilation rules. Can this transition to
WMC be implemented more efficiently, i.e., without fully grounding the instance?
Is the WFOMC algorithm better off constructing its own exponential-time solution
instead of relying on a WMC algorithm (and is that even possible)? The only way
WFOMC can become the standard approach to probabilistic inference in statistical
relational models is by being able to gracefully handle all instances, even if
it means abandoning efficiency guarantees.

Finally, ample opportunities remain to improve WMC encodings that already exist
as well as connect WMC to new problem domains. In particular, we showed how PBP
encodings of Bayesian networks are much smaller than the equivalent WMC
encodings and can be handled more efficiently by a WMC algorithm. Designing PBP
encodings for other applications of WMC and to new problem domains could be
similarly beneficial. Moreover, back in \cref{chapter:introduction} we compared
WMC to a range of other computational problems that ask to compute a sum of
products. Establishing efficient reductions among these problems could yield new
fixed-parameter tractable algorithms and/or improvements to the empirical state
of the art. Similarly, adapting a WMC algorithm to a semiring other than
$(\mathbb{R}_{\ge 0}, +, \cdot)$ could yield improvements to some of the related
problems outlined by \citet{DBLP:journals/japll/KimmigBR17}.

%% The two theoretical directions for future work we discuss here are on the arithmetic complexity of WMC/WFOMC and the various notions of completeness of WFOMC algorithms.
%% \begin{itemize}
%% \item arithmetic complexity
%%   \begin{itemize}
%%   \item prove that transitivity (sequence A006905 in the on-line encyclopedia of integer sequences\footnote{\url{https://oeis.org/A006905}}) cannot be computed in polynomial time.
%%   \end{itemize}
%% \item completeness: if an algebraic solution exists, are we able to find it? As we continue to expand the capabilities of WFOMC, the answer clearly is `not yet'. Two parts to this question:
%%   \begin{itemize}
%%   \item can a suitable FCG be constructed from the given node types?
%%   \item can the right combination of compilation rules lead to the construction of such an FCG?
%%   \item Another important avenue for future research is on the expressive power of various logics and languages. Mention $f(n) = f(f(n-1))$. A different kind of completeness: what kind of sequences and integer functions can be encoded in FOL? What about other kinds of logics and languages? \citep{Benthem2021INTERLEAVINGLA}
%%   \end{itemize}
%% \end{itemize}

\subsection{Computational Complexity}

Note that the execution of both WMC and WFOMC algorithms can be divided into two
parts:
\begin{itemize}
  \item looking for a \emph{solution} (i.e., an arithmetic circuit/expression
        that computes the required sum of products) and
  \item performing the numerical computations that produce the final answer.
\end{itemize}
(The two are typically much more intertwined in the case of propositional WMC.)
With this dichotomy in mind, one could ask: are the algorithms finding optimal
solutions? How much of the total running time depends on the complexity of the
solution, and how much on the algorithmic methods for finding one? Answers to
these questions would highlight the weaknesses of state-of-the-art algorithms
and direct the efforts of future research towards addressing these weaknesses.

On a more theoretical level, single-domain (W)FOMC problems compute sequences,
many of which are well-known to mathematicians. Since there is significant
interest in computing such sequences efficiently, the existence of many such
sequences with no efficient formulas suggests that a tractable solution might
not exist. However, we have no proof of that, i.e., no arithmetic circuit lower
bounds for sequences that have been known for decades and are easy to describe
in natural language and in logic. So far, the most notable hardness result
states that there exists a sentence in first-order logic with three variables
for which FOMC is $\#\P_1$-complete \citep{DBLP:conf/pods/BeameBGS15}. Having
similar hardness results for sentences that are both simple and practical would
be a significant advancement to the field.

\subsection{Random Instances}

In this thesis, we introduce two ways to generate random instances: one for
(probabilistic) logic programs and one for propositional formulas in CNF that
are then turned into WMC instances. As we provide the very first attempts at
testing WMC algorithms on random data, many opportunities for improvements and
future work remain.

An interesting opportunity to connect our work on random
logic programs in \cref{chapter:randomlps} and on (W)FOMC in
\cref{chapter:wfomc} is by adapting the constraint model to generate (W)FOMC
instances instead of logic programs. This way one could systematically search
for interesting instances that, e.g.,
\begin{itemize}
  \item reveal differences in the runtime complexity of various algorithms or
  \item demonstrate a gap between the performance of state-of-the-art WFOMC
        algorithms and formulas constructed by hand.
\end{itemize}

There is also ample opportunity for theoretical contributions. For instance, one
explanation for the surprising experimental results of \cref{chapter:randomlps}
is that all of the generated instances yielded easy WMC problems, and the
computational bottleneck was in the handling of the logic program before WMC\@.
We can state this idea as a (somewhat informal) conjecture.

\begin{conjecture}
  With high probability, the WMC instance that results from a random
  probabilistic logic program generated by the constraint model in
  \cref{chapter:randomlps} is tractable for some WMC algorithm.
\end{conjecture}

%% random CNFs with bounds on primal treewidth (what makes it tricky in my case
%% is that one random decision influences/conditions all subsequent decisions,
%% unlike in the case of Erd\H{o}s-R\'enyi random graphs)

\subsection{Artificial Intelligence and Combinatorics}

Our hope for the broader field of artificial intelligence is that---similarly to
the resurgence of symbolic artificial intelligence amidst deep neural networks
\citep{garnelo2019reconciling}---the field shifts some of its focus from numbers
and probabilities to structural concepts such as functions and relations. Once a
solution to, e.g., a WFOMC problem is formulated as a function $f$ rather than
the evaluation of $f$ on some particular input values, richer ways of reasoning
become available. For example, instead of asking whether a probability of some
event is above/below some threshold in a particular situation, one could ask for
conditions on the input values that are necessary for the probability to be
sufficiently high/low. Such reasoning capabilities have clear benefits to the
robustness of artificial agents and to explainability---another rapidly emerging
area of research
\citep{DBLP:journals/corr/abs-1909-03012,DBLP:journals/fdata/BelleP21,DBLP:journals/corr/abs-2202-10335}.

Finally, for the benefit of both artificial intelligence and combinatorics, we
would like to reiterate and expand on the notion of automatic enumerative
combinatorialist by \citet{DBLP:conf/ilp/BarvinekB0ZK21}. Perhaps (W)FOMC can
mature into an easy-to-use tool that can compute any function expressible in
first-order logic, in many cases providing a simple solution via a combination
of recursive functions. Similarly to how a constraint programmer describes the
constraints and asks the solver for a solution, a combinatorialist could
describe what needs to be counted in a logic-based format and receive recursive
or asymptotic solutions, generating functions, etc.

% Instead there has been some recent work on making connections between tractable
% circuits and explainability, and perhaps you can articulate something based on
% that.
%   \begin{itemize}
%     \item use of tractable circuits in explainability and verification of neural
%           networks \citep{DBLP:conf/pods/Darwiche20}
%     \item some of the tractable circuits are essentially WMC encodings
%           \citep{DBLP:conf/nips/ShenCD16}
%   \end{itemize}

%% \begin{itemize}
%% \item Making it more powerful
%%   \begin{itemize}
%%   \item new compilation rules?
%%   \item powers of negative one?
%%   \item log factors?
%%   \item solving recurrence relations in closed form?
%%   \item division?
%%   \end{itemize}
%% \item learning FCGs
%% \item Generalisations: sometimes restrictions become unnecessary and cause...
%% \item Reasoning about functions rather than numbers
%%   \begin{itemize}
%%   \item asymptotic solutions
%%   \item Potential application for expressing WFOMC as algebraic expressions over domain sizes. Q: Under what conditions over domain sizes would $\Pr(x) \ge \Pr(y)$ (or same for expectations)? We can solve this as an inequality. This could aid in probabilistic decision making, XAI, etc.
%%   \end{itemize}
%% \end{itemize}

%% Many open questions remain about the capabilities of logic-based methods for counting and performing sum-of-products computations, particularly in the context of first-order logic. The answer to a WFOMC instance depends on three things: the formula, the weights, and the domain sizes. By fixing the first two, the instance becomes a logical representation of a function $\mathbb{N}_0^n \to \mathbb{R}_{\ge 0}$, where $n$ is the number of domains. In particular, an unweighed formula $\phi$ that depends on only one domain $\Delta$ represents an integer sequence that we get by computing the model count of $\phi$ across all possible cardinalities of $\Delta$. This observation raises important open questions with the potential to improve WFOMC algorithms, contribute to other counting problems and find new efficiently-computable formulas to sequences of interest.

%% First, how complete is first-order logic in its ability to describe such functions? Can we identify conditions (e.g., monotonicity) that must be satisfied by a function for it to be representable as an instance of WFOMC? Can we find examples of simple sequences that are provably unrepresentable? Would a different kind of logic (e.g., second-order or modal) be more complete in this way? Answering these questions could help identify new areas of application of WFOMC and ensure that instances that are solvable in theory can be solved in practice as well.

%% Second, instead of finding (a more complex type of) arithmetic circuits that compute functions from their logical descriptions, is it possible to go in the other direction? That is, can we identify logical gadgets for all algebraic operations that could then be combined to form a WFOMC representation of a given function? This way, WFOMC could be used to automatically reformulate, e.g., functions that use an exponential amount of recursive calls to more tractable expressions.

%% Third, if a WFOMC algorithm identifies a recursive function (or a recurrence relation) as part of the solution, can we use one of the many recurrence-relation-solving techniques to replace recursive computations with a closed-form solution? More generally, there is ample opportunity to expand the set of algorithmic techniques and algebraic constructions used in WFOMC to one that is more complete (i.e., able to construct tractable solutions) and efficient. For instance, the use of negation is currently limited to Skolemization (i.e., removal of existential quantification), but one can construct instances whose most efficient solutions feature powers of negative one despite the formula having no existential quantification. Similarly to \cref{chapter:wfomc}, this could further extend the set of instances that can be handled in a lifted (i.e., tractable with respect to domain sizes) manner and improve the complexity of solutions that are already tractable.
