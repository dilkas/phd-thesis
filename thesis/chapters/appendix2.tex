\chapter{Proofs}

\begin{customthm}{1}
  The function $\mu_\nu$ is a measure.
\end{customthm}
\begin{proof}
  Note that $\mu_\nu(\bot) = 0$ since there are no atoms below $\bot$. Let $a, b
  \in 2^{2^{U}}$ be such that $a \land b = \bot$. By elementary properties of
  Boolean algebras, all atoms below $a \lor b$ are either below $a$ or below
  $b$. Moreover, none of them can be below both $a$ and $b$ because then they
  would have to be below $a \land b = \bot$. Thus
  \begin{align*}
    \mu_\nu(a \lor b) &= \sum_{\{u\} \le a \lor b} \nu(u) = \sum_{\{u\} \le a} \nu(u) + \sum_{\{u\} \le b} \nu(u) \\
                      &= \mu_\nu(a) + \mu_\nu(b)
  \end{align*}
  as required.
\end{proof}

\begin{customthm}{3}
  For any set $U$ and measure $\mu\colon 2^{2^U} \to \mathbb{R}_{\ge 0}$, there
  exists a set $V \supseteq U$, a factorable measure $\mu'\colon 2^{2^V} \to
  \mathbb{R}_{\ge 0}$, and a formula $f \in 2^{2^V}$ such that $\mu(x) = \mu'(x
  \land f)$ for all formulas $x \in 2^{2^U}$.
\end{customthm}
\begin{proof}
  Let $V = U \cup \{ f_m \mid m \in 2^U \}$, and $f = \bigwedge_{m \in 2^U} \{ m
  \} \leftrightarrow f_m$. We define weight function $\nu\colon 2^V \to
  \mathbb{R}_{\ge 0}$ as $\nu = \prod_{v \in V} \nu_v$, where $\nu_v(\{v\}) =
  \mu(\{m\})$ if $v = f_m$ for some $m \in 2^U$ and $\nu_v(x) = 1$ for all other
  $v \in V$ and $x \in 2^{\{v\}}$. Let $\mu'\colon 2^{2^V} \to \mathbb{R}_{\ge
    0}$ be the measure induced by $\nu$. It is enough to show that $\mu$ and $x
  \mapsto \mu'(x \land f)$ agree on the atoms in $2^{2^U}$. For any $\{ a \} \in
  2^{2^U}$,
  \begin{align*}
    \mu'(\{ a \} \land f) &= \sum_{\{ x \} \le \{ a \} \land f} \nu(x) = \nu(a \cup \{ f_a \}) \\
                          &= \nu_{f_a}(\{ f_a \}) = \mu(\{ a \})
  \end{align*}
  as required.
\end{proof}

\begin{customlemma}{1} \label{lemma:cpt2}
  Let $X \in \mathcal{V}$ be a random variable with parents $\mathrm{pa}(X) = \{ Y_1,
  \dots, Y_n \}$. Then $\mathrm{CPT}_X\colon 2^{\mathcal{E}^*(X)} \to
  \mathbb{R}_{\ge 0}$ is such that for any $x \in \im X$ and $(y_1, \dots, y_n)
  \in \prod_{i=1}^n \im Y_i$,
  \[
    \mathrm{CPT}_X (T) = \Pr(X = x \mid Y_1 = y_1, \dots, Y_n = y_n),
  \]
  where $T = \{ \lambda_{X=x} \} \cup \{ \lambda_{Y_i=y_i} \mid i = 1, \dots, n
  \}$.
\end{customlemma}
\begin{proof}
  If $X$ is binary, then $\mathrm{CPT}_X$ is a sum of $2\prod_{i=1}^n |\im
  Y_i|$ terms, one for each possible assignment of values to variables $X, Y_1,
  \dots, Y_n$. Exactly one of these terms is nonzero when applied to $T$, and
  it is equal to $\Pr(X = x \mid Y_1 = y_1, \dots, Y_n = y_n)$ by definition.

  If $X$ is not binary, then $\left( \sum_{i=1}^m [\lambda_{X = x_i}] \right)(T) = 1$, and
  \[
  \left( \prod_{i=1}^m \prod_{j=i+1}^m (\overline{[\lambda_{X = x_i}]} + \overline{[\lambda_{X = x_j}]}) \right)(T) = 1,
  \]
  so $\mathrm{CPT}_X(T) = \Pr(X = x \mid Y_1 = y_1, \dots, Y_n = y_n)$ by a similar argument as before.
\end{proof}

\begin{customlemma}{2} \label{lemma:full_distribution2}
  Let $\mathcal{V} = \{X_1, \dots, X_n\}$. Then
  \[
    \phi(T) =
    \begin{cases}
      \Pr(x_1, \dots, x_n) &
      \begin{aligned}
        &\text{if } T = \{ \lambda_{X_i=x_i} \}_{i = 1}^n \text{ for} \\
        &\text{some } \textstyle (x_i)_{i=1}^n \in \prod_{i=1}^n \im X_i
      \end{aligned} \\
      0 & \text{otherwise,}
    \end{cases}
  \]
  for all $T \in 2^U$.
\end{customlemma}
\begin{proof}
  If $T = \{ \lambda_{X=v_X} \mid X \in \mathcal{V} \}$ for some $(v_X)_{X
    \in \mathcal{V}} \in \prod_{X \in \mathcal{V}} \im X$, then
  \begin{align*}
    \phi(T) &= \prod_{X \in \mathcal{V}} \Pr \left( X=v_X \;\middle|\; \bigwedge_{Y \in \mathrm{pa}(X)} Y=v_Y \right) \\
            &= \Pr \left( \bigwedge_{X \in \mathcal{V}} X=v_X \right)
  \end{align*}
  by \cref{lemma:cpt2} and the definition of a Bayesian network. Otherwise there
  must be some non-binary random variable $X \in \mathcal{V}$ such that
  $|\mathcal{E}(X) \cap T| \ne 1$. If $\mathcal{E}(X) \cap T = \emptyset$, then
  $\left( \sum_{i=1}^m [\lambda_{X = x_i}] \right)(T) = 0$, and so
  $\mathrm{CPT}_X(T) = 0$, and $\phi(T) = 0$. If $|\mathcal{E}(X) \cap T| > 1$,
  then we must have two different values $x_1, x_2 \in \im X$ such that
  $\{\lambda_{X=x_1}, \lambda_{X=x_2} \} \subseteq T$ which means that
  $(\overline{[\lambda_{X=x_1}]} + \overline{[\lambda_{X=x_2}]})(T) = 0$, and
  so, again, $\mathrm{CPT}_X(T) = 0$, and $\phi(T) = 0$.
\end{proof}

\begin{customthm}{4}
  For any $X \in \mathcal{V}$ and $x \in \im X$,
  \[
    (\exists_U(\phi \cdot [\lambda_{X=x}]))(\emptyset) = \Pr(X = x).
  \]
\end{customthm}
\begin{proof}
  Let $\mathcal{V} = \{ X, Y_1, \dots, Y_n \}$. Then
  \begin{align*}
    (\exists_U (\phi \cdot [\lambda_{X=x}]))(\emptyset) &= \sum_{T \in 2^U} (\phi \cdot [\lambda_{X=x}])(T) \\
                                                        &= \sum_{\lambda_{X=x} \in T \in 2^U} \phi(T) \\
                                                        &= \sum_{\lambda_{X=x} \in T \in 2^U} \left( \prod_{Y \in \mathcal{V}} \mathrm{CPT}_Y \right)(T) \\
                                                        &= \sum_{(y_i)_{i=1}^n \in \prod_{i=1}^n \im Y_i} \Pr(x, y_1, \dots, y_n) \\
                                                        &= \Pr(X = x)
  \end{align*}
  by:
  \begin{itemize}
  \item the proof of Theorem~1 by \citet{DBLP:conf/aaai/DudekPV20};
  \item if $\lambda_{X=x} \not\in T \in 2^U$, then
    \[
    (\phi \cdot [\lambda_{X=x}])(T) = \phi(T) \cdot [\lambda_{X=x}](T \cap \{\lambda_{X=x} \}) = \phi(T) \cdot 0 = 0;
    \]
  \item \cref{lemma:full_distribution2};
  \item marginalisation of a probability distribution.
  \end{itemize}
\end{proof}
